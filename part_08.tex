\newpage
\begin{tikzpicture}[remember picture,overlay]
	  \node [rotate=0,scale=2,text opacity=0.2]
	      at (27,1.7) {Капранов~О.~Г.~\copyright~2010~~~Luga\TeX @yahoo.com};
\end{tikzpicture}
\vglue -18pt
\hspace{187pt}
\parbox{350pt}{%
\hypertarget{chapter4a}{\hyperlink{chapter4b}{\mbox{%
\begin{tikzpicture}
  \colorlet{even}{cyan!60!black}
  \colorlet{odd}{orange!100!black}
  \colorlet{links}{red!70!black}
  \colorlet{back}{yellow!20!white}
  \tikzset{
    box/.style={
      minimum height=15mm,
      inner sep=.7mm,
      outer sep=0mm,
      text width=120mm,
      text centered,
      font=\small\bfseries\sffamily,
      text=#1!50!black,
      draw=#1,
      line width=.25mm,
      top color=#1!5,
      bottom color=#1!40,
      shading angle=0,
      rounded corners=2.3mm,
      drop shadow={fill=#1!40!gray,fill opacity=.8},
      rotate=0,
    },
  }
  \node [box=even] {{%
  	\huge\textbf{Методичні рекомендації,}}
	\textbf{плани й завдання до семінарських і практичних занять}};
\end{tikzpicture}
}}}}\\[5pt]
\noindent
\begin{tikzpicture}
  \colorlet{even}{cyan!60!black}
  \colorlet{odd}{orange!100!black}
  \colorlet{links}{red!70!black}
  \colorlet{back}{yellow!20!white}
  \tikzset{
    box/.style={
      minimum height=15mm,
      inner sep=.7mm,
      outer sep=0mm,
      text width=120mm,
      text centered,
      font=\small\bfseries\sffamily,
      text=#1!50!black,
      draw=#1,
      line width=.25mm,
      top color=#1!5,
      bottom color=#1!40,
      shading angle=0,
      rounded corners=2.3mm,
      drop shadow={fill=#1!40!gray,fill opacity=.8},
      rotate=0,
    },
  }
	\node[box=links,xshift=3mm,yshift=1mm,
		minimum height=5pt,text width=325pt]
		at (0,-1.3) {\hyperlink{chapter4a}{\mbox{%
		\small \textcolor{black}{Тема 1.2. Співвідношення
			адміністративного права з іншими галузями}}}};
	\node[box=links,xshift=3mm,yshift=1mm,
		minimum height=5pt,text width=235pt]
		at (10.6,-1.3) {\hyperlink{chapter4b}{\mbox{%
		\small \textcolor{black}{Індивідуальні
		навчально-дослідницькі завдання}}}};
	\node[box=links,xshift=3mm,yshift=1mm,
		minimum height=5pt,text width=221pt]
		at (19.3,-1.3) {\hyperlink{chapter4b}{\mbox{%
		\small \textcolor{black}{
		Питання для самоконтролю та самоперевірки}}}};
	\node[box=links,xshift=3mm,yshift=1mm,
		minimum height=5pt,text width=140pt]
		at (26.35,-1.3) {\hyperlink{chapter4c}{\mbox{%
		\small \textcolor{black}{Додаткова література}}}};
\end{tikzpicture}
\vglue 5pt
\tikzfading[name=targetask, top color=transparent!90,
	bottom color=transparent!90,middle color=transparent!65]
\begin{tikzpicture}
\node [rounded corners,fill=magenta!50,minimum width=960pt,
minimum height=35pt,path fading=targetask] at (0,0) {\mbox{}};
\end{tikzpicture}
\begin{textblock}{39}(0.8,3.75)
\begin{tikzpicture}
	\node [text width=920pt] {\parbox{900pt}{%
	\textbf{Мета заняття:} оволодіння теоретичними положеннями й результатами науки
адміністративного права, її правовими категоріями щодо поняття, предмета,
методу й системи адміністративного права;
\mbox{}\hspace{35pt}розгляд організаційних і настановних питань щодо вивчення курсу
адміністративного права, порядку й форм проведення занять, форм педагогічного
контролю й т.п.}};
\end{tikzpicture}
 \end{textblock}
\begin{textblock}{40}(17,5)
\tikzstyle{abstract}=[rectangle, draw=black, rounded corners, fill=blue!40, drop shadow,
	text centered, text=white, text width=4cm,font=\large\bfseries\sffamily]
\tikzstyle{comment}=[rectangle, draw=black, rounded corners, fill=green, drop shadow,
	text centered, text=white, text width=10cm,
	font=\Large\bfseries\sffamily]
\tikzstyle{myarrow}=[->, >=open triangle 90, thick]
\tikzstyle{line}=[-, thick]
\begin{tikzpicture}[node distance=2cm]
    \node (Seminar) [abstract, rectangle split,
		rectangle split parts=2,text width=5cm] at (0,0)
        {
            Семінарське заняття
            \nodepart{second}1 год
        };

    \node (Thema) [abstract, rectangle split, rectangle split parts=2,
		text width=10cm] at (0,-2) {
            Співвідношення адміністративного права
            \nodepart{second}з іншими галузями
        };

    \node (AuxNode01) [text width=4cm] {};
    \node (Maindir) [abstract, rectangle split,
		rectangle split parts=2] at (0,-4)
        {
            Основні поняття
            \nodepart{second}:
        };

	\node (Maindirnames) [comment, rectangle split,
		rectangle split parts=2, text justified] at (0,-10)
	{
		\mbox{}
		\nodepart{second}\mbox{}
		\newline предмет права
		\newline галузь права
		\newline інститут права
		\newline публічне право
		\newline приватне право
		\newline принципи права
		\newline система права
		\newline метод правового регулювання
		\newline механізм правового регулювання
		\newline \mbox{}
	};
\end{tikzpicture}
\end{textblock}

\vglue 15pt

\tikzstyle{mybox} = [draw=red, fill=blue!20, very thick,
	rectangle, rounded corners, inner sep=10pt, inner ysep=20pt,
	font=\large\bfseries\sffamily]
\tikzstyle{fancytitle} =[fill=red, text=white,font=\large\bfseries\sffamily]

\begin{tikzpicture}
\node [mybox] (box) {%
	\begin{minipage}{0.50\textwidth}
		\begin{itemize}
			\item Особливості предмета й методу адміністративно-правового регулювання.
			\item Співвідношення адміністративного права із суміжними галузями права:
				\begin{itemize}
					\item конституційним правом;
					\item цивільним правом;
					\item кримінальним правом;
				\end{itemize}						
			\item трудовим правом.
			\item Роль адміністративного права на сучасному етапі розвитку суспільних
				відносин.
		\end{itemize}
	\end{minipage}
};
\node[fancytitle, right=10pt] at (box.north west) {Навчальні питання:};
\node[fancytitle, rounded corners] at (box.east) {{\bf ?}};
\end{tikzpicture}	

\vglue 15pt

\tikzstyle{mybox} = [draw=blue, fill=green!20, very thick,
	rectangle, rounded corners, inner sep=10pt, inner ysep=20pt,
	line width=1pt,font=\bfseries\sffamily]
\tikzstyle{fancytitle} =[fill=blue, text=white, ellipse,font=\bfseries\sffamily]

\begin{tikzpicture}[transform shape, rotate=0, baseline=-3.5cm]
	\node [mybox] (box) {%
		\begin{minipage}[t!]{0.5\textwidth}
При підготовці до заняття по даній темі курсанти та студенти повинні усвідомити
предмет, метод і систему, а також соціальне призначення адміністративного
права, ознайомитися з навчальною літературою, вміти розмежовувати категорії
„адміністративне право” як галузі права, науки й навчальної дисципліни,
з'ясувати особливості й зміни предмету, методу й системи галузі права в умовах
побудови демократичної правової держави й переходу до ринкових відносин.
Курсантам і студентам необхідно усвідомити місце адміністративного права серед
фундаментальних галузей права, його взаємозв'язок з конституційним, цивільним,
кримінальним, земельним, фінансовим, трудовим, житловим, природоохоронним й
іншими галузями права.

Визначаючи предмет галузі, необхідно визначити види суспільних відносин
управлінського характеру й підстави включення їх до предмету адміністративного
права. Важливо знати ознаки методів (прийомів і способів)
адміністративно-правового регулювання, юридичну характеристику його трьох
основних компонентів: припису, заборони й дозволу.


Особливу увагу курсанти та студенти повинні приділити питанням системи
адміністративного права, розподілу його на Загальну, Особливу й Спеціальну
частини, а також правовим інститутам державної служби, адміністративної
відповідальності, місцевого самоврядування.

З урахуванням Концепції реформи адміністративного права України варто вивчити
її положення, що стосуються збагачення й уточнення предмета
адміністративно-правового регулювання й системи адміністративного права,
викликані здійсненням адміністративної реформи й побудовою демократичної
правової держави в Україні.


Для кращого засвоєння теми пропонується виконати завдання для самостійної
роботи та індивідуальні навчально-дослідницькі завдання.


Рівень своїх знань з цієї теми пропонується перевірити шляхом надання
відповідей на питання для самоконтролю та самоперевірки.
		\end{minipage}
		};
\node[fancytitle] at (box.north) {Методичні рекомендації та пояснення};
\end{tikzpicture}
%%%-----------------------------------------------------------------------
\begin{textblock}{41}(25,-0.01)
\begin{tikzpicture}[even odd rule,rounded corners=2pt,x=10pt,y=10pt,drop shadow]
\filldraw[fill=yellow!90!black!40,drop shadow] (0,0)   rectangle (1,1)
	[xshift=5pt,yshift=5pt]   (0,0)   rectangle (1,1)
	[rotate=30]   (-1,-1) rectangle (2,2);
\node at (0,1.7) {\textbf{\thepage}};			      
\end{tikzpicture}
\end{textblock}
%%%--- Navigational panel top page
\begin{textblock}{42}(7.58,0.85)
\mbox{%%%--->
\Acrobatmenu{LastPage}{%
\tikz[baseline] \node[rectangle,inner sep=2pt,minimum height=3.1ex,
rounded corners,drop shadow,shadow scale=1,shadow xshift=.8ex,
shadow yshift=-.4ex,opacity=.7,fill=black!50,top color=red!90!black!50,
bottom color=red!80!black!80,draw=red!50!black!50,very thick,text=white,
text opacity=1,minimum width=3cm,font=\bfseries\sffamily] at (0,0) {К концу};
}\Acrobatmenu{GoBack}{%
\tikz[baseline] \node[rectangle,inner sep=2pt,minimum height=3.1ex,
rounded corners,drop shadow,shadow scale=1,shadow xshift=.8ex,
shadow yshift=-.4ex,opacity=.7,fill=black!50,top color=red!90!black!50,
bottom color=red!80!black!80,draw=red!50!black!50,very thick,text=white,
text opacity=1,minimum width=3cm,font=\bfseries\sffamily] at (4,0) {Назад};
}\Acrobatmenu{PrevPage}{%
\tikz[baseline] \node[rectangle,inner sep=2pt,minimum height=3.1ex,
rounded corners,drop shadow,shadow scale=1,shadow xshift=.8ex,
shadow yshift=-.4ex,opacity=.7,fill=black!50,top color=red!90!black!50,
bottom color=red!80!black!80,draw=red!50!black!50,very thick,text=white,
text opacity=1,minimum width=3cm,font=\bfseries\sffamily] at (8,0) {Предыдущий};
}\Acrobatmenu{NextPage}{%
\tikz[baseline] \node[rectangle,inner sep=2pt,minimum height=3.1ex,
rounded corners,drop shadow,shadow scale=1,shadow xshift=.8ex,
shadow yshift=-.4ex,opacity=.7,fill=black!50,top color=red!90!black!50,
bottom color=red!80!black!80,draw=red!50!black!50,very thick,text=white,
text opacity=1,minimum width=3cm,font=\bfseries\sffamily] at (12,0) {Следующий};
}\Acrobatmenu{GoForward}{%
\tikz[baseline] \node[rectangle,inner sep=2pt,minimum height=3.1ex,
rounded corners,drop shadow,shadow scale=1,shadow xshift=.8ex,
shadow yshift=-.4ex,opacity=.7,fill=black!50,top color=red!90!black!50,
bottom color=red!80!black!80,draw=red!50!black!50,very thick,text=white,
text opacity=1,minimum width=3cm,font=\bfseries\sffamily] at (16,0) {Вперед};
}\Acrobatmenu{FirstPage}{%
\tikz[baseline] \node[rectangle,inner sep=2pt,minimum height=3.1ex,
rounded corners,drop shadow,shadow scale=1,shadow xshift=.8ex,
shadow yshift=-.4ex,opacity=.7,fill=black!50,top color=red!90!black!50,
bottom color=red!80!black!80,draw=red!50!black!50,very thick,text=white,
text opacity=1,minimum width=3cm,font=\bfseries\sffamily] at (20,0) {К началу};
}\Acrobatmenu{FullScreen}{%
\tikz[baseline] \node[rectangle,inner sep=2pt,minimum height=3.1ex,
rounded corners,drop shadow,shadow scale=1,shadow xshift=.8ex,
shadow yshift=-.4ex,opacity=.7,fill=black!50,top color=red!90!black!50,
bottom color=red!80!black!80,draw=red!50!black!50,very thick,text=white,
text opacity=1,minimum width=3cm,font=\bfseries\sffamily] at (24,0) {Полный экран};
}\Acrobatmenu{Quit}{%
\tikz[baseline] \node[rectangle,inner sep=2pt,minimum height=3.1ex,
rounded corners,drop shadow,shadow scale=1,shadow xshift=.8ex,
shadow yshift=-.4ex,opacity=.7,fill=black!50,top color=red!90!black!50,
bottom color=red!80!black!80,draw=red!50!black!50,very thick,text=white,
text opacity=1,minimum width=3cm,font=\bfseries\sffamily] at (28,0) {Выход};
}	
}%%%---|
\end{textblock}
%%%-----------------------------------------------------------------------
%%%---> NEW PAGE ---------------------------------------------------------
\newpage
\begin{tikzpicture}[remember picture,overlay]
	  \node [rotate=0,scale=2,text opacity=0.2]
	      at (27,1.7) {Капранов~О.~Г.~\copyright~2010~~~Luga\TeX @yahoo.com};
\end{tikzpicture}
%%%-----------------------------------------------------------------------
\begin{textblock}{43}(25,-0.01)
\begin{tikzpicture}[even odd rule,rounded corners=2pt,x=10pt,y=10pt,drop shadow]
\filldraw[fill=yellow!90!black!40,drop shadow] (0,0)   rectangle (1,1)
	[xshift=5pt,yshift=5pt]   (0,0)   rectangle (1,1)
	[rotate=30]   (-1,-1) rectangle (2,2);
\node at (0,1.7) {\textbf{\thepage}};			      
\end{tikzpicture}
\end{textblock}
%%%--- Navigational panel top page
\begin{textblock}{44}(7.58,0.85)
\mbox{%%%--->
\Acrobatmenu{LastPage}{%
\tikz[baseline] \node[rectangle,inner sep=2pt,minimum height=3.1ex,
rounded corners,drop shadow,shadow scale=1,shadow xshift=.8ex,
shadow yshift=-.4ex,opacity=.7,fill=black!50,top color=red!90!black!50,
bottom color=red!80!black!80,draw=red!50!black!50,very thick,text=white,
text opacity=1,minimum width=3cm,font=\bfseries\sffamily] at (0,0) {К концу};
}\Acrobatmenu{GoBack}{%
\tikz[baseline] \node[rectangle,inner sep=2pt,minimum height=3.1ex,
rounded corners,drop shadow,shadow scale=1,shadow xshift=.8ex,
shadow yshift=-.4ex,opacity=.7,fill=black!50,top color=red!90!black!50,
bottom color=red!80!black!80,draw=red!50!black!50,very thick,text=white,
text opacity=1,minimum width=3cm,font=\bfseries\sffamily] at (4,0) {Назад};
}\Acrobatmenu{PrevPage}{%
\tikz[baseline] \node[rectangle,inner sep=2pt,minimum height=3.1ex,
rounded corners,drop shadow,shadow scale=1,shadow xshift=.8ex,
shadow yshift=-.4ex,opacity=.7,fill=black!50,top color=red!90!black!50,
bottom color=red!80!black!80,draw=red!50!black!50,very thick,text=white,
text opacity=1,minimum width=3cm,font=\bfseries\sffamily] at (8,0) {Предыдущий};
}\Acrobatmenu{NextPage}{%
\tikz[baseline] \node[rectangle,inner sep=2pt,minimum height=3.1ex,
rounded corners,drop shadow,shadow scale=1,shadow xshift=.8ex,
shadow yshift=-.4ex,opacity=.7,fill=black!50,top color=red!90!black!50,
bottom color=red!80!black!80,draw=red!50!black!50,very thick,text=white,
text opacity=1,minimum width=3cm,font=\bfseries\sffamily] at (12,0) {Следующий};
}\Acrobatmenu{GoForward}{%
\tikz[baseline] \node[rectangle,inner sep=2pt,minimum height=3.1ex,
rounded corners,drop shadow,shadow scale=1,shadow xshift=.8ex,
shadow yshift=-.4ex,opacity=.7,fill=black!50,top color=red!90!black!50,
bottom color=red!80!black!80,draw=red!50!black!50,very thick,text=white,
text opacity=1,minimum width=3cm,font=\bfseries\sffamily] at (16,0) {Вперед};
}\Acrobatmenu{FirstPage}{%
\tikz[baseline] \node[rectangle,inner sep=2pt,minimum height=3.1ex,
rounded corners,drop shadow,shadow scale=1,shadow xshift=.8ex,
shadow yshift=-.4ex,opacity=.7,fill=black!50,top color=red!90!black!50,
bottom color=red!80!black!80,draw=red!50!black!50,very thick,text=white,
text opacity=1,minimum width=3cm,font=\bfseries\sffamily] at (20,0) {К началу};
}\Acrobatmenu{FullScreen}{%
\tikz[baseline] \node[rectangle,inner sep=2pt,minimum height=3.1ex,
rounded corners,drop shadow,shadow scale=1,shadow xshift=.8ex,
shadow yshift=-.4ex,opacity=.7,fill=black!50,top color=red!90!black!50,
bottom color=red!80!black!80,draw=red!50!black!50,very thick,text=white,
text opacity=1,minimum width=3cm,font=\bfseries\sffamily] at (24,0) {Полный экран};
}\Acrobatmenu{Quit}{%
\tikz[baseline] \node[rectangle,inner sep=2pt,minimum height=3.1ex,
rounded corners,drop shadow,shadow scale=1,shadow xshift=.8ex,
shadow yshift=-.4ex,opacity=.7,fill=black!50,top color=red!90!black!50,
bottom color=red!80!black!80,draw=red!50!black!50,very thick,text=white,
text opacity=1,minimum width=3cm,font=\bfseries\sffamily] at (28,0) {Выход};
}	
}%%%---|
\end{textblock}
%%%-----------------------------------------------------------------------
\vglue -18pt
\hspace{187pt}
\parbox{350pt}{%
\hypertarget{chapter4b}{\hyperlink{chapter4c}{\mbox{%
\begin{tikzpicture}
  \colorlet{even}{cyan!60!black}
  \colorlet{odd}{orange!100!black}
  \colorlet{links}{red!70!black}
  \colorlet{back}{yellow!20!white}
  \tikzset{
    box/.style={
      minimum height=15mm,
      inner sep=.7mm,
      outer sep=0mm,
      text width=120mm,
      text centered,
      font=\small\bfseries\sffamily,
      text=#1!50!black,
      draw=#1,
      line width=.25mm,
      top color=#1!5,
      bottom color=#1!40,
      shading angle=0,
      rounded corners=2.3mm,
      drop shadow={fill=#1!40!gray,fill opacity=.8},
      rotate=0,
    },
  }
  \node [box=even] {{%
  	\huge\textbf{Методичні рекомендації,}}
	\textbf{плани й завдання до семінарських і практичних занять}};
\end{tikzpicture}
}}}}\\[5pt]
\noindent
\begin{tikzpicture}
  \colorlet{even}{cyan!60!black}
  \colorlet{odd}{orange!100!black}
  \colorlet{links}{red!70!black}
  \colorlet{back}{yellow!20!white}
  \tikzset{
    box/.style={
      minimum height=15mm,
      inner sep=.7mm,
      outer sep=0mm,
      text width=120mm,
      text centered,
      font=\small\bfseries\sffamily,
      text=#1!50!black,
      draw=#1,
      line width=.25mm,
      top color=#1!5,
      bottom color=#1!40,
      shading angle=0,
      rounded corners=2.3mm,
      drop shadow={fill=#1!40!gray,fill opacity=.8},
      rotate=0,
    },
  }
	\node[box=links,xshift=3mm,yshift=1mm,
		minimum height=5pt,text width=325pt]
		at (0,-1.3) {\hyperlink{chapter4a}{\mbox{%
		\small \textcolor{black}{Тема 1.2. Співвідношення
			адміністративного права з іншими галузями}}}};
	\node[box=links,xshift=3mm,yshift=1mm,
		minimum height=5pt,text width=235pt]
		at (10.6,-1.3) {\hyperlink{chapter4b}{\mbox{%
		\small \textcolor{black}{Індивідуальні
		навчально-дослідницькі завдання}}}};
	\node[box=links,xshift=3mm,yshift=1mm,
		minimum height=5pt,text width=221pt]
		at (19.3,-1.3) {\hyperlink{chapter4b}{\mbox{%
		\small \textcolor{black}{
		Питання для самоконтролю та самоперевірки}}}};
	\node[box=links,xshift=3mm,yshift=1mm,
		minimum height=5pt,text width=140pt]
		at (26.35,-1.3) {\hyperlink{chapter4c}{\mbox{%
		\small \textcolor{black}{Додаткова література}}}};
\end{tikzpicture}

\vfill

%%%---> Old version
%\begin{tikzpicture}[remember picture, note/.style={rectangle
%	callout,fill=#1}]
%	\node [note=green!50,opacity=.5,overlay,text opacity=1,
%		font=\large\bfseries\sffamily, callout relative pointer={(-5,-1)},
%			callout pointer width=1.3cm] at (15,1) {%
%Індивідуальні навчально-дослідницькі завдання:
%};
%\end{tikzpicture}
%
%\begin{itemize}
%	\item[] \tikz[baseline] \node[ball color=magenta,circle,text=black,
%			minimum size=4pt]
%		{1}; \quad {\large\textbf{%
%			Підготувати реферат за темою: <<Особливості предмету
%			галузі в світлі Концепції реформи адміністративного права
%			України>>.}}
%
%\item[] \tikz[baseline] \node[ball color=magenta,circle,text=black]
%		{2}; \quad {\large\textbf{%
%			Підготувати реферат за темою: <<Співвідношення методів
%			адміністративно-правового й цивільно-правового регулювання
%			суспільних відносин>>.}}
%
%\item[] \tikz[baseline] \node[ball color=magenta,circle,text=black]
%		{3}; \quad {\large\textbf{%
%			Підготувати доповідь за темою: <<Принципи адміністративного
%			права: удосконалення системи>>.}}
%\end{itemize}
%%%--->
%%%---> New version
\begin{flushleft}
\begin{tikzpicture}
\node (tbl) {
\begin{tabularx}{980pt}{l}
\arrayrulecolor{purple}
\multicolumn{1}{c}{\mbox{}\hspace{50pt}\mbox{\textcolor{white}{{%
	\large\bfseries\sffamily Індивідуальні навчально--дослідницькі
   		завдання}}}}\\[15pt]
\tikz[baseline] \node[ball color=green,circle, text=white] {{\bf 1}};\qquad
\tikz[baseline] \node[font=\large\bfseries\sffamily,
	text=black] {Підготувати реферат за темою: <<Особливості
   		предмету галузі в світлі Концепції реформи адміністративного
	   	права України>>};\\[15pt]
		\tikz[baseline] \node[ball color=green,circle, text=white] {{\bf 2}};\qquad
\tikz[baseline] \node[font=\large\bfseries\sffamily,
	text=black] {Підготувати реферат за темою: <<Співвідношення
   		методів адміністративно--правового й цивільно--правового
	   	регулювання суспільних відносин>>};\\[15pt]
		\tikz[baseline] \node[ball color=green,circle, text=white] {{\bf 3}};\qquad
\tikz[baseline] \node[font=\large\bfseries\sffamily,
	text=black] {Підготувати доповідь за темою: <<Принципи
   		адміністративного права: удосконалення системи>>};\\[15pt]
\end{tabularx}};

\begin{pgfonlayer}{background}
\draw[rounded corners,top color=red,bottom color=black,draw=white]
	($(tbl.north west)+(0.14,0)$) rectangle ($(tbl.north east)-(0.13,0.9)$);
\draw[rounded corners,top color=white,bottom color=black,
	middle color=red,draw=blue!20] ($(tbl.south west) +(0.12,0.5)$)
		rectangle ($(tbl.south east)-(0.12,0)$);
\draw[top color=blue!1,bottom color=blue!20,draw=white]
	($(tbl.north east)-(0.13,0.6)$) rectangle ($(tbl.south west)+(0.13,0.2)$);
\end{pgfonlayer}
\end{tikzpicture}
\end{flushleft}
%%%---> Old version
%\vglue 45pt
%
%\begin{tikzpicture}[remember picture, note/.style={rectangle
%	callout,fill=#1}]
%	\node [note=green!50,opacity=.5,overlay,text opacity=1,
%		font=\large\bfseries\sffamily, callout relative pointer={(-5,-1)},
%			callout pointer width=1.3cm] at (15,1) {%
%Питання для самоконтролю та самоперевірки:
%};
%\end{tikzpicture}
%
%\begin{itemize}
%\item[] \tikz[baseline] \node[ball color=magenta,circle,text=black]
%	{1}; \quad {\large\textbf{%
%У яких значеннях вживається термін <<адміністративне право України>>?
%}}
%\item[] \tikz[baseline] \node[ball color=magenta,circle,text=black]
%	{2}; \quad {\large\textbf{%
%Як співвідносяться конституційне й адміністративне права? Розкрийте їх
%поняття і значення.
%}}
%\item[] \tikz[baseline] \node[ball color=magenta,circle,text=black]
%	{3}; \quad {\large\textbf{%
%Поясніть значення термінів <<предмет адміністративного права>>, <<метод
%адміністративного права>>, <<механізм адміністративно-правового
%регулювання>>.
%}}
%\item[] \tikz[baseline] \node[ball color=magenta,circle,text=black]
%	{4}; \quad {\large\textbf{%
%Розкрийте співвідношення адміністративного права із суміжними галузями
%права.
%}}
%\item[] \tikz[baseline] \node[ball color=magenta,circle,text=black]
%	{5}; \quad {\large\textbf{%
%Розкрийте зміст поняття <<система адміністративного права>>?
%}}
%
%\item[] \tikz[baseline] \node[ball color=magenta,circle,text=black]
%	{6}; \quad {\large\textbf{%
%Дайте правову характеристику головних елементів Загальної й Особливої
%частин адміністративного права.
%}}
%\item[] \tikz[baseline] \node[ball color=magenta,circle,text=black]
%	{7}; \quad {\large\textbf{%
%Що таке Спеціальна частина адміністративного права?
%}}
%\item[] \tikz[baseline] \node[ball color=magenta,circle,text=black]
%	{8}; \quad {\large\textbf{%
%Що таке <<принципи права>>?
%}}
%\item[] \tikz[baseline] \node[ball color=magenta,circle,text=black]
%	{9}; \quad {\large\textbf{%
%Для чого необхідне вивчення адміністративного права? Аргументуйте свою
%відповідь.
%}}
%\item[] \tikz[baseline] \node[ball color=magenta,circle,text=black]
%	{10}; \quad {\large\textbf{%
%У чому виражаються публічний і приватний аспекти адміністративного права, їх
%взаємозв'язок?
%}}
%\item[] \tikz[baseline] \node[ball color=magenta,circle,text=black]
%	{11}; \quad {\large\textbf{%
%Дайте історичну характеристику розвитку галузі адміністративного права в
%Україні.
%}}
%\end{itemize}
%%%--->
%%%---> NEW Version
\vfill
\begin{flushleft}
\begin{tikzpicture}
\node (tbl) {
\begin{tabularx}{980pt}{l}
\arrayrulecolor{purple}
\multicolumn{1}{c}{\textcolor{white}{{\large\bfseries\sffamily Питання
для самоконтролю та самоперевірки}}}\\[15pt]
\tikz[baseline] \node[ball color=magenta, circle,
	minimum size=0.8cm, text=white] {{\bf 1}};\qquad
\tikz[baseline] \node[font=\large\bfseries\sffamily,
	text=black] {%
У яких значеннях вживається термін <<адміністративне право України>>?
};\\[18pt]
\tikz[baseline] \node[ball color=magenta, circle,
	minimum size=0.8cm, text=white] {{\bf 2}};\qquad
\tikz[baseline] \node[font=\large\bfseries\sffamily,
	text=black] {%
Як співвідносяться конституційне й адміністративне права? Розкрийте їх
поняття і значення.
};\\[18pt]
\tikz[baseline] \node[ball color=magenta, circle,
	minimum size=0.8cm, text=white] {{\bf 3}};\qquad
\tikz[baseline] \node[font=\large\bfseries\sffamily,
	text=black] {%
Поясніть значення термінів <<предмет адміністративного права>>, <<метод
адміністративного права>>, <<механізм адміністративно-правового
регулювання>>.
};\\[18pt]
\tikz[baseline] \node[ball color=magenta, circle,
	minimum size=0.8cm, text=white] {{\bf 4}};\qquad
\tikz[baseline] \node[font=\large\bfseries\sffamily,
	text=black] {%
Розкрийте співвідношення адміністративного права із суміжними галузями
права.
};\\[18pt]
\tikz[baseline] \node[ball color=magenta, circle,
	minimum size=0.8cm, text=white] {{\bf 5}};\qquad
\tikz[baseline] \node[font=\large\bfseries\sffamily,
	text=black] {%
Розкрийте зміст поняття <<система адміністративного права>>?
};\\[18pt]
\tikz[baseline] \node[ball color=magenta, circle,
	minimum size=0.8cm, text=white] {{\bf 6}};\qquad
\tikz[baseline] \node[font=\large\bfseries\sffamily,
	text=black] {%
Дайте правову характеристику головних елементів Загальної й Особливої
частин адміністративного права.
};\\[18pt]
\tikz[baseline] \node[ball color=magenta, circle,
	minimum size=0.8cm, text=white] {{\bf 7}};\qquad
\tikz[baseline] \node[font=\large\bfseries\sffamily,
	text=black] {%
Що таке Спеціальна частина адміністративного права?
};\\[18pt]
\tikz[baseline] \node[ball color=magenta, circle,
	minimum size=0.8cm, text=white] {{\bf 8}};\qquad
\tikz[baseline] \node[font=\large\bfseries\sffamily,
	text=black] {%
Що таке <<принципи права>>?
};\\[18pt]
\tikz[baseline] \node[ball color=magenta, circle,
	minimum size=0.8cm,text=white] {{\bf 9}};\qquad
\tikz[baseline] \node[font=\large\bfseries\sffamily,
	text=black] {%
Для чого необхідне вивчення адміністративного права? Аргументуйте свою
відповідь.
};\\[18pt]
\tikz[baseline] \node[ball color=magenta,circle,
	minimum size=0.8cm, text=white] {{\bf 10}};\qquad
\tikz[baseline] \node[font=\large\bfseries\sffamily,
	text=black] {%
У чому виражаються публічний і приватний аспекти адміністративного права, їх
взаємозв'язок?
};\\[18pt]
\tikz[baseline] \node[ball color=magenta, circle,
	minimum size=0.8cm, text=white] {{\bf 11}};\qquad
\tikz[baseline] \node[font=\large\bfseries\sffamily,
	text=black] {%
Дайте історичну характеристику розвитку галузі адміністративного права в
Україні.
};\\[18pt]
\end{tabularx}};

\begin{pgfonlayer}{background}
\draw[rounded corners,top color=red,bottom color=black,draw=white]
	($(tbl.north west)+(0.14,0)$) rectangle ($(tbl.north east)-(0.13,0.9)$);
\draw[rounded corners,top color=white,bottom color=black,
	middle color=red,draw=blue!20] ($(tbl.south west) +(0.12,0.5)$)
		rectangle ($(tbl.south east)-(0.12,0)$);
\draw[top color=blue!1,bottom color=blue!20,draw=white]
	($(tbl.north east)-(0.13,0.6)$) rectangle ($(tbl.south west)+(0.13,0.2)$);
\end{pgfonlayer}
\end{tikzpicture}
\end{flushleft}
\vfill
\begin{flushright}
\tikz[baseline] \node[rectangle,inner sep=2pt,minimum height=3.1ex,rounded corners,
drop shadow,shadow scale=1,shadow xshift=.8ex,shadow yshift=-.4ex,opacity=.7,
fill=black!50,top color=blue!90!black!50,bottom color=blue!80!black!80,
draw=blue!50!black!50,very thick,text=white,text opacity=1,minimum width=4cm]{%
Тестовые модули};\hspace{10pt}\mbox{}
\end{flushright}	
\vfill
%%%---> NEW PAGE ---------------------------------------------------------
\newpage
\begin{tikzpicture}[remember picture,overlay]
	  \node [rotate=0,scale=2,text opacity=0.2]
	      at (27,1.7) {Капранов~О.~Г.~\copyright~2010~~~Luga\TeX @yahoo.com};
\end{tikzpicture}
\vglue -18pt
\hspace{187pt}
\parbox{350pt}{%
\hypertarget{chapter4c}{\hyperlink{chapter4d}{\mbox{%
\begin{tikzpicture}
  \colorlet{even}{cyan!60!black}
  \colorlet{odd}{orange!100!black}
  \colorlet{links}{red!70!black}
  \colorlet{back}{yellow!20!white}
  \tikzset{
    box/.style={
      minimum height=15mm,
      inner sep=.7mm,
      outer sep=0mm,
      text width=120mm,
      text centered,
      font=\small\bfseries\sffamily,
      text=#1!50!black,
      draw=#1,
      line width=.25mm,
      top color=#1!5,
      bottom color=#1!40,
      shading angle=0,
      rounded corners=2.3mm,
      drop shadow={fill=#1!40!gray,fill opacity=.8},
      rotate=0,
    },
  }
  \node [box=even] {{%
  	\huge\textbf{Методичні рекомендації,}}
	\textbf{плани й завдання до семінарських і практичних занять}};
\end{tikzpicture}
}}}}\\[5pt]
\noindent
\begin{tikzpicture}
  \colorlet{even}{cyan!60!black}
  \colorlet{odd}{orange!100!black}
  \colorlet{links}{red!70!black}
  \colorlet{back}{yellow!20!white}
  \tikzset{
    box/.style={
      minimum height=15mm,
      inner sep=.7mm,
      outer sep=0mm,
      text width=120mm,
      text centered,
      font=\small\bfseries\sffamily,
      text=#1!50!black,
      draw=#1,
      line width=.25mm,
      top color=#1!5,
      bottom color=#1!40,
      shading angle=0,
      rounded corners=2.3mm,
      drop shadow={fill=#1!40!gray,fill opacity=.8},
      rotate=0,
    },
  }
	\node[box=links,xshift=3mm,yshift=1mm,
		minimum height=5pt,text width=325pt]
		at (0,-1.3) {\hyperlink{chapter4a}{\mbox{%
		\small \textcolor{black}{Тема 1.2. Співвідношення
			адміністративного права з іншими галузями}}}};
	\node[box=links,xshift=3mm,yshift=1mm,
		minimum height=5pt,text width=235pt]
		at (10.6,-1.3) {\hyperlink{chapter4b}{\mbox{%
		\small \textcolor{black}{Індивідуальні
		навчально-дослідницькі завдання}}}};
	\node[box=links,xshift=3mm,yshift=1mm,
		minimum height=5pt,text width=221pt]
		at (19.3,-1.3) {\hyperlink{chapter4b}{\mbox{%
		\small \textcolor{black}{
		Питання для самоконтролю та самоперевірки}}}};
	\node[box=links,xshift=3mm,yshift=1mm,
		minimum height=5pt,text width=140pt]
		at (26.35,-1.3) {\hyperlink{chapter4c}{\mbox{%
		\small \textcolor{black}{Додаткова література}}}};
\end{tikzpicture}
\vglue 25pt
%%%---> Old Version
%\begin{tikzpicture}
%	\node[name=s,shape=rectangle callout,
%		callout relative pointer={(1.25cm,-1cm)},
%			callout pointer width=2cm, inner xsep=2cm, inner ysep=1cm,
%				font=\Large\bfseries\sffamily, text centered, fill=black,
%					shading angle=45, drop shadow,
%						text=white] at (3,0) {Додаткова література};
%\end{tikzpicture}
%%%---<
%%%---> NEW Version
\hyperlink{chapter4d}{\mbox{%
\begin{tikzpicture}
	\node[name=s,shape=rectangle callout,
		callout relative pointer={(1.25cm,-1cm)},
		callout pointer width=2cm, inner xsep=2cm, inner ysep=1cm,
		font=\Large\bfseries\sffamily, text centered,
		shading angle=45, drop shadow,
		postaction={path fading=south,
		fading angle=45,fill=blue, opacity=.5},
		left color=black, right color=red, draw=white,
		line width=2mm, text=white, drop shadow,
		shadow scale=1.25, shadow xshift=0pt,
		shadow yshift=0pt]
	at (3,0) {Додаткова література};
\end{tikzpicture}
}}
%%%---<
\vfill
\begin{itemize}
\item[]
	\scalebox{1.9}{\iconbook}\qquad 
\tooltipanim{\large\bf Административное право зарубежных стран}{21}{21}
\quad{\large{\bf Учебник}. Под ред. А.Н. {\bf Козырина}, М.А.
{\bf Штатиной}. --- {\bf М}., {\bf 2003}. --- 464 с.}
\item[]
\scalebox{1.9}{\iconbook}\qquad
\tooltipanim{\large\bf Виконавча влада і адміністративне право}{22}{22}
\quad{\large За заг. ред В.Б.{\bf Авер'янова}. --- К.,
{\bf 2002}. --- 668 с.}
\item[]
\scalebox{1.9}{\iconarticle}\qquad~~{\large {\bf Авер'янов} В.Б.}
\tooltipanim{\large\bf Актуальні завдання реформування адміністративного права}{23}{23}
\quad{\large{\bf Право України}. --- {\bf 1999}. --- {\bf №8}. --- С. 8.}
\item[]
\scalebox{1.9}{\iconarticle}\qquad~~{\large {\bf Авер'янов} В.Б.}
\tooltipanim{\large\bf Реформування українського адміністративного права}{24}{24}
\quad{\large черговий етап. {\bf Право України}. --- {\bf 2000}. --- {\bf №7}.
--- С. 6.}
\item[]
\scalebox{1.9}{\iconarticle}\qquad~~{\large {\bf Авер`янов} В.Б.}
\tooltipanim{\large\bf Предмет адміністративного права}{25}{25}
\quad{\large нова доктринальна оцінка.
{\bf Право України}. --- {\bf 2004}. --- {\bf №10}. --- С. 25.}
\item[]
\scalebox{1.9}{\iconarticle}\qquad~~{\large {\bf Антонова} В.П.}
\tooltipanim{\large\bf Институты административного права}{23}{23}
\quad{\large (третьи <<Лазаревские чтения>>). {\bf Государство и право}.
	--- {\bf 1999}. --- {\bf №10}.  --- С. 5.}
\item[]
\scalebox{1.9}{\iconarticle}\qquad~~{\large {\bf Афанасьєв} К.К.}
\tooltipanim{\large\bf К вопросу о предмете административно--правового регулирования}{24}{24}
\quad{\large	{\bf Вісник ЛАВС МВС України}.  --- {\bf 2002}. ---  {\bf №3}.
--- С. 53-65.}
\item[]
\scalebox{1.9}{\iconarticle}\qquad~~{\large  {\bf Бельский} К.С.}
\tooltipanim{\large\bf О предмете и системе науки административного права}{25}{25}
\quad{\large	{\bf Государство и право}. --- {\bf 1998}. --- {\bf №10}. --- С. 18.}
\item[]
\scalebox{1.9}{\iconarticle}\qquad~~{\large {\bf Бельский} К.С.}
\tooltipanim{\large\bf О системе административного права}{23}{23}
\quad{\large{\bf Государство и право}.  --- {\bf 1998}. --- {\bf №3}. --- С. 5.}
\item[]
\scalebox{1.9}{\iconbook}\qquad \parbox{850pt}{%
{\large {\bf Головін} А.П., {\bf Нікітенко} О.І.}
\tooltipanim{\large\bf До питання про предмет та систему адміністративного права}{21}{21}
\quad{\large{\bf Українське адміністративне право}: актуальні проблеми реформування: {\bf Збірник наукових праць}. ---
{\bf Суми}: ВВП <<Мрія--1>> ЛТД: Ініціатива, {\bf 2000}.--282 с.}}
\item[]
\scalebox{1.9}{\iconarticle}\qquad~~\parbox{830pt}{%
{\large {\bf Данильева} И. Э.}
\tooltipanim{\large\bf Понятие и значение принципов права в регулировании административных правоотношений}{23}{23}\quad  {\large{\bf Вісник} Запорізького державного університету: Зб. наук. статей.
Юридичні науки. --- {\bf Запоріжжя}: Запорізький державний університет.
--- {\bf 2004}. --- {\bf №1}. --- С. 74--78.}}
\end{itemize}
\vfill
\begin{textblock}{45}(22.1,15)
	\hyperlink{chapter4d}{\mbox{%
	\tikz[every node/.style={font=\normalsize\bfseries\sffamily,signal,draw,
		text=white,signal to=nowhere}]
		\node[signal to=east, minimum width=35pt, postaction={path fading=south,
			fading angle=45,fill=blue,opacity=.5}, left color=black, right color=red,
				draw=white, line width=2mm, drop shadow,
					minimum height=8pt]
			{Додаткова література: 2~стр};
			}}
\end{textblock}
\vfill
\mbox{}
%%%-----------------------------------------------------------------------
\begin{textblock}{46}(25,-0.01)
\begin{tikzpicture}[even odd rule,rounded corners=2pt,x=10pt,y=10pt,drop shadow]
\filldraw[fill=yellow!90!black!40,drop shadow] (0,0)   rectangle (1,1)
	[xshift=5pt,yshift=5pt]   (0,0)   rectangle (1,1)
	[rotate=30]   (-1,-1) rectangle (2,2);
\node at (0,1.7) {\textbf{\thepage}};			      
\end{tikzpicture}
\end{textblock}
%%%--- Navigational panel top page
\begin{textblock}{47}(7.58,0.85)
\mbox{%%%--->
\Acrobatmenu{LastPage}{%
\tikz[baseline] \node[rectangle,inner sep=2pt,minimum height=3.1ex,
rounded corners,drop shadow,shadow scale=1,shadow xshift=.8ex,
shadow yshift=-.4ex,opacity=.7,fill=black!50,top color=red!90!black!50,
bottom color=red!80!black!80,draw=red!50!black!50,very thick,text=white,
text opacity=1,minimum width=3cm,font=\bfseries\sffamily] at (0,0) {К концу};
}\Acrobatmenu{GoBack}{%
\tikz[baseline] \node[rectangle,inner sep=2pt,minimum height=3.1ex,
rounded corners,drop shadow,shadow scale=1,shadow xshift=.8ex,
shadow yshift=-.4ex,opacity=.7,fill=black!50,top color=red!90!black!50,
bottom color=red!80!black!80,draw=red!50!black!50,very thick,text=white,
text opacity=1,minimum width=3cm,font=\bfseries\sffamily] at (4,0) {Назад};
}\Acrobatmenu{PrevPage}{%
\tikz[baseline] \node[rectangle,inner sep=2pt,minimum height=3.1ex,
rounded corners,drop shadow,shadow scale=1,shadow xshift=.8ex,
shadow yshift=-.4ex,opacity=.7,fill=black!50,top color=red!90!black!50,
bottom color=red!80!black!80,draw=red!50!black!50,very thick,text=white,
text opacity=1,minimum width=3cm,font=\bfseries\sffamily] at (8,0) {Предыдущий};
}\Acrobatmenu{NextPage}{%
\tikz[baseline] \node[rectangle,inner sep=2pt,minimum height=3.1ex,
rounded corners,drop shadow,shadow scale=1,shadow xshift=.8ex,
shadow yshift=-.4ex,opacity=.7,fill=black!50,top color=red!90!black!50,
bottom color=red!80!black!80,draw=red!50!black!50,very thick,text=white,
text opacity=1,minimum width=3cm,font=\bfseries\sffamily] at (12,0) {Следующий};
}\Acrobatmenu{GoForward}{%
\tikz[baseline] \node[rectangle,inner sep=2pt,minimum height=3.1ex,
rounded corners,drop shadow,shadow scale=1,shadow xshift=.8ex,
shadow yshift=-.4ex,opacity=.7,fill=black!50,top color=red!90!black!50,
bottom color=red!80!black!80,draw=red!50!black!50,very thick,text=white,
text opacity=1,minimum width=3cm,font=\bfseries\sffamily] at (16,0) {Вперед};
}\Acrobatmenu{FirstPage}{%
\tikz[baseline] \node[rectangle,inner sep=2pt,minimum height=3.1ex,
rounded corners,drop shadow,shadow scale=1,shadow xshift=.8ex,
shadow yshift=-.4ex,opacity=.7,fill=black!50,top color=red!90!black!50,
bottom color=red!80!black!80,draw=red!50!black!50,very thick,text=white,
text opacity=1,minimum width=3cm,font=\bfseries\sffamily] at (20,0) {К началу};
}\Acrobatmenu{FullScreen}{%
\tikz[baseline] \node[rectangle,inner sep=2pt,minimum height=3.1ex,
rounded corners,drop shadow,shadow scale=1,shadow xshift=.8ex,
shadow yshift=-.4ex,opacity=.7,fill=black!50,top color=red!90!black!50,
bottom color=red!80!black!80,draw=red!50!black!50,very thick,text=white,
text opacity=1,minimum width=3cm,font=\bfseries\sffamily] at (24,0) {Полный экран};
}\Acrobatmenu{Quit}{%
\tikz[baseline] \node[rectangle,inner sep=2pt,minimum height=3.1ex,
rounded corners,drop shadow,shadow scale=1,shadow xshift=.8ex,
shadow yshift=-.4ex,opacity=.7,fill=black!50,top color=red!90!black!50,
bottom color=red!80!black!80,draw=red!50!black!50,very thick,text=white,
text opacity=1,minimum width=3cm,font=\bfseries\sffamily] at (28,0) {Выход};
}	
}%%%---|
\end{textblock}
%%%-----------------------------------------------------------------------
%%%---> NEW PAGE ----------------------------------------------------------
\newpage
\begin{tikzpicture}[remember picture,overlay]
	  \node [rotate=0,scale=2,text opacity=0.2]
	      at (27,1.7) {Капранов~О.~Г.~\copyright~2010~~~Luga\TeX @yahoo.com};
\end{tikzpicture}
\vglue -18pt
\hspace{187pt}
\parbox{350pt}{%
\hypertarget{chapter4d}{\hyperlink{chapter5a}{\mbox{%
\begin{tikzpicture}
  \colorlet{even}{cyan!60!black}
  \colorlet{odd}{orange!100!black}
  \colorlet{links}{red!70!black}
  \colorlet{back}{yellow!20!white}
  \tikzset{
    box/.style={
      minimum height=15mm,
      inner sep=.7mm,
      outer sep=0mm,
      text width=120mm,
      text centered,
      font=\small\bfseries\sffamily,
      text=#1!50!black,
      draw=#1,
      line width=.25mm,
      top color=#1!5,
      bottom color=#1!40,
      shading angle=0,
      rounded corners=2.3mm,
      drop shadow={fill=#1!40!gray,fill opacity=.8},
      rotate=0,
    },
  }
  \node [box=even] {{%
  	\huge\textbf{Методичні рекомендації,}}
	\textbf{плани й завдання до семінарських і практичних занять}};
\end{tikzpicture}
}}}}\\[5pt]
\noindent
\begin{tikzpicture}
  \colorlet{even}{cyan!60!black}
  \colorlet{odd}{orange!100!black}
  \colorlet{links}{red!70!black}
  \colorlet{back}{yellow!20!white}
  \tikzset{
    box/.style={
      minimum height=15mm,
      inner sep=.7mm,
      outer sep=0mm,
      text width=120mm,
      text centered,
      font=\small\bfseries\sffamily,
      text=#1!50!black,
      draw=#1,
      line width=.25mm,
      top color=#1!5,
      bottom color=#1!40,
      shading angle=0,
      rounded corners=2.3mm,
      drop shadow={fill=#1!40!gray,fill opacity=.8},
      rotate=0,
    },
  }
	\node[box=links,xshift=3mm,yshift=1mm,
		minimum height=5pt,text width=325pt]
		at (0,-1.3) {\hyperlink{chapter4a}{\mbox{%
		\small \textcolor{black}{Тема 1.2. Співвідношення
			адміністративного права з іншими галузями}}}};
	\node[box=links,xshift=3mm,yshift=1mm,
		minimum height=5pt,text width=235pt]
		at (10.6,-1.3) {\hyperlink{chapter4b}{\mbox{%
		\small \textcolor{black}{Індивідуальні
		навчально-дослідницькі завдання}}}};
	\node[box=links,xshift=3mm,yshift=1mm,
		minimum height=5pt,text width=221pt]
		at (19.3,-1.3) {\hyperlink{chapter4b}{\mbox{%
		\small \textcolor{black}{
		Питання для самоконтролю та самоперевірки}}}};
	\node[box=links,xshift=3mm,yshift=1mm,
		minimum height=5pt,text width=140pt]
		at (26.35,-1.3) {\hyperlink{chapter4c}{\mbox{%
		\small \textcolor{black}{Додаткова література}}}};
\end{tikzpicture}
\vglue 25pt
%%%---> Old Version
%\begin{tikzpicture}
%	\node[name=s,shape=rectangle callout,
%		callout relative pointer={(1.25cm,-1cm)},
%			callout pointer width=2cm, inner xsep=2cm, inner ysep=1cm,
%				font=\Large\bfseries\sffamily, text centered,
%					shading angle=45] at (3,0) {Додаткова література};
%\end{tikzpicture}					
%%%---<
%%%---> NEW Version
\hyperlink{chapter4c}{\mbox{%
\begin{tikzpicture}
	\node[name=s,shape=rectangle callout,
		callout relative pointer={(1.25cm,-1cm)},
		callout pointer width=2cm, inner xsep=2cm, inner ysep=1cm,
		font=\Large\bfseries\sffamily, text centered,
		shading angle=45,
		postaction={path fading=south,
		fading angle=45,fill=blue, opacity=.5},
		left color=black, right color=red, draw=white,
		line width=2mm, text=white,
		shadow scale=3.25, shadow xshift=3pt,
		shadow yshift=3pt]
	at (3,0) {Додаткова література};
\end{tikzpicture}
}}
%%%---<
\vfill
\begin{itemize}
\item[]
	\scalebox{1.9}{\iconarticle}\qquad~~{\large {\bf Кондратьєв} Р., {\bf
	Чернего} О.I.}
\tooltipanim{\large\bf Принципи права та їх роль у регулюванні суспільних відносин}{23}{23}\quad {\large{\bf Право України}.  --- {\bf 2000}. --- {\bf №2}. --- С. 43.}
\item[]
	\scalebox{1.9}{\iconarticle}\qquad~~\parbox{850pt}{%
	{\large {\bf Константий} О.}
\tooltipanim{\large\bf Система адміністративного права як
	концептуальна основа здійснення адміністративного судочинства в
	Україні}{24}{24}\quad
{\large{\bf Право України}. --- {\bf 2004}. --- {\bf №12}. --- С. 20.}}
\item[]
	\scalebox{1.9}{\iconarticle}\qquad~~{\large {\bf Кравчук} І.}
\tooltipanim{\large\bf Адаптація права України до права Європейського Союзу}{25}{25}\quad
{\large	цілі, етапи, пріоритети. {\bf Право України}.  --- {\bf 2004}.
	--- {\bf №10}. --- С. 132.}
\item[]
	\scalebox{1.9}{\iconarticle}\qquad~~{\large {\bf Кубко} Є.}
\tooltipanim{\large\bf Про предмет адміністративного права}{23}{23}\quad
{\large{\bf Право України}. --- {\bf 2000}. --- {\bf №5}. --- С.3.}
\item[]
	\scalebox{1.9}{\iconarticle}\qquad~~{\large {\bf Ославський} М.}
\tooltipanim{\large\bf До питання необхідності здійснення адміністративної реформи в Україні}{24}{24}\quad {\large{\bf Право України}. --- {\bf 2004}. --- {\bf №9}. ---
	С. 40.}
\item[]
	\scalebox{1.9}{\iconarticle}\qquad~~\parbox{820pt}{%
	{\large {\bf Полешко} А.}
	\tooltipanim{\large\bf Напрями реформування адміністративного права}{25}{25}\quad
{\large	(за матеріалами Національної науковотеоретичної конференції).
	{\bf Право України}. --- {\bf 2000}. --- {\bf №8}. --- С. 35.}}
\item[]
	\scalebox{1.9}{\iconarticle}\qquad~~{\large {\bf Старилов} Ю.Н.}
\tooltipanim{\large\bf Как развивалась наука административного права в европейских странах}{23}{23}\quad {\large{\bf Журнал российского права}.  --- {\bf 1999}. ---
	{\bf №3/4}. --- С. 203.}
\item[]
	\scalebox{1.9}{\iconarticle}\qquad~~{\large {\bf Тихомиров} Ю.А.}
\tooltipanim{\large\bf О концепции развития административного права и процесса}{24}{24}\quad
{\large{\bf Государство и право}. --- {\bf 1998}. --- №1. --- С. 5.}
\item[]
	\scalebox{1.9}{\iconarticle}\qquad~~{\large {\bf Тимощук} В.}
\tooltipanim{\large\bf Адміністративне право в контексті європейського вибору України}{25}{25}\quad
{\large	(з міжнародної конференції). {\bf Право України}.  --- {\bf 2004}.
	--- {\bf №3}. --- С. 25.}
\item[]
\scalebox{1.9}{\iconarticle}\qquad~~{\large {\bf Хорощак} Н.}
\tooltipanim{\large\bf Нове в адміністративному праві}{23}{23}\quad
{\large{\bf Право України}.  --- {\bf 2004}. --- {\bf №8}.  --- С. 130.}
\item[]
\scalebox{1.9}{\iconarticle}\qquad~~{\large {\bf Шаповал} В.}
\tooltipanim{\large\bf Конституція України як форма адміністративного права}{24}{24}\quad
{\large{\bf Право України}. --- {\bf 2000}. --- {\bf №1}. --- С. 3.}
\end{itemize}
\vfill
\begin{textblock}{48}(22.1,15)
	\hyperlink{chapter4c}{\mbox{%
	\tikz[every node/.style={font=\normalsize\bfseries\sffamily,signal,draw,
		text=white}]
		\node[signal to=east,
			minimum width=35pt, postaction={path fading=south,
			fading angle=45,fill=blue,opacity=.5}, left color=black, right color=red,
			draw=white, line width=2mm, drop shadow, rotate=180]
			{\rotatebox{180}{\mbox{Додаткова література: 1~стр}}};
			}}		
\end{textblock}
\vfill
\mbox{}
%%%-----------------------------------------------------------------------
\begin{textblock}{49}(25,-0.01)
\begin{tikzpicture}[even odd rule,rounded corners=2pt,x=10pt,y=10pt,drop shadow]
\filldraw[fill=yellow!90!black!40,drop shadow] (0,0)   rectangle (1,1)
	[xshift=5pt,yshift=5pt]   (0,0)   rectangle (1,1)
	[rotate=30]   (-1,-1) rectangle (2,2);
\node at (0,1.7) {\textbf{\thepage}};			      
\end{tikzpicture}
\end{textblock}
%%%--- Navigational panel top page
\begin{textblock}{50}(7.58,0.85)
\mbox{%%%--->
\Acrobatmenu{LastPage}{%
\tikz[baseline] \node[rectangle,inner sep=2pt,minimum height=3.1ex,
rounded corners,drop shadow,shadow scale=1,shadow xshift=.8ex,
shadow yshift=-.4ex,opacity=.7,fill=black!50,top color=red!90!black!50,
bottom color=red!80!black!80,draw=red!50!black!50,very thick,text=white,
text opacity=1,minimum width=3cm,font=\bfseries\sffamily] at (0,0) {К концу};
}\Acrobatmenu{GoBack}{%
\tikz[baseline] \node[rectangle,inner sep=2pt,minimum height=3.1ex,
rounded corners,drop shadow,shadow scale=1,shadow xshift=.8ex,
shadow yshift=-.4ex,opacity=.7,fill=black!50,top color=red!90!black!50,
bottom color=red!80!black!80,draw=red!50!black!50,very thick,text=white,
text opacity=1,minimum width=3cm,font=\bfseries\sffamily] at (4,0) {Назад};
}\Acrobatmenu{PrevPage}{%
\tikz[baseline] \node[rectangle,inner sep=2pt,minimum height=3.1ex,
rounded corners,drop shadow,shadow scale=1,shadow xshift=.8ex,
shadow yshift=-.4ex,opacity=.7,fill=black!50,top color=red!90!black!50,
bottom color=red!80!black!80,draw=red!50!black!50,very thick,text=white,
text opacity=1,minimum width=3cm,font=\bfseries\sffamily] at (8,0) {Предыдущий};
}\Acrobatmenu{NextPage}{%
\tikz[baseline] \node[rectangle,inner sep=2pt,minimum height=3.1ex,
rounded corners,drop shadow,shadow scale=1,shadow xshift=.8ex,
shadow yshift=-.4ex,opacity=.7,fill=black!50,top color=red!90!black!50,
bottom color=red!80!black!80,draw=red!50!black!50,very thick,text=white,
text opacity=1,minimum width=3cm,font=\bfseries\sffamily] at (12,0) {Следующий};
}\Acrobatmenu{GoForward}{%
\tikz[baseline] \node[rectangle,inner sep=2pt,minimum height=3.1ex,
rounded corners,drop shadow,shadow scale=1,shadow xshift=.8ex,
shadow yshift=-.4ex,opacity=.7,fill=black!50,top color=red!90!black!50,
bottom color=red!80!black!80,draw=red!50!black!50,very thick,text=white,
text opacity=1,minimum width=3cm,font=\bfseries\sffamily] at (16,0) {Вперед};
}\Acrobatmenu{FirstPage}{%
\tikz[baseline] \node[rectangle,inner sep=2pt,minimum height=3.1ex,
rounded corners,drop shadow,shadow scale=1,shadow xshift=.8ex,
shadow yshift=-.4ex,opacity=.7,fill=black!50,top color=red!90!black!50,
bottom color=red!80!black!80,draw=red!50!black!50,very thick,text=white,
text opacity=1,minimum width=3cm,font=\bfseries\sffamily] at (20,0) {К началу};
}\Acrobatmenu{FullScreen}{%
\tikz[baseline] \node[rectangle,inner sep=2pt,minimum height=3.1ex,
rounded corners,drop shadow,shadow scale=1,shadow xshift=.8ex,
shadow yshift=-.4ex,opacity=.7,fill=black!50,top color=red!90!black!50,
bottom color=red!80!black!80,draw=red!50!black!50,very thick,text=white,
text opacity=1,minimum width=3cm,font=\bfseries\sffamily] at (24,0) {Полный экран};
}\Acrobatmenu{Quit}{%
\tikz[baseline] \node[rectangle,inner sep=2pt,minimum height=3.1ex,
rounded corners,drop shadow,shadow scale=1,shadow xshift=.8ex,
shadow yshift=-.4ex,opacity=.7,fill=black!50,top color=red!90!black!50,
bottom color=red!80!black!80,draw=red!50!black!50,very thick,text=white,
text opacity=1,minimum width=3cm,font=\bfseries\sffamily] at (28,0) {Выход};
}	
}%%%---|
\end{textblock}
%%%-----------------------------------------------------------------------
