%%%---> Table #1
\newpage
\begin{tikzpicture}[remember picture,overlay]
	  \node [rotate=0,scale=2,text opacity=0.2]
	      at (27,1.7) {Капранов~О.~Г.~\copyright~2010~~~Luga\TeX @yahoo.com};
\end{tikzpicture}
\vglue -18pt
\hspace{187pt}
\parbox{350pt}{%
\hypertarget{admintbl1}{\hyperlink{admintbl2}{\mbox{%
\begin{tikzpicture}
  \colorlet{even}{cyan!60!black}
  \colorlet{odd}{orange!100!black}
  \colorlet{links}{red!70!black}
  \colorlet{back}{yellow!20!white}
  \tikzset{
    box/.style={
      minimum height=15mm,
      inner sep=.7mm,
      outer sep=0mm,
      text width=120mm,
      text centered,
      font=\small\bfseries\sffamily,
      text=#1!50!black,
      draw=#1,
      line width=.25mm,
      top color=#1!5,
      bottom color=#1!40,
      shading angle=0,
      rounded corners=2.3mm,
      drop shadow={fill=#1!40!gray,fill opacity=.8},
      rotate=0,
    },
  }
  \node [box=even]{{%
  	\huge\textbf{Тематичний план за курсом <<Адміністративне право>>}}};
\end{tikzpicture}
}}}}
\hfill
\begin{flushright}
		\tikz \node [copy shadow={left color=green!50},tape,
			left color=green!50,draw=green,thick]
					 {{\large\textbf{Таблица №1}}};
\end{flushright}	
\hfill
\begin{center}
\begin{tikzpicture}
\node (tbl) {
\begin{tabularx}{976pt}{|>{\bfseries}c|>{\bfseries}p{430pt}|>{\bfseries}c|
	>{\bfseries}c|>{\bfseries}c|>{\bfseries}c|>{\bfseries}c|>{\bfseries}c|}
\arrayrulecolor{purple}
 & &\multicolumn{6}{c|}{\textcolor{white}{\bf Кількість годин за видами
 занять}}\\ \cline{3-8}\\
 & & &\multicolumn{5}{c|}{\raisebox{1.3ex}[0pt][0pt]{{\bf у тому числі:}}}\\ \cline{4-8}\\
 & & & & &\multicolumn{3}{c|}{\raisebox{1.3ex}[0pt][0pt]{{\bf із них:}}}\\ \cline{6-8}
\raisebox{5.5ex}[0pt][0pt]{Номера тем} &
 \mbox{}\hspace{130pt}\raisebox{5.5ex}[0pt][0pt]{Найменування розділів і тем} &
	\raisebox{1.5ex}[0pt][0pt]{Всього} &
		\raisebox{1.5ex}[0pt][0pt]{Самостійна робота} &
			\raisebox{1.5ex}[0pt][0pt]{Аудиторні} & 
				\raisebox{-0.8ex}[0pt][0pt]{Лекції} &
					\raisebox{-0.8ex}[0pt][0pt]{Семінарські заняття} &
						\raisebox{-0.8ex}[0pt][0pt]{Практичні заняття}\\[1ex]
\midrule
\multicolumn{8}{|>{\bfseries}c|}{ІІІ семестр}\\
\midrule
\multicolumn{8}{|>{\bfseries}c|}{ЗМІСТОВИЙ МОДУЛЬ I}\\
\midrule
\multicolumn{8}{|>{\bfseries}c|}{Загальна характеристика адміністративного
права, державного управління та виконавчої влади}\\
\midrule
1.0 & Предмет, метод і система адміністративного права & 9 & 3 & 6 & 4 & 2 &\\
\midrule
1.1 & Особливості предмета й методу адміністративно-правового регулювання & & & & 2 & &\\
\midrule
1.2 & Система адміністративного права & & & & 2 & &\\
\midrule
1.3 & Співвідношення адміністративного права з іншими галузями права & & & & & 2 &\\
\midrule
2.0 & Адміністративне право та державне управління & 9 & 5 & 4 & 2 & 2 &\\
\midrule
2.1 & Поняття та сутність державного управління & & & & 2 & &\\
\midrule
2.2 & Принципи та функції державного управління & & & & & 2 &\\
\midrule
3.0 & Адміністративно-правові норми та адміністративно-правові відносини & 9 & 5 & 4 & 2 & & 2\\
\midrule
3.1 & Поняття та особливості адміністративно-правових норм та
адміністративно-правових відносин & & & & 2 & &\\
\midrule
3.2 & Джерела адміністративного права & & & & & & 2\\
\midrule
4.0 & Суб'єкти адміністративного права & 27 & 9 & 18 & 10 & 4 & 4\\
\midrule
4.1 & Поняття та види суб'єктів адміністративного права & & & & 2 & &\\
\midrule
4.2 & Громадяни України як суб'єкти адміністративного права & & & & 2 & &\\
\midrule
4.3 & Спеціальні адміністративно-правові статуси індивідуальних суб'єктів
адміністративного права & & & & & & 2\\
\midrule
4.4 & Державні службовці --- суб'єкти адміністративного права & & & & 2 & &\\
\midrule
4.5 & Порівняльний аналіз статусів державних службовців & & & & & & 2\\
\midrule
4.6 & Органи виконавчої влади --- суб'єкти адміністративного права & & & & 2 & &\\
\midrule
4.7 & Поняття, ознаки та класифікація органів виконавчої влади & & & & & 2 &\\
\midrule
4.8 & Центральні та місцеві органи виконавчої влади & & & & 2 & &\\
\midrule
4.9 & Місце органів внутрішніх справ у системі виконавчої влади & & & & & 2 &\\[0.5ex]
\end{tabularx}};

\begin{pgfonlayer}{background}
\draw[rounded corners,top color=red,bottom color=black,draw=white]
	($(tbl.north west)+(0.14,0)$) rectangle ($(tbl.north east)-(0.13,0.9)$);
\draw[rounded corners,top color=white,bottom color=black,
	middle color=red,draw=blue!20] ($(tbl.south west) +(0.12,0.5)$)
		rectangle ($(tbl.south east)-(0.12,0)$);
\draw[top color=blue!1,bottom color=blue!20,draw=white]
	($(tbl.north east)-(0.13,0.6)$) rectangle ($(tbl.south west)+(0.13,0.2)$);
\end{pgfonlayer}
\end{tikzpicture}
\end{center}
\hfill
\centerline{%
\hyperlink{admintbl2}{\mbox{%
\tikz[baseline] \node[rectangle,inner sep=2pt,minimum height=3.1ex,rounded corners,
drop shadow,shadow scale=1,shadow xshift=.8ex,shadow yshift=-.4ex,opacity=.7,
fill=black!50,top color=blue!90!black!50,bottom color=blue!80!black!80,
draw=blue!50!black!50,very thick,text=white,text opacity=1,minimum width=4cm]{%
	Таблица №2};
	}}\qquad\qquad\qquad\qquad
\hyperlink{admintbl3}{\mbox{%
\tikz[baseline] \node[rectangle,inner sep=2pt,minimum height=3.1ex,rounded corners,
drop shadow,shadow scale=1,shadow xshift=.8ex,shadow yshift=-.4ex,opacity=.7,
fill=black!50,top color=blue!90!black!50,bottom color=blue!80!black!80,
draw=blue!50!black!50,very thick,text=white,text opacity=1,minimum width=4cm]{%
	Таблица №3};
	}}\qquad\qquad\qquad\qquad
\hyperlink{admintbl4}{\mbox{%
\tikz[baseline] \node[rectangle,inner sep=2pt,minimum height=3.1ex,rounded corners,
drop shadow,shadow scale=1,shadow xshift=.8ex,shadow yshift=-.4ex,opacity=.7,
fill=black!50,top color=blue!90!black!50,bottom color=blue!80!black!80,
draw=blue!50!black!50,very thick,text=white,text opacity=1,minimum width=4cm]{%
	Таблица №4};
	}}}
\hfill
\mbox{}
%%%-----------------------------------------------------------------------
\begin{textblock}{31}(25,-0.01)
\begin{tikzpicture}[even odd rule,rounded corners=2pt,x=10pt,y=10pt,drop shadow]
\filldraw[fill=yellow!90!black!40,drop shadow] (0,0)   rectangle (1,1)
	[xshift=5pt,yshift=5pt]   (0,0)   rectangle (1,1)
	[rotate=30]   (-1,-1) rectangle (2,2);
\node at (0,1.7) {\textbf{\thepage}};			      
\end{tikzpicture}
\end{textblock}
%%%--- Navigational panel top page
\begin{textblock}{32}(7.58,0.85)
\mbox{%%%--->
\Acrobatmenu{LastPage}{%
\tikz[baseline] \node[rectangle,inner sep=2pt,minimum height=3.1ex,
rounded corners,drop shadow,shadow scale=1,shadow xshift=.8ex,
shadow yshift=-.4ex,opacity=.7,fill=black!50,top color=red!90!black!50,
bottom color=red!80!black!80,draw=red!50!black!50,very thick,text=white,
text opacity=1,minimum width=3cm,font=\bfseries\sffamily] at (0,0) {К концу};
}\Acrobatmenu{GoBack}{%
\tikz[baseline] \node[rectangle,inner sep=2pt,minimum height=3.1ex,
rounded corners,drop shadow,shadow scale=1,shadow xshift=.8ex,
shadow yshift=-.4ex,opacity=.7,fill=black!50,top color=red!90!black!50,
bottom color=red!80!black!80,draw=red!50!black!50,very thick,text=white,
text opacity=1,minimum width=3cm,font=\bfseries\sffamily] at (4,0) {Назад};
}\Acrobatmenu{PrevPage}{%
\tikz[baseline] \node[rectangle,inner sep=2pt,minimum height=3.1ex,
rounded corners,drop shadow,shadow scale=1,shadow xshift=.8ex,
shadow yshift=-.4ex,opacity=.7,fill=black!50,top color=red!90!black!50,
bottom color=red!80!black!80,draw=red!50!black!50,very thick,text=white,
text opacity=1,minimum width=3cm,font=\bfseries\sffamily] at (8,0) {Предыдущий};
}\Acrobatmenu{NextPage}{%
\tikz[baseline] \node[rectangle,inner sep=2pt,minimum height=3.1ex,
rounded corners,drop shadow,shadow scale=1,shadow xshift=.8ex,
shadow yshift=-.4ex,opacity=.7,fill=black!50,top color=red!90!black!50,
bottom color=red!80!black!80,draw=red!50!black!50,very thick,text=white,
text opacity=1,minimum width=3cm,font=\bfseries\sffamily] at (12,0) {Следующий};
}\Acrobatmenu{GoForward}{%
\tikz[baseline] \node[rectangle,inner sep=2pt,minimum height=3.1ex,
rounded corners,drop shadow,shadow scale=1,shadow xshift=.8ex,
shadow yshift=-.4ex,opacity=.7,fill=black!50,top color=red!90!black!50,
bottom color=red!80!black!80,draw=red!50!black!50,very thick,text=white,
text opacity=1,minimum width=3cm,font=\bfseries\sffamily] at (16,0) {Вперед};
}\Acrobatmenu{FirstPage}{%
\tikz[baseline] \node[rectangle,inner sep=2pt,minimum height=3.1ex,
rounded corners,drop shadow,shadow scale=1,shadow xshift=.8ex,
shadow yshift=-.4ex,opacity=.7,fill=black!50,top color=red!90!black!50,
bottom color=red!80!black!80,draw=red!50!black!50,very thick,text=white,
text opacity=1,minimum width=3cm,font=\bfseries\sffamily] at (20,0) {К началу};
}\Acrobatmenu{FullScreen}{%
\tikz[baseline] \node[rectangle,inner sep=2pt,minimum height=3.1ex,
rounded corners,drop shadow,shadow scale=1,shadow xshift=.8ex,
shadow yshift=-.4ex,opacity=.7,fill=black!50,top color=red!90!black!50,
bottom color=red!80!black!80,draw=red!50!black!50,very thick,text=white,
text opacity=1,minimum width=3cm,font=\bfseries\sffamily] at (24,0) {Полный экран};
}\Acrobatmenu{Quit}{%
\tikz[baseline] \node[rectangle,inner sep=2pt,minimum height=3.1ex,
rounded corners,drop shadow,shadow scale=1,shadow xshift=.8ex,
shadow yshift=-.4ex,opacity=.7,fill=black!50,top color=red!90!black!50,
bottom color=red!80!black!80,draw=red!50!black!50,very thick,text=white,
text opacity=1,minimum width=3cm,font=\bfseries\sffamily] at (28,0) {Выход};
}	
}%%%---|
\end{textblock}
%%%-----------------------------------------------------------------------
%%%---> Table #2
\newpage
\begin{tikzpicture}[remember picture,overlay]
	  \node [rotate=0,scale=2,text opacity=0.2]
	      at (27,1.7) {Капранов~О.~Г.~\copyright~2010~~~Luga\TeX @yahoo.com};
\end{tikzpicture}
\vglue -18pt
\hspace{187pt}
\parbox{350pt}{%
\hypertarget{admintbl2}{\hyperlink{admintbl3}{\mbox{%
\begin{tikzpicture}
  \colorlet{even}{cyan!60!black}
  \colorlet{odd}{orange!100!black}
  \colorlet{links}{red!70!black}
  \colorlet{back}{yellow!20!white}
  \tikzset{
    box/.style={
      minimum height=15mm,
      inner sep=.7mm,
      outer sep=0mm,
      text width=120mm,
      text centered,
      font=\small\bfseries\sffamily,
      text=#1!50!black,
      draw=#1,
      line width=.25mm,
      top color=#1!5,
      bottom color=#1!40,
      shading angle=0,
      rounded corners=2.3mm,
      drop shadow={fill=#1!40!gray,fill opacity=.8},
      rotate=0,
    },
  }
  \node [box=even]{{%
  	\huge\textbf{Тематичний план за курсом <<Адміністративне право>>}}};
\end{tikzpicture}
}}}}
\hfill
\begin{flushright}
	\tikz \node [copy shadow={left color=green!50},tape,
		left color=green!50,draw=green,thick]
					 {{\large\textbf{Таблица №2}}};
\end{flushright}	
\hfill
\begin{center}
\begin{tikzpicture}
\node (tbl) {
\begin{tabularx}{976pt}{|>{\bfseries}c|>{\bfseries}p{430pt}|>{\bfseries}c|
	>{\bfseries}c|>{\bfseries}c|>{\bfseries}c|>{\bfseries}c|>{\bfseries}c|}
\arrayrulecolor{purple}
 & &\multicolumn{6}{c|}{\textcolor{white}{\bf Кількість годин за видами
 занять}}\\ \cline{3-8}\\
 & & &\multicolumn{5}{c|}{\raisebox{1.3ex}[0pt][0pt]{{\bf у тому числі:}}}\\ \cline{4-8}\\
 & & & & &\multicolumn{3}{c|}{\raisebox{1.3ex}[0pt][0pt]{{\bf із них:}}}\\ \cline{6-8}
\raisebox{5.5ex}[0pt][0pt]{Номера тем} &
 \mbox{}\hspace{130pt}\raisebox{5.5ex}[0pt][0pt]{Найменування розділів і тем} &
	\raisebox{1.5ex}[0pt][0pt]{Всього} &
		\raisebox{1.5ex}[0pt][0pt]{Самостійна робота} &
			\raisebox{1.5ex}[0pt][0pt]{Аудиторні} & 
				\raisebox{-0.8ex}[0pt][0pt]{Лекції} &
					\raisebox{-0.8ex}[0pt][0pt]{Семінарські заняття} &
						\raisebox{-0.8ex}[0pt][0pt]{Практичні заняття}\\[1ex]
\midrule
\multicolumn{8}{|>{\bfseries}c|}{Модуль-контроль №1}\\
\midrule
\multicolumn{8}{|>{\bfseries}c|}{ЗМІСТОВИЙ МОДУЛЬ II}\\
\midrule
\multicolumn{8}{|>{\bfseries}c|}{Адміністративно-правові форми і методи}\\
\midrule
5.0 & Характеристика адміністративно-правових форм і методів & 15 & 7 & 8 & 4 & 2 & 2\\
\midrule
5.1 & Поняття, види та особливості форм і методів управлінської діяльності & & & & 2 & &\\
\midrule
5.2 & Співвідношення адміністративно-правових форм і методів & & & & & 2 &\\
\midrule
5.3 & Юридичні акти управління & & & & 2 & &\\
\midrule
5.4 & Місце актів управління в системі правових актів & & & & 2 & &\\
\midrule
6.0 & Адміністративний примус & 12 & 6 & 6 & 2 & 2 & 2\\
\midrule
6.1 & Поняття, сутність та види адміністративного примусу & & & & 2 & &\\
\midrule
6.2 & Правові засади адміністративного примусу у сфері державного управління & & & & & 2 &\\
\midrule
6.3 & Адміністративний примус в діяльності органів внутрішніх справ & & & & & & 2\\
\midrule
7.0 & Адміністративно-процесуальне право & 9 & 5 & 4 & 2 & & 2\\
\midrule
7.1   & Поняття, особливості та структура адміністративного процесу & & & & 2 & &\\
\midrule
7.2 & Стадії провадження у справах про адміністративні правопорушення & & & & & & 2\\[0.5ex]
\end{tabularx}};

\begin{pgfonlayer}{background}
\draw[rounded corners,top color=red,bottom color=black,draw=white]
	($(tbl.north west)+(0.14,0)$) rectangle ($(tbl.north east)-(0.13,0.9)$);
\draw[rounded corners,top color=white,bottom color=black,
	middle color=red,draw=blue!20] ($(tbl.south west) +(0.12,0.5)$)
		rectangle ($(tbl.south east)-(0.12,0)$);
\draw[top color=blue!1,bottom color=blue!20,draw=white]
	($(tbl.north east)-(0.13,0.6)$) rectangle ($(tbl.south west)+(0.13,0.2)$);
\end{pgfonlayer}
\end{tikzpicture}
\end{center}
\hfill
\centerline{%
\hyperlink{admintbl1}{\mbox{%
\tikz[baseline] \node[rectangle,inner sep=2pt,minimum height=3.1ex,rounded corners,
drop shadow,shadow scale=1,shadow xshift=.8ex,shadow yshift=-.4ex,opacity=.7,
fill=black!50,top color=blue!90!black!50,bottom color=blue!80!black!80,
draw=blue!50!black!50,very thick,text=white,text opacity=1,minimum width=4cm]{%
	Таблица №1};
	}}\qquad\qquad\qquad\qquad
\hyperlink{admintbl3}{\mbox{%
\tikz[baseline] \node[rectangle,inner sep=2pt,minimum height=3.1ex,rounded corners,
drop shadow,shadow scale=1,shadow xshift=.8ex,shadow yshift=-.4ex,opacity=.7,
fill=black!50,top color=blue!90!black!50,bottom color=blue!80!black!80,
draw=blue!50!black!50,very thick,text=white,text opacity=1,minimum width=4cm]{%
	Таблица №3};
	}}\qquad\qquad\qquad\qquad
\hyperlink{admintbl4}{\mbox{%
\tikz[baseline] \node[rectangle,inner sep=2pt,minimum height=3.1ex,rounded corners,
drop shadow,shadow scale=1,shadow xshift=.8ex,shadow yshift=-.4ex,opacity=.7,
fill=black!50,top color=blue!90!black!50,bottom color=blue!80!black!80,
draw=blue!50!black!50,very thick,text=white,text opacity=1,minimum width=4cm]{%
	Таблица №4};
	}}}
\hfill
\mbox{}
%%%-----------------------------------------------------------------------
\begin{textblock}{33}(25,-0.01)
\begin{tikzpicture}[even odd rule,rounded corners=2pt,x=10pt,y=10pt,drop shadow]
\filldraw[fill=yellow!90!black!40,drop shadow] (0,0)   rectangle (1,1)
	[xshift=5pt,yshift=5pt]   (0,0)   rectangle (1,1)
	[rotate=30]   (-1,-1) rectangle (2,2);
\node at (0,1.7) {\textbf{\thepage}};			      
\end{tikzpicture}
\end{textblock}
%%%--- Navigational panel top page
\begin{textblock}{34}(7.58,0.85)
\mbox{%%%--->
\Acrobatmenu{LastPage}{%
\tikz[baseline] \node[rectangle,inner sep=2pt,minimum height=3.1ex,
rounded corners,drop shadow,shadow scale=1,shadow xshift=.8ex,
shadow yshift=-.4ex,opacity=.7,fill=black!50,top color=red!90!black!50,
bottom color=red!80!black!80,draw=red!50!black!50,very thick,text=white,
text opacity=1,minimum width=3cm,font=\bfseries\sffamily] at (0,0) {К концу};
}\Acrobatmenu{GoBack}{%
\tikz[baseline] \node[rectangle,inner sep=2pt,minimum height=3.1ex,
rounded corners,drop shadow,shadow scale=1,shadow xshift=.8ex,
shadow yshift=-.4ex,opacity=.7,fill=black!50,top color=red!90!black!50,
bottom color=red!80!black!80,draw=red!50!black!50,very thick,text=white,
text opacity=1,minimum width=3cm,font=\bfseries\sffamily] at (4,0) {Назад};
}\Acrobatmenu{PrevPage}{%
\tikz[baseline] \node[rectangle,inner sep=2pt,minimum height=3.1ex,
rounded corners,drop shadow,shadow scale=1,shadow xshift=.8ex,
shadow yshift=-.4ex,opacity=.7,fill=black!50,top color=red!90!black!50,
bottom color=red!80!black!80,draw=red!50!black!50,very thick,text=white,
text opacity=1,minimum width=3cm,font=\bfseries\sffamily] at (8,0) {Предыдущий};
}\Acrobatmenu{NextPage}{%
\tikz[baseline] \node[rectangle,inner sep=2pt,minimum height=3.1ex,
rounded corners,drop shadow,shadow scale=1,shadow xshift=.8ex,
shadow yshift=-.4ex,opacity=.7,fill=black!50,top color=red!90!black!50,
bottom color=red!80!black!80,draw=red!50!black!50,very thick,text=white,
text opacity=1,minimum width=3cm,font=\bfseries\sffamily] at (12,0) {Следующий};
}\Acrobatmenu{GoForward}{%
\tikz[baseline] \node[rectangle,inner sep=2pt,minimum height=3.1ex,
rounded corners,drop shadow,shadow scale=1,shadow xshift=.8ex,
shadow yshift=-.4ex,opacity=.7,fill=black!50,top color=red!90!black!50,
bottom color=red!80!black!80,draw=red!50!black!50,very thick,text=white,
text opacity=1,minimum width=3cm,font=\bfseries\sffamily] at (16,0) {Вперед};
}\Acrobatmenu{FirstPage}{%
\tikz[baseline] \node[rectangle,inner sep=2pt,minimum height=3.1ex,
rounded corners,drop shadow,shadow scale=1,shadow xshift=.8ex,
shadow yshift=-.4ex,opacity=.7,fill=black!50,top color=red!90!black!50,
bottom color=red!80!black!80,draw=red!50!black!50,very thick,text=white,
text opacity=1,minimum width=3cm,font=\bfseries\sffamily] at (20,0) {К началу};
}\Acrobatmenu{FullScreen}{%
\tikz[baseline] \node[rectangle,inner sep=2pt,minimum height=3.1ex,
rounded corners,drop shadow,shadow scale=1,shadow xshift=.8ex,
shadow yshift=-.4ex,opacity=.7,fill=black!50,top color=red!90!black!50,
bottom color=red!80!black!80,draw=red!50!black!50,very thick,text=white,
text opacity=1,minimum width=3cm,font=\bfseries\sffamily] at (24,0) {Полный экран};
}\Acrobatmenu{Quit}{%
\tikz[baseline] \node[rectangle,inner sep=2pt,minimum height=3.1ex,
rounded corners,drop shadow,shadow scale=1,shadow xshift=.8ex,
shadow yshift=-.4ex,opacity=.7,fill=black!50,top color=red!90!black!50,
bottom color=red!80!black!80,draw=red!50!black!50,very thick,text=white,
text opacity=1,minimum width=3cm,font=\bfseries\sffamily] at (28,0) {Выход};
}	
}%%%---|
\end{textblock}
%%%-----------------------------------------------------------------------
%%%---> Table #3
\newpage
\begin{tikzpicture}[remember picture,overlay]
	  \node [rotate=0,scale=2,text opacity=0.2]
	      at (27,1.7) {Капранов~О.~Г.~\copyright~2010~~~Luga\TeX @yahoo.com};
\end{tikzpicture}
\vglue -18pt
\hspace{187pt}
\parbox{350pt}{%
\hypertarget{admintbl3}{\hyperlink{admintbl4}{\mbox{%
\begin{tikzpicture}
  \colorlet{even}{cyan!60!black}
  \colorlet{odd}{orange!100!black}
  \colorlet{links}{red!70!black}
  \colorlet{back}{yellow!20!white}
  \tikzset{
    box/.style={
      minimum height=15mm,
      inner sep=.7mm,
      outer sep=0mm,
      text width=120mm,
      text centered,
      font=\small\bfseries\sffamily,
      text=#1!50!black,
      draw=#1,
      line width=.25mm,
      top color=#1!5,
      bottom color=#1!40,
      shading angle=0,
      rounded corners=2.3mm,
      drop shadow={fill=#1!40!gray,fill opacity=.8},
      rotate=0,
    },
  }
  \node [box=even]{{%
  	\huge\textbf{Тематичний план за курсом <<Адміністративне право>>}}};
\end{tikzpicture}
}}}}
\hfill
\begin{flushright}
	\tikz \node [copy shadow={left color=green!50},tape,
		left color=green!50,draw=green,thick]
					 {{\large\textbf{Таблица №3}}};
\end{flushright}	
\hfill
\begin{center}
\begin{tikzpicture}
\node (tbl) {
\begin{tabularx}{976pt}{|>{\bfseries}c|>{\bfseries}p{430pt}|>{\bfseries}c|
	>{\bfseries}c|>{\bfseries}c|>{\bfseries}c|>{\bfseries}c|>{\bfseries}c|}
\arrayrulecolor{purple}
 & &\multicolumn{6}{c|}{\textcolor{white}{\bf Кількість годин за видами
 занять}}\\ \cline{3-8}\\
 & & &\multicolumn{5}{c|}{\raisebox{1.3ex}[0pt][0pt]{{\bf у тому числі:}}}\\ \cline{4-8}\\
 & & & & &\multicolumn{3}{c|}{\raisebox{1.3ex}[0pt][0pt]{{\bf із них:}}}\\ \cline{6-8}
\raisebox{5.5ex}[0pt][0pt]{Номера тем} &
 \mbox{}\hspace{130pt}\raisebox{5.5ex}[0pt][0pt]{Найменування розділів і тем} &
	\raisebox{1.5ex}[0pt][0pt]{Всього} &
		\raisebox{1.5ex}[0pt][0pt]{Самостійна робота} &
			\raisebox{1.5ex}[0pt][0pt]{Аудиторні} & 
				\raisebox{-0.8ex}[0pt][0pt]{Лекції} &
					\raisebox{-0.8ex}[0pt][0pt]{Семінарські заняття} &
						\raisebox{-0.8ex}[0pt][0pt]{Практичні заняття}\\[1ex]
\midrule
\multicolumn{8}{|>{\bfseries}c|}{Модуль-контроль №2}\\
\midrule
\multicolumn{8}{|>{\bfseries}c|}{ЗМІСТОВИЙ МОДУЛЬ III}\\
\midrule
\multicolumn{8}{|>{\bfseries}c|}{Основи адміністративно-правової організації
	управління в економічній, соціально-культурній та адміністративно-політичній сфері}\\
\midrule
8.0  & Забезпечення законності в державному управлінні & 9 & 5 & 4 & 2 & & 2\\
\midrule
8.1  & Законність і дисципліна в сфері виконавчої влади & & & & 2 & &\\
\midrule
8.2  & Контроль і нагляд у державному управлінні & & & & & & 2\\
\midrule
9.0  & Особливості адміністративно-правової організації управління & 9 & 5 & 4 & 2 & 2 &\\
\midrule
9.1  & Адміністративно-правова організація управління економікою,
соціально-культурною та адміністративно-політичною сферами & & & & 2 & &\\
\midrule
9.2  & Сутність й особливості міжгалузевого управління & & & & & 2 &\\
\midrule
10.0 & Адміністративне право та управління економікою & 9 & 5 & 2 & 2 & &\\
\midrule
10.1 & Адміністративно-правове регулювання в окремих галузях економіки & & & & 2 & &\\
\midrule
11.0 & Адміністративне право та управління соціально-культурною сферою & 9 & 5 & 4 & 2 & 2 &\\
\midrule
11.1 & Адміністративно-правове регулювання в окремих галузях
	соціально-культурної сфери & & & & 2 & &\\
\midrule
11.2 & Особливості управління соціально-культурним комплексом & & & & & 2 &\\
\midrule
12.0 & Адміністративне право та управління адміністративно-політичною сферою & 9 & 5 & 4 & 2 & & 2\\
\midrule
12.1 & Адміністративно-правове регулювання в сфері безпеки, оборони та юстиції & & & & 2 & &\\
\midrule
12.2 & Особливості управління в адміністративно-політичній сфері & & & & & & 2\\
\midrule
12.3 & Адміністративне право та управління в сфері охорони правопорядку & 9 & 7 & 4 & 2 & 2 &\\
\midrule
13.0 & Адміністративне право та управління в сфері охорони правопорядку & 9 & 7 & 4 & 2 & 2 &\\
\midrule
13.1 & Управління в сфері внутрішніх справ & & & & 2 & &\\
\midrule
13.2 & Особливості адміністративно-правових відносин за участю ОВС & & & & & 2 &\\[0.5ex]
\end{tabularx}};

\begin{pgfonlayer}{background}
\draw[rounded corners,top color=red,bottom color=black,draw=white]
	($(tbl.north west)+(0.14,0)$) rectangle ($(tbl.north east)-(0.13,0.9)$);
\draw[rounded corners,top color=white,bottom color=black,
	middle color=red,draw=blue!20] ($(tbl.south west) +(0.12,0.5)$)
		rectangle ($(tbl.south east)-(0.12,0)$);
\draw[top color=blue!1,bottom color=blue!20,draw=white]
	($(tbl.north east)-(0.13,0.6)$) rectangle ($(tbl.south west)+(0.13,0.2)$);
\end{pgfonlayer}
\end{tikzpicture}
\end{center}
\hfill
\centerline{%
\hyperlink{admintbl1}{\mbox{%
\tikz[baseline] \node[rectangle,inner sep=2pt,minimum height=3.1ex,rounded corners,
drop shadow,shadow scale=1,shadow xshift=.8ex,shadow yshift=-.4ex,opacity=.7,
fill=black!50,top color=blue!90!black!50,bottom color=blue!80!black!80,
draw=blue!50!black!50,very thick,text=white,text opacity=1,minimum width=4cm]{%
	Таблица №1};
	}}\qquad\qquad\qquad\qquad
\hyperlink{admintbl2}{\mbox{%
\tikz[baseline] \node[rectangle,inner sep=2pt,minimum height=3.1ex,rounded corners,
drop shadow,shadow scale=1,shadow xshift=.8ex,shadow yshift=-.4ex,opacity=.7,
fill=black!50,top color=blue!90!black!50,bottom color=blue!80!black!80,
draw=blue!50!black!50,very thick,text=white,text opacity=1,minimum width=4cm]{%
	Таблица №2};
	}}\qquad\qquad\qquad\qquad
\hyperlink{admintbl4}{\mbox{%
\tikz[baseline] \node[rectangle,inner sep=2pt,minimum height=3.1ex,rounded corners,
drop shadow,shadow scale=1,shadow xshift=.8ex,shadow yshift=-.4ex,opacity=.7,
fill=black!50,top color=blue!90!black!50,bottom color=blue!80!black!80,
draw=blue!50!black!50,very thick,text=white,text opacity=1,minimum width=4cm]{%
	Таблица №4};
	}}}
\hfill
\mbox{}
%%%-----------------------------------------------------------------------
\begin{textblock}{35}(25,-0.01)
\begin{tikzpicture}[even odd rule,rounded corners=2pt,x=10pt,y=10pt,drop shadow]
\filldraw[fill=yellow!90!black!40,drop shadow] (0,0)   rectangle (1,1)
	[xshift=5pt,yshift=5pt]   (0,0)   rectangle (1,1)
	[rotate=30]   (-1,-1) rectangle (2,2);
\node at (0,1.7) {\textbf{\thepage}};			      
\end{tikzpicture}
\end{textblock}
%%%--- Navigational panel top page
\begin{textblock}{36}(7.58,0.85)
\mbox{%%%--->
\Acrobatmenu{LastPage}{%
\tikz[baseline] \node[rectangle,inner sep=2pt,minimum height=3.1ex,
rounded corners,drop shadow,shadow scale=1,shadow xshift=.8ex,
shadow yshift=-.4ex,opacity=.7,fill=black!50,top color=red!90!black!50,
bottom color=red!80!black!80,draw=red!50!black!50,very thick,text=white,
text opacity=1,minimum width=3cm,font=\bfseries\sffamily] at (0,0) {К концу};
}\Acrobatmenu{GoBack}{%
\tikz[baseline] \node[rectangle,inner sep=2pt,minimum height=3.1ex,
rounded corners,drop shadow,shadow scale=1,shadow xshift=.8ex,
shadow yshift=-.4ex,opacity=.7,fill=black!50,top color=red!90!black!50,
bottom color=red!80!black!80,draw=red!50!black!50,very thick,text=white,
text opacity=1,minimum width=3cm,font=\bfseries\sffamily] at (4,0) {Назад};
}\Acrobatmenu{PrevPage}{%
\tikz[baseline] \node[rectangle,inner sep=2pt,minimum height=3.1ex,
rounded corners,drop shadow,shadow scale=1,shadow xshift=.8ex,
shadow yshift=-.4ex,opacity=.7,fill=black!50,top color=red!90!black!50,
bottom color=red!80!black!80,draw=red!50!black!50,very thick,text=white,
text opacity=1,minimum width=3cm,font=\bfseries\sffamily] at (8,0) {Предыдущий};
}\Acrobatmenu{NextPage}{%
\tikz[baseline] \node[rectangle,inner sep=2pt,minimum height=3.1ex,
rounded corners,drop shadow,shadow scale=1,shadow xshift=.8ex,
shadow yshift=-.4ex,opacity=.7,fill=black!50,top color=red!90!black!50,
bottom color=red!80!black!80,draw=red!50!black!50,very thick,text=white,
text opacity=1,minimum width=3cm,font=\bfseries\sffamily] at (12,0) {Следующий};
}\Acrobatmenu{GoForward}{%
\tikz[baseline] \node[rectangle,inner sep=2pt,minimum height=3.1ex,
rounded corners,drop shadow,shadow scale=1,shadow xshift=.8ex,
shadow yshift=-.4ex,opacity=.7,fill=black!50,top color=red!90!black!50,
bottom color=red!80!black!80,draw=red!50!black!50,very thick,text=white,
text opacity=1,minimum width=3cm,font=\bfseries\sffamily] at (16,0) {Вперед};
}\Acrobatmenu{FirstPage}{%
\tikz[baseline] \node[rectangle,inner sep=2pt,minimum height=3.1ex,
rounded corners,drop shadow,shadow scale=1,shadow xshift=.8ex,
shadow yshift=-.4ex,opacity=.7,fill=black!50,top color=red!90!black!50,
bottom color=red!80!black!80,draw=red!50!black!50,very thick,text=white,
text opacity=1,minimum width=3cm,font=\bfseries\sffamily] at (20,0) {К началу};
}\Acrobatmenu{FullScreen}{%
\tikz[baseline] \node[rectangle,inner sep=2pt,minimum height=3.1ex,
rounded corners,drop shadow,shadow scale=1,shadow xshift=.8ex,
shadow yshift=-.4ex,opacity=.7,fill=black!50,top color=red!90!black!50,
bottom color=red!80!black!80,draw=red!50!black!50,very thick,text=white,
text opacity=1,minimum width=3cm,font=\bfseries\sffamily] at (24,0) {Полный экран};
}\Acrobatmenu{Quit}{%
\tikz[baseline] \node[rectangle,inner sep=2pt,minimum height=3.1ex,
rounded corners,drop shadow,shadow scale=1,shadow xshift=.8ex,
shadow yshift=-.4ex,opacity=.7,fill=black!50,top color=red!90!black!50,
bottom color=red!80!black!80,draw=red!50!black!50,very thick,text=white,
text opacity=1,minimum width=3cm,font=\bfseries\sffamily] at (28,0) {Выход};
}	
}%%%---|
\end{textblock}
%%%-----------------------------------------------------------------------
%%%---> Table #4
\newpage
\begin{tikzpicture}[remember picture,overlay]
	  \node [rotate=0,scale=2,text opacity=0.2]
	      at (27,1.7) {Капранов~О.~Г.~\copyright~2010~~~Luga\TeX @yahoo.com};
\end{tikzpicture}
\vglue -18pt
\hspace{187pt}
\parbox{350pt}{%
\hypertarget{admintbl4}{\hyperlink{chapter4a}{\mbox{%
\begin{tikzpicture}
  \colorlet{even}{cyan!60!black}
  \colorlet{odd}{orange!100!black}
  \colorlet{links}{red!70!black}
  \colorlet{back}{yellow!20!white}
  \tikzset{
    box/.style={
      minimum height=15mm,
      inner sep=.7mm,
      outer sep=0mm,
      text width=120mm,
      text centered,
      font=\small\bfseries\sffamily,
      text=#1!50!black,
      draw=#1,
      line width=.25mm,
      top color=#1!5,
      bottom color=#1!40,
      shading angle=0,
      rounded corners=2.3mm,
      drop shadow={fill=#1!40!gray,fill opacity=.8},
      rotate=0,
    },
  }
  \node [box=even]{{%
  	\huge\textbf{Тематичний план за курсом <<Адміністративне право>>}}};
\end{tikzpicture}
}}}}
\hfill
\begin{flushright}
	\tikz \node [copy shadow={left color=green!50},tape,
		left color=green!50,draw=green,thick]
					 {{\large\textbf{Таблица №4}}};
\end{flushright}	
\hfill
\begin{center}
\begin{tikzpicture}
\node (tbl) {
\begin{tabularx}{976pt}{|>{\bfseries}c|>{\bfseries}p{430pt}|>{\bfseries}c|
	>{\bfseries}c|>{\bfseries}c|>{\bfseries}c|>{\bfseries}c|>{\bfseries}c|}
\arrayrulecolor{purple}
 & &\multicolumn{6}{c|}{\textcolor{white}{\bf Кількість годин за видами
 занять}}\\ \cline{3-8}\\
 & & &\multicolumn{5}{c|}{\raisebox{1.3ex}[0pt][0pt]{{\bf у тому числі:}}}\\ \cline{4-8}\\
 & & & & &\multicolumn{3}{c|}{\raisebox{1.3ex}[0pt][0pt]{{\bf із них:}}}\\ \cline{6-8}
\raisebox{5.5ex}[0pt][0pt]{Номера тем} &
 \mbox{}\hspace{130pt}\raisebox{5.5ex}[0pt][0pt]{Найменування розділів і тем} &
	\raisebox{1.5ex}[0pt][0pt]{Всього} &
		\raisebox{1.5ex}[0pt][0pt]{Самостійна робота} &
			\raisebox{1.5ex}[0pt][0pt]{Аудиторні} & 
				\raisebox{-0.8ex}[0pt][0pt]{Лекції} &
					\raisebox{-0.8ex}[0pt][0pt]{Семінарські заняття} &
						\raisebox{-0.8ex}[0pt][0pt]{Практичні заняття}\\[1ex]
\midrule
\multicolumn{8}{|>{\bfseries}c|}{Модуль-контроль №3}\\
\midrule
\multicolumn{8}{|>{\bfseries}c|}{ЗМІСТОВИЙ МОДУЛЬ IV}\\
\midrule
\multicolumn{8}{|>{\bfseries}c|}{Адміністративна відповідальність}\\
\midrule
14.0 & Загально-правові засади інституту адміністративної відповідальності & 12 & 6 & 6 & 2 & 4 &\\
\midrule
14.1 & Поняття адміністративної відповідальності та її особливості & & & & 2 & &\\
\midrule
14.2 & Поняття, ознаки і юридичний склад адміністративного правопорушення & & & & & 2 &\\
\midrule
14.3 & Система адміністративних стягнень & & & & & 2 &\\
\midrule
15.0 & Антикорупційне законодавство & 6 & 2 & 4 & 2 & & 2\\
\midrule
15.1 & Характеристика корупційних правопорушень & & & & 2 & &\\
\midrule
15.2 & Адміністративна відповідальність за корупційні правопорушення & & & & & & 2\\
\midrule
16.0 & Адміністративні правопорушення, що посягають на громадський порядок,
громадську безпеку та встановлений порядок управління & 12 & 4 & 8 & 2 & 2 & 4\\
\midrule
16.1 & Юридична характеристика адміністративних право-порушень, що посягають на
громадський порядок, громадську безпеку та встановлений порядок управління & & & & 2 & 2 &\\
\midrule
16.2 & Адміністративні правопорушення, що посягають на громадський порядок та
громадську безпеку. & & & & & & 2\\
\midrule
16.3 & Адміністративні правопорушення, що посягають на встановлений порядок
управління. & & & & & & 2\\
\midrule
17.0 & Юридична характеристика адміністративних правопорушень в окремих сферах
економіки, соціально-культурного комплексу та у галузі
адміністративно-політичної діяльності & 24 & 16 & 8 & 2 & 2 & 4\\
\midrule
17.1 & Особливості складів адміністративних правопорушень в окремих галузях
економічної, соціально-культурної та адміністративно-політичної діяльності & & & & 2 & &\\
\midrule
17.2 & Адміністративні правопорушення в галузі охорони праці та здоров'я населення & & & & & 2 &\\
\midrule
17.3 & Адміністративні правопорушення на транспорті, в галузі шляхового
господарства і зв'язку. & & & & & 2 &\\
\midrule
17.4   & Адміністративні правопорушення в галузі торгівлі, фінансів і
	підприємницької діяльності & & & & & & 2\\
\midrule
\multicolumn{8}{|>{\bfseries}c|}{Модуль-контроль №4}\\
\midrule
 & Разом & 198 & 100 & 98 & 46 & 26 & 26\\[0.5ex]
\end{tabularx}};

\begin{pgfonlayer}{background}
\draw[rounded corners,top color=red,bottom color=black,draw=white]
	($(tbl.north west)+(0.14,0)$) rectangle ($(tbl.north east)-(0.13,0.9)$);
\draw[rounded corners,top color=white,bottom color=black,
	middle color=red,draw=blue!20] ($(tbl.south west) +(0.12,0.5)$)
		rectangle ($(tbl.south east)-(0.12,0)$);
\draw[top color=blue!1,bottom color=blue!20,draw=white]
	($(tbl.north east)-(0.13,0.6)$) rectangle ($(tbl.south west)+(0.13,0.2)$);
\end{pgfonlayer}
\end{tikzpicture}
\end{center}
\hfill
\centerline{%
\hyperlink{admintbl1}{\mbox{%
\tikz[baseline] \node[rectangle,inner sep=2pt,minimum height=3.1ex,rounded corners,
drop shadow,shadow scale=1,shadow xshift=.8ex,shadow yshift=-.4ex,opacity=.7,
fill=black!50,top color=blue!90!black!50,bottom color=blue!80!black!80,
draw=blue!50!black!50,very thick,text=white,text opacity=1,minimum width=4cm]{%
	Таблица №1};
	}}\qquad\qquad\qquad\qquad
\hyperlink{admintbl2}{\mbox{%
\tikz[baseline] \node[rectangle,inner sep=2pt,minimum height=3.1ex,rounded corners,
drop shadow,shadow scale=1,shadow xshift=.8ex,shadow yshift=-.4ex,opacity=.7,
fill=black!50,top color=blue!90!black!50,bottom color=blue!80!black!80,
draw=blue!50!black!50,very thick,text=white,text opacity=1,minimum width=4cm]{%
	Таблица №2};
	}}\qquad\qquad\qquad\qquad
\hyperlink{admintbl3}{\mbox{%
\tikz[baseline] \node[rectangle,inner sep=2pt,minimum height=3.1ex,rounded corners,
drop shadow,shadow scale=1,shadow xshift=.8ex,shadow yshift=-.4ex,opacity=.7,
fill=black!50,top color=blue!90!black!50,bottom color=blue!80!black!80,
draw=blue!50!black!50,very thick,text=white,text opacity=1,minimum width=4cm]{%
	Таблица №3};
	}}}
\hfill
\mbox{}
%%%-----------------------------------------------------------------------
\begin{textblock}{37}(25,-0.01)
\begin{tikzpicture}[even odd rule,rounded corners=2pt,x=10pt,y=10pt,drop shadow]
\filldraw[fill=yellow!90!black!40,drop shadow] (0,0)   rectangle (1,1)
	[xshift=5pt,yshift=5pt]   (0,0)   rectangle (1,1)
	[rotate=30]   (-1,-1) rectangle (2,2);
\node at (0,1.7) {\textbf{\thepage}};			      
\end{tikzpicture}
\end{textblock}
%%%--- Navigational panel top page
\begin{textblock}{38}(7.58,0.85)
\mbox{%%%--->
\Acrobatmenu{LastPage}{%
\tikz[baseline] \node[rectangle,inner sep=2pt,minimum height=3.1ex,
rounded corners,drop shadow,shadow scale=1,shadow xshift=.8ex,
shadow yshift=-.4ex,opacity=.7,fill=black!50,top color=red!90!black!50,
bottom color=red!80!black!80,draw=red!50!black!50,very thick,text=white,
text opacity=1,minimum width=3cm,font=\bfseries\sffamily] at (0,0) {К концу};
}\Acrobatmenu{GoBack}{%
\tikz[baseline] \node[rectangle,inner sep=2pt,minimum height=3.1ex,
rounded corners,drop shadow,shadow scale=1,shadow xshift=.8ex,
shadow yshift=-.4ex,opacity=.7,fill=black!50,top color=red!90!black!50,
bottom color=red!80!black!80,draw=red!50!black!50,very thick,text=white,
text opacity=1,minimum width=3cm,font=\bfseries\sffamily] at (4,0) {Назад};
}\Acrobatmenu{PrevPage}{%
\tikz[baseline] \node[rectangle,inner sep=2pt,minimum height=3.1ex,
rounded corners,drop shadow,shadow scale=1,shadow xshift=.8ex,
shadow yshift=-.4ex,opacity=.7,fill=black!50,top color=red!90!black!50,
bottom color=red!80!black!80,draw=red!50!black!50,very thick,text=white,
text opacity=1,minimum width=3cm,font=\bfseries\sffamily] at (8,0) {Предыдущий};
}\Acrobatmenu{NextPage}{%
\tikz[baseline] \node[rectangle,inner sep=2pt,minimum height=3.1ex,
rounded corners,drop shadow,shadow scale=1,shadow xshift=.8ex,
shadow yshift=-.4ex,opacity=.7,fill=black!50,top color=red!90!black!50,
bottom color=red!80!black!80,draw=red!50!black!50,very thick,text=white,
text opacity=1,minimum width=3cm,font=\bfseries\sffamily] at (12,0) {Следующий};
}\Acrobatmenu{GoForward}{%
\tikz[baseline] \node[rectangle,inner sep=2pt,minimum height=3.1ex,
rounded corners,drop shadow,shadow scale=1,shadow xshift=.8ex,
shadow yshift=-.4ex,opacity=.7,fill=black!50,top color=red!90!black!50,
bottom color=red!80!black!80,draw=red!50!black!50,very thick,text=white,
text opacity=1,minimum width=3cm,font=\bfseries\sffamily] at (16,0) {Вперед};
}\Acrobatmenu{FirstPage}{%
\tikz[baseline] \node[rectangle,inner sep=2pt,minimum height=3.1ex,
rounded corners,drop shadow,shadow scale=1,shadow xshift=.8ex,
shadow yshift=-.4ex,opacity=.7,fill=black!50,top color=red!90!black!50,
bottom color=red!80!black!80,draw=red!50!black!50,very thick,text=white,
text opacity=1,minimum width=3cm,font=\bfseries\sffamily] at (20,0) {К началу};
}\Acrobatmenu{FullScreen}{%
\tikz[baseline] \node[rectangle,inner sep=2pt,minimum height=3.1ex,
rounded corners,drop shadow,shadow scale=1,shadow xshift=.8ex,
shadow yshift=-.4ex,opacity=.7,fill=black!50,top color=red!90!black!50,
bottom color=red!80!black!80,draw=red!50!black!50,very thick,text=white,
text opacity=1,minimum width=3cm,font=\bfseries\sffamily] at (24,0) {Полный экран};
}\Acrobatmenu{Quit}{%
\tikz[baseline] \node[rectangle,inner sep=2pt,minimum height=3.1ex,
rounded corners,drop shadow,shadow scale=1,shadow xshift=.8ex,
shadow yshift=-.4ex,opacity=.7,fill=black!50,top color=red!90!black!50,
bottom color=red!80!black!80,draw=red!50!black!50,very thick,text=white,
text opacity=1,minimum width=3cm,font=\bfseries\sffamily] at (28,0) {Выход};
}	
}%%%---|
\end{textblock}
%%%-----------------------------------------------------------------------
