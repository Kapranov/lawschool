%\begin{tikzpicture}
%	\draw [preaction={fill=black,opacity=.5, transform
%	canvas={xshift=1mm,yshift=-1mm}}] [fill=light-blue] (0,0) rectangle (23,12);
%\end{tikzpicture}
%
%\newpage
%\begin{tikzpicture}
%	\draw
%	[preaction={fill=gray,opacity=.3, transform canvas={xshift=3mm,yshift=-3mm}}] 
%	[postaction={path fading=south,fill=gray}]
%	[postaction={path fading=south,fading angle=45,fill=blue,opacity=.5}]
%	[left color=black,right color=red,draw=gray,line width=2mm]
%	(0,0) rectangle (23,12);
%\end{tikzpicture}
%
%%%-----------------------------------------------------------------------
%%%-----------------------------------------------------------------------
%%%-----------------------------------------------------------------------
\newpage

\begin{flushright}
	\vglue 25pt
\begin{tikzpicture}
	\draw [preaction={fill=black,opacity=.5, transform
	canvas={xshift=1mm,yshift=-1mm}}] [fill=light-blue!30!white,top
		color=light-blue!30!white, bottom color=gray,
		line width=1pt] (0,0) rectangle (23,12);
	\node[fill=white,minimum width=215pt, minimum height=25pt,
		top color=white,bottom color=white,line width=0mm] at (16.5,12.0) {};	
	\node[opacity=0.5, font=\huge\bfseries\sffamily, text=white, text
		opacity=1, drop shadow,fill=light-blue!50!black,
		fill,draw] at (16.5,12.1) {Додаткова література};
	\node[draw opacity=0.8, fill opacity=0.5, text opacity=1, rotate=90,
		font=\large\bfseries\sffamily, text=white, text opacity=1,
		fill=red!50!black,fill,draw]
	at (-0.3,4.3) {Адміністративне право України};
	\node[cloud callout, cloud puffs=15, aspect=2.5, cloud puff arc=120,
	shading=ball, text=white, font=\large\bfseries\sffamily, drop shadow]
	at (1,13.2) {Luga\TeX};
	\pgfimage[interpolate=true,height=12cm]{107.jpg}
	\node[text width=350pt, text justified] at (7,7) {%
		{\bf Рыжиков}~Ю. {\bf Работа над диссертацией по техническим
		наукам}.\\[5pt]
		2-е изд., перераб. и доп Издательство БХВ-Петербург, 2007 г.
		512 стр.\\[5pt]
		SBN: 5-9775-0138-2\\[15pt]
		\parbox{340pt}{%
		Книга представляет собой свод методических рекомендаций по
		написанию и оформлению диссертаций. В ней приведены
		требования к ученым и к диссертациям; даны определения
		базовых понятий науковедения; описана методика постановки
		задачи, сбора материала, написания глав диссертации,
		подготовки к защите. Дан обзор теоретического вооружения
		<<технического>> ученого (логика, прикладная математика,
		программирование) и его технологической оснастки (пакеты
		математических программ, система подготовки математических
		рукописей \LaTeX, Visio). Большое внимание уделяется
		литературной отделке рукописи, приводятся многочисленные
		примеры стилистических погрешностей и рекомендации по их
		устранению. Для аспирантов, докторантов и соискателей
		ученых степеней, студентов технических вузов и
		преподавателей.}};
\end{tikzpicture}
\end{flushright}
%%%---> PAGE 2
\newpage

\begin{flushright}
	\vglue 62pt
\begin{tikzpicture}
	\draw [preaction={fill=black,opacity=.5, transform
	canvas={xshift=1mm,yshift=-1mm}}] [fill=light-blue!30!white,top
		color=light-blue!30!white, bottom color=gray,
		line width=1pt] (0,0) rectangle (23,12);
	\node[fill=white,minimum width=215pt, minimum height=25pt,
		top color=white,bottom color=white,line width=0mm] at (16.5,12.0) {};	
	\node[opacity=0.5, font=\huge\bfseries\sffamily, text=white, text
		opacity=1, drop shadow,fill=light-blue!50!black,
		fill,draw] at (16.5,12.1) {Додаткова література};
	\node[draw opacity=0.8, fill opacity=0.5, text opacity=1, rotate=90,
		font=\large\bfseries\sffamily, text=white, text opacity=1,
		fill=red!50!black,fill,draw]
	at (-0.3,4.3) {Адміністративне право України};
	\pgfimage[interpolate=true,height=12cm]{107.jpg}
	\node[text width=350pt, text justified] at (7,7) {%
		{\bf Рыжиков}~Ю. {\bf Работа над диссертацией по техническим
		наукам}.\\[5pt]
		2-е изд., перераб. и доп Издательство БХВ-Петербург, 2007 г.
		512 стр.\\[5pt]
		SBN: 5-9775-0138-2\\[15pt]
		\parbox{340pt}{%
		Книга представляет собой свод методических рекомендаций по
		написанию и оформлению диссертаций. В ней приведены
		требования к ученым и к диссертациям; даны определения
		базовых понятий науковедения; описана методика постановки
		задачи, сбора материала, написания глав диссертации,
		подготовки к защите. Дан обзор теоретического вооружения
		<<технического>> ученого (логика, прикладная математика,
		программирование) и его технологической оснастки (пакеты
		математических программ, система подготовки математических
		рукописей \LaTeX, Visio). Большое внимание уделяется
		литературной отделке рукописи, приводятся многочисленные
		примеры стилистических погрешностей и рекомендации по их
		устранению. Для аспирантов, докторантов и соискателей
		ученых степеней, студентов технических вузов и
		преподавателей.}};
\end{tikzpicture}
\end{flushright}
%%%-----------------------------------------------------------------------
%%%-----------------------------------------------------------------------
%%%-----------------------------------------------------------------------
%%%---> PAGE 4
\newpage
\begin{flushright}
	\vglue 77.5pt
\begin{tikzpicture}
	\draw [preaction={fill=black,opacity=.5, transform
	canvas={xshift=1mm,yshift=-1mm}}] [fill=white,line width=1pt]
		(0,0) rectangle (23,12);
	\pgftext[at=\pgfpoint{2.2cm}{0.1cm},left,base]
		{\pgfimage[interpolate=true,width=18cm]{out2.jpg}}
	\draw[fill=light-blue] (0,12) rectangle (23,10); 
	\node[rectangle, rounded corners=20pt,inner sep=11pt,
		fill=light-blue!50!black, font=\large\bfseries\sffamily]
		at (20,10) {\color{white}Журнальные статьи};
	\node[draw opacity=0.8, fill opacity=0.5, text opacity=1, rotate=90,
		font=\large\bfseries\sffamily, text=white, text opacity=1,
		fill=red!50!black,fill,draw]
	at (-0.3,4.3) {Адміністративне право України};
%	\pgfimage[interpolate=true,width=15cm]{out1.jpg}
%	\pgfimage[interpolate=true,width=15cm]{out2.jpg}
%\pgftext[at=\pgfpoint{1cm}{3cm},left,base]
%	{\pgfimage[interpolate=true,width=20cm]{out1.jpg}}
\end{tikzpicture}
\end{flushright}

%%%---> PAGE 4
\newpage

\begin{flushright}
\begin{tikzpicture}
	\draw [preaction={fill=black,opacity=.5, transform
	canvas={xshift=1mm,yshift=-1mm}}] [fill=white,line width=1pt]
		(0,0) rectangle (15,20);
	\pgftext[at=\pgfpoint{0.5cm}{-0.8cm},left,base]
		{\pgfimage[interpolate=true,height=20cm]{out1.pdf}}
	\draw[fill=light-blue] (0,20) rectangle (15,18); 
	\node[rectangle, rounded corners=20pt,inner sep=11pt,
		fill=light-blue!50!black, font=\large\bfseries\sffamily]
		at (11,18) {\color{white}Журнальные статьи};
	\node[draw opacity=0.8, fill opacity=0.5, text opacity=1, rotate=90,
		font=\large\bfseries\sffamily, text=white, text opacity=1,
		fill=red!50!black,fill,draw]
	at (-0.3,4.3) {Адміністративне право України};
	\node[cloud callout, cloud puffs=15, aspect=2.5, cloud puff arc=120,
	shading=ball, text=white, font=\large\bfseries\sffamily, drop shadow]
	at (1,21.2) {Luga\TeX};
\end{tikzpicture}
\end{flushright}

%%%---> PAGE 5
\newpage

\begin{flushright}
\begin{tikzpicture}
	\draw [preaction={fill=black,opacity=.5, transform
	canvas={xshift=1mm,yshift=-1mm}}] [fill=white,line width=1pt]
		(0,0) rectangle (15,20);
	\pgftext[at=\pgfpoint{0.5cm}{-0.8cm},left,base]
		{\pgfimage[interpolate=true,height=20cm]{out5.pdf}}
	\draw[fill=light-blue] (0,20) rectangle (15,18); 
	\node[rectangle, rounded corners=20pt,inner sep=11pt,
		fill=light-blue!50!black, font=\large\bfseries\sffamily]
		at (11,18) {\color{white}Журнальные статьи};
	\node[draw opacity=0.8, fill opacity=0.5, text opacity=1, rotate=90,
		font=\large\bfseries\sffamily, text=white, text opacity=1,
		fill=red!50!black,fill,draw]
	at (-0.3,4.3) {Адміністративне право України};
\end{tikzpicture}
\end{flushright}
%%%-----------------------------------------------------------------------
%%%-----------------------------------------------------------------------
%%%---> Minmap graph

%\def\lecture#1#2#3#4#5#6{
%\node [annotation, #3, scale=0.65, text width=4cm, inner sep=2mm] at (#4) {
%Lecture #1: \textcolor{orange}{\textbf{#2}}
%\list{--}{\topsep=2pt\itemsep=0pt\parsep=0pt
%\parskip=0pt\labelwidth=8pt\leftmargin=8pt
%\itemindent=0pt\labelsep=2pt}
%	#5
%	\endlist
%	};
%}

\begin{flushright}
\begin{tikzpicture} [mindmap,
		every node/.style={concept, circular drop shadow, execute at begin node=\hskip0pt},
		root concept/.append style={
		concept color=black, fill=white, line width=1ex, text=black},
		text=white, grow cyclic,
		level 1/.append style={level distance=6cm,sibling angle=45},
		level 2/.append style={level distance=3cm,sibling
		angle=35},scale=0.9]
		\clip (-8,-8) rectangle ++(16,16);
		\node [root concept, text width=115pt,
			font=\Large\bfseries\sffamily] (mainform) {Форми\\
				поточного\\ контролю} % root
		child [concept color=red] { node [text width=90pt, font=\small\bfseries\sffamily]
			{опитування під час проведення
			семінарських і практичних занять}
%		child { node {Problem Measures} }
		}
		child [concept color=blue] { node [text width=80pt,
			font=\small\bfseries\sffamily] {проведення контрольних робіт}
%		child { node {Turing Machines} }
		}
		child [concept color=orange] { node [text width=90pt,
			font=\small\bfseries\sffamily] {проведення\\ опитування з\\
			застосуванням\\ спеціальних\\ комп'ютерних\\ програм}
%		child { node {Complexity Measures} }
		}
		child [concept color=green!50!black] { node [text width=90pt,
			font=\small\bfseries\sffamily] {заслуховування\\
			рефератів,\\ доповідей}
			}
%		child { node {Exact Algorithms} }
		child [concept color=magenta!50!black] { node [font=\small\bfseries\sffamily] {вирішення задач}
			}
		child [concept color=yellow!50!black] { node
		[font=\small\bfseries\sffamily] {оцінювання виконання вправ}
			}
		child [concept color=cyan!50!black] { node
		[font=\small\bfseries\sffamily] {оцінювання вирішення
		практичних завдань} };
%%%---> It's work
%\begin{scope}[every annotation/.style={fill=black!40}]
%\node [annotation, above] at (mainform.north) {
%	Lecture 1: Computational Problems
%\begin{itemize}
%\item Knowledge of several key problems
%\item Knowledge of problem encondings
%\item Being able to formalize problems
%\end{itemize}
%};
%\end{scope}
%
%\lecture{1}{mainform}{above,xshift=-3mm}
%{mainform.north}{
%\item Knowledge of several key problems
%\item Knowledge of problem encondings
%\item Being able to formalize problems
%}{2009-04-08}

\begin{pgfonlayer}{background}
%%%---> original size
%\clip (-1.5,-5) rectangle ++(4,10);
%\clip (-6.3,-7.8) rectangle ++(14.3,15.2);
\colorlet{upperleft}{green!50!black!25}
\colorlet{upperright}{orange!25}
\colorlet{lowerleft}{red!25}
\colorlet{lowerright}{blue!25}

% The large rectangles:
\fill [upperleft] (mainform) rectangle ++(-33,20);
\fill [upperright] (mainform) rectangle ++(20,20);
\fill [lowerleft] (mainform) rectangle ++(-33,-20);
\fill [lowerright] (mainform) rectangle ++(20,-20);

% The shadings:
\shade [left color=upperleft,right color=upperright]
([xshift=-1cm]mainform) rectangle ++(2,20);
\shade [left color=lowerleft,right color=lowerright]
([xshift=-1cm]mainform) rectangle ++(2,-33);
\shade [top color=upperleft,bottom color=lowerleft]
([yshift=-1cm]mainform) rectangle ++(-33,2);
\shade [top color=upperright,bottom color=lowerright]
([yshift=-1cm]mainform) rectangle ++(20,2);
\end{pgfonlayer}
\end{tikzpicture}
\end{flushright}
%%%-----------------------------------------------------------------------
\newpage
\begin{center}
\begin{tikzpicture}
\node (tbl) {
\begin{tabularx}{430pt}{c}
\textcolor{white}{\bf График диаграмма форм поточного контролю:}\\[1ex]
\begin{tikzpicture} [mindmap,
		every node/.style={concept, circular drop shadow, execute at begin node=\hskip0pt},
		root concept/.append style={
		concept color=black, fill=white, line width=1ex, text=black},
		text=white, grow cyclic,
		level 1/.append style={level distance=6cm,sibling angle=45},
		level 2/.append style={level distance=3cm,sibling
		angle=35},scale=0.9]
		\clip (-8,-8) rectangle ++(16,16);
		\node [root concept, text width=115pt,
			font=\Large\bfseries\sffamily] (mainform) {Форми\\
				поточного\\ контролю} % root
		child [concept color=red] { node [text width=90pt, font=\small\bfseries\sffamily]
			{опитування під час проведення
			семінарських і практичних занять}
%		child { node {Problem Measures} }
		}
		child [concept color=blue] { node [text width=80pt,
			font=\small\bfseries\sffamily] {проведення контрольних робіт}
%		child { node {Turing Machines} }
		}
		child [concept color=orange] { node [text width=90pt,
			font=\small\bfseries\sffamily] {проведення\\ опитування з\\
			застосуванням\\ спеціальних\\ комп'ютерних\\ програм}
%		child { node {Complexity Measures} }
		}
		child [concept color=green!50!black] { node [text width=90pt,
			font=\small\bfseries\sffamily] {заслуховування\\
			рефератів,\\ доповідей}
			}
%		child { node {Exact Algorithms} }
		child [concept color=magenta!50!black] { node [font=\small\bfseries\sffamily] {вирішення задач}
			}
		child [concept color=yellow!50!black] { node
		[font=\small\bfseries\sffamily] {оцінювання виконання вправ}
			}
		child [concept color=cyan!50!black] { node
		[font=\small\bfseries\sffamily] {оцінювання вирішення
		практичних завдань} };
\begin{pgfonlayer}{background}
%%%---> original size
%\clip (-1.5,-5) rectangle ++(4,10);
%\clip (-6.3,-7.8) rectangle ++(14.3,15.2);
\clip (-8.21,-8.05) rectangle ++(16.7,16.15);
\colorlet{upperleft}{green!50!black!25}
\colorlet{upperright}{orange!25}
\colorlet{lowerleft}{red!25}
\colorlet{lowerright}{blue!25}

% The large rectangles:
\fill [upperleft] (mainform) rectangle ++(-33,20);
\fill [upperright] (mainform) rectangle ++(20,20);
\fill [lowerleft] (mainform) rectangle ++(-33,-20);
\fill [lowerright] (mainform) rectangle ++(20,-20);

% The shadings:
\shade [left color=upperleft,right color=upperright]
([xshift=-1cm]mainform) rectangle ++(2,20);
\shade [left color=lowerleft,right color=lowerright]
([xshift=-1cm]mainform) rectangle ++(2,-33);
\shade [top color=upperleft,bottom color=lowerleft]
([yshift=-1cm]mainform) rectangle ++(-33,2);
\shade [top color=upperright,bottom color=lowerright]
([yshift=-1cm]mainform) rectangle ++(20,2);
\end{pgfonlayer}

\end{tikzpicture}
\end{tabularx}
};
\begin{pgfonlayer}{background}
\draw[rounded corners,top color=green,bottom color=black,draw=white]
	($(tbl.north west)+(0.14,0)$) rectangle ($(tbl.north east)-(0.13,0.9)$);
\draw[rounded corners,top color=white,bottom color=black,
	middle color=green,draw=blue!20] ($(tbl.south west) +(0.12,0.5)$)
		rectangle ($(tbl.south east)-(0.12,0)$);
\draw[top color=blue!1,bottom color=blue!20,draw=white]
	($(tbl.north east)-(0.13,0.6)$) rectangle ($(tbl.south west)+(0.13,0.2)$);
\end{pgfonlayer}
\end{tikzpicture}
\end{center}
%%%-----------------------------------------------------------------------
%%%---> Calendar
\newpage

\def\lecture#1#2#3#4#5#6{
% As before:
\node [annotation, #3, scale=0.65, text width=4cm, inner sep=2mm, fill=white] at (#4) {
Lecture #1: \textcolor{orange}{\textbf{#2}}
\list{--}{\topsep=2pt\itemsep=0pt\parsep=0pt
\parskip=0pt\labelwidth=8pt\leftmargin=8pt
\itemindent=0pt\labelsep=2pt}
	#5
	\endlist
	};
	% New:
	\node [anchor=base west] at (cal-#6.base east) {\textcolor{orange}{\textbf{#2}}};
}

\noindent
\begin{tikzpicture}[anchor=mid]
	\begin{scope}[
		mindmap,
		every node/.style={concept, circular drop shadow,execute at begin node=\hskip0pt},
		root concept/.append style={
		concept color=black,
		fill=white, line width=1ex,
		text=black, font=\large\scshape},
		text=white,
		computational problems/.style={concept color=red,faded/.style={concept color=red!50}},
		computational models/.style={concept color=blue,faded/.style={concept color=blue!50}},
		measuring complexity/.style={concept color=orange,faded/.style={concept color=orange!50}},
		solving problems/.style={concept color=green!50!black,faded/.style={concept color=green!50!black!50}},
		grow cyclic,
		level 1/.append style={level distance=4.5cm,sibling angle=90,font=\scshape},
		level 2/.append style={level distance=3cm,sibling angle=45,font=\scriptsize}]
		\node [root concept, font=\bfseries\sffamily, text width=115pt]
		(Computational Complexity) {Административное\par право} % root
		child [computational problems] { node [yshift=-1cm] (Computational Problems) {Computational Problems}
		child { node (Problem Measures) {Problem Measures} }
		child { node (Problem Aspects) {Problem Aspects} }
		child [faded] { node (problem Domains) {Problem Domains} }
		child { node (Key Problems) {Key Problems} }
		}
		child [computational models] { node [yshift=-1cm] (Computational Models) {Computational Models}
		child { node (Turing Machines) {Turing Machines} }
		child [faded] { node (Random-Access Machines) {Random-Access Machines} }
		child { node (Circuits) {Circuits} }
		child [faded] { node (Binary Decision Diagrams) {Binary Decision Diagrams} }
		child { node (Oracle Machines) {Oracle Machines} }
		child { node (Programming in Logic) {Programming in Logic} }
		}
		child [measuring complexity] { node [yshift=1cm] (Measuring Complexity) {Measuring Complexity}
		child { node (Complexity Measures) {Complexity Measures} }
		child { node (Classifying Complexity) {Classifying Complexity} }
		child { node (Comparing Complexity) {Comparing Complexity} }
		child [faded] { node (Describing Complexity) {Describing Complexity} }
		}
		child [solving problems] { node [yshift=1cm] (Solving Problems) {Solving Problems}
		child { node (Exact Algorithms) {Exact Algorithms} }
		child { node (Randomization) {Randomization} }
		child { node (Fixed-Parameter Algorithms) {Fixed-Parameter Algorithms} }
		child { node (Parallel Computation) {Parallel Computation} }
		child { node (Partial Solutions) {Partial Solutions} }
		child { node (Approximation) {Approximation} }
		};
	\end{scope}
%%%---> Left side calendar	
	\tiny
	\calendar (mycalsep) [day list downward,
%	month text=\%mt\ \%y0,
	month text=Сентябрь\quad 2010,
	month yshift=3.5em,
	name=cal,
%	at={(-.5\textwidth-5mm,.5\textheight-1cm)},
%	at={(-.48\textwidth-5mm,.48\textheight-4cm)},
at={(-17,9)},
	dates=2010-09-01 to 2010-09-last]
%	if (equals=2010-09-25) {\draw (0,0) circle (4pt);}
	if (weekend)
	[white]
	if (day of month=1) {
	\node at (.0em,1.5em) [anchor=base west] {\large\bfseries\sffamily\tikzmonthtext};
	};
	\draw[red] (cal-2010-09-20) circle (4pt);
	\calendar (mycalocb) [day list downward,
%	month text=\%mt\ \%y0,
	month text=Октябрь \quad 2010,
	month yshift=3.5em,
	name=cal,
%	at={(-.5\textwidth-5mm,.5\textheight-1cm)},
%	at={(-.48\textwidth-5mm,.48\textheight-4cm)},
at={(-17,-2.5)},
	dates=2010-10-01 to 2010-10-last]
%	if (at most=2010-10-04) [nodes={strike out,draw}]
	if (weekend)
	[white]
	if (day of month=1) {
	\node at (.0em,1.5em) [anchor=base west] {\large\bfseries\sffamily\tikzmonthtext};
	};
	\node[starburst,drop shadow,fill=white,draw] at (10,5) {Экзамен};
	\node[starburst, fill=yellow, draw=red, line width=2pt] at (15,3) {\bf Модуль!};
%%%---> Right side	calendar
	\calendar (mycalnov) [day list downward,
%	month text=\%mt\ \%y0,
	month text=Ноябрь \quad 2010,
	name=cal,
%	at={(-.5\textwidth-5mm,.5\textheight-1cm)},
%	at={(.48\textwidth-15mm,.48\textheight-4cm)},
at={(15.2,9)},
	dates=2010-11-01 to 2010-11-last]
	if (weekend)
	[white]
	if (day of month=1) {
	\node at (-1.5em,1.5em) [anchor=base east] {\large\bfseries\sffamily\tikzmonthtext};
	};
	\calendar (mycaldec) [day list downward,
%	month text=\%mt\ \%y0,
	month text=Декабрь \quad 2010,
	name=cal,
%	at={(-.5\textwidth-5mm,.5\textheight-1cm)},
%	at={(.48\textwidth-15mm,.48\textheight-4cm)},
at={(15.2,-2.5)},
	dates=2010-12-01 to 2010-12-last]
	if (weekend)
	[white]
	if (day of month=1) {
	\node at (-1.5em,1.5em) [anchor=base east] {\large\bfseries\sffamily\tikzmonthtext};
	};
%%%--->	
	\lecture{1}{Тема 1.2. Співвідношення адміністративного права з іншими
	галузями}{above,xshift=-5mm,yshift=5mm}{Computational Problems.north}{
\item Мета заняття;
\item Основні поняття;
\item Навчальні питання;
\item Методичні рекомендації та пояснення;
\item Індивідуальні навчально-дослідницькі завдання;
\item Питання для самоконтролю та самоперевірки;
\item Додаткова література.	
	}{2010-09-08}
	\lecture{2}{Тема 2.2. Принципи й функції державного управління}{above left}
	{Computational Models.west}{
\item Мета заняття;
\item Основні поняття;
\item Навчальні питання;
\item Методичні рекомендації та пояснення;
\item Індивідуальні навчально-дослідницькі завдання;
\item Питання для самоконтролю та самоперевірки;
\item Додаткова література.	
	}{2010-09-15}
%%%	
	\begin{pgfonlayer}{background}
		\clip[xshift=-1cm] (-.5\textwidth,-.5\textheight) rectangle ++(\textwidth,\textheight);
		\colorlet{upperleft}{green!50!black!25}
		\colorlet{upperright}{orange!25}
		\colorlet{lowerleft}{red!25}
		\colorlet{lowerright}{blue!25}
		% The large rectangles:
		\fill [upperleft] (Computational Complexity) rectangle ++(-20,20);
		\fill [upperright] (Computational Complexity) rectangle ++(20,20);
		\fill [lowerleft] (Computational Complexity) rectangle ++(-20,-20);
		\fill [lowerright] (Computational Complexity) rectangle ++(20,-20);
		% The shadings:
		\shade [left color=upperleft,right color=upperright]
		([xshift=-1cm]Computational Complexity) rectangle ++(2,20);
		\shade [left color=lowerleft,right color=lowerright]
		([xshift=-1cm]Computational Complexity) rectangle ++(2,-20);
		\shade [top color=upperleft,bottom color=lowerleft]
		([yshift=-1cm]Computational Complexity) rectangle ++(-20,2);
		\shade [top color=upperright,bottom color=lowerright]
		([yshift=-1cm]Computational Complexity) rectangle ++(20,2);
	\end{pgfonlayer}
\end{tikzpicture}

%%%---> New  Page calendar 2010
\sffamily
\colorlet{winter}{blue}
\colorlet{spring}{green!60!black}
\colorlet{summer}{orange}
\colorlet{fall}{red}
% A counter, since TikZ is not clever enough (yet) to handle
% arbitrary angle systems.
\newcount\mycount
\begin{center}
\begin{tikzpicture}
	[transform shape,
	every day/.style={anchor=mid,font=\fontsize{6}{6}\selectfont}]
	\node [text width=150pt] (Computational Complexity) {\large\bfseries\sffamily Учебный год \the\year};
	\foreach \month/\monthcolor in
	{1/winter,2/winter,3/spring,4/spring,5/spring,6/summer,
	7/summer,8/summer,9/fall,10/fall,11/fall,12/winter}
	{
	% Computer angle:
	\mycount=\month
	\advance\mycount by -1
	\multiply\mycount by 30
	\advance\mycount by -90
	% The actual calendar
	\calendar at (\the\mycount:6.4cm)
	[
	dates=\the\year-\month-01 to \the\year-\month-last,
	]
	if (day of month=1) {\color{\monthcolor}\tikzmonthcode}
	if (Sunday) [red]
	if (all)
	{
	% Again, compute angle
	\mycount=1
	\advance\mycount by -\pgfcalendarcurrentday
	\multiply\mycount by 11
	\advance\mycount by 90
	\pgftransformshift{\pgfpointpolar{\mycount}{1.4cm}}
	};
	}
	\begin{pgfonlayer}{background}
		\clip[xshift=-1cm] (-.5\textwidth,-.5\textheight) rectangle ++(\textwidth,\textheight);
		\colorlet{upperleft}{green!50!black!25}
		\colorlet{upperright}{orange!25}
		\colorlet{lowerleft}{red!25}
		\colorlet{lowerright}{blue!25}
		% The large rectangles:
		\fill [upperleft] (Computational Complexity) rectangle ++(-20,20);
		\fill [upperright] (Computational Complexity) rectangle ++(20,20);
		\fill [lowerleft] (Computational Complexity) rectangle ++(-20,-20);
		\fill [lowerright] (Computational Complexity) rectangle ++(20,-20);
		% The shadings:
		\shade [left color=upperleft,right color=upperright]
		([xshift=-1cm]Computational Complexity) rectangle ++(2,20);
		\shade [left color=lowerleft,right color=lowerright]
		([xshift=-1cm]Computational Complexity) rectangle ++(2,-20);
		\shade [top color=upperleft,bottom color=lowerleft]
		([yshift=-1cm]Computational Complexity) rectangle ++(-20,2);
		\shade [top color=upperright,bottom color=lowerright]
		([yshift=-1cm]Computational Complexity) rectangle ++(20,2);
	\end{pgfonlayer}
\end{tikzpicture}

\sffamily\scriptsize
\tikz
\calendar [dates=2000-01-01 to 2000-12-31,
month list,month label left,month yshift=1.25em]
if (Sunday) [black!50];

\vglue 15pt

\begin{tikzpicture}[decoration={footprints,foot length=20pt}]
	\fill [decorate] (0,0) -- (3,0);
\end{tikzpicture}

\vglue 15pt

\begin{tikzpicture}[decoration={footprints,stride length=50pt}]
	\fill [decorate] (0,0) -- (3,0);
\end{tikzpicture}

\vglue 15pt

\begin{tikzpicture}[decoration={footprints,foot sep=10pt}]
	\fill [decorate] (0,0) -- (3,0);
\end{tikzpicture}

\vglue 15pt

\begin{tikzpicture}[decoration={footprints,foot angle=60}]
	\fill [decorate] (0,0) -- (3,0);
\end{tikzpicture}

\vglue 15pt

\begin{tikzpicture}
	\path[mindmap,concept color=black,text=white]
	node[concept] {Computer Science}
	[clockwise from=0]
	child[concept color=green!50!black] {
	node[concept] {practical}
	[clockwise from=90]
	child { node[concept] {algorithms} }
	child { node[concept] {data structures} }
	child { node[concept] {pro\-gramming languages} }
	child { node[concept] {software engineer\-ing} }
	}
	child[concept color=blue] {
	node[concept] {applied}
	[clockwise from=-30]
	child { node[concept] {databases} }
	child { node[concept] {WWW} }
	}
	child[concept color=red] { node[concept] {technical} }
	child[concept color=orange] { node[concept] {theoretical} };
\end{tikzpicture}

\newpage

\begin{minipage}{3.5cm}\raggedright
	\color{red}Red text,%
	\begin{colormixin}{25!white}
		washed-out red text,
		\color{blue} washed-out blue text,
		\begin{colormixin}{25!black}
			dark washed-out blue text,
			\color{green} dark washed-out green text,%
		\end{colormixin}
		back to washed-out blue text,%
	\end{colormixin}
	and back to red.
\end{minipage}%

\vglue 15pt

\pgfmathsetseed{1}
\foreach \col in {black,red,green,blue}
{
\begin{tikzpicture}[x=10pt,y=10pt,ultra thick,baseline,line cap=round]
	\coordinate (current point) at (0,0);
	\coordinate (old velocity) at (0,0);
	\coordinate (new velocity) at (rand,rand);
	\foreach \i in {0,1,...,100}
	{
	\draw[\col!\i] (current point)
	.. controls ++([scale=-1]old velocity) and
	++(new velocity) .. ++(rand,rand)
	coordinate (current point);
	\coordinate (old velocity) at (new velocity);
	\coordinate (new velocity) at (rand,rand);
	}
\end{tikzpicture}
}

\begin{tikzpicture}
	\node[chamfered rectangle, white, fill=red, double=red, draw, very
	thick]
	{\bf STOP!};
\end{tikzpicture}

\begin{tikzpicture}
	\foreach \i in {1,...,4}
	\node[starburst,drop shadow,fill=white,draw] at (0,\i) {Burst \i};
\end{tikzpicture}

\begin{tikzpicture}
	\node [forbidden sign,line width=1ex,draw=red,fill=white] {Smoking};
\end{tikzpicture}

\begin{tikzpicture}[aspect=1, every node/.style={cloud, cloud puffs=11,
	draw}]
	\node [fill=gray!20] {rain};
	\node [cloud ignores aspect, fill=white] at (1.5,0) {snow};
\end{tikzpicture}

\begin{tikzpicture}
	\node[starburst, fill=yellow, draw=red, line width=2pt] {\bf BANG!};
\end{tikzpicture}

\begin{tikzpicture}
	\draw[help lines] grid(3,2);
	\node[starburst, draw, minimum width=3cm, minimum height=2cm]
	at (1.5, 1) {\bf BOOM!};
\end{tikzpicture}

\begin{tikzpicture}
	[every node/.style={signal, draw, text=white, signal to=nowhere}]
	\node[fill=green!65!black, signal to=east] at (0,1) {To East};
	\node[fill=red!65!black, signal from=east] at (0,0) {From East};
\end{tikzpicture}

\begin{tikzpicture}
	\node[cloud callout, cloud puffs=15, aspect=2.5, cloud puff arc=120,
	shading=ball,text=white] {\bf Imagine...};
\end{tikzpicture}

\begin{tikzpicture}
	\draw[help lines] (0,0) grid (3,2);
	\node [cross out,draw=red] at (1.5,1) {cross out};
\end{tikzpicture}

\end{center}
%%%-----------------------------------------------------------------------
%%%---> Cover Tikz pgfmanual
%%%-----------------------------------------------------------------------
%%%-----------------------------------------------------------------------
%%%---> Article.pdf
%\newpage
%\begin{tikzpicture}
%	\draw [preaction={fill=black,opacity=.5, transform
%	canvas={xshift=1mm,yshift=-1mm}}] [fill=white,line width=1pt]
%		(0,0) rectangle (15,20);
%	\pgftext[at=\pgfpoint{0.5cm}{-1.5cm},left,base]
%		{\pgfimage[interpolate=true,height=20cm]{page-003.pdf}}
%	\draw[fill=light-blue] (0,20) rectangle (15,18); 
%	\node[rectangle, rounded corners=20pt,inner sep=11pt,
%		fill=light-blue!50!black, font=\large\bfseries\sffamily]
%		at (11,18) {\color{white}Журнальные статьи};
%	\node[draw opacity=0.8, fill opacity=0.5, text opacity=1, rotate=90,
%		font=\large\bfseries\sffamily, text=white, text opacity=1,
%		fill=red!50!black,fill,draw]
%	at (-0.3,4.3) {Адміністративне право України};
%	\node at (1, 19) {%
%	\begin{tikzpicture}[even odd rule,rounded
%		corners=2pt,x=10pt,y=10pt,drop shadow]
%	\filldraw[fill=yellow!90!black!40,drop shadow] (0,0)   rectangle (1,1)
%		[xshift=5pt,yshift=5pt] (0,0) rectangle (1,1)
%		[rotate=30] (-1,-1) rectangle (2,2);
%		\node at (0,1.7) {\textbf{\thepage}};
%\end{tikzpicture}
%	};
%\end{tikzpicture}
%
%
%%%-----------------------------------------------------------------------
%%%-----------------------------------------------------------------------
%%%-----------------------------------------------------------------------
%
%%%---> Bottom shadow page 167
%\newpage
%\begin{tikzpicture}
%	[
%	% Define an interesting style
%	button/.style={
%	% First preaction: Fuzzy shadow
%	preaction={fill=black,path fading=circle with fuzzy edge 20 percent,
%	opacity=.5,transform canvas={xshift=1mm,yshift=-1mm}},
%	% Second preaction: Background pattern
%	preaction={pattern=#1,
%	path fading=circle with fuzzy edge 15 percent},
%	% Third preaction: Make background shiny
%	preaction={top color=white,
%	bottom color=black!50,
%	shading angle=45,
%	path fading=circle with fuzzy edge 15 percent,
%	opacity=0.2},
%	% Fourth preaction: Make edge especially shiny
%	preaction={path fading=fuzzy ring 15 percent,
%	top color=black!5,
%	bottom color=black!80,
%	shading angle=45},
%	inner sep=2ex
%	},
%	button/.default=horizontal lines light blue,
%	circle
%	]
%%	\draw [help lines] (0,0) grid (4,3);
%	\node [button,minimum size=5cm] at (15.2,1) {\bf\huge Luga\TeX};
%	\node [button=crosshatch dots light steel blue, text=white,minimum
%		size=3cm] at (1,1.5) {\bf Luga\TeX};
%\end{tikzpicture}
%\vfill
%\begin{tikzpicture}
%	\node [circular drop shadow={shadow scale=1.05},minimum size=3.13cm,
%		fill=blue!20,draw,thick,circle] {Hello!};
%\end{tikzpicture}
%
%\vfill
%% \huge\bfseries\sffamily
%\begin{tikzfadingfrompicture}[name=tikz]
%	\node [text=transparent!20]
%	{\fontfamily{ptm}\fontsize{65}{65}\Huge\bfseries\selectfont Ti\emph{k}Z};
%\end{tikzfadingfrompicture}
% Now we use the fading in another picture:
%\begin{tikzpicture}
%%	\fill [black!20] (-2,-1) rectangle (2,1);
%%	\pattern [pattern=checkerboard,pattern color=black!30]
%%	(-2,-1) rectangle (2,1);
%	\shade[path fading=tikz,fit fading=false, left color=blue, 
%		right color=black] (-2,-1) rectangle (2,1);
%\end{tikzpicture}
%\vfill
%\begin{tikzpicture}
%	% Checker board
%	\fill [black!20] (0,0) rectangle (4,4);
%	\path [pattern=checkerboard,pattern color=black!30] (0,0) rectangle
%	(4,4);
%	\shade [ball color=blue,path fading=south] (2,2) circle (1.8);
%\end{tikzpicture}
%
%\vfill
%
%\tikzfading[name=fade inside,
%inner color=transparent!80,
%outer color=transparent!30]
%\begin{tikzpicture}
%	% Checker board
%	\fill [black!20] (0,0) rectangle (4,4);
%	\path [pattern=checkerboard,pattern color=black!30] (0,0) rectangle
%	(4,4);
%	\shade [ball color=red] (3,3) circle (0.8);
%	\shade [ball color=white,path fading=fade inside] (2,2) circle (1.8);
%\end{tikzpicture}
%
%\vfill
%
%\tikzfading[name=middle,
%top color=transparent!50,
%bottom color=transparent!50,
%middle color=transparent!20]
%\begin{tikzpicture}
%	\node [circle,circular drop shadow,
%	pattern=horizontal lines dark blue,
%	path fading=south,
%	minimum size=3.6cm] {};
%	\pattern [path fading=north,
%	pattern=horizontal lines dark gray]
%	(0,0) circle (1.8cm);
%	\pattern [path fading=middle,
%	pattern=crosshatch dots light steel blue]
%	(0,0) circle (1.8cm);
%\end{tikzpicture}
%
%\newpage
%%\vfill
%%\tikz \calendar[dates=2000-01-01 to 2000-01-31,week list];
%%\vfill
%\sffamily\scriptsize
%\tikz
%\calendar [dates=2010-01-01 to 2010-12-31, month list,month label left,month yshift=1.25em]
%if (Sunday) [black!50];
%
%\newpage
%
%\tikz[mindmap,concept color=red!50]
%\node [concept] {Root concept}
%child[grow=right] {node[concept] {Child concept}};
%
%\newpage
%
%\pgfmathsetseed{1}
%\foreach \col in {black,red,green,blue}
%{
%\begin{tikzpicture}[x=10pt,y=10pt,ultra thick,baseline,line cap=round]
%	\coordinate (current point) at (0,0);
%	\coordinate (old velocity) at (0,0);
%	\coordinate (new velocity) at (rand,rand);
%	\foreach \i in {0,1,...,100}
%	{
%	\draw[\col!\i] (current point)
%	.. controls ++([scale=-1]old velocity) and
%	++(new velocity) .. ++(rand,rand)
%	coordinate (current point);
%	\coordinate (old velocity) at (new velocity);
%	\coordinate (new velocity) at (rand,rand);
%	}
%\end{tikzpicture}
%}
