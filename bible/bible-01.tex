%%%---> PAGE 0
%\newpage
%\begin{tikzpicture}
%	\draw
%	[preaction={fill=gray,opacity=.3, transform canvas={xshift=3mm,yshift=-3mm}}] 
%	[postaction={path fading=south,fill=gray}]
%	[postaction={path fading=south,fading angle=45,fill=blue,opacity=.5}]
%	[left color=black,right color=red,draw=gray,line width=2mm]
%	(0,0) rectangle (23,12);
%\end{tikzpicture}
%
%%%---> PAGE 1
\newpage

\begin{flushright}
	\vglue 25pt
\begin{tikzpicture}
	\draw [preaction={fill=black,opacity=.5, transform
	canvas={xshift=1mm,yshift=-1mm}}] [fill=light-blue!30!white,top
		color=light-blue!30!white, bottom color=gray,
		line width=1pt] (0,0) rectangle (23,12);
	\node[fill=white,minimum width=215pt, minimum height=25pt,
		top color=white,bottom color=white,line width=0mm] at (16.5,12.0) {};	
	\node[opacity=0.5, font=\huge\bfseries\sffamily, text=white, text
		opacity=1, drop shadow,fill=light-blue!50!black,
		fill,draw] at (16.5,12.1) {Додаткова література};
	\node[draw opacity=0.8, fill opacity=0.5, text opacity=1, rotate=90,
		font=\large\bfseries\sffamily, text=white, text opacity=1,
		fill=red!50!black,fill,draw]
	at (-0.3,4.3) {Адміністративне право України};
	\node[cloud callout, cloud puffs=15, aspect=2.5, cloud puff arc=120,
	shading=ball, text=white, font=\large\bfseries\sffamily, drop shadow]
	at (1,13.2) {Luga\TeX};
	\pgfimage[interpolate=true,height=12cm]{107.jpg}
	\node[text width=350pt, text justified] at (7,7) {%
		{\bf Рыжиков}~Ю. {\bf Работа над диссертацией по техническим
		наукам}.\\[5pt]
		2-е изд., перераб. и доп Издательство БХВ-Петербург, 2007 г.
		512 стр.\\[5pt]
		SBN: 5-9775-0138-2\\[15pt]
		\parbox{340pt}{%
		Книга представляет собой свод методических рекомендаций по
		написанию и оформлению диссертаций. В ней приведены
		требования к ученым и к диссертациям; даны определения
		базовых понятий науковедения; описана методика постановки
		задачи, сбора материала, написания глав диссертации,
		подготовки к защите. Дан обзор теоретического вооружения
		<<технического>> ученого (логика, прикладная математика,
		программирование) и его технологической оснастки (пакеты
		математических программ, система подготовки математических
		рукописей \LaTeX, Visio). Большое внимание уделяется
		литературной отделке рукописи, приводятся многочисленные
		примеры стилистических погрешностей и рекомендации по их
		устранению. Для аспирантов, докторантов и соискателей
		ученых степеней, студентов технических вузов и
		преподавателей.}};
\end{tikzpicture}
\end{flushright}
%%%---> PAGE 2
\newpage

\begin{flushright}
	\vglue 62pt
\begin{tikzpicture}
	\draw [preaction={fill=black,opacity=.5, transform
	canvas={xshift=1mm,yshift=-1mm}}] [fill=light-blue!30!white,top
		color=light-blue!30!white, bottom color=gray,
		line width=1pt] (0,0) rectangle (23,12);
	\node[fill=white,minimum width=215pt, minimum height=25pt,
		top color=white,bottom color=white,line width=0mm] at (16.5,12.0) {};	
	\node[opacity=0.5, font=\huge\bfseries\sffamily, text=white, text
		opacity=1, drop shadow,fill=light-blue!50!black,
		fill,draw] at (16.5,12.1) {Додаткова література};
	\node[draw opacity=0.8, fill opacity=0.5, text opacity=1, rotate=90,
		font=\large\bfseries\sffamily, text=white, text opacity=1,
		fill=red!50!black,fill,draw]
	at (-0.3,4.3) {Адміністративне право України};
	\pgfimage[interpolate=true,height=12cm]{107.jpg}
	\node[text width=350pt, text justified] at (7,7) {%
		{\bf Рыжиков}~Ю. {\bf Работа над диссертацией по техническим
		наукам}.\\[5pt]
		2-е изд., перераб. и доп Издательство БХВ-Петербург, 2007 г.
		512 стр.\\[5pt]
		SBN: 5-9775-0138-2\\[15pt]
		\parbox{340pt}{%
		Книга представляет собой свод методических рекомендаций по
		написанию и оформлению диссертаций. В ней приведены
		требования к ученым и к диссертациям; даны определения
		базовых понятий науковедения; описана методика постановки
		задачи, сбора материала, написания глав диссертации,
		подготовки к защите. Дан обзор теоретического вооружения
		<<технического>> ученого (логика, прикладная математика,
		программирование) и его технологической оснастки (пакеты
		математических программ, система подготовки математических
		рукописей \LaTeX, Visio). Большое внимание уделяется
		литературной отделке рукописи, приводятся многочисленные
		примеры стилистических погрешностей и рекомендации по их
		устранению. Для аспирантов, докторантов и соискателей
		ученых степеней, студентов технических вузов и
		преподавателей.}};
\end{tikzpicture}
\end{flushright}
%%%---> PAGE 3
%\newpage
%\begin{flushright}
%	\vglue 77.5pt
%\begin{tikzpicture}
%	\draw [preaction={fill=black,opacity=.5, transform
%	canvas={xshift=1mm,yshift=-1mm}}] [fill=white,line width=1pt]
%		(0,0) rectangle (23,12);
%	\pgftext[at=\pgfpoint{2.2cm}{0.1cm},left,base]
%		{\pgfimage[interpolate=true,width=18cm]{out2.jpg}}
%	\draw[fill=light-blue] (0,12) rectangle (23,10); 
%	\node[rectangle, rounded corners=20pt,inner sep=11pt,
%		fill=light-blue!50!black, font=\large\bfseries\sffamily]
%		at (20,10) {\color{white}Журнальные статьи};
%	\node[draw opacity=0.8, fill opacity=0.5, text opacity=1, rotate=90,
%		font=\large\bfseries\sffamily, text=white, text opacity=1,
%		fill=red!50!black,fill,draw]
%	at (-0.3,4.3) {Адміністративне право України};
%%	\pgfimage[interpolate=true,width=15cm]{out1.jpg}
%%	\pgfimage[interpolate=true,width=15cm]{out2.jpg}
%%\pgftext[at=\pgfpoint{1cm}{3cm},left,base]
%%	{\pgfimage[interpolate=true,width=20cm]{out1.jpg}}
%\end{tikzpicture}
%\end{flushright}
%
%%%---> PAGE 4
\newpage

\begin{flushright}
\begin{tikzpicture}
	\draw [preaction={fill=black,opacity=.5, transform
	canvas={xshift=1mm,yshift=-1mm}}] [fill=white,line width=1pt]
		(0,0) rectangle (15,20);
	\pgftext[at=\pgfpoint{0.5cm}{-0.8cm},left,base]
		{\pgfimage[interpolate=true,height=20cm]{out1.pdf}}
	\draw[fill=light-blue] (0,20) rectangle (15,18); 
	\node[rectangle, rounded corners=20pt,inner sep=11pt,
		fill=light-blue!50!black, font=\large\bfseries\sffamily]
		at (11,18) {\color{white}Журнальные статьи};
	\node[draw opacity=0.8, fill opacity=0.5, text opacity=1, rotate=90,
		font=\large\bfseries\sffamily, text=white, text opacity=1,
		fill=red!50!black,fill,draw]
	at (-0.3,4.3) {Адміністративне право України};
	\node[cloud callout, cloud puffs=15, aspect=2.5, cloud puff arc=120,
	shading=ball, text=white, font=\large\bfseries\sffamily, drop shadow]
	at (1,21.2) {Luga\TeX};
\end{tikzpicture}
\end{flushright}

%%%---> PAGE 5
\newpage

\begin{flushright}
\begin{tikzpicture}
	\draw [preaction={fill=black,opacity=.5, transform
	canvas={xshift=1mm,yshift=-1mm}}] [fill=white,line width=1pt]
		(0,0) rectangle (15,20);
	\pgftext[at=\pgfpoint{0.5cm}{-0.8cm},left,base]
		{\pgfimage[interpolate=true,height=20cm]{out5.pdf}}
	\draw[fill=light-blue] (0,20) rectangle (15,18); 
	\node[rectangle, rounded corners=20pt,inner sep=11pt,
		fill=light-blue!50!black, font=\large\bfseries\sffamily]
		at (11,18) {\color{white}Журнальные статьи};
	\node[draw opacity=0.8, fill opacity=0.5, text opacity=1, rotate=90,
		font=\large\bfseries\sffamily, text=white, text opacity=1,
		fill=red!50!black,fill,draw]
	at (-0.3,4.3) {Адміністративне право України};
\end{tikzpicture}
\end{flushright}
