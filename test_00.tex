%%%----------------------------------------------------------------------------
\newpage
\tikzfading[name=middle,top color=transparent!50,bottom color=transparent!50,
	middle color=transparent!20]
\begin{tikzpicture}
	\node [circle,circular drop shadow,
		pattern=horizontal lines dark blue,
		path fading=south,
		minimum size=3.6cm] {};
		\pattern [path fading=north,
			pattern=horizontal lines dark gray]
			(0,0) circle (1.8cm);
		\pattern [path fading=middle,
			pattern=crosshatch dots light steel blue]
			(0,0) circle (1.8cm);
\end{tikzpicture}
% Now we use the fading in another picture:
\vglue 15pt
%%%---> Tikz TEXT
%\begin{tikzfadingfrompicture}[name=tikz]
%	  \node [text=transparent!20]
%	      {\fontfamily{ptm}\fontsize{75}{75}\bfseries\selectfont Lu\emph{ga}\TeX};
%\end{tikzfadingfrompicture}
% Now we use the fading in another picture:
%\begin{tikzpicture}
%\shade[path fading=tikz,fit fading=false,
%	left color=blue,right color=black] (-3,-1) rectangle (3,1);
%\end{tikzpicture}
\vglue 35pt

\begin{tikzpicture}[rounded corners,ultra thick]
%	\shade[top color=yellow,bottom color=black] (0,0) rectangle +(2,1);
%	\shade[left color=yellow,right color=black] (3,0) rectangle +(2,1);
%	\shadedraw[inner color=yellow,outer color=black,draw=yellow] (0,0) rectangle +(5,1);
	%\node [inner color=yellow,outer color=black,draw=yellow] {%
	\node [rectangle,very thick,
		bottom color=blue!80!black!30,
		top color=white,
		draw=blue!50!black!50,
		drop shadow,
		%path fading=south,
		] at (0,0) {%
		\textcolor{black}{\textbf{Интерактивный навчально-методичний посібник}}};
	\pattern [path fading=north] ;
	\pattern [path fading=middle] ;
\end{tikzpicture}

\vglue 15pt

\begin{tikzpicture}
	\node[rectangle,inner sep=2pt,minimum height=3.1ex,rounded corners,
	drop shadow,shadow scale=1,shadow xshift=.8ex,shadow yshift=-.4ex,opacity=.9,
	fill=black!50,top color=white,bottom color=blue!80!black!30,
	draw=blue!50!black!50,very thick,text=black,text opacity=1]
	(inbottom) {%
	\textbf{Интерактивный навчально-методичний посібник}
	};
	\pattern [path fading=north] ;
	\pattern [path fading=middle] ;
\end{tikzpicture}

\vglue 15pt

\begin{tikzpicture}
	\node[rectangle,inner sep=2pt,minimum height=3.1ex,rounded corners,
	drop shadow,shadow scale=1,shadow xshift=.8ex,shadow yshift=-.4ex,opacity=.7,
	fill=black!50,top color=blue!90!black!50,bottom color=blue!80!black!80,
	draw=blue!50!black!50,very thick,text=white,text opacity=1]
	(inbottom) {%
	\textbf{Интерактивный навчально-методичний посібник}
	};
\end{tikzpicture}

\vglue 15pt

\begin{tikzpicture}
	\node[rectangle,inner sep=2pt,minimum height=3.1ex,rounded corners,
	drop shadow,shadow scale=1,shadow xshift=.8ex,shadow yshift=-.4ex,opacity=.7,
	fill=black!50,top color=green!30!black!100,bottom color=green!80!black!80,
	draw=green!30!black!100,very thick,text=white,text opacity=1]
	(inbottom) {%
	\textbf{Интерактивный навчально-методичний посібник}
	};
\end{tikzpicture}

\vglue 15pt

\begin{tikzpicture}
	\node[rectangle,inner sep=2pt,minimum height=3.1ex,rounded corners,
	drop shadow,shadow scale=1,shadow xshift=.8ex,shadow yshift=-.4ex,opacity=.7,
	fill=black!50,top color=magenta!30!black!100,bottom color=magenta!80!black!80,
	draw=magenta!30!black!100,very thick,text=white,text opacity=1]
	(inbottom) {%
	\textbf{Интерактивный навчально-методичний посібник}
	};
\end{tikzpicture}

\vglue 15pt

\begin{tikzpicture}
	\node[rectangle,inner sep=2pt,minimum height=3.1ex,rounded corners,
	drop shadow,shadow scale=1,shadow xshift=.8ex,shadow yshift=-.4ex,opacity=.7,
	fill=black!50,top color=yellow!30!black!100,bottom color=yellow!80!black!80,
	draw=yellow!30!black!100,very thick,text=white,text opacity=1]
	(inbottom) {%
	\textbf{Интерактивный навчально-методичний посібник}
	};
\end{tikzpicture}

\vglue 15pt

\begin{tikzpicture}[rounded corners,ultra thick]
	\node[rectangle,inner sep=2pt,minimum height=3.1ex,rounded corners,
	drop shadow,shadow scale=1,shadow xshift=.8ex,shadow yshift=-.4ex,opacity=.7,
	fill=black!50,top color=yellow,bottom color=yellow,
	draw=yellow!90!black!30,very thick,text=black,text opacity=1]
	(inbottom) {%
	\textbf{Интерактивный навчально-методичний посібник}
	};
\end{tikzpicture}

\vglue 15pt

\begin{tikzpicture}[rounded corners,ultra thick]
	\node[drop shadow,shadow scale=1,shadow xshift=.8ex,
	shadow yshift=-.4ex,top color=white,bottom color=blue!35,
	text=black,text opacity=1,rectangle,very thick]
	(inbottom) {%
	\textbf{Интерактивный навчально-методичний посібник}
	};
	\pattern [path fading=north] ;
	\pattern [path fading=middle] ;
\end{tikzpicture}

\vglue 15pt

\begin{tikzpicture}
	\node[rectangle,path fading=south,text=white] (inbottom) {%
	{%
	\mbox{}\hspace{5pt}\textbf{Интерактивный навчально-методичний посібник}
	\hspace{3pt}\mbox{}}
	};
\begin{pgfonlayer}{background}
\draw[rounded corners,top color=red,bottom color=black,draw=white,drop shadow,
	shadow scale=1, shadow xshift=.8ex, shadow yshift=-.4ex,
	opacity=.5, fill=black!50]
	($(inbottom.north west)+(0.18,0)$) rectangle ($(inbottom.north east)-(0.18,0.6)$);
%\draw[rounded corners,top color=white,bottom color=black,
%	middle color=red,draw=blue!20] ($(inbottom.south west) +(0.12,0.5)$)
%		rectangle ($(inbottom.south east)-(0.12,0)$);
%\draw[top color=blue!1,bottom color=blue!20,draw=white]
%	($(inbottom.north east)-(0.13,0.6)$) rectangle ($(inbottom.south west)+(0.13,0.2)$);
\end{pgfonlayer}
\end{tikzpicture}

\vglue 15pt

\begin{tikzpicture}
	\draw [pattern=fivepointed stars]
		[preaction={fill=black,opacity=.5,
			transform canvas={xshift=1mm,yshift=-1mm}}]
		[preaction={top color=blue,bottom color=white}]
			(0,0) rectangle (1,2)
				(1,2) circle (5mm);
\end{tikzpicture}

\begin{tikzpicture}
	[
	% Define an interesting style
		button/.style={
	% First preaction: Fuzzy shadow
		preaction={fill=black,path fading=circle with fuzzy edge 20 percent,
		opacity=.5,transform canvas={xshift=1mm,yshift=-1mm}},
	% Second preaction: Background pattern
		preaction={pattern=#1,
		path fading=circle with fuzzy edge 15 percent},
	% Third preaction: Make background shiny
		preaction={top color=white,
		bottom color=black!50,
		shading angle=45,
		path fading=circle with fuzzy edge 15 percent,
		opacity=0.2},
	% Fourth preaction: Make edge especially shiny
		preaction={path fading=fuzzy ring 15 percent,
		top color=black!5,
		bottom color=black!80,
		shading angle=45},
		inner sep=2ex
},button/.default=horizontal lines light blue,circle]
%	\draw [help lines] (0,0) grid (4,3);
\node [button=crosshatch dots light steel blue,text=white] at (1,1) {\textbf{Пуск}};
\node [button] at (3,1) {\textbf{Пуск}};
\node [button=crosshatch dots light steel blue,text=white] at (1,1) {\textbf{Пуск}};
\end{tikzpicture}

\vglue 15pt

\begin{tikzpicture}[even odd rule,rounded corners=2pt,x=10pt,y=10pt,drop shadow]
\filldraw[fill=yellow!90!black!40,drop shadow] (0,0)   rectangle (1,1)
	[xshift=5pt,yshift=5pt]   (0,0)   rectangle (1,1)
	[rotate=30]   (-1,-1) rectangle (2,2);
\node at (0,1.7) {\textbf{\thepage}};			      
\end{tikzpicture}

\vglue 15pt
\begin{tikzpicture}[
	nonterminal/.style={
	% The shape:
	rectangle,
	% The size:
	minimum size=6mm,
	% The border:
	very thick,
	draw=red!50!black!50,         % 50% red and 50% black,
	% and that mixed with 50% white
	% The filling:
	top color=white,              % a shading that is white at the top\ldots
	bottom color=red!50!black!20, % and something else at the bottom
	% Font
	font=\itshape
}]
\node [nonterminal] {\textbf{Капранов Олег}};
\end{tikzpicture}

\vglue 15pt

\tikzfading[name=fade out,inner color=transparent!0,outer color=transparent!100]
% Now we use the fading in another picture:
\begin{tikzpicture}
% Background
%\fill [black!20] (-1.2,-1.2) rectangle (1.2,1.2);
%\path [pattern=checkerboard,pattern color=black!30]
%	(-1.2,-1.2) rectangle (1.2,1.2);
\fill [blue,path fading=fade out] (-1,-1) rectangle (1,1);
\end{tikzpicture}

\vglue 15pt

\begin{tikzpicture}
% Checker board
%\fill [black!20] (0,0) rectangle (4,4);
%\path [pattern=checkerboard,pattern color=black!30] (0,0) rectangle (4,4);
\shade [ball color=blue,path fading=south] (2,2) circle (1.8);
\end{tikzpicture}

\vglue 15pt

\tikzfading[name=fade inside,inner color=transparent!80,outer color=transparent!30]
\begin{tikzpicture}
% Checker board
%\fill [black!20] (0,0) rectangle (4,4);
%\path [pattern=checkerboard,pattern color=black!30] (0,0) rectangle (4,4);
\shade [ball color=red] (3,3) circle (0.8);
\shade [ball color=white,path fading=fade inside] (2,2) circle (1.8);
\end{tikzpicture}
\begin{textblock}{4}(25,0)
\begin{tikzpicture}[even odd rule,rounded corners=2pt,x=10pt,y=10pt,drop shadow]
\filldraw[fill=yellow!90!black!40,drop shadow] (0,0)   rectangle (1,1)
	[xshift=5pt,yshift=5pt]   (0,0)   rectangle (1,1)
	[rotate=30]   (-1,-1) rectangle (2,2);
\node at (0,1.7) {\textbf{\thepage}};			      
\end{tikzpicture}
\end{textblock}

\pgfmathsetseed{1}
\foreach \col in {black,red,green,blue}
{
\begin{tikzpicture}[x=10pt,y=10pt,ultra thick,baseline,line cap=round]
	\coordinate (current point) at (0,0);
	\coordinate (old velocity) at (0,0);
	\coordinate (new velocity) at (rand,rand);
	\foreach \i in {0,1,...,100}
	{
	\draw[\col!\i] (current point)
	.. controls ++([scale=-1]old velocity) and
	++(new velocity) .. ++(rand,rand)
	coordinate (current point);
	\coordinate (old velocity) at (new velocity);
	\coordinate (new velocity) at (rand,rand);
	}
\end{tikzpicture}
}
