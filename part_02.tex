\disableTemplate{covers}
\disableTemplate{lawordercover}
\disableTiling
\newpage
%%%---> Main baground overlay
\begin{tikzpicture}[remember picture,overlay]
\begin{pgfonlayer}{background}
%%%---> original size
%\clip (-1.5,-5) rectangle ++(4,10);
%\clip (-6.3,-7.8) rectangle ++(14.3,15.2);
\colorlet{upperleft}{red!25}
\colorlet{upperright}{green!50!black!25}
\colorlet{lowerleft}{blue!25}
\colorlet{lowerright}{red!25}

% The large rectangles:
\fill [upperleft] (17.7,-11) rectangle ++(-20,16);
\fill [upperright] (17.7,-11) rectangle ++(18,16);
\fill [lowerleft] (17.7,-11) rectangle ++(-20,-13);
\fill [lowerright] (17.7,-11) rectangle ++(18,-13);

% The shadings:
\shade [left color=upperleft,right color=upperright]
([xshift=-1cm]17.7,-11) rectangle ++(2,16);
\shade [left color=lowerleft,right color=lowerright]
([xshift=-1cm]17.7,-11) rectangle ++(2,-13);
\shade [top color=upperleft,bottom color=lowerleft]
([yshift=-1cm]17.7,-11) rectangle ++(-20,2);
\shade [top color=upperright,bottom color=lowerright]
([yshift=-1cm]17.7,-11) rectangle ++(18,2);
\end{pgfonlayer}
\end{tikzpicture}
\begin{tikzpicture}[remember picture,overlay]
	  \node [rotate=0,scale=2,text opacity=0.2]
	      at (27,1.7) {Капранов~О.~Г.~\copyright~2010~~~Luga\TeX @yahoo.com};
\end{tikzpicture}
\vglue -18pt
\hspace{187pt}
\parbox{350pt}{%
\hypertarget{studyplan}{\hyperlink{studyear}{%
\begin{tikzpicture}
  \colorlet{even}{cyan!60!black}
  \colorlet{odd}{orange!100!black}
  \colorlet{links}{red!70!black}
  \colorlet{back}{yellow!20!white}
  \tikzset{
    box/.style={
      minimum height=15mm,
      inner sep=.7mm,
      outer sep=0mm,
      text width=120mm,
      text centered,
      font=\small\bfseries\sffamily,
      text=#1!50!black,
      draw=#1,
      line width=.25mm,
      top color=#1!5,
      bottom color=#1!40,
      shading angle=0,
      rounded corners=2.3mm,
      drop shadow={fill=#1!40!gray,fill opacity=.8},
      rotate=0,
    },
  }
  \node [box=odd]{{\huge\textbf{Учебный план\quad 2010--2011}}};
\end{tikzpicture}
}}}\\

\def\lecture#1#2#3#4#5#6{
% As before:
\node [annotation, #3, scale=0.65, text width=4cm, inner sep=2mm, fill=white] at (#4) {
Lecture #1: \textcolor{orange}{\textbf{#2}}
\list{--}{\topsep=2pt\itemsep=0pt\parsep=0pt
\parskip=0pt\labelwidth=8pt\leftmargin=8pt
\itemindent=0pt\labelsep=2pt}
	#5
	\endlist
	};
	% New:
	\node [anchor=base west] at (cal-#6.base east) {\textcolor{orange}{\textbf{#2}}};
}

\def\onlylecture#1#2{
	% New:
	\node [anchor=base west] at (cal-#2.base east) {\textcolor{orange}{\textbf{#1}}};
}
\def\onlylectureright#1#2{
	% New:
	\node [anchor=base east] at (cal-#2.base west) {\textcolor{orange}{\textbf{#1}}};
}

\noindent
\begin{tikzpicture}[anchor=mid]
	\begin{scope}[
		mindmap,
		every node/.style={concept, circular drop shadow,execute at begin node=\hskip0pt},
		root concept/.append style={
		concept color=black,
		fill=white, line width=1ex,
		text=black, font=\large\scshape},
		text=white,
		computational problems/.style={concept color=red,faded/.style={concept color=red!50}},
		computational models/.style={concept color=blue,faded/.style={concept color=blue!50}},
		measuring complexity/.style={concept color=orange,faded/.style={concept color=orange!50}},
		solving problems/.style={concept color=green!50!black,faded/.style={concept color=green!50!black!50}},
		grow cyclic,
		level 1/.append style={level distance=4.5cm,sibling angle=90,font=\scshape},
		level 2/.append style={level distance=3cm,sibling angle=45,font=\scriptsize}]
		\node [root concept, font=\bfseries\sffamily, text width=115pt]
		(Computational Complexity) {Административное\par право} % root
		child [computational problems] { node [yshift=-1cm] (Computational Problems) {Computational Problems}
		child { node (Problem Measures) {Problem Measures} }
		child { node (Problem Aspects) {Problem Aspects} }
		child [faded] { node (problem Domains) {Problem Domains} }
		child { node (Key Problems) {Key Problems} }
		}
		child [computational models] { node [yshift=-1cm] (Computational Models) {Computational Models}
		child { node (Turing Machines) {Turing Machines} }
		child [faded] { node (Random-Access Machines) {Random-Access Machines} }
		child { node (Circuits) {Circuits} }
		child [faded] { node (Binary Decision Diagrams) {Binary Decision Diagrams} }
		child { node (Oracle Machines) {Oracle Machines} }
		child { node (Programming in Logic) {Programming in Logic} }
		}
		child [measuring complexity] { node [yshift=1cm] (Measuring Complexity) {Measuring Complexity}
		child { node (Complexity Measures) {Complexity Measures} }
		child { node (Classifying Complexity) {Classifying Complexity} }
		child { node (Comparing Complexity) {Comparing Complexity} }
		child [faded] { node (Describing Complexity) {Describing Complexity} }
		}
		child [solving problems] { node [yshift=1cm] (Solving Problems) {Solving Problems}
		child { node (Exact Algorithms) {Exact Algorithms} }
		child { node (Randomization) {Randomization} }
		child { node (Fixed-Parameter Algorithms) {Fixed-Parameter Algorithms} }
		child { node (Parallel Computation) {Parallel Computation} }
		child { node (Partial Solutions) {Partial Solutions} }
		child { node (Approximation) {Approximation} }
		};
	\end{scope}
%%%---> Left side calendar	
	\tiny
	\calendar (mycalsep) [day list downward,
	month text=\%mt\ \%y0,
%	month text=Сентябрь\quad 2010,
	month yshift=3.5em,
	name=cal,
%	at={(-.5\textwidth-5mm,.5\textheight-1cm)},
%	at={(-.48\textwidth-5mm,.48\textheight-4cm)},
at={(-18,9)},
	dates=2010-09-01 to 2010-09-last]
%	if (equals=2010-09-25) {\draw (0,0) circle (4pt);}
	if (weekend)
	[white]
	if (day of month=1) {
	\node at (.0em,1.5em) [anchor=base west] {\large\bfseries\sffamily\tikzmonthtext};
	};
	\draw[red] (cal-2010-09-20) circle (4pt);
	\calendar (mycalocb) [day list downward,
	month text=\%mt\ \%y0,
%	month text=Октябрь \quad 2010,
	month yshift=3.5em,
	name=cal,
%	at={(-.5\textwidth-5mm,.5\textheight-1cm)},
%	at={(-.48\textwidth-5mm,.48\textheight-4cm)},
at={(-18,-2.5)},
	dates=2010-10-01 to 2010-10-last]
%	if (at most=2010-10-04) [nodes={strike out,draw}]
	if (weekend)
	[white]
	if (day of month=1) {
	\node at (.0em,1.5em) [anchor=base west] {\large\bfseries\sffamily\tikzmonthtext};
	};
	\node[starburst,drop shadow,fill=white,draw] at (13.3,5.1) {ТЕСТЫ};
	\node[signal, draw, text=white, fill=red!65!black, signal to=nowhere,
		signal from=west] at (-15,4) {ТЕСТЫ};
	\node[starburst, fill=yellow, draw=red, line width=2pt,
		drop shadow] at (-15,2.3) {\bf МОДУЛЬ};
	\node[starburst, fill=yellow, draw=blue, line width=2pt,
		drop shadow] at (13.3,3.4) {\bf МОДУЛЬ};
	\node[starburst, fill=yellow, draw=cyan, line width=2pt,
		drop shadow] at (-15.2,-8.8) {\bf МОДУЛЬ};
	\node[starburst, fill=yellow, draw=white, line width=2pt,
		drop shadow] at (13.3,-7.6) {\bf МОДУЛЬ};
	\node[chamfered rectangle, white, fill=red, double=red,
		draw, very thick] at (13.3,-9.4) {\bf ЭКЗАМЕН};
	\node[cloud callout, cloud puffs=15, aspect=2.5,
		cloud puff arc=120, shading=ball,text=white] at (13.3, -6.2)
			{\bf ТЕСТЫ};
%%%---> Right side	calendar
	\calendar (mycalnov) [day list downward,
	month text=\%mt\ \%y0,
%	month text=Ноябрь \quad 2010,
	name=cal,
%	at={(-.5\textwidth-5mm,.5\textheight-1cm)},
%	at={(.48\textwidth-15mm,.48\textheight-4cm)},
at={(16.2,9)},
	dates=2010-11-01 to 2010-11-last]
	if (weekend)
	[white]
	if (day of month=1) {
	\node at (-1.5em,1.5em) [anchor=base east] {\large\bfseries\sffamily\tikzmonthtext};
	};
	\calendar (mycaldec) [day list downward,
	month text=\%mt\ \%y0,
%	month text=Декабрь \quad 2010,
	name=cal,
%	at={(-.5\textwidth-5mm,.5\textheight-1cm)},
%	at={(.48\textwidth-15mm,.48\textheight-4cm)},
at={(16.2,-2.5)},
	dates=2010-12-01 to 2010-12-last]
	if (weekend)
	[white]
	if (day of month=1) {
	\node at (-1.5em,1.5em) [anchor=base east] {\large\bfseries\sffamily\tikzmonthtext};
	};
%%%--->	\onlylecture{}{2010-09-16}
\onlylecture{Тема 3.2. Джерела адміністративного права}{2010-09-16}
\onlylecture{Тема 4.3. Спеціальні адміністративно-правові статуси індивідуальних
суб'єктів адміністративного права}{2010-09-17}
\onlylecture{Тема 4.5. Порівняльний аналіз статусів державних
службовців}{2010-10-01}
\onlylecture{Тема 4.5. Порівняльний аналіз статусів державних службовців}{2010-10-05}
\onlylecture{Тема 4.7. Поняття, ознаки та класифікація органів виконавчої
влади}{2010-10-07}
\onlylecture{Тема 4.9. Місце органів внутрішніх справ у системі органів
виконавчої влади}{2010-10-11}
\onlylecture{Тема 5.2. Співвідношення адміністративно-правових форм і
методів}{2010-10-14}
\onlylecture{Тема 5.4. Місце актів управління в системі правових актів}{2010-10-18}
\onlylecture{Тема 6.2. Правові засади адміністративного примусу у сфері
державного управління}{2010-10-20}
\onlylecture{Тема 6.3. Адміністративний примус у діяльності органів
внутрішніх справ}{2010-10-22}
\onlylectureright{Тема 7.2. Стадії провадження у справах про адміністративні
правопорушення}{2010-11-01}
\onlylectureright{Тема 8.2. Контроль і нагляд у державному управлінні}{2010-11-03}
\onlylectureright{Тема 9.2. Сутність й особливості міжгалузевого управління}{2010-11-05}
\onlylectureright{Тема 11.2. Особливості управління соціально-культурним
комплексом}{2010-11-09}
\onlylectureright{Тема 12.2. Особливості управління в адміністративно-політичній сфері}{2010-11-10}
\onlylectureright{Тема 13.2. Особливості адміністративно-правових відносин за
участю ОВС}{2010-11-12}
\onlylectureright{Тема 14.2. Поняття, ознаки та юридичний склад адміністративного
правопорушення}{2010-11-29}
\onlylectureright{Тема 14.3. Система адміністративних стягнень}{2010-11-30}

\onlylectureright{Тема 15.2. Адміністративна відповідальність за корупційні
правопорушення}{2010-12-01}
\onlylectureright{Тема 16.1. Юридична характеристика адміністративних
правопорушень, що посягають на}{2010-12-02}
\onlylectureright{громадський порядок, громадську безпеку та
встановлений порядок управління}{2010-12-03}
\onlylectureright{Тема 16.2. Адміністративні правопорушення, що посягають на
громадський}{2010-12-06}
\onlylectureright{порядок і громадську безпеку}{2010-12-07}
\onlylectureright{Тема 16.3. Адміністративні правопорушення, що посягають на
встановлений порядок управління}{2010-12-08}
\onlylectureright{Тема 17.2. Адміністративні правопорушення в галузі охорони
труда та здоров'я населення}{2010-12-09}
\onlylectureright{Тема 17.3. Адміністративні правопорушення на транспорті, в
галузі шляхового господарства і зв`язку}{2010-12-10}
\onlylectureright{Тема 17.4. Адміністративні правопорушення в галузі торгівлі,
фінансів і підприємницької діяльності}{2010-12-13}
\onlylectureright{Рекомендації щодо науково-дослідницької роботи}{2010-12-14}
%\onlylecture{}{2010-09-16}
%\onlylecture{}{2010-09-16}
%\onlylecture{}{2010-09-16}
%\onlylecture{}{2010-09-16}
%\onlylecture{}{2010-09-16}
%\onlylecture{}{2010-09-16}
%\onlylecture{}{2010-09-16}
%\onlylecture{}{2010-09-16}
%\onlylecture{}{2010-09-16}
%\onlylecture{}{2010-09-16}
%\onlylecture{}{2010-09-16}
%\onlylecture{}{2010-09-16}
%\onlylecture{}{2010-09-16}
%\onlylecture{}{2010-09-16}
%\onlylecture{}{2010-09-16}
%\onlylecture{}{2010-09-16}
%\onlylecture{}{2010-09-16}
%\onlylecture{}{2010-09-16}
%%%--->	
	\lecture{1}{Тема 1.2. Співвідношення адміністративного права з іншими
	галузями}{above,xshift=-5mm,yshift=5mm}{Computational Problems.north}{
\item Мета заняття;
\item Основні поняття;
\item Навчальні питання;
\item Методичні рекомендації та пояснення;
\item Індивідуальні навчально-дослідницькі завдання;
\item Питання для самоконтролю та самоперевірки;
\item Додаткова література.	
	}{2010-09-08}
	\lecture{2}{Тема 2.2. Принципи й функції державного управління}{above left}
	{Computational Models.west}{
\item Мета заняття;
\item Основні поняття;
\item Навчальні питання;
\item Методичні рекомендації та пояснення;
\item Індивідуальні навчально-дослідницькі завдання;
\item Питання для самоконтролю та самоперевірки;
\item Додаткова література.	
	}{2010-09-15}
%%%---> New version

%%%---> Old version	
%	\begin{pgfonlayer}{background}
%		\clip[xshift=-1cm] (-.5\textwidth,-.5\textheight) rectangle ++(\textwidth,\textheight);
%		\colorlet{upperleft}{green!50!black!25}
%		\colorlet{upperright}{orange!25}
%		\colorlet{lowerleft}{red!25}
%		\colorlet{lowerright}{blue!25}
%		% The large rectangles:
%		\fill [upperleft] (Computational Complexity) rectangle ++(-20,20);
%		\fill [upperright] (Computational Complexity) rectangle ++(20,20);
%		\fill [lowerleft] (Computational Complexity) rectangle ++(-20,-20);
%		\fill [lowerright] (Computational Complexity) rectangle ++(20,-20);
%		% The shadings:
%		\shade [left color=upperleft,right color=upperright]
%		([xshift=-1cm]Computational Complexity) rectangle ++(2,20);
%		\shade [left color=lowerleft,right color=lowerright]
%		([xshift=-1cm]Computational Complexity) rectangle ++(2,-20);
%		\shade [top color=upperleft,bottom color=lowerleft]
%		([yshift=-1cm]Computational Complexity) rectangle ++(-20,2);
%		\shade [top color=upperright,bottom color=lowerright]
%		([yshift=-1cm]Computational Complexity) rectangle ++(20,2);
%	\end{pgfonlayer}
\end{tikzpicture}
%%%-----------------------------------------------------------------------
\begin{textblock}{5}(25,-0.01)
\begin{tikzpicture}[even odd rule,rounded corners=2pt,x=10pt,y=10pt,drop shadow]
\filldraw[fill=yellow!90!black!40,drop shadow] (0,0)   rectangle (1,1)
	[xshift=5pt,yshift=5pt]   (0,0)   rectangle (1,1)
	[rotate=30]   (-1,-1) rectangle (2,2);
\node at (0,1.7) {\textbf{\thepage}};			      
\end{tikzpicture}
\end{textblock}
%%%--- Navigational panel top page
\begin{textblock}{6}(7.58,0.85)
\mbox{%%%--->
\Acrobatmenu{LastPage}{%
\tikz[baseline] \node[rectangle,inner sep=2pt,minimum height=3.1ex,
rounded corners,drop shadow,shadow scale=1,shadow xshift=.8ex,
shadow yshift=-.4ex,opacity=.7,fill=black!50,top color=red!90!black!50,
bottom color=red!80!black!80,draw=red!50!black!50,very thick,text=white,
text opacity=1,minimum width=3cm,font=\bfseries\sffamily] at (0,0) {К концу};
}\Acrobatmenu{GoBack}{%
\tikz[baseline] \node[rectangle,inner sep=2pt,minimum height=3.1ex,
rounded corners,drop shadow,shadow scale=1,shadow xshift=.8ex,
shadow yshift=-.4ex,opacity=.7,fill=black!50,top color=red!90!black!50,
bottom color=red!80!black!80,draw=red!50!black!50,very thick,text=white,
text opacity=1,minimum width=3cm,font=\bfseries\sffamily] at (4,0) {Назад};
}\Acrobatmenu{PrevPage}{%
\tikz[baseline] \node[rectangle,inner sep=2pt,minimum height=3.1ex,
rounded corners,drop shadow,shadow scale=1,shadow xshift=.8ex,
shadow yshift=-.4ex,opacity=.7,fill=black!50,top color=red!90!black!50,
bottom color=red!80!black!80,draw=red!50!black!50,very thick,text=white,
text opacity=1,minimum width=3cm,font=\bfseries\sffamily] at (8,0) {Предыдущий};
}\Acrobatmenu{NextPage}{%
\tikz[baseline] \node[rectangle,inner sep=2pt,minimum height=3.1ex,
rounded corners,drop shadow,shadow scale=1,shadow xshift=.8ex,
shadow yshift=-.4ex,opacity=.7,fill=black!50,top color=red!90!black!50,
bottom color=red!80!black!80,draw=red!50!black!50,very thick,text=white,
text opacity=1,minimum width=3cm,font=\bfseries\sffamily] at (12,0) {Следующий};
}\Acrobatmenu{GoForward}{%
\tikz[baseline] \node[rectangle,inner sep=2pt,minimum height=3.1ex,
rounded corners,drop shadow,shadow scale=1,shadow xshift=.8ex,
shadow yshift=-.4ex,opacity=.7,fill=black!50,top color=red!90!black!50,
bottom color=red!80!black!80,draw=red!50!black!50,very thick,text=white,
text opacity=1,minimum width=3cm,font=\bfseries\sffamily] at (16,0) {Вперед};
}\Acrobatmenu{FirstPage}{%
\tikz[baseline] \node[rectangle,inner sep=2pt,minimum height=3.1ex,
rounded corners,drop shadow,shadow scale=1,shadow xshift=.8ex,
shadow yshift=-.4ex,opacity=.7,fill=black!50,top color=red!90!black!50,
bottom color=red!80!black!80,draw=red!50!black!50,very thick,text=white,
text opacity=1,minimum width=3cm,font=\bfseries\sffamily] at (20,0) {К началу};
}\Acrobatmenu{FullScreen}{%
\tikz[baseline] \node[rectangle,inner sep=2pt,minimum height=3.1ex,
rounded corners,drop shadow,shadow scale=1,shadow xshift=.8ex,
shadow yshift=-.4ex,opacity=.7,fill=black!50,top color=red!90!black!50,
bottom color=red!80!black!80,draw=red!50!black!50,very thick,text=white,
text opacity=1,minimum width=3cm,font=\bfseries\sffamily] at (24,0) {Полный экран};
}\Acrobatmenu{Quit}{%
\tikz[baseline] \node[rectangle,inner sep=2pt,minimum height=3.1ex,
rounded corners,drop shadow,shadow scale=1,shadow xshift=.8ex,
shadow yshift=-.4ex,opacity=.7,fill=black!50,top color=red!90!black!50,
bottom color=red!80!black!80,draw=red!50!black!50,very thick,text=white,
text opacity=1,minimum width=3cm,font=\bfseries\sffamily] at (28,0) {Выход};
}	
}%%%---|
\end{textblock}
\begin{textblock}{7}(13.39,1.305)	%(-0.38,-0.153)
\begin{tikzpicture}[remember picture,overlay]
	\node {\mbox{\includegraphics[scale=0.99]{./laworder_bg_01}}};
\end{tikzpicture}
\end{textblock}	
%%%----------------------------------------------------------------------|
