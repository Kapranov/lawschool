\newpage
\begin{tikzpicture}[remember picture,overlay]
	  \node [rotate=0,scale=2,text opacity=0.2]
	      at (27,1.7) {Капранов~О.~Г.~\copyright~2010~~~Luga\TeX @yahoo.com};
\end{tikzpicture}
\vglue -18pt
\hspace{187pt}
\parbox{350pt}{%
\hypertarget{chapter5a}{\hyperlink{chapter5b}{\mbox{%
\begin{tikzpicture}
  \colorlet{even}{cyan!60!black}
  \colorlet{odd}{orange!100!black}
  \colorlet{links}{red!70!black}
  \colorlet{back}{yellow!20!white}
  \tikzset{
    box/.style={
      minimum height=15mm,
      inner sep=.7mm,
      outer sep=0mm,
      text width=120mm,
      text centered,
      font=\small\bfseries\sffamily,
      text=#1!50!black,
      draw=#1,
      line width=.25mm,
      top color=#1!5,
      bottom color=#1!40,
      shading angle=0,
      rounded corners=2.3mm,
      drop shadow={fill=#1!40!gray,fill opacity=.8},
      rotate=0,
    },
  }
  \node [box=even] {{%
  	\huge\textbf{Методичні рекомендації,}}
	\textbf{плани й завдання до семінарських і практичних занять}};
\end{tikzpicture}
}}}}\\[5pt]
\noindent
\begin{tikzpicture}
  \colorlet{even}{cyan!60!black}
  \colorlet{odd}{orange!100!black}
  \colorlet{links}{red!70!black}
  \colorlet{back}{yellow!20!white}
  \tikzset{
    box/.style={
      minimum height=15mm,
      inner sep=.7mm,
      outer sep=0mm,
      text width=120mm,
      text centered,
      font=\small\bfseries\sffamily,
      text=#1!50!black,
      draw=#1,
      line width=.25mm,
      top color=#1!5,
      bottom color=#1!40,
      shading angle=0,
      rounded corners=2.3mm,
      drop shadow={fill=#1!40!gray,fill opacity=.8},
      rotate=0,
    },
  }
	\node[box=links,xshift=3mm,yshift=1mm,
		minimum height=5pt,text width=325pt]
		at (0,-1.3) {\hyperlink{chapter5a}{\mbox{%
		\small \textcolor{black}{Тема 2.2. Принципи й функції державного
		управління}}}};
	\node[box=links,xshift=3mm,yshift=1mm,
		minimum height=5pt,text width=235pt]
		at (10.6,-1.3) {\hyperlink{chapter5b}{\mbox{%
		\small \textcolor{black}{Індивідуальні
		навчально-дослідницькі завдання}}}};
	\node[box=links,xshift=3mm,yshift=1mm,
		minimum height=5pt,text width=221pt]
		at (19.3,-1.3) {\hyperlink{chapter5b}{\mbox{%
		\small \textcolor{black}{
		Питання для самоконтролю та самоперевірки}}}};
	\node[box=links,xshift=3mm,yshift=1mm,
		minimum height=5pt,text width=140pt]
		at (26.35,-1.3) {\hyperlink{chapter5c}{\mbox{%
		\small \textcolor{black}{Додаткова література}}}};
\end{tikzpicture}
\vglue 5pt
\tikzfading[name=targetask, top color=transparent!90,
	bottom color=transparent!90,middle color=transparent!65]
\begin{tikzpicture}
\node [rounded corners,fill=magenta!50,minimum width=960pt,
minimum height=35pt,path fading=targetask] at (0,0) {\mbox{}};
\end{tikzpicture}
\begin{textblock}{51}(0.8,3.75)
\begin{tikzpicture}
	\node [text width=920pt] {\parbox{900pt}{%
	\textbf{Мета заняття:}
	визначити поняття, сутність і види управлінської діяльності,
	співвідношення соціального й державного управління, виконавчої
	влади й державного управління, охарактеризувати види й зміст
	принципів і функцій державного управління.
}};
\end{tikzpicture}
 \end{textblock}
\begin{textblock}{52}(17,5)
\tikzstyle{abstract}=[rectangle, draw=black, rounded corners, fill=blue!40, drop shadow,
	text centered, text=white, text width=4cm,font=\large\bfseries\sffamily]
\tikzstyle{comment}=[rectangle, draw=black, rounded corners, fill=green, drop shadow,
	text centered, text=white, text width=10cm,
	font=\Large\bfseries\sffamily]
\tikzstyle{myarrow}=[->, >=open triangle 90, thick]
\tikzstyle{line}=[-, thick]
\begin{tikzpicture}[node distance=2cm]
    \node (Seminar) [abstract, rectangle split,
		rectangle split parts=2,text width=5cm] at (0,0)
        {
            Семінарське заняття
            \nodepart{second}2 год
        };

    \node (Thema) [abstract, rectangle split, rectangle split parts=2,
		text width=10cm] at (0,-2) {
            Принципи й функції
            \nodepart{second}державного управління
        };

    \node (AuxNode01) [text width=4cm] {};
    \node (Maindir) [abstract, rectangle split,
		rectangle split parts=2] at (0,-4)
        {
            Основні поняття
            \nodepart{second}:
        };

	\node (Maindirnames) [comment, rectangle split,
		rectangle split parts=2, text justified] at (0,-10)
	{
		\mbox{}
		\nodepart{second}\mbox{}
		\newline управління
		\newline організація
		\newline соціальне управління
		\newline державне управління
		\newline державна влада
		\newline виконавча влада
		\newline управлінська діяльність
		\newline управлінський цикл
		\newline управлінський апарат
		\newline компетенція
		\newline \mbox{}
	};
\end{tikzpicture}
\end{textblock}

\vglue 15pt

\tikzstyle{mybox} = [draw=red, fill=blue!20, very thick,
	rectangle, rounded corners, inner sep=10pt, inner ysep=20pt,
	font=\large\bfseries\sffamily]
\tikzstyle{fancytitle} =[fill=red, text=white,font=\large\bfseries\sffamily]

\begin{tikzpicture}
\node [mybox] (box) {%
	\begin{minipage}{0.50\textwidth}
		\begin{itemize}
			\item Поняття, сутність і види управління.
			\item Ознаки державного управління.
			\item Класифікація принципів державного управління.
			\item Система й зміст функцій управлінської діяльності.
		\end{itemize}
	\end{minipage}
};
\node[fancytitle, right=10pt] at (box.north west) {Навчальні питання:};
\node[fancytitle, rounded corners] at (box.east) {{\bf ?}};
\end{tikzpicture}	

\vglue 15pt

\tikzstyle{mybox} = [draw=blue, fill=green!20, very thick,
	rectangle, rounded corners, inner sep=10pt, inner ysep=20pt,
	line width=1pt,font=\bfseries\sffamily]
\tikzstyle{fancytitle} =[fill=blue, text=white, ellipse,font=\bfseries\sffamily]

\begin{tikzpicture}[transform shape, rotate=0, baseline=-3.5cm]
	\node [mybox] (box) {%
		\begin{minipage}[t!]{0.5\textwidth}
У ході підготовки до заняття по даній темі курсанти та студенти повинні
усвідомити поняття й види управління, роль і місце державного управління,
принципи поділу державної влади на законодавчу, виконавчу й судову. Необхідно
з'ясувати поняття і зміст функцій державного управління, уміти співвідносити
поняття <<виконавча влада>> й <<державне управління>>. При цьому на основі аналізу
нормативно--правових актів курсанти та студенти повинні показати застосування в
законодавстві цих державно-правових категорій.


Будучи різновидом соціального управління, державне управління зберігає його
характеристики. Одночасно, йому властиві численні особливості, що відображають
його специфіку. Визначаючи сутність державного управління, системність
управлінського процесу, необхідно усвідомити його принципи, класифікацію й
зміст функцій, виділяти серед них функції орієнтування, забезпечення й
оперативного управління системою.


Поділяючи принципи державного управління на правові й організаційно-правові,
слід вивчити й уміти надати характеристику принципам законності, демократизму,
планування, диференціації й фіксації повноважень, відповідальності в рамках
компетенції, єдності галузевих, міжгалузевих і територіальних, лінійних і
функціональних засад управління й ін.

Із практичної точки зору курсанти та студенти повинні вміти відмежовувати
управління від інших видів державної діяльності, визначати підсистеми
державного управління стосовно окремих його галузей, давати оцінку ролі
адміністративного права в регулюванні відносин управлінського характеру.


Для кращого засвоєння теми пропонується виконати завдання для самостійної
роботи та індивідуальні навчально--дослідницькі завдання.


Рівень своїх знань з цієї теми пропонується перевірити шляхом надання
відповідей на питання для самоконтролю та самоперевірки.
		\end{minipage}
		};
\node[fancytitle] at (box.north) {Методичні рекомендації та пояснення};
\end{tikzpicture}
%%%-----------------------------------------------------------------------
\begin{textblock}{53}(25,-0.01)
\begin{tikzpicture}[even odd rule,rounded corners=2pt,x=10pt,y=10pt,drop shadow]
\filldraw[fill=yellow!90!black!40,drop shadow] (0,0)   rectangle (1,1)
	[xshift=5pt,yshift=5pt]   (0,0)   rectangle (1,1)
	[rotate=30]   (-1,-1) rectangle (2,2);
\node at (0,1.7) {\textbf{\thepage}};			      
\end{tikzpicture}
\end{textblock}
%%%--- Navigational panel top page
\begin{textblock}{54}(7.58,0.85)
\mbox{%%%--->
\Acrobatmenu{LastPage}{%
\tikz[baseline] \node[rectangle,inner sep=2pt,minimum height=3.1ex,
rounded corners,drop shadow,shadow scale=1,shadow xshift=.8ex,
shadow yshift=-.4ex,opacity=.7,fill=black!50,top color=red!90!black!50,
bottom color=red!80!black!80,draw=red!50!black!50,very thick,text=white,
text opacity=1,minimum width=3cm,font=\bfseries\sffamily] at (0,0) {К концу};
}\Acrobatmenu{GoBack}{%
\tikz[baseline] \node[rectangle,inner sep=2pt,minimum height=3.1ex,
rounded corners,drop shadow,shadow scale=1,shadow xshift=.8ex,
shadow yshift=-.4ex,opacity=.7,fill=black!50,top color=red!90!black!50,
bottom color=red!80!black!80,draw=red!50!black!50,very thick,text=white,
text opacity=1,minimum width=3cm,font=\bfseries\sffamily] at (4,0) {Назад};
}\Acrobatmenu{PrevPage}{%
\tikz[baseline] \node[rectangle,inner sep=2pt,minimum height=3.1ex,
rounded corners,drop shadow,shadow scale=1,shadow xshift=.8ex,
shadow yshift=-.4ex,opacity=.7,fill=black!50,top color=red!90!black!50,
bottom color=red!80!black!80,draw=red!50!black!50,very thick,text=white,
text opacity=1,minimum width=3cm,font=\bfseries\sffamily] at (8,0) {Предыдущий};
}\Acrobatmenu{NextPage}{%
\tikz[baseline] \node[rectangle,inner sep=2pt,minimum height=3.1ex,
rounded corners,drop shadow,shadow scale=1,shadow xshift=.8ex,
shadow yshift=-.4ex,opacity=.7,fill=black!50,top color=red!90!black!50,
bottom color=red!80!black!80,draw=red!50!black!50,very thick,text=white,
text opacity=1,minimum width=3cm,font=\bfseries\sffamily] at (12,0) {Следующий};
}\Acrobatmenu{GoForward}{%
\tikz[baseline] \node[rectangle,inner sep=2pt,minimum height=3.1ex,
rounded corners,drop shadow,shadow scale=1,shadow xshift=.8ex,
shadow yshift=-.4ex,opacity=.7,fill=black!50,top color=red!90!black!50,
bottom color=red!80!black!80,draw=red!50!black!50,very thick,text=white,
text opacity=1,minimum width=3cm,font=\bfseries\sffamily] at (16,0) {Вперед};
}\Acrobatmenu{FirstPage}{%
\tikz[baseline] \node[rectangle,inner sep=2pt,minimum height=3.1ex,
rounded corners,drop shadow,shadow scale=1,shadow xshift=.8ex,
shadow yshift=-.4ex,opacity=.7,fill=black!50,top color=red!90!black!50,
bottom color=red!80!black!80,draw=red!50!black!50,very thick,text=white,
text opacity=1,minimum width=3cm,font=\bfseries\sffamily] at (20,0) {К началу};
}\Acrobatmenu{FullScreen}{%
\tikz[baseline] \node[rectangle,inner sep=2pt,minimum height=3.1ex,
rounded corners,drop shadow,shadow scale=1,shadow xshift=.8ex,
shadow yshift=-.4ex,opacity=.7,fill=black!50,top color=red!90!black!50,
bottom color=red!80!black!80,draw=red!50!black!50,very thick,text=white,
text opacity=1,minimum width=3cm,font=\bfseries\sffamily] at (24,0) {Полный экран};
}\Acrobatmenu{Quit}{%
\tikz[baseline] \node[rectangle,inner sep=2pt,minimum height=3.1ex,
rounded corners,drop shadow,shadow scale=1,shadow xshift=.8ex,
shadow yshift=-.4ex,opacity=.7,fill=black!50,top color=red!90!black!50,
bottom color=red!80!black!80,draw=red!50!black!50,very thick,text=white,
text opacity=1,minimum width=3cm,font=\bfseries\sffamily] at (28,0) {Выход};
}	
}%%%---|
\end{textblock}
%%%-----------------------------------------------------------------------
%%%---> NEW PAGE ---------------------------------------------------------
\newpage
\begin{tikzpicture}[remember picture,overlay]
	  \node [rotate=0,scale=2,text opacity=0.2]
	      at (27,1.7) {Капранов~О.~Г.~\copyright~2010~~~Luga\TeX @yahoo.com};
\end{tikzpicture}
%%%-----------------------------------------------------------------------
\begin{textblock}{55}(25,-0.01)
\begin{tikzpicture}[even odd rule,rounded corners=2pt,x=10pt,y=10pt,drop shadow]
\filldraw[fill=yellow!90!black!40,drop shadow] (0,0)   rectangle (1,1)
	[xshift=5pt,yshift=5pt]   (0,0)   rectangle (1,1)
	[rotate=30]   (-1,-1) rectangle (2,2);
\node at (0,1.7) {\textbf{\thepage}};			      
\end{tikzpicture}
\end{textblock}
%%%--- Navigational panel top page
\begin{textblock}{56}(7.58,0.85)
\mbox{%%%--->
\Acrobatmenu{LastPage}{%
\tikz[baseline] \node[rectangle,inner sep=2pt,minimum height=3.1ex,
rounded corners,drop shadow,shadow scale=1,shadow xshift=.8ex,
shadow yshift=-.4ex,opacity=.7,fill=black!50,top color=red!90!black!50,
bottom color=red!80!black!80,draw=red!50!black!50,very thick,text=white,
text opacity=1,minimum width=3cm,font=\bfseries\sffamily] at (0,0) {К концу};
}\Acrobatmenu{GoBack}{%
\tikz[baseline] \node[rectangle,inner sep=2pt,minimum height=3.1ex,
rounded corners,drop shadow,shadow scale=1,shadow xshift=.8ex,
shadow yshift=-.4ex,opacity=.7,fill=black!50,top color=red!90!black!50,
bottom color=red!80!black!80,draw=red!50!black!50,very thick,text=white,
text opacity=1,minimum width=3cm,font=\bfseries\sffamily] at (4,0) {Назад};
}\Acrobatmenu{PrevPage}{%
\tikz[baseline] \node[rectangle,inner sep=2pt,minimum height=3.1ex,
rounded corners,drop shadow,shadow scale=1,shadow xshift=.8ex,
shadow yshift=-.4ex,opacity=.7,fill=black!50,top color=red!90!black!50,
bottom color=red!80!black!80,draw=red!50!black!50,very thick,text=white,
text opacity=1,minimum width=3cm,font=\bfseries\sffamily] at (8,0) {Предыдущий};
}\Acrobatmenu{NextPage}{%
\tikz[baseline] \node[rectangle,inner sep=2pt,minimum height=3.1ex,
rounded corners,drop shadow,shadow scale=1,shadow xshift=.8ex,
shadow yshift=-.4ex,opacity=.7,fill=black!50,top color=red!90!black!50,
bottom color=red!80!black!80,draw=red!50!black!50,very thick,text=white,
text opacity=1,minimum width=3cm,font=\bfseries\sffamily] at (12,0) {Следующий};
}\Acrobatmenu{GoForward}{%
\tikz[baseline] \node[rectangle,inner sep=2pt,minimum height=3.1ex,
rounded corners,drop shadow,shadow scale=1,shadow xshift=.8ex,
shadow yshift=-.4ex,opacity=.7,fill=black!50,top color=red!90!black!50,
bottom color=red!80!black!80,draw=red!50!black!50,very thick,text=white,
text opacity=1,minimum width=3cm,font=\bfseries\sffamily] at (16,0) {Вперед};
}\Acrobatmenu{FirstPage}{%
\tikz[baseline] \node[rectangle,inner sep=2pt,minimum height=3.1ex,
rounded corners,drop shadow,shadow scale=1,shadow xshift=.8ex,
shadow yshift=-.4ex,opacity=.7,fill=black!50,top color=red!90!black!50,
bottom color=red!80!black!80,draw=red!50!black!50,very thick,text=white,
text opacity=1,minimum width=3cm,font=\bfseries\sffamily] at (20,0) {К началу};
}\Acrobatmenu{FullScreen}{%
\tikz[baseline] \node[rectangle,inner sep=2pt,minimum height=3.1ex,
rounded corners,drop shadow,shadow scale=1,shadow xshift=.8ex,
shadow yshift=-.4ex,opacity=.7,fill=black!50,top color=red!90!black!50,
bottom color=red!80!black!80,draw=red!50!black!50,very thick,text=white,
text opacity=1,minimum width=3cm,font=\bfseries\sffamily] at (24,0) {Полный экран};
}\Acrobatmenu{Quit}{%
\tikz[baseline] \node[rectangle,inner sep=2pt,minimum height=3.1ex,
rounded corners,drop shadow,shadow scale=1,shadow xshift=.8ex,
shadow yshift=-.4ex,opacity=.7,fill=black!50,top color=red!90!black!50,
bottom color=red!80!black!80,draw=red!50!black!50,very thick,text=white,
text opacity=1,minimum width=3cm,font=\bfseries\sffamily] at (28,0) {Выход};
}	
}%%%---|
\end{textblock}
%%%-----------------------------------------------------------------------
\vglue -18pt
\hspace{187pt}
\parbox{350pt}{%
\hypertarget{chapter5b}{\hyperlink{chapter5c}{\mbox{%
\begin{tikzpicture}
  \colorlet{even}{cyan!60!black}
  \colorlet{odd}{orange!100!black}
  \colorlet{links}{red!70!black}
  \colorlet{back}{yellow!20!white}
  \tikzset{
    box/.style={
      minimum height=15mm,
      inner sep=.7mm,
      outer sep=0mm,
      text width=120mm,
      text centered,
      font=\small\bfseries\sffamily,
      text=#1!50!black,
      draw=#1,
      line width=.25mm,
      top color=#1!5,
      bottom color=#1!40,
      shading angle=0,
      rounded corners=2.3mm,
      drop shadow={fill=#1!40!gray,fill opacity=.8},
      rotate=0,
    },
  }
  \node [box=even] {{%
  	\huge\textbf{Методичні рекомендації,}}
	\textbf{плани й завдання до семінарських і практичних занять}};
\end{tikzpicture}
}}}}\\[5pt]
\noindent
\begin{tikzpicture}
  \colorlet{even}{cyan!60!black}
  \colorlet{odd}{orange!100!black}
  \colorlet{links}{red!70!black}
  \colorlet{back}{yellow!20!white}
  \tikzset{
    box/.style={
      minimum height=15mm,
      inner sep=.7mm,
      outer sep=0mm,
      text width=120mm,
      text centered,
      font=\small\bfseries\sffamily,
      text=#1!50!black,
      draw=#1,
      line width=.25mm,
      top color=#1!5,
      bottom color=#1!40,
      shading angle=0,
      rounded corners=2.3mm,
      drop shadow={fill=#1!40!gray,fill opacity=.8},
      rotate=0,
    },
  }
	\node[box=links,xshift=3mm,yshift=1mm,
		minimum height=5pt,text width=325pt]
		at (0,-1.3) {\hyperlink{chapter5a}{\mbox{%
		\small \textcolor{black}{Тема 2.2. Принципи й функції
	   	державного управління}}}};
	\node[box=links,xshift=3mm,yshift=1mm,
		minimum height=5pt,text width=235pt]
		at (10.6,-1.3) {\hyperlink{chapter5b}{\mbox{%
		\small \textcolor{black}{Індивідуальні
		навчально-дослідницькі завдання}}}};
	\node[box=links,xshift=3mm,yshift=1mm,
		minimum height=5pt,text width=221pt]
		at (19.3,-1.3) {\hyperlink{chapter5b}{\mbox{%
		\small \textcolor{black}{
		Питання для самоконтролю та самоперевірки}}}};
	\node[box=links,xshift=3mm,yshift=1mm,
		minimum height=5pt,text width=140pt]
		at (26.35,-1.3) {\hyperlink{chapter5c}{\mbox{%
		\small \textcolor{black}{Додаткова література}}}};
\end{tikzpicture}

\vfill

%%%---> Old version
%\begin{tikzpicture}[remember picture, note/.style={rectangle
%	callout,fill=#1}]
%	\node [note=green!50,opacity=.5,overlay,text opacity=1,
%		font=\large\bfseries\sffamily, callout relative pointer={(-5,-1)},
%			callout pointer width=1.3cm] at (15,1) {%
%Індивідуальні навчально-дослідницькі завдання:
%};
%\end{tikzpicture}
%
%\begin{itemize}
%	\item[] \tikz[baseline] \node[ball color=magenta,circle,text=black,
%			minimum size=4pt]
%		{1}; \quad {\large\textbf{%
%			Підготувати реферат за темою: <<Особливості предмету
%			галузі в світлі Концепції реформи адміністративного права
%			України>>.}}
%
%\item[] \tikz[baseline] \node[ball color=magenta,circle,text=black]
%		{2}; \quad {\large\textbf{%
%			Підготувати реферат за темою: <<Співвідношення методів
%			адміністративно-правового й цивільно-правового регулювання
%			суспільних відносин>>.}}
%
%\item[] \tikz[baseline] \node[ball color=magenta,circle,text=black]
%		{3}; \quad {\large\textbf{%
%			Підготувати доповідь за темою: <<Принципи адміністративного
%			права: удосконалення системи>>.}}
%\end{itemize}
%%%--->
%%%---> New version
\begin{flushleft}
\begin{tikzpicture}
\node (tbl) {
\begin{tabularx}{980pt}{l}
\arrayrulecolor{purple}
\multicolumn{1}{c}{\mbox{}\hspace{50pt}\mbox{\textcolor{white}{{%
	\large\bfseries\sffamily Індивідуальні навчально--дослідницькі
   		завдання}}}}\\[15pt]
\tikz[baseline] \node[ball color=green,circle, text=white] {{\bf 1}};\qquad
\tikz[baseline] \node[font=\large\bfseries\sffamily,
	text=black] {
Підготувати реферат за темою: <<Роль державного управління в світлі сучасного
реформування суспільства>>.
	   	};\\[20pt]
		\tikz[baseline] \node[ball color=green,circle, text=white] {{\bf 2}};\qquad
\tikz[baseline] \node[font=\large\bfseries\sffamily,
	text=black] {
Підготувати реферат за темою: <<Управління в області внутрішніх справ як
підсистема державного управління>>.
		};\\[20pt]
		\tikz[baseline] \node[ball color=green,circle, text=white] {{\bf 3}};\qquad
\tikz[baseline] \node[font=\large\bfseries\sffamily,
	text=black] {
Підготувати доповідь за темою: <<Проблеми визначення принципів державного
управління на сучасному етапі>>.
	};\\[20pt]
		\tikz[baseline] \node[ball color=green,circle, text=white] {{\bf 3}};\qquad
\tikz[baseline] \node[font=\large\bfseries\sffamily,
	text=black] {
Скласти бібліографію за темою заняття з урахуванням нових надходжень до
бібліотеки.
	};\\[20pt]
		\tikz[baseline] \node[ball color=green,circle, text=white] {{\bf 3}};\qquad
\tikz[baseline] \node[font=\large\bfseries\sffamily, text width=820pt,
	text=black] {
	Підготувати рецензію на наукову статтю: Конопльов В. Організаційно--правовий
	механізм підвищення ефективності управлінської діяльності. Право України. ---
	2005. --- № 9. --- С. 111.
	};\\[20pt]
\end{tabularx}};

\begin{pgfonlayer}{background}
\draw[rounded corners,top color=red,bottom color=black,draw=white]
	($(tbl.north west)+(0.14,0)$) rectangle ($(tbl.north east)-(0.13,0.9)$);
\draw[rounded corners,top color=white,bottom color=black,
	middle color=red,draw=blue!20] ($(tbl.south west) +(0.12,0.5)$)
		rectangle ($(tbl.south east)-(0.12,0)$);
\draw[top color=blue!1,bottom color=blue!20,draw=white]
	($(tbl.north east)-(0.13,0.6)$) rectangle ($(tbl.south west)+(0.13,0.2)$);
\end{pgfonlayer}
\end{tikzpicture}
\end{flushleft}
%%%---> Old version
%\vglue 45pt
%
%\begin{tikzpicture}[remember picture, note/.style={rectangle
%	callout,fill=#1}]
%	\node [note=green!50,opacity=.5,overlay,text opacity=1,
%		font=\large\bfseries\sffamily, callout relative pointer={(-5,-1)},
%			callout pointer width=1.3cm] at (15,1) {%
%Питання для самоконтролю та самоперевірки:
%};
%\end{tikzpicture}
%
%\begin{itemize}
%\item[] \tikz[baseline] \node[ball color=magenta,circle,text=black]
%	{1}; \quad {\large\textbf{%
%У яких значеннях вживається термін <<адміністративне право України>>?
%}}
%\item[] \tikz[baseline] \node[ball color=magenta,circle,text=black]
%	{2}; \quad {\large\textbf{%
%Як співвідносяться конституційне й адміністративне права? Розкрийте їх
%поняття і значення.
%}}
%\item[] \tikz[baseline] \node[ball color=magenta,circle,text=black]
%	{3}; \quad {\large\textbf{%
%Поясніть значення термінів <<предмет адміністративного права>>, <<метод
%адміністративного права>>, <<механізм адміністративно-правового
%регулювання>>.
%}}
%\item[] \tikz[baseline] \node[ball color=magenta,circle,text=black]
%	{4}; \quad {\large\textbf{%
%Розкрийте співвідношення адміністративного права із суміжними галузями
%права.
%}}
%\item[] \tikz[baseline] \node[ball color=magenta,circle,text=black]
%	{5}; \quad {\large\textbf{%
%Розкрийте зміст поняття <<система адміністративного права>>?
%}}
%
%\item[] \tikz[baseline] \node[ball color=magenta,circle,text=black]
%	{6}; \quad {\large\textbf{%
%Дайте правову характеристику головних елементів Загальної й Особливої
%частин адміністративного права.
%}}
%\item[] \tikz[baseline] \node[ball color=magenta,circle,text=black]
%	{7}; \quad {\large\textbf{%
%Що таке Спеціальна частина адміністративного права?
%}}
%\item[] \tikz[baseline] \node[ball color=magenta,circle,text=black]
%	{8}; \quad {\large\textbf{%
%Що таке <<принципи права>>?
%}}
%\item[] \tikz[baseline] \node[ball color=magenta,circle,text=black]
%	{9}; \quad {\large\textbf{%
%Для чого необхідне вивчення адміністративного права? Аргументуйте свою
%відповідь.
%}}
%\item[] \tikz[baseline] \node[ball color=magenta,circle,text=black]
%	{10}; \quad {\large\textbf{%
%У чому виражаються публічний і приватний аспекти адміністративного права, їх
%взаємозв'язок?
%}}
%\item[] \tikz[baseline] \node[ball color=magenta,circle,text=black]
%	{11}; \quad {\large\textbf{%
%Дайте історичну характеристику розвитку галузі адміністративного права в
%Україні.
%}}
%\end{itemize}
%%%--->
%%%---> NEW Version
\vfill
\begin{flushleft}
\begin{tikzpicture}
\node (tbl) {
\begin{tabularx}{980pt}{l}
\arrayrulecolor{purple}
\multicolumn{1}{c}{\mbox{}\hspace{50pt}\mbox{\textcolor{white}{{\large\bfseries\sffamily Питання
для самоконтролю та самоперевірки}}}}\\[15pt]
\tikz[baseline] \node[ball color=magenta, circle,
	minimum size=0.8cm, text=white] {{\bf 1}};\qquad
\tikz[baseline] \node[font=\large\bfseries\sffamily,
	text=black] {%
Розкрийте зміст і співвідношення понять <<організація>> й <<управління>>.
};\\[18pt]
\tikz[baseline] \node[ball color=magenta, circle,
	minimum size=0.8cm, text=white] {{\bf 2}};\qquad
\tikz[baseline] \node[font=\large\bfseries\sffamily,
	text=black] {%
Назвіть основні характерні риси державного управління.
};\\[18pt]
\tikz[baseline] \node[ball color=magenta, circle,
	minimum size=0.8cm, text=white] {{\bf 3}};\qquad
\tikz[baseline] \node[font=\large\bfseries\sffamily,
	text=black] {%
Порівняєте державне управління з іншими формами державної діяльності.
};\\[18pt]
\tikz[baseline] \node[ball color=magenta, circle,
	minimum size=0.8cm, text=white] {{\bf 4}};\qquad
\tikz[baseline] \node[font=\large\bfseries\sffamily,
	text=black] {%
Наведіть характеристику принципів законності й доцільності
управлінської діяльності.
};\\[18pt]
\tikz[baseline] \node[ball color=magenta, circle,
	minimum size=0.8cm, text=white] {{\bf 5}};\qquad
\tikz[baseline] \node[font=\large\bfseries\sffamily,
	text=black] {%
Співвіднесіть поняття <<державне управління>> й <<виконавча влада>>.
};\\[18pt]
\tikz[baseline] \node[ball color=magenta, circle,
	minimum size=0.8cm, text=white] {{\bf 6}};\qquad
\tikz[baseline] \node[font=\large\bfseries\sffamily,
	text=black] {%
Визначить роль адміністративного права в здійсненні державного
управління і їх взаємозв'язок.
};\\[18pt]
\end{tabularx}};

\begin{pgfonlayer}{background}
\draw[rounded corners,top color=red,bottom color=black,draw=white]
	($(tbl.north west)+(0.14,0)$) rectangle ($(tbl.north east)-(0.13,0.9)$);
\draw[rounded corners,top color=white,bottom color=black,
	middle color=red,draw=blue!20] ($(tbl.south west) +(0.12,0.5)$)
		rectangle ($(tbl.south east)-(0.12,0)$);
\draw[top color=blue!1,bottom color=blue!20,draw=white]
	($(tbl.north east)-(0.13,0.6)$) rectangle ($(tbl.south west)+(0.13,0.2)$);
\end{pgfonlayer}
\end{tikzpicture}
\end{flushleft}
\vfill
\begin{flushright}
\tikz[baseline] \node[rectangle,inner sep=2pt,minimum height=3.1ex,rounded corners,
drop shadow,shadow scale=1,shadow xshift=.8ex,shadow yshift=-.4ex,opacity=.7,
fill=black!50,top color=blue!90!black!50,bottom color=blue!80!black!80,
draw=blue!50!black!50,very thick,text=white,text opacity=1,minimum width=4cm]{%
Тестовые модули};\hspace{10pt}\mbox{}
\end{flushright}	
\vfill
%%%-----------------------------------------------------------------------
%%%---> NEW PAGE ---------------------------------------------------------
\newpage
\begin{tikzpicture}[remember picture,overlay]
	  \node [rotate=0,scale=2,text opacity=0.2]
	      at (27,1.7) {Капранов~О.~Г.~\copyright~2010~~~Luga\TeX @yahoo.com};
\end{tikzpicture}
\vglue -18pt
\hspace{187pt}
\parbox{350pt}{%
\hypertarget{chapter5c}{\hyperlink{chapter5d}{\mbox{%
\begin{tikzpicture}
  \colorlet{even}{cyan!60!black}
  \colorlet{odd}{orange!100!black}
  \colorlet{links}{red!70!black}
  \colorlet{back}{yellow!20!white}
  \tikzset{
    box/.style={
      minimum height=15mm,
      inner sep=.7mm,
      outer sep=0mm,
      text width=120mm,
      text centered,
      font=\small\bfseries\sffamily,
      text=#1!50!black,
      draw=#1,
      line width=.25mm,
      top color=#1!5,
      bottom color=#1!40,
      shading angle=0,
      rounded corners=2.3mm,
      drop shadow={fill=#1!40!gray,fill opacity=.8},
      rotate=0,
    },
  }
  \node [box=even] {{%
  	\huge\textbf{Методичні рекомендації,}}
	\textbf{плани й завдання до семінарських і практичних занять}};
\end{tikzpicture}
}}}}\\[5pt]
\noindent
\begin{tikzpicture}
  \colorlet{even}{cyan!60!black}
  \colorlet{odd}{orange!100!black}
  \colorlet{links}{red!70!black}
  \colorlet{back}{yellow!20!white}
  \tikzset{
    box/.style={
      minimum height=15mm,
      inner sep=.7mm,
      outer sep=0mm,
      text width=120mm,
      text centered,
      font=\small\bfseries\sffamily,
      text=#1!50!black,
      draw=#1,
      line width=.25mm,
      top color=#1!5,
      bottom color=#1!40,
      shading angle=0,
      rounded corners=2.3mm,
      drop shadow={fill=#1!40!gray,fill opacity=.8},
      rotate=0,
    },
  }
	\node[box=links,xshift=3mm,yshift=1mm,
		minimum height=5pt,text width=325pt]
		at (0,-1.3) {\hyperlink{chapter5a}{\mbox{%
		\small \textcolor{black}{Тема 2.2. Принципи й функції
	   	державного управління}}}};
	\node[box=links,xshift=3mm,yshift=1mm,
		minimum height=5pt,text width=235pt]
		at (10.6,-1.3) {\hyperlink{chapter5b}{\mbox{%
		\small \textcolor{black}{Індивідуальні
		навчально-дослідницькі завдання}}}};
	\node[box=links,xshift=3mm,yshift=1mm,
		minimum height=5pt,text width=221pt]
		at (19.3,-1.3) {\hyperlink{chapter5b}{\mbox{%
		\small \textcolor{black}{
		Питання для самоконтролю та самоперевірки}}}};
	\node[box=links,xshift=3mm,yshift=1mm,
		minimum height=5pt,text width=140pt]
		at (26.35,-1.3) {\hyperlink{chapter5c}{\mbox{%
		\small \textcolor{black}{Додаткова література}}}};
\end{tikzpicture}
\vglue 25pt
%%%---> Old Version
%\begin{tikzpicture}
%	\node[name=s,shape=rectangle callout,
%		callout relative pointer={(1.25cm,-1cm)},
%			callout pointer width=2cm, inner xsep=2cm, inner ysep=1cm,
%				font=\Large\bfseries\sffamily, text centered, fill=black,
%					shading angle=45, drop shadow,
%						text=white] at (3,0) {Додаткова література};
%\end{tikzpicture}
%%%---<
%%%---> NEW Version
\hyperlink{chapter5d}{\mbox{%
\begin{tikzpicture}
	\node[name=s,shape=rectangle callout,
		callout relative pointer={(1.25cm,-1cm)},
		callout pointer width=2cm, inner xsep=2cm, inner ysep=1cm,
		font=\Large\bfseries\sffamily, text centered,
		shading angle=45, drop shadow,
		postaction={path fading=south,
		fading angle=45,fill=blue, opacity=.5},
		left color=black, right color=red, draw=white,
		line width=2mm, text=white, drop shadow,
		shadow scale=1.25, shadow xshift=0pt,
		shadow yshift=0pt]
	at (3,0) {Додаткова література};
\end{tikzpicture}
}}
%%%---<
\vfill
\begin{itemize}
\item[]
	\scalebox{1.9}{\iconbook}\qquad {\large {\bf Авер'янов} В.Б.,
	{\bf Крупчан} О.Д.}
	\tooltipanim{\large\bf Виконавча влада}{21}{21}\quad{\large конституційні
	засади і
	шляхи реформування. --- {\bf Харків}. {\bf 1998}.}
\item[]
	\scalebox{1.9}{\iconbook}\qquad {\large {\bf Атаманчук} Г.В.}
	\tooltipanim{\large\bf Теория государственного управления}{22}{22}\quad
	{\large Курс лекций. --- {\bf М}.,
	{\bf 1997}.}
\item[]
	\scalebox{1.9}{\iconbook}\qquad
	\tooltipanim{\large\bf Виконавча влада і адміністративне
	право}{21}{21}\quad {\large За заг. ред {\bf Авер'янова} В.Б. --- {\bf К}.,
	{\bf 2002}. --- 668 с.}
\item[]
	\scalebox{1.9}{\iconbook}\qquad
	\tooltipanim{\large\bf Державне управління в Україні}{22}{22}\quad
	{\large Навчальний посібник. За заг. ред. {\bf Авер'янова} В.Б.
	--- {\bf К}., {\bf 1999}. --- 266 с.}
\item[]
\scalebox{1.9}{\iconbook}\qquad
\tooltipanim{\large\bf Державне управління в Україні}{21}{21}\quad
{\large наукові, правові, кадрові та організаційні засади.
Навчальний посібник. За заг. ред.\\
\mbox{}\hspace{43pt} {\bf Нижник} Н.Р. {\bf Олуйка} В.М.  --- {\bf Львів}., {\bf 2002}. --- 352 с.}
\item[]
	\scalebox{1.9}{\iconbook}\qquad 
	\tooltipanim{\large\bf Державне управління}{22}{22}\quad
	{\large	проблеми адміністративно-правової теорії та практики.
	За заг. ред. {\bf Авер'янова} В.Б. --- {\bf К}., {\bf 2003}. --- 384 с.}
\item[]
	\scalebox{1.9}{\iconbook}\qquad
	\tooltipanim{\large\bf Державне управління та адміністративне право в сучасній Україні}{21}{21}\quad {\large актуальні проблеми
	реформування. За заг. ред. {\bf Авер'янова}, {\bf Коліушка}.
	--- {\bf К}., {\bf 1999}. --- 50 с.}
\item[]
	\scalebox{1.9}{\iconbook}\qquad {\large{\bf Фалмер} Р.М.}
	\tooltipanim{\large\bf Энциклопедия современного
	управления}{22}{22}\quad {\large В {\bf 5}-и частях. 
	--- {\bf М}., {\bf 1992}.}
\item[]
	\scalebox{1.9}{\iconbook}\qquad {\large {\bf Щекин} Г.В.}
	\tooltipanim{\large\bf Теория социального управления}{21}{21}\quad
	{\large --- {\bf К}., {\bf 1996}.}
\item[]
	\scalebox{1.9}{\iconarticle}\qquad~~{\large\bf
	Авер`янов В.} \tooltipanim{\large\bf Ще раз про зміст і співвідношення понять <<державне управління>> і <<виконавча влада>>}{23}{23}\quad
{\large проблемні нотатки.\\
\mbox{}\hspace{43pt} {\bf Право України}. --- {\bf 2004}. --- {\bf №5}. --- С. 113.}
\item[]
	\scalebox{1.9}{\iconarticle}\qquad~~{\large {\bf Бульба} О.}
\tooltipanim{\large\bf Європейська інтеграція України та питання реалізації поділу влади}{24}{24}\quad {\large{\bf Право України}. --- {\bf 2004}. --- {\bf 12}. --- С. 8.}
\item[]
	\scalebox{1.9}{\iconarticle}\qquad~~{\large {\bf Васькович} Й.}
\tooltipanim{\large\bf Проблеми та перспективи побудови правової держави в Україні}{25}{25}\quad
{\large	{\bf Право України}. --- {\bf 2000}. --- {\bf №1}. --- С. 32.}
\item[]
\scalebox{1.9}{\iconarticle}\qquad~~{\large\bf Дерець В.}
\tooltipanim{\large\bf Реординаційні відносини як окремий вид управлінських
відносин між органами виконавчої влади}{23}{23}\quad
{\large Стан і перспективи реформування\\
\mbox{}\hspace{45pt} адміністративного права.
( IV Національна конференція). {\bf Право
України}. --- {\bf 2005}. --- {\bf №5}. --- С. 35.}
\end{itemize}
\vfill
\begin{textblock}{57}(22.1,15)
	\hyperlink{chapter5d}{\mbox{%
	\tikz[every node/.style={font=\normalsize\bfseries\sffamily,signal,draw,
		text=white,signal to=nowhere}]
		\node[signal to=east, minimum width=35pt, postaction={path fading=south,
			fading angle=45,fill=blue,opacity=.5}, left color=black, right color=red,
				draw=white, line width=1mm, drop shadow,
					minimum height=8pt]
			{Додаткова література: 2~стр};
			}}
\end{textblock}
\vfill
\mbox{}
%%%-----------------------------------------------------------------------
\begin{textblock}{58}(25,-0.01)
\begin{tikzpicture}[even odd rule,rounded corners=2pt,x=10pt,y=10pt,drop shadow]
\filldraw[fill=yellow!90!black!40,drop shadow] (0,0)   rectangle (1,1)
	[xshift=5pt,yshift=5pt]   (0,0)   rectangle (1,1)
	[rotate=30]   (-1,-1) rectangle (2,2);
\node at (0,1.7) {\textbf{\thepage}};			      
\end{tikzpicture}
\end{textblock}
%%%--- Navigational panel top page
\begin{textblock}{59}(7.58,0.85)
\mbox{%%%--->
\Acrobatmenu{LastPage}{%
\tikz[baseline] \node[rectangle,inner sep=2pt,minimum height=3.1ex,
rounded corners,drop shadow,shadow scale=1,shadow xshift=.8ex,
shadow yshift=-.4ex,opacity=.7,fill=black!50,top color=red!90!black!50,
bottom color=red!80!black!80,draw=red!50!black!50,very thick,text=white,
text opacity=1,minimum width=3cm,font=\bfseries\sffamily] at (0,0) {К концу};
}\Acrobatmenu{GoBack}{%
\tikz[baseline] \node[rectangle,inner sep=2pt,minimum height=3.1ex,
rounded corners,drop shadow,shadow scale=1,shadow xshift=.8ex,
shadow yshift=-.4ex,opacity=.7,fill=black!50,top color=red!90!black!50,
bottom color=red!80!black!80,draw=red!50!black!50,very thick,text=white,
text opacity=1,minimum width=3cm,font=\bfseries\sffamily] at (4,0) {Назад};
}\Acrobatmenu{PrevPage}{%
\tikz[baseline] \node[rectangle,inner sep=2pt,minimum height=3.1ex,
rounded corners,drop shadow,shadow scale=1,shadow xshift=.8ex,
shadow yshift=-.4ex,opacity=.7,fill=black!50,top color=red!90!black!50,
bottom color=red!80!black!80,draw=red!50!black!50,very thick,text=white,
text opacity=1,minimum width=3cm,font=\bfseries\sffamily] at (8,0) {Предыдущий};
}\Acrobatmenu{NextPage}{%
\tikz[baseline] \node[rectangle,inner sep=2pt,minimum height=3.1ex,
rounded corners,drop shadow,shadow scale=1,shadow xshift=.8ex,
shadow yshift=-.4ex,opacity=.7,fill=black!50,top color=red!90!black!50,
bottom color=red!80!black!80,draw=red!50!black!50,very thick,text=white,
text opacity=1,minimum width=3cm,font=\bfseries\sffamily] at (12,0) {Следующий};
}\Acrobatmenu{GoForward}{%
\tikz[baseline] \node[rectangle,inner sep=2pt,minimum height=3.1ex,
rounded corners,drop shadow,shadow scale=1,shadow xshift=.8ex,
shadow yshift=-.4ex,opacity=.7,fill=black!50,top color=red!90!black!50,
bottom color=red!80!black!80,draw=red!50!black!50,very thick,text=white,
text opacity=1,minimum width=3cm,font=\bfseries\sffamily] at (16,0) {Вперед};
}\Acrobatmenu{FirstPage}{%
\tikz[baseline] \node[rectangle,inner sep=2pt,minimum height=3.1ex,
rounded corners,drop shadow,shadow scale=1,shadow xshift=.8ex,
shadow yshift=-.4ex,opacity=.7,fill=black!50,top color=red!90!black!50,
bottom color=red!80!black!80,draw=red!50!black!50,very thick,text=white,
text opacity=1,minimum width=3cm,font=\bfseries\sffamily] at (20,0) {К началу};
}\Acrobatmenu{FullScreen}{%
\tikz[baseline] \node[rectangle,inner sep=2pt,minimum height=3.1ex,
rounded corners,drop shadow,shadow scale=1,shadow xshift=.8ex,
shadow yshift=-.4ex,opacity=.7,fill=black!50,top color=red!90!black!50,
bottom color=red!80!black!80,draw=red!50!black!50,very thick,text=white,
text opacity=1,minimum width=3cm,font=\bfseries\sffamily] at (24,0) {Полный экран};
}\Acrobatmenu{Quit}{%
\tikz[baseline] \node[rectangle,inner sep=2pt,minimum height=3.1ex,
rounded corners,drop shadow,shadow scale=1,shadow xshift=.8ex,
shadow yshift=-.4ex,opacity=.7,fill=black!50,top color=red!90!black!50,
bottom color=red!80!black!80,draw=red!50!black!50,very thick,text=white,
text opacity=1,minimum width=3cm,font=\bfseries\sffamily] at (28,0) {Выход};
}	
}%%%---|
\end{textblock}
%%%-----------------------------------------------------------------------
%%%---> NEW PAGE ----------------------------------------------------------
\newpage
\begin{tikzpicture}[remember picture,overlay]
	  \node [rotate=0,scale=2,text opacity=0.2]
	      at (27,1.7) {Капранов~О.~Г.~\copyright~2010~~~Luga\TeX @yahoo.com};
\end{tikzpicture}
\vglue -18pt
\hspace{187pt}
\parbox{350pt}{%
\hypertarget{chapter5d}{\hyperlink{intro}{\mbox{%
\begin{tikzpicture}
  \colorlet{even}{cyan!60!black}
  \colorlet{odd}{orange!100!black}
  \colorlet{links}{red!70!black}
  \colorlet{back}{yellow!20!white}
  \tikzset{
    box/.style={
      minimum height=15mm,
      inner sep=.7mm,
      outer sep=0mm,
      text width=120mm,
      text centered,
      font=\small\bfseries\sffamily,
      text=#1!50!black,
      draw=#1,
      line width=.25mm,
      top color=#1!5,
      bottom color=#1!40,
      shading angle=0,
      rounded corners=2.3mm,
      drop shadow={fill=#1!40!gray,fill opacity=.8},
      rotate=0,
    },
  }
  \node [box=even] {{%
  	\huge\textbf{Методичні рекомендації,}}
	\textbf{плани й завдання до семінарських і практичних занять}};
\end{tikzpicture}
}}}}\\[5pt]
\noindent
\begin{tikzpicture}
  \colorlet{even}{cyan!60!black}
  \colorlet{odd}{orange!100!black}
  \colorlet{links}{red!70!black}
  \colorlet{back}{yellow!20!white}
  \tikzset{
    box/.style={
      minimum height=15mm,
      inner sep=.7mm,
      outer sep=0mm,
      text width=120mm,
      text centered,
      font=\small\bfseries\sffamily,
      text=#1!50!black,
      draw=#1,
      line width=.25mm,
      top color=#1!5,
      bottom color=#1!40,
      shading angle=0,
      rounded corners=2.3mm,
      drop shadow={fill=#1!40!gray,fill opacity=.8},
      rotate=0,
    },
  }
	\node[box=links,xshift=3mm,yshift=1mm,
		minimum height=5pt,text width=325pt]
		at (0,-1.3) {\hyperlink{chapter5a}{\mbox{%
		\small \textcolor{black}{Тема 2.2. Принципи й функції
			державного управління}}}};
	\node[box=links,xshift=3mm,yshift=1mm,
		minimum height=5pt,text width=235pt]
		at (10.6,-1.3) {\hyperlink{chapter5b}{\mbox{%
		\small \textcolor{black}{Індивідуальні
		навчально-дослідницькі завдання}}}};
	\node[box=links,xshift=3mm,yshift=1mm,
		minimum height=5pt,text width=221pt]
		at (19.3,-1.3) {\hyperlink{chapter5b}{\mbox{%
		\small \textcolor{black}{
		Питання для самоконтролю та самоперевірки}}}};
	\node[box=links,xshift=3mm,yshift=1mm,
		minimum height=5pt,text width=140pt]
		at (26.35,-1.3) {\hyperlink{chapter5c}{\mbox{%
		\small \textcolor{black}{Додаткова література}}}};
\end{tikzpicture}
\vglue 25pt
%%%---> Old Version
%\begin{tikzpicture}
%	\node[name=s,shape=rectangle callout,
%		callout relative pointer={(1.25cm,-1cm)},
%			callout pointer width=2cm, inner xsep=2cm, inner ysep=1cm,
%				font=\Large\bfseries\sffamily, text centered,
%					shading angle=45] at (3,0) {Додаткова література};
%\end{tikzpicture}					
%%%---<
%%%---> NEW Version
\hyperlink{chapter5c}{\mbox{%
\begin{tikzpicture}
	\node[name=s,shape=rectangle callout,
		callout relative pointer={(1.25cm,-1cm)},
		callout pointer width=2cm, inner xsep=2cm, inner ysep=1cm,
		font=\Large\bfseries\sffamily, text centered,
		shading angle=45,
		postaction={path fading=south,
		fading angle=45,fill=blue, opacity=.5},
		left color=black, right color=red, draw=white,
		line width=2mm, text=white,
		shadow scale=3.25, shadow xshift=3pt,
		shadow yshift=3pt]
	at (3,0) {Додаткова література};
\end{tikzpicture}
}}
%%%---<
\vfill
\begin{itemize}
\item[]
	\scalebox{1.9}{\iconarticle}\qquad~~{\large
	{\bf Конопльов} В.}
	\tooltipanim{\large\bf Організаційно-правовий механізм підвищення
	ефективності управлінської діяльності}{23}{23}\quad
	{\large{\bf Право України}.
	--- {\bf 2005}. --- {\bf №9}. --- C. 35.}
\item[]
	\scalebox{1.9}{\iconarticle}\qquad~~{\large {\bf Журавський} В.}
\tooltipanim{\large\bf Щодо реформи адміністративно-територіального устрою України}{24}{24}
\quad{\large{\bf Право України}. --- {\bf 2005}. --- {\bf №8}. --- C. 16.}
\item[]
	\scalebox{1.9}{\iconarticle}\qquad~~{\large {\bf Капустинський} В.}
\tooltipanim{\large\bf Організаційно-правовий механізм підвищення ефективності
управлінської діяльності}{25}{25}\quad
{\large{\bf Право України}. --- {\bf 2005}.  --- {\bf №9}. --- С. 111.}
\item[]
	\scalebox{1.9}{\iconarticle}\qquad~~{\large {\bf Лагода} О.}
\tooltipanim{\large\bf Законність як правовий орієнтир системи
управлінської діяльності}{23}{23}\quad
{\large	{\bf Право України}. --- {\bf 2005}. --- {\bf №10}. --- С. 95.}
\item[]
	\scalebox{1.9}{\iconarticle}\qquad~~{\large {\bf Лазарєва} Н.}
	\tooltipanim{\large\bf Державні послуги у сфері освіти}{24}{24}\quad
{\large погляди на питання. {\bf Право України}. --- {\bf 2005}. ---
{\bf №11}. --- C. 17.}
\item[]
	\scalebox{1.9}{\iconarticle}\qquad~~{\large {\bf Ней} Н.}
	\tooltipanim{\large\bf Право України}{25}{25}\quad
	{\large--- {\bf 2004}. --- {\bf №12}. --- C. 28.}
\item[]
	\scalebox{1.9}{\iconarticle}\qquad~~{\large {\bf Развадовський} В.}
\tooltipanim{\large\bf Функції державного управління транспортною системою України}{23}{23}\quad
{\large	{\bf Право України}. --- {\bf 2004}. --- {\bf №5}. --- C. 121.}
\item[]
	\scalebox{1.9}{\iconarticle}\qquad~~{\large {\bf Рыжов} В.С.}
	\tooltipanim{\large\bf К судьбе государственного
	управления}{24}{24}\quad
	{\large{\bf Государство и право}.
	--- {\bf 1999}. --- {\bf №2}. --- С. 14.}
\item[]
	\scalebox{1.9}{\iconarticle}\qquad~~{\large
	{\bf Скомороха} В.} \tooltipanim{\large\bf Адміністративна реформа в
	Україні}{25}{25}\quad
	{\large	потрібне законодавче забезпечення (правові аспекти поділу і
	розмежування влади).\\
	\mbox{}\hspace{48pt}{\bf Право України}. --- {\bf 1999} --- {\bf №8}.
	--- C. 8.}
\item[]
	\scalebox{1.9}{\iconarticle}\qquad~~{\large {\bf Титарчук} В.}
	\tooltipanim{\large\bf Вдосконалення державного апарату}{23}{23}\quad
	{\large (окремі питання).
	{\bf Право України}. --- {\bf 1999}. --- {\bf №3}. --- C. 15.}
\item[]
	\scalebox{1.9}{\iconarticle}\qquad~~{\large {\bf Фролова} О.}
\tooltipanim{\large\bf Проблеми реформування інформаційно--методичного забезпечення управління}{24}{24}\quad{\large {\bf Право України}. --- {\bf 2004}. --- {\bf №12}.
	--- С. 87.}
\end{itemize}
\vfill
\begin{textblock}{60}(22.1,15)
	\hyperlink{chapter5c}{\mbox{%
	\tikz[every node/.style={font=\normalsize\bfseries\sffamily,signal,draw,
		text=white}]
		\node[signal to=east,
			minimum width=35pt, postaction={path fading=south,
			fading angle=45,fill=blue,opacity=.5}, left color=black, right color=red,
			draw=white, line width=1mm, drop shadow, rotate=180]
			{\rotatebox{180}{\mbox{Додаткова література: 1~стр}}};
			}}		
\end{textblock}
\vfill
\mbox{}
%%%-----------------------------------------------------------------------
\begin{textblock}{61}(25,-0.01)
\begin{tikzpicture}[even odd rule,rounded corners=2pt,x=10pt,y=10pt,drop shadow]
\filldraw[fill=yellow!90!black!40,drop shadow] (0,0)   rectangle (1,1)
	[xshift=5pt,yshift=5pt]   (0,0)   rectangle (1,1)
	[rotate=30]   (-1,-1) rectangle (2,2);
\node at (0,1.7) {\textbf{\thepage}};			      
\end{tikzpicture}
\end{textblock}
%%%--- Navigational panel top page
\begin{textblock}{62}(7.58,0.85)
\mbox{%%%--->
\Acrobatmenu{LastPage}{%
\tikz[baseline] \node[rectangle,inner sep=2pt,minimum height=3.1ex,
rounded corners,drop shadow,shadow scale=1,shadow xshift=.8ex,
shadow yshift=-.4ex,opacity=.7,fill=black!50,top color=red!90!black!50,
bottom color=red!80!black!80,draw=red!50!black!50,very thick,text=white,
text opacity=1,minimum width=3cm,font=\bfseries\sffamily] at (0,0) {К концу};
}\Acrobatmenu{GoBack}{%
\tikz[baseline] \node[rectangle,inner sep=2pt,minimum height=3.1ex,
rounded corners,drop shadow,shadow scale=1,shadow xshift=.8ex,
shadow yshift=-.4ex,opacity=.7,fill=black!50,top color=red!90!black!50,
bottom color=red!80!black!80,draw=red!50!black!50,very thick,text=white,
text opacity=1,minimum width=3cm,font=\bfseries\sffamily] at (4,0) {Назад};
}\Acrobatmenu{PrevPage}{%
\tikz[baseline] \node[rectangle,inner sep=2pt,minimum height=3.1ex,
rounded corners,drop shadow,shadow scale=1,shadow xshift=.8ex,
shadow yshift=-.4ex,opacity=.7,fill=black!50,top color=red!90!black!50,
bottom color=red!80!black!80,draw=red!50!black!50,very thick,text=white,
text opacity=1,minimum width=3cm,font=\bfseries\sffamily] at (8,0) {Предыдущий};
}\Acrobatmenu{NextPage}{%
\tikz[baseline] \node[rectangle,inner sep=2pt,minimum height=3.1ex,
rounded corners,drop shadow,shadow scale=1,shadow xshift=.8ex,
shadow yshift=-.4ex,opacity=.7,fill=black!50,top color=red!90!black!50,
bottom color=red!80!black!80,draw=red!50!black!50,very thick,text=white,
text opacity=1,minimum width=3cm,font=\bfseries\sffamily] at (12,0) {Следующий};
}\Acrobatmenu{GoForward}{%
\tikz[baseline] \node[rectangle,inner sep=2pt,minimum height=3.1ex,
rounded corners,drop shadow,shadow scale=1,shadow xshift=.8ex,
shadow yshift=-.4ex,opacity=.7,fill=black!50,top color=red!90!black!50,
bottom color=red!80!black!80,draw=red!50!black!50,very thick,text=white,
text opacity=1,minimum width=3cm,font=\bfseries\sffamily] at (16,0) {Вперед};
}\Acrobatmenu{FirstPage}{%
\tikz[baseline] \node[rectangle,inner sep=2pt,minimum height=3.1ex,
rounded corners,drop shadow,shadow scale=1,shadow xshift=.8ex,
shadow yshift=-.4ex,opacity=.7,fill=black!50,top color=red!90!black!50,
bottom color=red!80!black!80,draw=red!50!black!50,very thick,text=white,
text opacity=1,minimum width=3cm,font=\bfseries\sffamily] at (20,0) {К началу};
}\Acrobatmenu{FullScreen}{%
\tikz[baseline] \node[rectangle,inner sep=2pt,minimum height=3.1ex,
rounded corners,drop shadow,shadow scale=1,shadow xshift=.8ex,
shadow yshift=-.4ex,opacity=.7,fill=black!50,top color=red!90!black!50,
bottom color=red!80!black!80,draw=red!50!black!50,very thick,text=white,
text opacity=1,minimum width=3cm,font=\bfseries\sffamily] at (24,0) {Полный экран};
}\Acrobatmenu{Quit}{%
\tikz[baseline] \node[rectangle,inner sep=2pt,minimum height=3.1ex,
rounded corners,drop shadow,shadow scale=1,shadow xshift=.8ex,
shadow yshift=-.4ex,opacity=.7,fill=black!50,top color=red!90!black!50,
bottom color=red!80!black!80,draw=red!50!black!50,very thick,text=white,
text opacity=1,minimum width=3cm,font=\bfseries\sffamily] at (28,0) {Выход};
}	
}%%%---|
\end{textblock}
%%%-----------------------------------------------------------------------
