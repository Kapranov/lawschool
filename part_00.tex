\documentclass[english,russian]{article}
\usepackage{pscyr}
\usepackage{ucs}
\usepackage[T2A]{fontenc}
\usepackage[utf8x]{inputenc}
\usepackage[russian]{babel}
\usepackage[pdftex,usetemplates,nodirectory,russian,%
unicode,useui,pro,xcolor,usenames,dvipsnames,x11names]{web}
%%%-----------------------------------------------------------------------
\usepackage[pdftex,execJS]{insdljs}
\usepackage{aeb_tilebg}
\usepackage[pdftex]{attachfile2}
\usepackage{geometry}
\usepackage{translator}
\usepackage{amssymb,latexsym,amsmath}
\usepackage{eucal,eufrak,mathrsfs}
\usepackage{tipa}
\usepackage{texnames}
\usepackage{textcomp}
\usepackage{setspace}
\usepackage{tabularx}
\usepackage{colortbl}
\usepackage{booktabs}
\usepackage{pgf,pgfshade}
\usepackage{pgfmath}
\usepackage{xxcolor}
\usepackage{marvosym}
\usepackage{url}
\usepackage{lipsum}
\usepackage{marginnote}
\usepackage[pdftex]{graphicx}
\usepackage[filename=articles,mouseover]{fancytooltips}
\usepackage{pdfcomment}
\usepackage{pdfpages}
\usepackage{cooltooltips}
\usepackage{ifthen}
\usepackage{animate}
\usepackage{ocg}
\usepackage[colorinlistoftodos, shadow]{todonotes}
\usepackage[absolute,overlay]{textpos}
%%%---> don't work with tikz
%%% \usepackage[3D,final]{movie15}
\usepackage{pstricks}
\usepackage{pst-blur}
\usepackage{tikz}
%%%-----------------------------------------------------------------------
\def\delayinterval{3000}
%%%-----------------------------------------------------------------------
%%%--> \screensize{<height>}{<width>}
%\screensize{4.72in}{5.67in}
\screensize{10.665in}{14.220in}
%%%---> \margins[<panel_width>]{<left>}{<right>}{<top>}{<bottom>}
\margins{.25in}{.25in}{120pt}{.15in}

%\geometry{left=5mm}	% левое поле
%\geometry{right=5mm}	% правое поле
%\geometry{top=15mm}	% верхнее поле
%\geometry{bottom=5mm}	% нижнее поле

\pgfdeclarelayer{background}
\pgfdeclarelayer{foreground}
\pgfsetlayers{background,main,foreground}

%\usetikzlibrary{snakes,backgrounds}
\usetikzlibrary{calc,patterns,shapes}
\usetikzlibrary{trees,matrix}
\usetikzlibrary{positioning,arrows}
\usetikzlibrary{shadows}
\usetikzlibrary{calendar}
\usetikzlibrary{mindmap}
\usetikzlibrary{patterns}
\usetikzlibrary{decorations.text}
\usetikzlibrary{decorations.shapes}
\usetikzlibrary{decorations.text}
\usetikzlibrary{decorations.footprints}
\usetikzlibrary{decorations.pathmorphing}
\usetikzlibrary{folding}
%%%------------------------------------------------------------------------
\sectionLayout{%
	afterskip=12pt,
	halign=c,
	color=blue,
	shadowcolor=blue
}
%%%-----------------------------------------------------------------------
\definecolor{myborder}{rgb}{0.8,1,1}
\definecolor{gray9}{gray}{.9}
\definecolor{orange}{rgb}{1,.549,0}
\definecolor{panelbackground}{gray}{.8}
\definecolor{gray6}{gray}{.4}
\definecolor{gray3}{gray}{.3}
\definecolor{boldtxt}{rgb}{0.3,0.3,0.1}

\definecolor{hellgelb}{rgb}{1,1,0.85}
\definecolor{colkeys}{rgb}{0,0,1}
\definecolor{colIdentifier}{rgb}{0,0,0}
\definecolor{colComments}{rgb}{1,0,0}
\definecolor{colString}{rgb}{0,0.5,0}
\definecolor{light-blue}{rgb}{0.8,0.85,1}
\definecolor{mygray}{gray}{0.75}
\definecolor{grayfifteen}{gray}{.85}
\definecolor{logoblue}{rgb}{0,0,0.267}

\definecolor{buttonbackground}{rgb}{0,.624,.820}
\definecolor{buttonshadow}{rgb}{.001,0,.502}
\definecolor{button}{rgb}{1,.549,.0}
\definecolor{buttondisable}{gray}{.7}

\definecolor{grayfifteen}{gray}{.85}
\definecolor{logoblue}{rgb}{0,0,0.267}
\definecolor{monred}{rgb}{0.8 0.9 1} 

\def\Black{\color{black}}
\def\nBlue{\color{buttonshadow}}
%%%-----------------------------------------------------------------------
\pagestyle{empty}
%%%-----------------------------------------------------------------------
\hypersetup{bookmarksopen=false,
			pdftex=true,
			pdfmark=true,
			linktocpage,
			colorlinks=true,
			unicode,
			linkcolor=black,
			anchorcolor=cyan,
			citecolor=green,
			urlcolor=blue,
			pagecolor=white,
			bookmarks=false,
			bookmarksopen=false,
			pdfcenterwindow=yes,
			pdfmenubar=false,
			pdftoolbar=false,
			pdfpagemode=SinglePage,
			pdfpagemode=FullScreen,
			pdfstartpage=10,
			pdfwindowui=true,
			pdfpagetransition={/S /Split /D 5 /M /O /Dm /V}
			%pdfpagetransition={/S /Split /D 5 /M /O /Dm /H}
			%pdfpagetransition={/S /Wipe /Di /90}
			}
%%%-----------------------------------------------------------------------
\graphicspath{{./img/}} 
%%%-----------------------------------------------------------------------
\newcommand{\Eq}[1]{(\ref{#1})}
%%%---> Tikz Box create --------------------------------------------------
\newlength{\boxw}
\newlength{\boxh}
\newlength{\shadowsize}
\newlength{\boxroundness}
\newlength{\tmpa}
\newsavebox{\shadowblockbox}

\setlength{\shadowsize}{6pt}
\setlength{\boxroundness}{3pt}

\newenvironment{shadowblock}[1]%
{\begin{lrbox}{\shadowblockbox}\begin{minipage}{#1}}%
{\end{minipage}\end{lrbox}%
\settowidth{\boxw}{\usebox{\shadowblockbox}}%
\settodepth{\tmpa}{\usebox{\shadowblockbox}}%
\settoheight{\boxh}{\usebox{\shadowblockbox}}%
\addtolength{\boxh}{\tmpa}%
\begin{tikzpicture}
    \addtolength{\boxw}{\boxroundness * 2}
    \addtolength{\boxh}{\boxroundness * 2}

    \foreach \x in {0,.05,...,1}
    {
        \setlength{\tmpa}{\shadowsize * \real{\x}}
        \fill[xshift=\shadowsize - 1pt,yshift=-\shadowsize + 1pt,
				black,opacity=0.04,rounded corners=\boxroundness]
            (\tmpa, \tmpa) rectangle +(\boxw - \tmpa - \tmpa,
                \boxh - \tmpa - \tmpa);
    }

    \filldraw[left color=blue!50, draw=black!50,thick, rounded corners=\boxroundness]
        (0, 0) rectangle (\boxw, \boxh);
    \draw node[xshift=\boxroundness,yshift=\boxroundness,
        inner sep=0pt,outer sep=0pt,anchor=south west]
             (0,0) {\usebox{\shadowblockbox}};
\end{tikzpicture}}

\newenvironment{shadowblockb}[1]%
{\begin{lrbox}{\shadowblockbox}\begin{minipage}{#1}}%
{\end{minipage}\end{lrbox}%
\settowidth{\boxw}{\usebox{\shadowblockbox}}%
\settodepth{\tmpa}{\usebox{\shadowblockbox}}%
\settoheight{\boxh}{\usebox{\shadowblockbox}}%
\addtolength{\boxh}{\tmpa}%
\begin{tikzpicture}
    \addtolength{\boxw}{\boxroundness * 2}
    \addtolength{\boxh}{\boxroundness * 2}

    \foreach \x in {0,.05,...,1}
    {
        \setlength{\tmpa}{\shadowsize * \real{\x}}
        \fill[xshift=\shadowsize - 1pt,yshift=-\shadowsize + 1pt,
				black,opacity=.04,rounded corners=\boxroundness]
            (\tmpa, \tmpa) rectangle +(\boxw - \tmpa - \tmpa,
                \boxh - \tmpa - \tmpa);
    }

    \filldraw[color=yellow!50, draw=black!50, rounded corners=\boxroundness]
        (0, 0) rectangle (\boxw, \boxh);
    \draw node[xshift=\boxroundness,yshift=\boxroundness,
        inner sep=0pt,outer sep=0pt,anchor=south west]
             (0,0) {\usebox{\shadowblockbox}};
\end{tikzpicture}}
%%%-----------------------------------------------------------------------
%%%---> Tikz tables matrix
\newcommand*\up{\textcolor{green}{\ensuremath{\blacktriangle}}}
\newcommand*\down{\textcolor{red}{\ensuremath{\blacktriangledown}}}
\newcommand*\const{\textcolor{darkgray}{\textbf{--}}}
\newcommand*\head[1]{\textbf{#1}}
\newenvironment{matrixtable}[2]{%
  \begin{tikzpicture}[matrix of nodes/.style={
    execute at begin cell=\node\bgroup\strut,
    execute at end cell=\egroup;}]
  \matrix (m) [matrix of nodes,top color=blue!20,
    bottom color=blue!80,draw=white,
    nodes={draw,top color=blue!10,bottom color=blue!35,
    draw,inner sep=2pt,minimum height=3.1ex},
    column sep=1ex,row sep=0.6ex,inner sep=2ex,
    rounded corners,column 1/.style={minimum width=#1},
	column 2/.style={minimum width=#2}]}
%    column 3/.style={minimum width=#3},
%   column 4/.style={minimum width=#4}]}%
{;\end{tikzpicture}}
%%%-----------------------------------------------------------------------
%\newcommand*\up{\textcolor{green}{%
%	\ensuremath{\blacktriangle}}}
%\newcommand*\down{\textcolor{red}{%
%	\ensuremath{\blacktriangledown}}}
%\newcommand*\const{\textcolor{darkgray}{\textbf{--}}}	
%%%-----------------------------------------------------------------------
%\newcommand\lawordercover
%{%
%	\parbox[b][\paperheight][c]{\textscreenwidth}%
%	%
%	{\centering\includegraphics[width=\paperwidth,height=\paperheight]{%
%	./img/laworder_bg_01}}%
%}
%%%-----------------------------------------------------------------------
\newcommand\lawordercover
{%
	\put(0,633){\includegraphics[scale=0.99]{./laworder_bg_01}}
}
%\AddToTemplate{lawordercover}
%%%-----------------------------------------------------------------------
%\setTileBgGraphic[scale=0.9]{title_01}
%%%---> That's fine
%\setTileBgGraphic[scale=0.99]{title_01}
\setTileBgGraphic[scale=2.99]{title_41}
\maxiterations{17}
%\setTileBgGraphic[scale=0.49]{title_70}
%\maxiterations{17}
%\setTileBgGraphic[scale=0.39]{title_30}
%\maxiterations{27}
%\setTileBgGraphic[scale=0.39]{title_31}
%\maxiterations{27}
%\setTileBgGraphic[scale=0.49]{title_04}
%\maxiterations{27}
%%%-----------------------------------------------------------------------
\newcommand\covers
{%
	\parbox[b][\paperheight][c]{\textscreenwidth}%
		%
		{\centering\includegraphics[width=\paperwidth,height=\paperheight]{%
			./img/ready_05}}%
}
\newcommand\susebgcard
{%
	\parbox[b][\paperheight][c]{\textscreenwidth}%
		%
		{\centering\includegraphics[width=\paperwidth,height=\paperheight]{%
			./img/ready_06}}%
}

%%%-----------------------------------------------------------------------
\graphicspath{{./img/}} 
%%%-----------------------------------------------------------------------
\begin{document}
%\AddToTemplate{lawordercover}
%%%-------------------- Logo LugaTeX Interactive -------------------------
\newcommand{\logo}[5]
{
 	\colorlet{border}{#1}
 	\colorlet{trunk}{#2}
 	\colorlet{leaf a}{#3}
 	\colorlet{leaf b}{#4}
 	\begin{tikzpicture}
 	\scriptsize\scshape
 	\draw[border,line width=1ex,yshift=.3cm,
 	yscale=1.45,xscale=1.05,looseness=1.42]
 	(1,0) to [out=90, in=0] (0,1) to [out=180,in=90] (-1,0)
 	to [out=-90,in=-180] (0,-1) to [out=0, in=-90] (1,0) -- cycle;
 	\coordinate (root) [grow cyclic,rotate=90]
 	child {
 		child [line cap=round] foreach \a in {0,1} {
 			child foreach \b in {0,1} {
 				child foreach \c in {0,1} {
 					child foreach \d in {0,1} {
 						child foreach \leafcolor in {leaf a,leaf b}
 						{ edge from parent [color=\leafcolor,-#5] }
 					} } }
 		} edge from parent [shorten >=-1pt,serif cm-,line cap=butt]
 	};
 	\node [align=center,below] at (0pt,-.5ex)
 	{ \textcolor{border}{L}uga\TeX \\ \textcolor{border}{I}nteractive \\
 	   \textcolor{border}{S}ystem };
 	\end{tikzpicture}
}
%%%-----------------------------------------------------------------------
%%%---> icon book
\newcommand{\iconbook}{%
\begin{colormixin}{130}%
  \pgfdeclarehorizontalshading{cover}{20pt}{%
    rgb(0pt)=(0.84,.5,.5);
    rgb(1.8pt)=(0.82,.48,.48);
    rgb(1.9pt)=(0.83,.66,.65);
    rgb(2.1pt)=(0.83,.66,.65);
    rgb(3pt)=(0.69,.25,.3);
    rgb(8pt)=(0.45,0.05,0.05)}%
                                %
  \pgfdeclareverticalshading{side}{10pt}{%
    rgb(0pt)=(0.78,.78,.78);
    rgb(2.5pt)=(0.60,.60,.60);
    rgb(5pt)=(0.25,.25,.25)}%
  \noindent\hbox{%
    \begin{pgfpicture}{0pt}{1pt}{14pt}{11pt}
      \pgfsetxvec{\pgfpoint{1pt}{0pt}}
      \pgfsetyvec{\pgfpoint{0pt}{1pt}}
      \pgfsetlinewidth{0.4pt}
      \pgfsetroundjoin
      
      \pgfsetlinewidth{0.8pt}
      \color[gray]{0.5}
      \pgfmoveto{\pgfxy(6.5,11.5)}
      \pgflineto{\pgfxy(1,10.5)}
      \pgfcurveto{\pgfxy(0.6,9.75)}{\pgfxy(0.6,8.75)}{\pgfxy(1,8)}
      \pgflineto{\pgfxy(6.5,2)}
      \pgflineto{\pgfxy(13,3)}
      \pgfcurveto{\pgfxy(12,4)}{\pgfxy(12,5)}{\pgfxy(13,6)}
      \pgfclosepath
      
      \pgfmoveto{\pgfxy(6.5,2)}  
      \pgfcurveto{\pgfxy(6,3)}{\pgfxy(6,4)}{\pgfxy(6.5,5)}
      \pgflineto{\pgfxy(13,6)}
      \pgfstroke

      \begin{pgfscope}
        \pgfmoveto{\pgfxy(6.5,11.5)}
        \pgflineto{\pgfxy(1,10.5)}
        \pgfcurveto{\pgfxy(0.6,9.75)}{\pgfxy(0.6,8.75)}{\pgfxy(1,8)}
        \pgflineto{\pgfxy(6.5,2)}
        \pgfcurveto{\pgfxy(6,3)}{\pgfxy(6,4)}{\pgfxy(6.5,5)}
        \pgflineto{\pgfxy(13,6)}
        \pgfclosepath
        \pgfclip

        \pgfputat{\pgfxy(8.5,0)}
        {%
          \begin{pgfrotateby}{\pgfdegree{45}}
            \pgfbox[left,base]{\pgfuseshading{cover}}
          \end{pgfrotateby}
        }
      \end{pgfscope}      
      
      \begin{pgfscope}
        \pgfmoveto{\pgfxy(6.5,2)}  
        \pgfcurveto{\pgfxy(6,3)}{\pgfxy(6,4)}{\pgfxy(6.5,5)}
        \pgflineto{\pgfxy(13,6)}
        \pgfcurveto{\pgfxy(12,5)}{\pgfxy(12,4)}{\pgfxy(13,3)}
        \pgfclosepath
        \pgfclip

        \pgfputat{\pgfxy(7.5,0)}
        {%
          \begin{pgfrotateby}{\pgfdegree{30}}
            \pgfbox[left,base]{\pgfuseshading{side}}
          \end{pgfrotateby}
        }
      \end{pgfscope}      
      
      \pgfsetlinewidth{0.4pt}
      \color[gray]{0.2}
      \pgfmoveto{\pgfxy(6.5,11.5)}
      \pgflineto{\pgfxy(1,10.5)}
      \pgfcurveto{\pgfxy(0.6,9.75)}{\pgfxy(0.6,8.75)}{\pgfxy(1,8)}
      \pgflineto{\pgfxy(6.5,2)}
      \pgflineto{\pgfxy(13,3)}
      \pgfcurveto{\pgfxy(12,4)}{\pgfxy(12,5)}{\pgfxy(13,6)}
      \pgfclosepath
      
      \pgfmoveto{\pgfxy(6.5,2)}  
      \pgfcurveto{\pgfxy(6,3)}{\pgfxy(6,4)}{\pgfxy(6.5,5)}
      \pgflineto{\pgfxy(13,6)}
      \pgfstroke
    \end{pgfpicture}%
  }%
\end{colormixin}%
}
%%%---> icon article
\newcommand{\iconarticle}{%
\begin{colormixin}{20}%
  \pgfdeclareverticalshading{shadow}{20pt}{%
    rgb(0pt)=(.2,.2,.2);
    rgb(11pt)=(1,1,1)}%
  \pgfdeclareverticalshading{paper}{20pt}{%
    rgb(0pt)=(0.8,0.8,0.5);
    rgb(15pt)=(1,1,1)}%
  \pgfdeclareverticalshading{pic}{2.5pt}{%
    rgb(0pt)=(0.25,0.75,0.25);
    rgb(1.5pt)=(0.75,0.25,0.25);
    rgb(3.5pt)=(0.25,0.25,0.75)}%
  \pgfdeclareverticalshading{corner}{2pt}{%
    rgb(0pt)=(0.5,0.5,0);
    rgb(2pt)=(0.8,0.8,0.8)}%
  \noindent\hbox{%
    \begin{pgfpicture}{-1pt}{-2pt}{10pt}{12pt}
      \pgfsetxvec{\pgfpoint{1pt}{0pt}}
      \pgfsetyvec{\pgfpoint{0pt}{1pt}}
      \pgfsetlinewidth{0.4pt}

                                %    \begin{pgfscope}
                                %      \color[gray]{0.7}
                                %      \pgfmoveto{\pgfxy(0.6,-1)}
                                %      \pgflineto{\pgfxy(9,-1)}
                                %      \pgflineto{\pgfxy(9,8.2)}
                                %      \pgflineto{\pgfxy(6.8,10.4)}
                                %      \pgflineto{\pgfxy(0.6,10.4)}
                                %      \pgfclip

                                %      \pgfputat{\pgfxy(0.6,-10)}
                                %      {%
                                %        \begin{pgfrotateby}{\pgfdegree{45}}
                                %          \pgfbox[left,base]{\pgfuseshading{shadow}}
                                %        \end{pgfrotateby}
                                %      }
                                %    \end{pgfscope}      

      \begin{pgfscope}
        \pgfmoveto{\pgfxy(0,0)}
        \pgflineto{\pgfxy(8,0)}
        \pgflineto{\pgfxy(8,9)}
        \pgflineto{\pgfxy(6,9)}
        \pgflineto{\pgfxy(6,11)}
        \pgflineto{\pgfxy(0,11)}
        \pgfclip

        \pgfputat{\pgfxy(0,-10)}
        {%
          \begin{pgfrotateby}{\pgfdegree{45}}
            \pgfbox[left,base]{\pgfuseshading{paper}}
          \end{pgfrotateby}
        }
      \end{pgfscope}
      
      \begin{pgfscope}
        \pgfmoveto{\pgfxy(8,9)}
        \pgflineto{\pgfxy(6,9)}
        \pgflineto{\pgfxy(6,11)}
        \pgfclip

        \pgfputat{\pgfxy(6,9)}{\pgfbox[left,base]{\pgfuseshading{corner}}}
      \end{pgfscope}

      \pgfmoveto{\pgfxy(0,0)}
      \pgflineto{\pgfxy(8,0)}
      \pgflineto{\pgfxy(8,9)}
      \pgflineto{\pgfxy(6,11)}
      \pgflineto{\pgfxy(0,11)}
      \pgfclosepath
      \pgfstroke
      
      \color[gray]{0.5}
      \pgfxyline(1,9.5)(6,9.5)
      \color[gray]{0.6}
      \pgfxyline(2,8)(6,8)
      \pgfxyline(2,7)(6,7)
      
      \color[gray]{0.7}
      \pgfxyline(1,5.5)(3.5,5.5)
      \pgfxyline(1,4.5)(3.5,4.5)
      \pgfxyline(1,3.5)(3.5,3.5)
      \pgfxyline(1,2.5)(3.5,2.5)
      \pgfxyline(1,1.5)(3.5,1.5)

      \pgfputat{\pgfxy(4.5,2.25)}{\pgfbox[left,base]{\pgfuseshading{pic}}}
      \pgfxyline(4.5,1.5)(7,1.5)

      \color{black}
      \pgfmoveto{\pgfxy(8,9)}
      \pgflineto{\pgfxy(6,9)}
      \pgflineto{\pgfxy(6,11)}
      \pgfstroke
    \end{pgfpicture}%
  }%
\end{colormixin}%
}
%%%-----------------------------------------------------------------------
%\multido{\i=1+1}{1}{\null\newpage}
\AddToTemplate{covers}
\AddToTemplate{lawordercover}
\newpage
\begin{tikzpicture}[remember picture,overlay]
	  \node [rotate=0,scale=2,text opacity=0.2]
	      at (27,1.7) {Капранов~О.~Г.~\copyright~2010~~~Luga\TeX @yahoo.com};
\end{tikzpicture}
\vglue -18pt
\hspace{187pt}
\parbox{350pt}{%
\hypertarget{interactive}{\hyperlink{studyplan}{%
\begin{tikzpicture}
  \colorlet{even}{cyan!60!black}
  \colorlet{odd}{orange!100!black}
  \colorlet{links}{red!70!black}
  \colorlet{back}{yellow!20!white}
  \tikzset{
    box/.style={
      minimum height=15mm,
      inner sep=.7mm,
      outer sep=0mm,
      text width=120mm,
      text centered,
      font=\huge\bfseries\sffamily,
      text=#1!50!black,
      draw=#1,
      line width=.25mm,
      top color=#1!5,
      bottom color=#1!40,
      shading angle=0,
      rounded corners=2.3mm,
      drop shadow={fill=#1!40!gray,fill opacity=.8},
      rotate=0,
    },
  }
\node[box=links,text=monred!15,anchor=south east,xshift=0.1mm,yshift=1mm,fill=blue!65!black]
	{Навчально-методичний $\text{посібник}^{\text{%
	\pdfcomment[open=true,color=yellow,icolor=red,icon=Comment, subject={Аннотация},author={%
	АДМІНІСТРАТИВНЕ ПРАВО УКРАЇНИ}]{%
	Рекомендовано вченою радою Луганського державного університету внутрішніх
	справ імені Е.О. Дідоренка (протокол від 2008 року).
	%Укладачі: АФАНАСЬЄВ К.К., канд. юрид. наук, доц., професор кафедри
	%адміністративного права та адміністративної діяльності ЛДУВС імені Е.О.
	%Дідоренка, БЕНИЦЬКИЙ О.М., викладач кафедри адміністративного права та
	%адміністративної діяльності ЛДУВС імені Е.О. Дідоренка, капітан міліції,
	%ГОЛОВІН А.П.., канд. юрид. наук, доцент, начальник кафедри адміністративного
	%права та адміністративної діяльності ЛДУВС імені Е.О. Дідоренка, полковник
	%міліції, САЄНКО С.І., канд. юрид. наук, старший викладач кафедри
	%адміністративного права та адміністративної діяльності ЛДУВС імені Е.О.
	%Дідоренка, капітан міліції.
	%Відповідальний редактор: А.П. Головін, канд. юрид. наук, доцент, начальник
	%кафедри адміністративного права та адміністративної діяльності ЛДУВС імені Е.О.
	%Дідоренка.
	%Рецензенти: ЛЕВЧЕНКОВ О.І., канд. юрид. наук, доц., засл. юрист України,
	%проректор з наукової роботи Луганського державного університету внутрішніх
	%справ імені Е.О. Дідоренка; ШАПОВАЛОВА О.В., докт. юрид. наук, доц., завідувач
	%кафедри господарського права СНУ імені Володимира Даля
	%Афанасьєв К.К., Беницький О.М., Головін А.П., Саєнко С.І.
	Адміністративне право України: Навчально-методичний посібник / МВС України,
	Луганський державний університет внутрішніх справ ім. Е.О. Дідоренка; [Відп.
	ред. А.П. Головін]. - Луганськ: РВВ ЛДУВС, 2008. - с.}}}$};
	\node [font=\Large\bfseries\sffamily, text centered, text width=120mm,
   	rectangle, inner sep=2pt, very thick, inner color=transparent!80,
   	outer color=transparent!30, draw opacity=0.9, fill opacity=0.3,
   	line width=1.6pt, text opacity=1, text=blue!30!black]	
	at (-6,-0.3) {Электронно-интерактивный документ};
\end{tikzpicture}
}}}
\hspace{18pt}
\parbox{300pt}{%
\mbox{}\hspace{22pt}\vglue -13pt\parbox{400pt}{%
\mbox{\href{ftp://10.2.61.3/}{\mbox{\includegraphics[scale=.2]{adobe_02.png}}}}\hspace{13pt}
\mbox{\href{ftp://10.2.61.3/}{\mbox{\includegraphics[scale=.2]{adobe_03.png}}}}\hspace{13pt}
\mbox{\href{ftp://10.2.61.3/}{\mbox{\includegraphics[scale=.2]{adobe_01.png}}}}
}}\\
%%%----> Navigational Panel bottom
\vglue 421pt
\begin{flushleft}
	\mbox{}\hspace{193.4pt}	
\hyperlink{studyplan}{%
\begin{tikzpicture}[baseline]
\colorlet{even}{cyan!60!black}
\colorlet{odd}{orange!100!black}
\colorlet{links}{red!70!black}
\colorlet{back}{yellow!20!white}
\colorlet{magic}{magenta!20!black}
\tikzset{
	box/.style={
		minimum height=18mm,
		minimum width=19mm,
		inner sep=.7mm,
		outer sep=0mm,
		text width=16mm,
		text centered,
		font=\normalsize\bfseries\sffamily,
%		text=#1!50!black,
		text=white,
		draw=#1,
		line width=.35mm,
		top color=#1!5,
		bottom color=#1!40,
		shading angle=0,
		rounded corners=2.3mm,
		drop shadow={fill=#1!40!gray,fill opacity=.8},
		rotate=0,
		opacity=1,
		text opacity=1,
	},
}
\node[box=links,text=monred!15,anchor=south east,xshift=0.1mm,yshift=1mm,fill=blue!65!black]
{\textcolor{black}{Учебный план}};
\end{tikzpicture}}\hspace{15.3pt}
\hyperlink{mymedia}{
\begin{tikzpicture}[baseline]
\colorlet{even}{cyan!60!black}
\colorlet{odd}{orange!100!black}
\colorlet{links}{red!70!black}
\colorlet{back}{yellow!20!white}
\colorlet{magic}{magenta!20!black}
\tikzset{
	box/.style={
		minimum height=18mm,
		minimum width=19mm,
		inner sep=.7mm,
		outer sep=0mm,
		text width=16mm,
		text centered,
		font=\normalsize\bfseries\sffamily,
%		text=#1!50!black,
		text=white,
		draw=#1,
		line width=.35mm,
		top color=#1!5,
		bottom color=#1!40,
		shading angle=0,
		rounded corners=2.3mm,
		drop shadow={fill=#1!40!gray,fill opacity=.8},
		rotate=0,
		opacity=1,
		text opacity=1,
	},
}
\node[box=links,text=monred!15,anchor=south east,xshift=0.1mm,yshift=1mm,fill=blue!65!black]
{\textcolor{black}{Видео лекция}};
\end{tikzpicture}}\hspace{15.3pt}
\hyperlink{preface}{
\begin{tikzpicture}[baseline]
\colorlet{even}{cyan!60!black}
\colorlet{odd}{orange!100!black}
\colorlet{links}{red!70!black}
\colorlet{back}{yellow!20!white}
\colorlet{magic}{magenta!20!black}
\tikzset{
	box/.style={
		minimum height=18mm,
		minimum width=19mm,
		inner sep=.7mm,
		outer sep=0mm,
		text width=16mm,
		text centered,
		font=\normalsize\bfseries\sffamily,
%		text=#1!50!black,
		text=white,
		draw=#1,
		line width=.35mm,
		top color=#1!5,
		bottom color=#1!40,
		shading angle=0,
		rounded corners=2.3mm,
		drop shadow={fill=#1!40!gray,fill opacity=.8},
		rotate=0,
		opacity=1,
		text opacity=1,
	},
}
\node[box=links,text=monred!15,anchor=south east,xshift=0.1mm,yshift=1mm,fill=blue!65!black]
{\mbox{\textcolor{black}{Загальні}} \scalebox{0.8}{\textcolor{black}{положення}}};
\end{tikzpicture}}\hspace{18.5pt}
\hyperlink{forms}{%
\begin{tikzpicture}[baseline]
\colorlet{even}{cyan!60!black}
\colorlet{odd}{orange!100!black}
\colorlet{links}{red!70!black}
\colorlet{back}{yellow!20!white}
\colorlet{magic}{magenta!20!black}
\tikzset{
	box/.style={
		minimum height=18mm,
		minimum width=19mm,
		inner sep=.7mm,
		outer sep=0mm,
		text width=16mm,
		text centered,
		font=\normalsize\bfseries\sffamily,
%		text=#1!50!black,
		text=white,
		draw=#1,
		line width=.35mm,
		top color=#1!5,
		bottom color=#1!40,
		shading angle=0,
		rounded corners=2.3mm,
		drop shadow={fill=#1!40!gray,fill opacity=.8},
		rotate=0,
		opacity=1,
		text opacity=1,
	},
}
\node[box=links,text=monred!15,anchor=south east,xshift=0.1mm,yshift=1mm,fill=blue!65!black]
{\mbox{\textcolor{black}{Форми}} \mbox{\textcolor{black}{та}} \mbox{\textcolor{black}{критерії}}};
\end{tikzpicture}}\hspace{18.3pt}
\hyperlink{admintbl1}{%
\begin{tikzpicture}[baseline]
\colorlet{even}{cyan!60!black}
\colorlet{odd}{orange!100!black}
\colorlet{links}{red!70!black}
\colorlet{back}{yellow!20!white}
\colorlet{magic}{magenta!20!black}
\tikzset{
	box/.style={
		minimum height=18mm,
		minimum width=19mm,
		inner sep=.7mm,
		outer sep=0mm,
		text width=16mm,
		text centered,
		font=\normalsize\bfseries\sffamily,
%		text=#1!50!black,
		text=white,
		draw=#1,
		line width=.35mm,
		top color=#1!5,
		bottom color=#1!40,
		shading angle=0,
		rounded corners=2.3mm,
		drop shadow={fill=#1!40!gray,fill opacity=.8},
		rotate=0,
		opacity=1,
		text opacity=1,
	},
}
\node[box=links,text=monred!15,anchor=south east,xshift=0.1mm,yshift=1mm,fill=blue!65!black]
{\mbox{\textcolor{black}{План}} \mbox{\textcolor{black}{за курсом}}};
\end{tikzpicture}}\hspace{18.5pt}
\hyperlink{chapter4a}{%
\begin{tikzpicture}[baseline]
\colorlet{even}{cyan!60!black}
\colorlet{odd}{orange!100!black}
\colorlet{links}{red!70!black}
\colorlet{back}{yellow!20!white}
\colorlet{magic}{magenta!20!black}
\tikzset{
	box/.style={
		minimum height=18mm,
		minimum width=19mm,
		inner sep=.7mm,
		outer sep=0mm,
		text width=16mm,
		text centered,
		font=\normalsize\bfseries\sffamily,
%		text=#1!50!black,
		text=white,
		draw=#1,
		line width=.35mm,
		top color=#1!5,
		bottom color=#1!40,
		shading angle=0,
		rounded corners=2.3mm,
		drop shadow={fill=#1!40!gray,fill opacity=.8},
		rotate=0,
		opacity=1,
		text opacity=1,
	},
}
\node[box=links,text=monred!15,anchor=south east,xshift=0.1mm,yshift=1mm,fill=blue!65!black]
{\mbox{\textcolor{black}{Тема~1.1}}};
\end{tikzpicture}}\hspace{18.5pt}
\hyperlink{chapter5a}{%
\begin{tikzpicture}[baseline]
\colorlet{even}{cyan!60!black}
\colorlet{odd}{orange!100!black}
\colorlet{links}{red!70!black}
\colorlet{back}{yellow!20!white}
\colorlet{magic}{magenta!20!black}
\tikzset{
	box/.style={
		minimum height=18mm,
		minimum width=19mm,
		inner sep=.7mm,
		outer sep=0mm,
		text width=16mm,
		text centered,
		font=\normalsize\bfseries\sffamily,
%		text=#1!50!black,
		text=white,
		draw=#1,
		line width=.35mm,
		top color=#1!5,
		bottom color=#1!40,
		shading angle=0,
		rounded corners=2.3mm,
		drop shadow={fill=#1!40!gray,fill opacity=.8},
		rotate=0,
		opacity=1,
		text opacity=1,
	},
}
\node[box=links,text=monred!15,anchor=south east,xshift=0.1mm,yshift=1mm,fill=blue!65!black]
{\mbox{\textcolor{black}{Тема~2.2}}};
\end{tikzpicture}}\hspace{18.6pt}
\hyperlink{mylugatex}{%
\begin{tikzpicture}[baseline]
\colorlet{even}{cyan!60!black}
\colorlet{odd}{orange!100!black}
\colorlet{links}{red!70!black}
\colorlet{back}{yellow!20!white}
\colorlet{magic}{magenta!20!black}
\tikzset{
	box/.style={
		minimum height=18mm,
		minimum width=19mm,
		inner sep=.7mm,
		outer sep=0mm,
		text width=16mm,
		text centered,
		font=\normalsize\bfseries\sffamily,
%		text=#1!50!black,
		text=white,
		draw=#1,
		line width=.35mm,
		top color=#1!5,
		bottom color=#1!40,
		shading angle=0,
		rounded corners=2.3mm,
		drop shadow={fill=#1!40!gray,fill opacity=.8},
		rotate=0,
		opacity=1,
		text opacity=1,
	},
}
\node[box=links,text=monred!15,anchor=south east,xshift=0.1mm,yshift=1mm,fill=blue!65!black]
{\mbox{\textcolor{black}{Lu\emph{g}a\TeX}}};
\end{tikzpicture}}
\end{flushleft}
%%%--->
\vglue 15pt
\centerline{%
\mbox{\href{ftp://10.2.61.3/}{\mbox{\includegraphics[scale=0.8]{facebook.png}}}}
\mbox{\href{ftp://10.2.61.3/}{\mbox{\includegraphics[scale=0.8]{twitter.png}}}}
}
\vglue 10pt
\hspace{347pt}
\hyperlink{mymedia}{%
\mbox{%
\tikz[baseline] \node[rectangle,inner sep=2pt,minimum height=3.1ex,
rounded corners,drop shadow,shadow scale=1,shadow xshift=.8ex,
shadow yshift=-.4ex,opacity=.7,fill=black!50,top color=blue!90!black!50,
bottom color=blue!80!black!80,draw=blue!50!black!50,very thick,text=white,
text opacity=1,minimum width=9cm]{%
Интерактивный навчально-методичний посібник};}}
%%%---|
\begin{textblock}{1}(25,-0.01)
\begin{tikzpicture}[even odd rule,rounded corners=2pt,x=10pt,y=10pt,drop shadow]
\filldraw[fill=yellow!90!black!40,drop shadow] (0,0)   rectangle (1,1)
	[xshift=5pt,yshift=5pt]   (0,0)   rectangle (1,1)
	[rotate=30]   (-1,-1) rectangle (2,2);
\node at (0,1.7) {\textbf{\thepage}};			      
\end{tikzpicture}
\end{textblock}
%%%--- Navigational panel top page
\begin{textblock}{2}(7.58,0.85)
\mbox{%%%--->
\Acrobatmenu{LastPage}{%
\tikz[baseline] \node[rectangle,inner sep=2pt,minimum height=3.1ex,
rounded corners,drop shadow,shadow scale=1,shadow xshift=.8ex,
shadow yshift=-.4ex,opacity=.7,fill=black!50,top color=red!90!black!50,
bottom color=red!80!black!80,draw=red!50!black!50,very thick,text=white,
text opacity=1,minimum width=3cm,font=\bfseries\sffamily] at (0,0) {К концу};
}\Acrobatmenu{GoBack}{%
\tikz[baseline] \node[rectangle,inner sep=2pt,minimum height=3.1ex,
rounded corners,drop shadow,shadow scale=1,shadow xshift=.8ex,
shadow yshift=-.4ex,opacity=.7,fill=black!50,top color=red!90!black!50,
bottom color=red!80!black!80,draw=red!50!black!50,very thick,text=white,
text opacity=1,minimum width=3cm,font=\bfseries\sffamily] at (4,0) {Назад};
}\Acrobatmenu{PrevPage}{%
\tikz[baseline] \node[rectangle,inner sep=2pt,minimum height=3.1ex,
rounded corners,drop shadow,shadow scale=1,shadow xshift=.8ex,
shadow yshift=-.4ex,opacity=.7,fill=black!50,top color=red!90!black!50,
bottom color=red!80!black!80,draw=red!50!black!50,very thick,text=white,
text opacity=1,minimum width=3cm,font=\bfseries\sffamily] at (8,0) {Предыдущий};
}\Acrobatmenu{NextPage}{%
\tikz[baseline] \node[rectangle,inner sep=2pt,minimum height=3.1ex,
rounded corners,drop shadow,shadow scale=1,shadow xshift=.8ex,
shadow yshift=-.4ex,opacity=.7,fill=black!50,top color=red!90!black!50,
bottom color=red!80!black!80,draw=red!50!black!50,very thick,text=white,
text opacity=1,minimum width=3cm,font=\bfseries\sffamily] at (12,0) {Следующий};
}\Acrobatmenu{GoForward}{%
\tikz[baseline] \node[rectangle,inner sep=2pt,minimum height=3.1ex,
rounded corners,drop shadow,shadow scale=1,shadow xshift=.8ex,
shadow yshift=-.4ex,opacity=.7,fill=black!50,top color=red!90!black!50,
bottom color=red!80!black!80,draw=red!50!black!50,very thick,text=white,
text opacity=1,minimum width=3cm,font=\bfseries\sffamily] at (16,0) {Вперед};
}\Acrobatmenu{FirstPage}{%
\tikz[baseline] \node[rectangle,inner sep=2pt,minimum height=3.1ex,
rounded corners,drop shadow,shadow scale=1,shadow xshift=.8ex,
shadow yshift=-.4ex,opacity=.7,fill=black!50,top color=red!90!black!50,
bottom color=red!80!black!80,draw=red!50!black!50,very thick,text=white,
text opacity=1,minimum width=3cm,font=\bfseries\sffamily] at (20,0) {К началу};
}\Acrobatmenu{FullScreen}{%
\tikz[baseline] \node[rectangle,inner sep=2pt,minimum height=3.1ex,
rounded corners,drop shadow,shadow scale=1,shadow xshift=.8ex,
shadow yshift=-.4ex,opacity=.7,fill=black!50,top color=red!90!black!50,
bottom color=red!80!black!80,draw=red!50!black!50,very thick,text=white,
text opacity=1,minimum width=3cm,font=\bfseries\sffamily] at (24,0) {Полный экран};
}\Acrobatmenu{Quit}{%
\tikz[baseline] \node[rectangle,inner sep=2pt,minimum height=3.1ex,
rounded corners,drop shadow,shadow scale=1,shadow xshift=.8ex,
shadow yshift=-.4ex,opacity=.7,fill=black!50,top color=red!90!black!50,
bottom color=red!80!black!80,draw=red!50!black!50,very thick,text=white,
text opacity=1,minimum width=3cm,font=\bfseries\sffamily] at (28,0) {Выход};
}	
}%%%---|
\end{textblock}
%%%----------------------------------------------------------------------|
\begin{textblock}{3}(11,10.3)
\pdfmarkupcomment[subject={Аннотация},author={Полная версия издания для печати},
	opacity=0.5,open=true,color=white]{%
	\textattachfile{Article.pdf}{\scalebox{2}{\iconbook}}
		}
	{Adobe Reader}
\hspace{3pt}\tikz
\node [copy shadow={left color=red!50}, left color=red!50,draw=white,very thin]
{{\bf {\large Печатная версия}}};
\end{textblock}
%%%----------------------------------------------------------------------|
\begin{textblock}{4}(5.05,3.65)
\begin{tikzpicture}
\colorlet{even}{cyan!60!black}
\colorlet{odd}{orange!100!black}
\colorlet{links}{red!70!black}
\colorlet{back}{yellow!20!white}
\tikzset{
	box/.style={
		minimum height=50mm,
		inner sep=.7mm,
		outer sep=0mm,
		text width=212mm,
		text centered,
		font=\huge\bfseries\sffamily,
		text=#1!50!black,
		draw=#1,
		line width=.25mm,
		top color=#1!5,
		bottom color=#1!40,
		shading angle=0,
		rounded corners=2.3mm,
		drop shadow={fill=#1!40!gray,fill opacity=.8},
		rotate=0,
		opacity=.6,
		text opacity=1,
	},
}
\node [box=odd] {%
		МВС УКРАЇНИ\\
		ЛУГАНСЬКИЙ ДЕРЖАВНИЙ УНІВЕРСИТЕТ\\
		ВНУТРІШНІХ СПРАВ ІМЕНІ Е.О. ДІДОРЕНКА\\[70pt]
		Кафедра адміністративного права та\\
		адміністративної діяльності\\[70pt]
		АДМІНІСТРАТИВНЕ ПРАВО УКРАЇНИ\\[100pt]
		Луганськ 2010};
\end{tikzpicture}
\end{textblock}

\disableTemplate{covers}
\disableTemplate{lawordercover}
\disableTiling
\newpage
%%%---> Main baground overlay
\begin{tikzpicture}[remember picture,overlay]
\begin{pgfonlayer}{background}
%%%---> original size
%\clip (-1.5,-5) rectangle ++(4,10);
%\clip (-6.3,-7.8) rectangle ++(14.3,15.2);
\colorlet{upperleft}{red!25}
\colorlet{upperright}{green!50!black!25}
\colorlet{lowerleft}{blue!25}
\colorlet{lowerright}{red!25}

% The large rectangles:
\fill [upperleft] (17.7,-11) rectangle ++(-20,16);
\fill [upperright] (17.7,-11) rectangle ++(18,16);
\fill [lowerleft] (17.7,-11) rectangle ++(-20,-13);
\fill [lowerright] (17.7,-11) rectangle ++(18,-13);

% The shadings:
\shade [left color=upperleft,right color=upperright]
([xshift=-1cm]17.7,-11) rectangle ++(2,16);
\shade [left color=lowerleft,right color=lowerright]
([xshift=-1cm]17.7,-11) rectangle ++(2,-13);
\shade [top color=upperleft,bottom color=lowerleft]
([yshift=-1cm]17.7,-11) rectangle ++(-20,2);
\shade [top color=upperright,bottom color=lowerright]
([yshift=-1cm]17.7,-11) rectangle ++(18,2);
\end{pgfonlayer}
\end{tikzpicture}
\begin{tikzpicture}[remember picture,overlay]
	  \node [rotate=0,scale=2,text opacity=0.2]
	      at (27,1.7) {Капранов~О.~Г.~\copyright~2010~~~Luga\TeX @yahoo.com};
\end{tikzpicture}
\vglue -18pt
\hspace{187pt}
\parbox{350pt}{%
\hypertarget{studyplan}{\hyperlink{studyear}{%
\begin{tikzpicture}
  \colorlet{even}{cyan!60!black}
  \colorlet{odd}{orange!100!black}
  \colorlet{links}{red!70!black}
  \colorlet{back}{yellow!20!white}
  \tikzset{
    box/.style={
      minimum height=15mm,
      inner sep=.7mm,
      outer sep=0mm,
      text width=120mm,
      text centered,
      font=\small\bfseries\sffamily,
      text=#1!50!black,
      draw=#1,
      line width=.25mm,
      top color=#1!5,
      bottom color=#1!40,
      shading angle=0,
      rounded corners=2.3mm,
      drop shadow={fill=#1!40!gray,fill opacity=.8},
      rotate=0,
    },
  }
  \node [box=odd]{{\huge\textbf{Учебный план\quad 2010--2011}}};
\end{tikzpicture}
}}}\\

\def\lecture#1#2#3#4#5#6{
% As before:
\node [annotation, #3, scale=0.65, text width=4cm, inner sep=2mm, fill=white] at (#4) {
Lecture #1: \textcolor{orange}{\textbf{#2}}
\list{--}{\topsep=2pt\itemsep=0pt\parsep=0pt
\parskip=0pt\labelwidth=8pt\leftmargin=8pt
\itemindent=0pt\labelsep=2pt}
	#5
	\endlist
	};
	% New:
	\node [anchor=base west] at (cal-#6.base east) {\textcolor{orange}{\textbf{#2}}};
}

\def\onlylecture#1#2{
	% New:
	\node [anchor=base west] at (cal-#2.base east) {\textcolor{orange}{\textbf{#1}}};
}
\def\onlylectureright#1#2{
	% New:
	\node [anchor=base east] at (cal-#2.base west) {\textcolor{orange}{\textbf{#1}}};
}

\noindent
\begin{tikzpicture}[anchor=mid]
	\begin{scope}[
		mindmap,
		every node/.style={concept, circular drop shadow,execute at begin node=\hskip0pt},
		root concept/.append style={
		concept color=black,
		fill=white, line width=1ex,
		text=black, font=\large\scshape},
		text=white,
		computational problems/.style={concept color=red,faded/.style={concept color=red!50}},
		computational models/.style={concept color=blue,faded/.style={concept color=blue!50}},
		measuring complexity/.style={concept color=orange,faded/.style={concept color=orange!50}},
		solving problems/.style={concept color=green!50!black,faded/.style={concept color=green!50!black!50}},
		grow cyclic,
		level 1/.append style={level distance=4.5cm,sibling angle=90,font=\scshape},
		level 2/.append style={level distance=3cm,sibling angle=45,font=\scriptsize}]
		\node [root concept, font=\bfseries\sffamily, text width=115pt]
		(Computational Complexity) {Административное\par право} % root
		child [computational problems] { node [yshift=-1cm] (Computational Problems) {Computational Problems}
		child { node (Problem Measures) {Problem Measures} }
		child { node (Problem Aspects) {Problem Aspects} }
		child [faded] { node (problem Domains) {Problem Domains} }
		child { node (Key Problems) {Key Problems} }
		}
		child [computational models] { node [yshift=-1cm] (Computational Models) {Computational Models}
		child { node (Turing Machines) {Turing Machines} }
		child [faded] { node (Random-Access Machines) {Random-Access Machines} }
		child { node (Circuits) {Circuits} }
		child [faded] { node (Binary Decision Diagrams) {Binary Decision Diagrams} }
		child { node (Oracle Machines) {Oracle Machines} }
		child { node (Programming in Logic) {Programming in Logic} }
		}
		child [measuring complexity] { node [yshift=1cm] (Measuring Complexity) {Measuring Complexity}
		child { node (Complexity Measures) {Complexity Measures} }
		child { node (Classifying Complexity) {Classifying Complexity} }
		child { node (Comparing Complexity) {Comparing Complexity} }
		child [faded] { node (Describing Complexity) {Describing Complexity} }
		}
		child [solving problems] { node [yshift=1cm] (Solving Problems) {Solving Problems}
		child { node (Exact Algorithms) {Exact Algorithms} }
		child { node (Randomization) {Randomization} }
		child { node (Fixed-Parameter Algorithms) {Fixed-Parameter Algorithms} }
		child { node (Parallel Computation) {Parallel Computation} }
		child { node (Partial Solutions) {Partial Solutions} }
		child { node (Approximation) {Approximation} }
		};
	\end{scope}
%%%---> Left side calendar	
	\tiny
	\calendar (mycalsep) [day list downward,
	month text=\%mt\ \%y0,
%	month text=Сентябрь\quad 2010,
	month yshift=3.5em,
	name=cal,
%	at={(-.5\textwidth-5mm,.5\textheight-1cm)},
%	at={(-.48\textwidth-5mm,.48\textheight-4cm)},
at={(-18,9)},
	dates=2010-09-01 to 2010-09-last]
%	if (equals=2010-09-25) {\draw (0,0) circle (4pt);}
	if (weekend)
	[white]
	if (day of month=1) {
	\node at (.0em,1.5em) [anchor=base west] {\large\bfseries\sffamily\tikzmonthtext};
	};
	\draw[red] (cal-2010-09-20) circle (4pt);
	\calendar (mycalocb) [day list downward,
	month text=\%mt\ \%y0,
%	month text=Октябрь \quad 2010,
	month yshift=3.5em,
	name=cal,
%	at={(-.5\textwidth-5mm,.5\textheight-1cm)},
%	at={(-.48\textwidth-5mm,.48\textheight-4cm)},
at={(-18,-2.5)},
	dates=2010-10-01 to 2010-10-last]
%	if (at most=2010-10-04) [nodes={strike out,draw}]
	if (weekend)
	[white]
	if (day of month=1) {
	\node at (.0em,1.5em) [anchor=base west] {\large\bfseries\sffamily\tikzmonthtext};
	};
	\node[starburst,drop shadow,fill=white,draw] at (13.3,5.1) {ТЕСТЫ};
	\node[signal, draw, text=white, fill=red!65!black, signal to=nowhere,
		signal from=west] at (-15,4) {ТЕСТЫ};
	\node[starburst, fill=yellow, draw=red, line width=2pt,
		drop shadow] at (-15,2.3) {\bf МОДУЛЬ};
	\node[starburst, fill=yellow, draw=blue, line width=2pt,
		drop shadow] at (13.3,3.4) {\bf МОДУЛЬ};
	\node[starburst, fill=yellow, draw=cyan, line width=2pt,
		drop shadow] at (-15.2,-8.8) {\bf МОДУЛЬ};
	\node[starburst, fill=yellow, draw=white, line width=2pt,
		drop shadow] at (13.3,-7.6) {\bf МОДУЛЬ};
	\node[chamfered rectangle, white, fill=red, double=red,
		draw, very thick] at (13.3,-9.4) {\bf ЭКЗАМЕН};
	\node[cloud callout, cloud puffs=15, aspect=2.5,
		cloud puff arc=120, shading=ball,text=white] at (13.3, -6.2)
			{\bf ТЕСТЫ};
%%%---> Right side	calendar
	\calendar (mycalnov) [day list downward,
	month text=\%mt\ \%y0,
%	month text=Ноябрь \quad 2010,
	name=cal,
%	at={(-.5\textwidth-5mm,.5\textheight-1cm)},
%	at={(.48\textwidth-15mm,.48\textheight-4cm)},
at={(16.2,9)},
	dates=2010-11-01 to 2010-11-last]
	if (weekend)
	[white]
	if (day of month=1) {
	\node at (-1.5em,1.5em) [anchor=base east] {\large\bfseries\sffamily\tikzmonthtext};
	};
	\calendar (mycaldec) [day list downward,
	month text=\%mt\ \%y0,
%	month text=Декабрь \quad 2010,
	name=cal,
%	at={(-.5\textwidth-5mm,.5\textheight-1cm)},
%	at={(.48\textwidth-15mm,.48\textheight-4cm)},
at={(16.2,-2.5)},
	dates=2010-12-01 to 2010-12-last]
	if (weekend)
	[white]
	if (day of month=1) {
	\node at (-1.5em,1.5em) [anchor=base east] {\large\bfseries\sffamily\tikzmonthtext};
	};
%%%--->	\onlylecture{}{2010-09-16}
\onlylecture{Тема 3.2. Джерела адміністративного права}{2010-09-16}
\onlylecture{Тема 4.3. Спеціальні адміністративно-правові статуси індивідуальних
суб'єктів адміністративного права}{2010-09-17}
\onlylecture{Тема 4.5. Порівняльний аналіз статусів державних
службовців}{2010-10-01}
\onlylecture{Тема 4.5. Порівняльний аналіз статусів державних службовців}{2010-10-05}
\onlylecture{Тема 4.7. Поняття, ознаки та класифікація органів виконавчої
влади}{2010-10-07}
\onlylecture{Тема 4.9. Місце органів внутрішніх справ у системі органів
виконавчої влади}{2010-10-11}
\onlylecture{Тема 5.2. Співвідношення адміністративно-правових форм і
методів}{2010-10-14}
\onlylecture{Тема 5.4. Місце актів управління в системі правових актів}{2010-10-18}
\onlylecture{Тема 6.2. Правові засади адміністративного примусу у сфері
державного управління}{2010-10-20}
\onlylecture{Тема 6.3. Адміністративний примус у діяльності органів
внутрішніх справ}{2010-10-22}
\onlylectureright{Тема 7.2. Стадії провадження у справах про адміністративні
правопорушення}{2010-11-01}
\onlylectureright{Тема 8.2. Контроль і нагляд у державному управлінні}{2010-11-03}
\onlylectureright{Тема 9.2. Сутність й особливості міжгалузевого управління}{2010-11-05}
\onlylectureright{Тема 11.2. Особливості управління соціально-культурним
комплексом}{2010-11-09}
\onlylectureright{Тема 12.2. Особливості управління в адміністративно-політичній сфері}{2010-11-10}
\onlylectureright{Тема 13.2. Особливості адміністративно-правових відносин за
участю ОВС}{2010-11-12}
\onlylectureright{Тема 14.2. Поняття, ознаки та юридичний склад адміністративного
правопорушення}{2010-11-29}
\onlylectureright{Тема 14.3. Система адміністративних стягнень}{2010-11-30}

\onlylectureright{Тема 15.2. Адміністративна відповідальність за корупційні
правопорушення}{2010-12-01}
\onlylectureright{Тема 16.1. Юридична характеристика адміністративних
правопорушень, що посягають на}{2010-12-02}
\onlylectureright{громадський порядок, громадську безпеку та
встановлений порядок управління}{2010-12-03}
\onlylectureright{Тема 16.2. Адміністративні правопорушення, що посягають на
громадський}{2010-12-06}
\onlylectureright{порядок і громадську безпеку}{2010-12-07}
\onlylectureright{Тема 16.3. Адміністративні правопорушення, що посягають на
встановлений порядок управління}{2010-12-08}
\onlylectureright{Тема 17.2. Адміністративні правопорушення в галузі охорони
труда та здоров'я населення}{2010-12-09}
\onlylectureright{Тема 17.3. Адміністративні правопорушення на транспорті, в
галузі шляхового господарства і зв`язку}{2010-12-10}
\onlylectureright{Тема 17.4. Адміністративні правопорушення в галузі торгівлі,
фінансів і підприємницької діяльності}{2010-12-13}
\onlylectureright{Рекомендації щодо науково-дослідницької роботи}{2010-12-14}
%\onlylecture{}{2010-09-16}
%\onlylecture{}{2010-09-16}
%\onlylecture{}{2010-09-16}
%\onlylecture{}{2010-09-16}
%\onlylecture{}{2010-09-16}
%\onlylecture{}{2010-09-16}
%\onlylecture{}{2010-09-16}
%\onlylecture{}{2010-09-16}
%\onlylecture{}{2010-09-16}
%\onlylecture{}{2010-09-16}
%\onlylecture{}{2010-09-16}
%\onlylecture{}{2010-09-16}
%\onlylecture{}{2010-09-16}
%\onlylecture{}{2010-09-16}
%\onlylecture{}{2010-09-16}
%\onlylecture{}{2010-09-16}
%\onlylecture{}{2010-09-16}
%\onlylecture{}{2010-09-16}
%%%--->	
	\lecture{1}{Тема 1.2. Співвідношення адміністративного права з іншими
	галузями}{above,xshift=-5mm,yshift=5mm}{Computational Problems.north}{
\item Мета заняття;
\item Основні поняття;
\item Навчальні питання;
\item Методичні рекомендації та пояснення;
\item Індивідуальні навчально-дослідницькі завдання;
\item Питання для самоконтролю та самоперевірки;
\item Додаткова література.	
	}{2010-09-08}
	\lecture{2}{Тема 2.2. Принципи й функції державного управління}{above left}
	{Computational Models.west}{
\item Мета заняття;
\item Основні поняття;
\item Навчальні питання;
\item Методичні рекомендації та пояснення;
\item Індивідуальні навчально-дослідницькі завдання;
\item Питання для самоконтролю та самоперевірки;
\item Додаткова література.	
	}{2010-09-15}
%%%---> New version

%%%---> Old version	
%	\begin{pgfonlayer}{background}
%		\clip[xshift=-1cm] (-.5\textwidth,-.5\textheight) rectangle ++(\textwidth,\textheight);
%		\colorlet{upperleft}{green!50!black!25}
%		\colorlet{upperright}{orange!25}
%		\colorlet{lowerleft}{red!25}
%		\colorlet{lowerright}{blue!25}
%		% The large rectangles:
%		\fill [upperleft] (Computational Complexity) rectangle ++(-20,20);
%		\fill [upperright] (Computational Complexity) rectangle ++(20,20);
%		\fill [lowerleft] (Computational Complexity) rectangle ++(-20,-20);
%		\fill [lowerright] (Computational Complexity) rectangle ++(20,-20);
%		% The shadings:
%		\shade [left color=upperleft,right color=upperright]
%		([xshift=-1cm]Computational Complexity) rectangle ++(2,20);
%		\shade [left color=lowerleft,right color=lowerright]
%		([xshift=-1cm]Computational Complexity) rectangle ++(2,-20);
%		\shade [top color=upperleft,bottom color=lowerleft]
%		([yshift=-1cm]Computational Complexity) rectangle ++(-20,2);
%		\shade [top color=upperright,bottom color=lowerright]
%		([yshift=-1cm]Computational Complexity) rectangle ++(20,2);
%	\end{pgfonlayer}
\end{tikzpicture}
%%%-----------------------------------------------------------------------
\begin{textblock}{5}(25,-0.01)
\begin{tikzpicture}[even odd rule,rounded corners=2pt,x=10pt,y=10pt,drop shadow]
\filldraw[fill=yellow!90!black!40,drop shadow] (0,0)   rectangle (1,1)
	[xshift=5pt,yshift=5pt]   (0,0)   rectangle (1,1)
	[rotate=30]   (-1,-1) rectangle (2,2);
\node at (0,1.7) {\textbf{\thepage}};			      
\end{tikzpicture}
\end{textblock}
%%%--- Navigational panel top page
\begin{textblock}{6}(7.58,0.85)
\mbox{%%%--->
\Acrobatmenu{LastPage}{%
\tikz[baseline] \node[rectangle,inner sep=2pt,minimum height=3.1ex,
rounded corners,drop shadow,shadow scale=1,shadow xshift=.8ex,
shadow yshift=-.4ex,opacity=.7,fill=black!50,top color=red!90!black!50,
bottom color=red!80!black!80,draw=red!50!black!50,very thick,text=white,
text opacity=1,minimum width=3cm,font=\bfseries\sffamily] at (0,0) {К концу};
}\Acrobatmenu{GoBack}{%
\tikz[baseline] \node[rectangle,inner sep=2pt,minimum height=3.1ex,
rounded corners,drop shadow,shadow scale=1,shadow xshift=.8ex,
shadow yshift=-.4ex,opacity=.7,fill=black!50,top color=red!90!black!50,
bottom color=red!80!black!80,draw=red!50!black!50,very thick,text=white,
text opacity=1,minimum width=3cm,font=\bfseries\sffamily] at (4,0) {Назад};
}\Acrobatmenu{PrevPage}{%
\tikz[baseline] \node[rectangle,inner sep=2pt,minimum height=3.1ex,
rounded corners,drop shadow,shadow scale=1,shadow xshift=.8ex,
shadow yshift=-.4ex,opacity=.7,fill=black!50,top color=red!90!black!50,
bottom color=red!80!black!80,draw=red!50!black!50,very thick,text=white,
text opacity=1,minimum width=3cm,font=\bfseries\sffamily] at (8,0) {Предыдущий};
}\Acrobatmenu{NextPage}{%
\tikz[baseline] \node[rectangle,inner sep=2pt,minimum height=3.1ex,
rounded corners,drop shadow,shadow scale=1,shadow xshift=.8ex,
shadow yshift=-.4ex,opacity=.7,fill=black!50,top color=red!90!black!50,
bottom color=red!80!black!80,draw=red!50!black!50,very thick,text=white,
text opacity=1,minimum width=3cm,font=\bfseries\sffamily] at (12,0) {Следующий};
}\Acrobatmenu{GoForward}{%
\tikz[baseline] \node[rectangle,inner sep=2pt,minimum height=3.1ex,
rounded corners,drop shadow,shadow scale=1,shadow xshift=.8ex,
shadow yshift=-.4ex,opacity=.7,fill=black!50,top color=red!90!black!50,
bottom color=red!80!black!80,draw=red!50!black!50,very thick,text=white,
text opacity=1,minimum width=3cm,font=\bfseries\sffamily] at (16,0) {Вперед};
}\Acrobatmenu{FirstPage}{%
\tikz[baseline] \node[rectangle,inner sep=2pt,minimum height=3.1ex,
rounded corners,drop shadow,shadow scale=1,shadow xshift=.8ex,
shadow yshift=-.4ex,opacity=.7,fill=black!50,top color=red!90!black!50,
bottom color=red!80!black!80,draw=red!50!black!50,very thick,text=white,
text opacity=1,minimum width=3cm,font=\bfseries\sffamily] at (20,0) {К началу};
}\Acrobatmenu{FullScreen}{%
\tikz[baseline] \node[rectangle,inner sep=2pt,minimum height=3.1ex,
rounded corners,drop shadow,shadow scale=1,shadow xshift=.8ex,
shadow yshift=-.4ex,opacity=.7,fill=black!50,top color=red!90!black!50,
bottom color=red!80!black!80,draw=red!50!black!50,very thick,text=white,
text opacity=1,minimum width=3cm,font=\bfseries\sffamily] at (24,0) {Полный экран};
}\Acrobatmenu{Quit}{%
\tikz[baseline] \node[rectangle,inner sep=2pt,minimum height=3.1ex,
rounded corners,drop shadow,shadow scale=1,shadow xshift=.8ex,
shadow yshift=-.4ex,opacity=.7,fill=black!50,top color=red!90!black!50,
bottom color=red!80!black!80,draw=red!50!black!50,very thick,text=white,
text opacity=1,minimum width=3cm,font=\bfseries\sffamily] at (28,0) {Выход};
}	
}%%%---|
\end{textblock}
\begin{textblock}{7}(13.39,1.305)	%(-0.38,-0.153)
\begin{tikzpicture}[remember picture,overlay]
	\node {\mbox{\includegraphics[scale=0.99]{./laworder_bg_01}}};
\end{tikzpicture}
\end{textblock}	
%%%----------------------------------------------------------------------|

%%%--->\newpage
\AddToTemplate{lawordercover}
\enableTiling
\begin{textblock}{67}(25,-0.01)
%\begin{textblock}{5}(25,0)
\begin{tikzpicture}[even odd rule,rounded corners=2pt,x=10pt,y=10pt,drop shadow]
\filldraw[fill=yellow!90!black!40,drop shadow] (0,0)   rectangle (1,1)
	[xshift=5pt,yshift=5pt]   (0,0)   rectangle (1,1)
	[rotate=30]   (-1,-1) rectangle (2,2);
\node at (0,1.7) {\textbf{\thepage}};			      
\end{tikzpicture}
\end{textblock}
%%%--- Navigational panel top page
\begin{textblock}{68}(7.58,0.85)
\mbox{%%%--->
\tikz[baseline] \node[rectangle,inner sep=2pt,minimum height=3.1ex,
rounded corners,drop shadow,shadow scale=1,shadow xshift=.8ex,
shadow yshift=-.4ex,opacity=.7,fill=black!50,top color=red!90!black!50,
bottom color=red!80!black!80,draw=red!50!black!50,very thick,text=white,
text opacity=1,minimum width=3cm,font=\bfseries\sffamily] at (0,0) {К концу};
\tikz[baseline] \node[rectangle,inner sep=2pt,minimum height=3.1ex,
rounded corners,drop shadow,shadow scale=1,shadow xshift=.8ex,
shadow yshift=-.4ex,opacity=.7,fill=black!50,top color=red!90!black!50,
bottom color=red!80!black!80,draw=red!50!black!50,very thick,text=white,
text opacity=1,minimum width=3cm,font=\bfseries\sffamily] at (4,0) {Назад};
\tikz[baseline] \node[rectangle,inner sep=2pt,minimum height=3.1ex,
rounded corners,drop shadow,shadow scale=1,shadow xshift=.8ex,
shadow yshift=-.4ex,opacity=.7,fill=black!50,top color=red!90!black!50,
bottom color=red!80!black!80,draw=red!50!black!50,very thick,text=white,
text opacity=1,minimum width=3cm,font=\bfseries\sffamily] at (8,0) {Предыдущий};
\tikz[baseline] \node[rectangle,inner sep=2pt,minimum height=3.1ex,
rounded corners,drop shadow,shadow scale=1,shadow xshift=.8ex,
shadow yshift=-.4ex,opacity=.7,fill=black!50,top color=red!90!black!50,
bottom color=red!80!black!80,draw=red!50!black!50,very thick,text=white,
text opacity=1,minimum width=3cm,font=\bfseries\sffamily] at (12,0) {Следующий};
\tikz[baseline] \node[rectangle,inner sep=2pt,minimum height=3.1ex,
rounded corners,drop shadow,shadow scale=1,shadow xshift=.8ex,
shadow yshift=-.4ex,opacity=.7,fill=black!50,top color=red!90!black!50,
bottom color=red!80!black!80,draw=red!50!black!50,very thick,text=white,
text opacity=1,minimum width=3cm,font=\bfseries\sffamily] at (16,0) {Вперед};
\tikz[baseline] \node[rectangle,inner sep=2pt,minimum height=3.1ex,
rounded corners,drop shadow,shadow scale=1,shadow xshift=.8ex,
shadow yshift=-.4ex,opacity=.7,fill=black!50,top color=red!90!black!50,
bottom color=red!80!black!80,draw=red!50!black!50,very thick,text=white,
text opacity=1,minimum width=3cm,font=\bfseries\sffamily] at (20,0) {К началу};
\tikz[baseline] \node[rectangle,inner sep=2pt,minimum height=3.1ex,
rounded corners,drop shadow,shadow scale=1,shadow xshift=.8ex,
shadow yshift=-.4ex,opacity=.7,fill=black!50,top color=red!90!black!50,
bottom color=red!80!black!80,draw=red!50!black!50,very thick,text=white,
text opacity=1,minimum width=3cm,font=\bfseries\sffamily] at (24,0) {Полный экран};
\tikz[baseline] \node[rectangle,inner sep=2pt,minimum height=3.1ex,
rounded corners,drop shadow,shadow scale=1,shadow xshift=.8ex,
shadow yshift=-.4ex,opacity=.7,fill=black!50,top color=red!90!black!50,
bottom color=red!80!black!80,draw=red!50!black!50,very thick,text=white,
text opacity=1,minimum width=3cm,font=\bfseries\sffamily] at (28,0) {Выход};
}%%%---|
\end{textblock}
%%%----------------------------------------------------------------------|
\begin{center}
\Acrobatmenu{LastPage}{%
\tikz[baseline] \node[rectangle,inner sep=2pt,minimum height=3.1ex,
rounded corners,drop shadow,shadow scale=1,shadow xshift=.8ex,
shadow yshift=-.4ex,opacity=.7,fill=black!50,top color=red!90!black!50,
bottom color=red!80!black!80,draw=red!50!black!50,very thick,text=white,
text opacity=1,minimum width=3cm,font=\bfseries\sffamily] at (0,0) {К концу};
}\Acrobatmenu{GoBack}{%
\tikz[baseline] \node[rectangle,inner sep=2pt,minimum height=3.1ex,
rounded corners,drop shadow,shadow scale=1,shadow xshift=.8ex,
shadow yshift=-.4ex,opacity=.7,fill=black!50,top color=red!90!black!50,
bottom color=red!80!black!80,draw=red!50!black!50,very thick,text=white,
text opacity=1,minimum width=3cm,font=\bfseries\sffamily] at (4,0) {Назад};
}\Acrobatmenu{PrevPage}{%
\tikz[baseline] \node[rectangle,inner sep=2pt,minimum height=3.1ex,
rounded corners,drop shadow,shadow scale=1,shadow xshift=.8ex,
shadow yshift=-.4ex,opacity=.7,fill=black!50,top color=red!90!black!50,
bottom color=red!80!black!80,draw=red!50!black!50,very thick,text=white,
text opacity=1,minimum width=3cm,font=\bfseries\sffamily] at (8,0) {Предыдущий};
}\Acrobatmenu{NextPage}{%
\tikz[baseline] \node[rectangle,inner sep=2pt,minimum height=3.1ex,
rounded corners,drop shadow,shadow scale=1,shadow xshift=.8ex,
shadow yshift=-.4ex,opacity=.7,fill=black!50,top color=red!90!black!50,
bottom color=red!80!black!80,draw=red!50!black!50,very thick,text=white,
text opacity=1,minimum width=3cm,font=\bfseries\sffamily] at (12,0) {Следующий};
}\Acrobatmenu{GoForward}{%
\tikz[baseline] \node[rectangle,inner sep=2pt,minimum height=3.1ex,
rounded corners,drop shadow,shadow scale=1,shadow xshift=.8ex,
shadow yshift=-.4ex,opacity=.7,fill=black!50,top color=red!90!black!50,
bottom color=red!80!black!80,draw=red!50!black!50,very thick,text=white,
text opacity=1,minimum width=3cm,font=\bfseries\sffamily] at (16,0) {Вперед};
}\Acrobatmenu{FirstPage}{%
\tikz[baseline] \node[rectangle,inner sep=2pt,minimum height=3.1ex,
rounded corners,drop shadow,shadow scale=1,shadow xshift=.8ex,
shadow yshift=-.4ex,opacity=.7,fill=black!50,top color=red!90!black!50,
bottom color=red!80!black!80,draw=red!50!black!50,very thick,text=white,
text opacity=1,minimum width=3cm,font=\bfseries\sffamily] at (20,0) {К началу};
}\Acrobatmenu{FullScreen}{%
\tikz[baseline] \node[rectangle,inner sep=2pt,minimum height=3.1ex,
rounded corners,drop shadow,shadow scale=1,shadow xshift=.8ex,
shadow yshift=-.4ex,opacity=.7,fill=black!50,top color=red!90!black!50,
bottom color=red!80!black!80,draw=red!50!black!50,very thick,text=white,
text opacity=1,minimum width=3cm,font=\bfseries\sffamily] at (24,0) {Полный экран};
}\Acrobatmenu{Quit}{%
\tikz[baseline] \node[rectangle,inner sep=2pt,minimum height=3.1ex,
rounded corners,drop shadow,shadow scale=1,shadow xshift=.8ex,
shadow yshift=-.4ex,opacity=.7,fill=black!50,top color=red!90!black!50,
bottom color=red!80!black!80,draw=red!50!black!50,very thick,text=white,
text opacity=1,minimum width=3cm,font=\bfseries\sffamily] at (28,0) {Выход};
}	
\end{center}	



ПРИВЕТ !!!!

\newpage
%%%---> New  Page calendar 2010
%%%---> Main baground overlay
\begin{tikzpicture}[remember picture,overlay]
	\draw [line width=1mm,opacity=.25] 
		(current page.center) circle (1cm);

		\begin{pgfonlayer}{background}
			\colorlet{upperleft}{green!50!black!25}
			\colorlet{upperright}{orange!25}
			\colorlet{lowerleft}{red!25}
			\colorlet{lowerright}{blue!25}
			
			% The large rectangles:
			\fill [upperleft] (current page.center) rectangle ++(-18.2,15);
			\fill [upperright] (current page.center) rectangle ++(18.2,15);
			\fill [lowerleft] (current page.center) rectangle ++(-18.2,-15);
			\fill [lowerright] (current page.center) rectangle ++(18.2,-15);

			% The shadings:
			\shade [left color=upperleft,right color=upperright]
			([xshift=-1cm]current page.center) rectangle ++(2,15);
			\shade [left color=lowerleft,right color=lowerright]
			([xshift=-1cm]current page.center) rectangle ++(2,-15);
			\shade [top color=upperleft,bottom color=lowerleft]
			([yshift=-1cm]current page.center) rectangle ++(-18.2,2);
			\shade [top color=upperright,bottom color=lowerright]
			([yshift=-1cm]current page.center) rectangle ++(18.2,2);
		\end{pgfonlayer}
\end{tikzpicture}
\begin{tikzpicture}[remember picture,overlay]
	  \node [rotate=0,scale=2,text opacity=0.2]
	      at (27,1.7) {Капранов~О.~Г.~\copyright~2010~~~Luga\TeX @yahoo.com};
\end{tikzpicture}
\vglue -18pt
\hspace{187pt}
\parbox{350pt}{%
\hypertarget{studyear}{\hyperlink{mymedia}{%
\begin{tikzpicture}
  \colorlet{even}{cyan!60!black}
  \colorlet{odd}{orange!100!black}
  \colorlet{links}{red!70!black}
  \colorlet{back}{yellow!20!white}
  \tikzset{
    box/.style={
      minimum height=15mm,
      inner sep=.7mm,
      outer sep=0mm,
      text width=120mm,
      text centered,
      font=\small\bfseries\sffamily,
      text=#1!50!black,
      draw=#1,
      line width=.25mm,
      top color=#1!5,
      bottom color=#1!40,
      shading angle=0,
      rounded corners=2.3mm,
      drop shadow={fill=#1!40!gray,fill opacity=.8},
      rotate=0,
    },
  }
  \node [box=odd]{{\huge\textbf{Учебный год\quad 2010--2011}}};
\end{tikzpicture}
}}}\\

\begin{center}
\sffamily
\colorlet{winter}{blue}
\colorlet{spring}{green!60!black}
\colorlet{summer}{orange}
\colorlet{fall}{red}
% A counter, since TikZ is not clever enough (yet) to handle
% arbitrary angle systems.
%
% anchor=mid
\newcount\mycount
\begin{tikzpicture}
	[transform shape,
	every day/.style={font=\fontsize{6}{6}\selectfont}]
	\node [circular drop shadow={shadow scale=1.05},minimum size=4.33cm,
	decorate, decoration=zigzag, fill=blue!20,draw,thick,circle] 
   	 (maintime)
   		{\large\bfseries\sffamily Учебный год \the\year};
	\foreach \month/\monthcolor in
	{1/winter,2/winter,3/spring,4/spring,5/spring,6/summer,
	7/summer,8/summer,9/fall,10/fall,11/fall,12/winter}
	{
	% Computer angle:
	\mycount=\month
	\advance\mycount by -1
	\multiply\mycount by 30
	\advance\mycount by -90
	% The actual calendar
	\calendar at (\the\mycount:6.4cm)
	[
	dates=\the\year-\month-01 to \the\year-\month-last,
	]
	if (day of month=1) {\color{\monthcolor}\tikzmonthcode}
	if (Sunday) [red]
	if (all)
	{
	% Again, compute angle
	\mycount=1
	\advance\mycount by -\pgfcalendarcurrentday
	\multiply\mycount by 11
	\advance\mycount by 90
	\pgftransformshift{\pgfpointpolar{\mycount}{1.4cm}}
	};
	}

%%%---> New version
%	\begin{pgfonlayer}{background}
%	\colorlet{upperleft}{green!50!black!25}
%	\colorlet{upperright}{orange!25}
%	\colorlet{lowerleft}{red!25}
%	\colorlet{lowerright}{blue!25}
%
%	\fill [upperleft] (maintime) rectangle ++(-10,8.66);
%	\fill [upperright] (maintime) rectangle ++(10,8.66);
%	\fill [lowerleft] (maintime) rectangle ++(-10,-8.66);
%	\fill [lowerright] (maintime) rectangle ++(10,-8.66);
%
%	\shade [left color=upperleft,right color=upperright]
%	([xshift=-1cm]maintime) rectangle ++(2,8.66);
%	\shade [left color=lowerleft,right color=lowerright]
%	([xshift=-1cm]maintime) rectangle ++(2,-8.66);
%	\shade [top color=upperleft,bottom color=lowerleft]
%	([yshift=-1cm]maintime) rectangle ++(-10,2);
%	\shade [top color=upperright,bottom color=lowerright]
%	([yshift=-1cm]maintime) rectangle ++(10,2);
%\end{pgfonlayer}
%%%---> Old version
%	\begin{pgfonlayer}{background}
%		\clip[xshift=-1cm] (-.5\textwidth,-.5\textheight) rectangle ++(\textwidth,\textheight);
%		\colorlet{upperleft}{green!50!black!25}
%		\colorlet{upperright}{orange!25}
%		\colorlet{lowerleft}{red!25}
%		\colorlet{lowerright}{blue!25}
%		% The large rectangles:
%		\fill [upperleft] (Computational Complexity) rectangle ++(-20,20);
%		\fill [upperright] (Computational Complexity) rectangle ++(20,20);
%		\fill [lowerleft] (Computational Complexity) rectangle ++(-20,-20);
%		\fill [lowerright] (Computational Complexity) rectangle ++(20,-20);
%		% The shadings:
%		\shade [left color=upperleft,right color=upperright]
%		([xshift=-1cm]Computational Complexity) rectangle ++(2,20);
%		\shade [left color=lowerleft,right color=lowerright]
%		([xshift=-1cm]Computational Complexity) rectangle ++(2,-20);
%		\shade [top color=upperleft,bottom color=lowerleft]
%		([yshift=-1cm]Computational Complexity) rectangle ++(-20,2);
%		\shade [top color=upperright,bottom color=lowerright]
%		([yshift=-1cm]Computational Complexity) rectangle ++(20,2);
%	\end{pgfonlayer}
\end{tikzpicture}
\end{center}

\begin{textblock}{8}(0.4,13)
\sffamily\scriptsize
\tikz
\calendar [dates=2010-01-01 to 2010-12-31,
month list,month label left,month yshift=1.25em]
if (Sunday) [white];
\end{textblock}
\begin{textblock}{9}(14,13)
\sffamily\scriptsize
\tikz
\calendar [dates=2011-01-01 to 2011-12-31,
month list,month label left,month yshift=1.25em]
if (Sunday) [white];
\end{textblock}

\begin{textblock}{10}(1,3)
\begin{tikzpicture}
	\calendar
	[
	dates=2010-10-01 to 2010-11-last,
	week list,inner sep=2pt,month label above centered,
	month text=\%mt \%y0
	]
	if (at most=2010-10-29) [nodes={strike out,draw}]
	if (weekend) [black!50,nodes={draw=none}]
	;
\end{tikzpicture}
\end{textblock}
\begin{textblock}{11}(22.2,3)
	\begin{tikzpicture}
		\calendar
		[
		dates=\year-\month-\day+-25 to \year-\month-\day+25,
		week list,inner sep=2pt,month label above centered,
		month text=\textit{\%mt \%y0}
		]
		if (at least=\year-\month-\day) {}
		else [nodes={strike out,draw}]
		if (at most=\year-\month-\day+7)
		[green!50!black]
		if (between=\year-\month-\day+8 and \year-\month-\day+10)
		[red]
		if (Sunday)
		[gray,nodes={draw=none}]
		;
	\end{tikzpicture}
\end{textblock}

\begin{textblock}{12}(4.8,12)
	\begin{tikzpicture}
	\node [draw=blue,thick,fill=blue!50,
		font=\large\bfseries\sffamily] {\bf Учебный год: 2010};
	\end{tikzpicture}
\end{textblock}
\begin{textblock}{13}(18.3,12)
	\begin{tikzpicture}
	\node [draw=red,thick,fill=red!50,
		font=\large\bfseries\sffamily] {\bf Учебный год: 2011};
	\end{tikzpicture}
\end{textblock}
%%%-----------------------------------------------------------------------
\begin{textblock}{14}(25,-0.01)
\begin{tikzpicture}[even odd rule,rounded corners=2pt,x=10pt,y=10pt,drop shadow]
\filldraw[fill=yellow!90!black!40,drop shadow] (0,0)   rectangle (1,1)
	[xshift=5pt,yshift=5pt]   (0,0)   rectangle (1,1)
	[rotate=30]   (-1,-1) rectangle (2,2);
\node at (0,1.7) {\textbf{\thepage}};			      
\end{tikzpicture}
\end{textblock}
%%%--- Navigational panel top page
\begin{textblock}{15}(7.58,0.85)
\mbox{%%%--->
\Acrobatmenu{LastPage}{%
\tikz[baseline] \node[rectangle,inner sep=2pt,minimum height=3.1ex,
rounded corners,drop shadow,shadow scale=1,shadow xshift=.8ex,
shadow yshift=-.4ex,opacity=.7,fill=black!50,top color=red!90!black!50,
bottom color=red!80!black!80,draw=red!50!black!50,very thick,text=white,
text opacity=1,minimum width=3cm,font=\bfseries\sffamily] at (0,0) {К концу};
}\Acrobatmenu{GoBack}{%
\tikz[baseline] \node[rectangle,inner sep=2pt,minimum height=3.1ex,
rounded corners,drop shadow,shadow scale=1,shadow xshift=.8ex,
shadow yshift=-.4ex,opacity=.7,fill=black!50,top color=red!90!black!50,
bottom color=red!80!black!80,draw=red!50!black!50,very thick,text=white,
text opacity=1,minimum width=3cm,font=\bfseries\sffamily] at (4,0) {Назад};
}\Acrobatmenu{PrevPage}{%
\tikz[baseline] \node[rectangle,inner sep=2pt,minimum height=3.1ex,
rounded corners,drop shadow,shadow scale=1,shadow xshift=.8ex,
shadow yshift=-.4ex,opacity=.7,fill=black!50,top color=red!90!black!50,
bottom color=red!80!black!80,draw=red!50!black!50,very thick,text=white,
text opacity=1,minimum width=3cm,font=\bfseries\sffamily] at (8,0) {Предыдущий};
}\Acrobatmenu{NextPage}{%
\tikz[baseline] \node[rectangle,inner sep=2pt,minimum height=3.1ex,
rounded corners,drop shadow,shadow scale=1,shadow xshift=.8ex,
shadow yshift=-.4ex,opacity=.7,fill=black!50,top color=red!90!black!50,
bottom color=red!80!black!80,draw=red!50!black!50,very thick,text=white,
text opacity=1,minimum width=3cm,font=\bfseries\sffamily] at (12,0) {Следующий};
}\Acrobatmenu{GoForward}{%
\tikz[baseline] \node[rectangle,inner sep=2pt,minimum height=3.1ex,
rounded corners,drop shadow,shadow scale=1,shadow xshift=.8ex,
shadow yshift=-.4ex,opacity=.7,fill=black!50,top color=red!90!black!50,
bottom color=red!80!black!80,draw=red!50!black!50,very thick,text=white,
text opacity=1,minimum width=3cm,font=\bfseries\sffamily] at (16,0) {Вперед};
}\Acrobatmenu{FirstPage}{%
\tikz[baseline] \node[rectangle,inner sep=2pt,minimum height=3.1ex,
rounded corners,drop shadow,shadow scale=1,shadow xshift=.8ex,
shadow yshift=-.4ex,opacity=.7,fill=black!50,top color=red!90!black!50,
bottom color=red!80!black!80,draw=red!50!black!50,very thick,text=white,
text opacity=1,minimum width=3cm,font=\bfseries\sffamily] at (20,0) {К началу};
}\Acrobatmenu{FullScreen}{%
\tikz[baseline] \node[rectangle,inner sep=2pt,minimum height=3.1ex,
rounded corners,drop shadow,shadow scale=1,shadow xshift=.8ex,
shadow yshift=-.4ex,opacity=.7,fill=black!50,top color=red!90!black!50,
bottom color=red!80!black!80,draw=red!50!black!50,very thick,text=white,
text opacity=1,minimum width=3cm,font=\bfseries\sffamily] at (24,0) {Полный экран};
}\Acrobatmenu{Quit}{%
\tikz[baseline] \node[rectangle,inner sep=2pt,minimum height=3.1ex,
rounded corners,drop shadow,shadow scale=1,shadow xshift=.8ex,
shadow yshift=-.4ex,opacity=.7,fill=black!50,top color=red!90!black!50,
bottom color=red!80!black!80,draw=red!50!black!50,very thick,text=white,
text opacity=1,minimum width=3cm,font=\bfseries\sffamily] at (28,0) {Выход};
}
}%%%---|
\end{textblock}
\begin{textblock}{16}(13.39,1.305)
\begin{tikzpicture}[remember picture,overlay]
	\node {\mbox{\includegraphics[scale=0.99]{./laworder_bg_01}}};
\end{tikzpicture}
\end{textblock}	
%%%----------------------------------------------------------------------|

\AddToTemplate{lawordercover}
\enableTiling
\newpage
\disableTemplate{covers}
\begin{tikzpicture}[remember picture,overlay]
	  \node [rotate=0,scale=2,text opacity=0.2]
	      at (27,1.7) {Капранов~О.~Г.~\copyright~2010~~~Luga\TeX @yahoo.com};
\end{tikzpicture}
%%%-----------------------------------------------------------------------
\begin{textblock}{17}(25,-0.01)
%\begin{textblock}{5}(25,0)
\begin{tikzpicture}[even odd rule,rounded corners=2pt,x=10pt,y=10pt,drop shadow]
\filldraw[fill=yellow!90!black!40,drop shadow] (0,0)   rectangle (1,1)
	[xshift=5pt,yshift=5pt]   (0,0)   rectangle (1,1)
	[rotate=30]   (-1,-1) rectangle (2,2);
\node at (0,1.7) {\textbf{\thepage}};			      
\end{tikzpicture}
\end{textblock}
%%%--- Navigational panel top page
\begin{textblock}{18}(7.58,0.85)
\mbox{%%%--->
\Acrobatmenu{LastPage}{%
\tikz[baseline] \node[rectangle,inner sep=2pt,minimum height=3.1ex,
rounded corners,drop shadow,shadow scale=1,shadow xshift=.8ex,
shadow yshift=-.4ex,opacity=.7,fill=black!50,top color=red!90!black!50,
bottom color=red!80!black!80,draw=red!50!black!50,very thick,text=white,
text opacity=1,minimum width=3cm,font=\bfseries\sffamily] at (0,0) {К концу};
}\Acrobatmenu{GoBack}{%
\tikz[baseline] \node[rectangle,inner sep=2pt,minimum height=3.1ex,
rounded corners,drop shadow,shadow scale=1,shadow xshift=.8ex,
shadow yshift=-.4ex,opacity=.7,fill=black!50,top color=red!90!black!50,
bottom color=red!80!black!80,draw=red!50!black!50,very thick,text=white,
text opacity=1,minimum width=3cm,font=\bfseries\sffamily] at (4,0) {Назад};
}\Acrobatmenu{PrevPage}{%
\tikz[baseline] \node[rectangle,inner sep=2pt,minimum height=3.1ex,
rounded corners,drop shadow,shadow scale=1,shadow xshift=.8ex,
shadow yshift=-.4ex,opacity=.7,fill=black!50,top color=red!90!black!50,
bottom color=red!80!black!80,draw=red!50!black!50,very thick,text=white,
text opacity=1,minimum width=3cm,font=\bfseries\sffamily] at (8,0) {Предыдущий};
}\Acrobatmenu{NextPage}{%
\tikz[baseline] \node[rectangle,inner sep=2pt,minimum height=3.1ex,
rounded corners,drop shadow,shadow scale=1,shadow xshift=.8ex,
shadow yshift=-.4ex,opacity=.7,fill=black!50,top color=red!90!black!50,
bottom color=red!80!black!80,draw=red!50!black!50,very thick,text=white,
text opacity=1,minimum width=3cm,font=\bfseries\sffamily] at (12,0) {Следующий};
}\Acrobatmenu{GoForward}{%
\tikz[baseline] \node[rectangle,inner sep=2pt,minimum height=3.1ex,
rounded corners,drop shadow,shadow scale=1,shadow xshift=.8ex,
shadow yshift=-.4ex,opacity=.7,fill=black!50,top color=red!90!black!50,
bottom color=red!80!black!80,draw=red!50!black!50,very thick,text=white,
text opacity=1,minimum width=3cm,font=\bfseries\sffamily] at (16,0) {Вперед};
}\Acrobatmenu{FirstPage}{%
\tikz[baseline] \node[rectangle,inner sep=2pt,minimum height=3.1ex,
rounded corners,drop shadow,shadow scale=1,shadow xshift=.8ex,
shadow yshift=-.4ex,opacity=.7,fill=black!50,top color=red!90!black!50,
bottom color=red!80!black!80,draw=red!50!black!50,very thick,text=white,
text opacity=1,minimum width=3cm,font=\bfseries\sffamily] at (20,0) {К началу};
}\Acrobatmenu{FullScreen}{%
\tikz[baseline] \node[rectangle,inner sep=2pt,minimum height=3.1ex,
rounded corners,drop shadow,shadow scale=1,shadow xshift=.8ex,
shadow yshift=-.4ex,opacity=.7,fill=black!50,top color=red!90!black!50,
bottom color=red!80!black!80,draw=red!50!black!50,very thick,text=white,
text opacity=1,minimum width=3cm,font=\bfseries\sffamily] at (24,0) {Полный экран};
}\Acrobatmenu{Quit}{%
\tikz[baseline] \node[rectangle,inner sep=2pt,minimum height=3.1ex,
rounded corners,drop shadow,shadow scale=1,shadow xshift=.8ex,
shadow yshift=-.4ex,opacity=.7,fill=black!50,top color=red!90!black!50,
bottom color=red!80!black!80,draw=red!50!black!50,very thick,text=white,
text opacity=1,minimum width=3cm,font=\bfseries\sffamily] at (28,0) {Выход};
}
}%%%---|
\end{textblock}
%%%----------------------------------------------------------------------|
\vglue -18pt
\hspace{187pt}
\parbox{350pt}{%
\hypertarget{mymedia}{\hyperlink{preface}{%
\begin{tikzpicture}
  \colorlet{even}{cyan!60!black}
  \colorlet{odd}{orange!100!black}
  \colorlet{links}{red!70!black}
  \colorlet{back}{yellow!20!white}
  \tikzset{
    box/.style={
      minimum height=15mm,
      inner sep=.7mm,
      outer sep=0mm,
      text width=120mm,
      text centered,
      font=\small\bfseries\sffamily,
      text=#1!50!black,
      draw=#1,
      line width=.25mm,
      top color=#1!5,
      bottom color=#1!40,
      shading angle=0,
      rounded corners=2.3mm,
      drop shadow={fill=#1!40!gray,fill opacity=.8},
      rotate=0,
    },
  }
  \node [box=even]{{\huge\textbf{Мультимедийные лекции}}};
\end{tikzpicture}
}}}\\
\begin{flushleft}
\parbox{520pt}{%
%%%---> Old version
%	\mbox{\includegraphics[scale=.99]{video_02.png}}\par
%%%----------------------------------------------------------------------|
\begin{tikzfadingfrompicture}[name=myvideoa]
	\node [text=transparent!1]
	{\fontsize{25}{25}\bfseries\sffamily\selectfont Видеол\emph{е}кция};
\end{tikzfadingfrompicture}
\begin{tikzfadingfrompicture}[name=myvideob]
	\node [text=transparent!1]
	{\fontsize{25}{25}\normalsize\bfseries\sffamily\selectfont полный курс};
\end{tikzfadingfrompicture}
\begin{tikzfadingfrompicture}[name=myvideoc]
	\node [text=transparent!1]
	{\fontsize{25}{25}\large\bfseries\sffamily\selectfont Административное право};
\end{tikzfadingfrompicture}
\begin{tikzpicture}
	\node {%
	\mbox{\includegraphics[scale=.99]{video_02.png}}	
	};
	\node[rectangle, inner sep=2pt, very thick, draw=black,
	inner color=transparent!80, outer color=transparent!30,
	draw opacity=0.8, fill opacity=0.2, line width=1.6pt,
	minimum width=16.48cm, minimum height=51.4ex
	] at (-0.32,0.26) {\mbox{}};
%	\shade[path fading=myvideoa,fit fading=false,
%	left color=blue!80!black,right color=black]
%	(-15.8,-1) rectangle (3.7,1);
%	\shade[path fading=myvideob,fit fading=false,
%	left color=blue!80!black,right color=black]
%	(-2.8,-1) rectangle (2.7,1);
%	\shade[path fading=myvideoc,fit fading=false,
%	left color=red!80!black,right color=black]
%	(-2.8,-1) rectangle (2.7,1);
	\node [%
	minimum height=65mm,
	inner sep=.7mm,
	outer sep=0mm,
	text width=65mm,
	text centered,
	font=\huge\bfseries\sffamily,
	text=white!100!black,
	line width=.25mm,
	rounded corners=3.3mm,
	%rounded corners=2.3mm,
	shading angle=45,
%	draw=orange!100!black,
%	top color=orange!100!black!5,
%	bottom color=orange!100!black!40,
%	drop shadow={fill=orange!100!black!40!gray,fill opacity=.8},
	draw=gray,
	inner color=transparent!80,
	outer color=transparent!30,
	draw opacity=0.8,
	fill opacity=0.2,
%	opacity=1,
	text opacity=1,
%	fill=blue!65!black
	] at (-4.7,0.5) {Видеолекция\\[-13pt]
	{\small 02:45:59}\\
	{\normalsize <<Административное право>>}\\[-13pt]
	{\small полный курс}\\
	{\normalsize Горбачев Борис Иванович}\\[-13pt]
	{\small Доцент, кандидат юридических наук}
	};
\end{tikzpicture}\par
\begin{center}
\parbox{420pt}{%
	Видеолекция в формате \textbf{AVI} по предмету
	<<\textcolor{magenta}{\bf Административное право}>>,
	лекция была записана в течении семестра с сентября по
	декабрь 2010 года, и смонтирована в единый файл.
	Для просмотра, нажмите мышкой по центру окна Видеолекция
	полный курс.}
\vglue 25pt
\parbox{420pt}{%
Аудиолекция по предмету <<\textcolor{magenta}{\bf Административное право}>>
с комментариями, звуковое сопровождение Капланов Олег Георгиевич,
в формате \textbf{MP3}, продолжительность \textbf{8} часов \textbf{43}
минуты. Для прослушивания лекции подвидите мышку к ссылке
\textbf{Аудиолекция}, и нажмите на нее.
Аудиолекция воспроизводится автоматически в фоновом режиме, вы можете
одновременно прослушивать лекцию и работать с интерактивными документами.
}
\vglue 25pt
\mbox{%
\begin{tikzpicture}[rounded corners,ultra thick]
	\node [rectangle,very thick,
		bottom color=blue!80!black!30,
		top color=white,
		draw=blue!50!black!50,
		minimum width=9cm,
		drop shadow,
		] at (0,0) {%
		\textcolor{black}{\textbf{Аудиолекция в формате MP3}}};
	\pattern [path fading=north] ;
	\pattern [top color=transparent!50,bottom color=transparent!50,color=transparent!20] ;
\end{tikzpicture}
}
\vglue 25pt
\parbox{420pt}{%
Электронный навчально-методичний посібник по предмету
<<\textcolor{magenta}{\bf Административное право}>>,
с автоматическим перелистованием страниц. Подвидите мышку к кнопке
<<\textbf{Интерактивный навчально-методичний посібник}>>,
и документ будет отображатся с задержкой для каждой страницы
\textbf{30 секунд}.
}
\vglue 25pt
\mbox{}\hspace{5pt}\mbox{1}{43}\par
\begin{tikzpicture}[rounded corners,ultra thick]
	\node [rectangle,very thick,
		bottom color=blue!80!black!30,
		top color=white,
		draw=blue!50!black!50,
		minimum width=9cm,
		drop shadow,
		] at (0,0) {%
		\textcolor{black}{\textbf{Интерактивный навчально-методичний посібник}}};
	\pattern [path fading=north] ;
	\pattern [top color=transparent!50,bottom color=transparent!50,color=transparent!20] ;
\end{tikzpicture}
}{1}{10}
}
\end{center}
}
\hspace{48pt}
\parbox{410pt}{%
\begin{tikzpicture}
\node (tbl) {
\begin{tabularx}{400pt}{rccccc}
\arrayrulecolor{purple}
 & \textcolor{white}{\bf Автор} & \textcolor{white}{\bf Название} & \textcolor{white}{\bf WMV}&
	\textcolor{white}{\bf MP3}&\textcolor{white}{\bf PDF}\\[0.5ex]
 & & & & &\\
	& \parbox{65pt}{\includegraphics[scale=.69]{faces_01.jpg}}
	& \parbox{175pt}{Арлинский Юрий Моисеевич\par Зав. кафедрою, професор,\par
		доктор фіз.-мат. наук\par
			Функціональний аналіз} & 
	\mbox{\includegraphics[scale=.69]{wmv.png}} &
	\mbox{\includegraphics[scale=.69]{mp3.png}} &
	\mbox{\includegraphics[scale=.69]{pdf.png}}\\
  &  & & & &\\
	& \parbox{65pt}{\includegraphics[scale=.69]{faces_02.jpg}}
	& \parbox{175pt}{Балицька Татьяна Юревна\par Доцент, кандидат технічних наук\par
	Математичний аналіз} & 
	\mbox{\includegraphics[scale=.69]{wmv.png}} &
	\mbox{\includegraphics[scale=.69]{mp3.png}} &
	\mbox{\includegraphics[scale=.69]{pdf.png}}\\
  & & & & &\\
	& \parbox{65pt}{\includegraphics[scale=.69]{faces_03.jpg}}
	& \parbox{175pt}{Кочевський А.О.\par Доцент, кандидат технічних наук\par
		Чисельні методи} & 
	\mbox{\includegraphics[scale=.69]{wmv.png}} &
	\mbox{\includegraphics[scale=.69]{mp3.png}} &
	\mbox{\includegraphics[scale=.69]{pdf.png}}\\
  & & & & &\\
	& \parbox{65pt}{\includegraphics[scale=.69]{faces_04.jpg}}
	& \parbox{175pt}{Щестюк Н.Ю.\par Доцент, канд. фіз.-мат. наук\par
		Теорія функцій комплексної змінної} & 
	\mbox{\includegraphics[scale=.69]{wmv.png}} &
	\mbox{\includegraphics[scale=.69]{mp3.png}} &
	\mbox{\includegraphics[scale=.69]{pdf.png}}\\
  & & & & &\\
	& \parbox{65pt}{\includegraphics[scale=.69]{faces_05.jpg}}
	& \parbox{175pt}{Деордиця Ю.С.\par Професор, кандидат технічних наук\par
		Економiко-математичне моделювання} & 
	\mbox{\includegraphics[scale=.69]{wmv.png}} &
	\mbox{\includegraphics[scale=.69]{mp3.png}} &
	\mbox{\includegraphics[scale=.69]{pdf.png}}\\
  & & & & &\\
	& \parbox{65pt}{\includegraphics[scale=.69]{faces_06.jpg}}
	& \parbox{175pt}{Чабанова Т.Д.\par Доцент, кандидат педагогічних наук\par
		Методика викладання математики} & 
	\mbox{\includegraphics[scale=.69]{wmv.png}} &
	\mbox{\includegraphics[scale=.69]{mp3.png}} &
	\mbox{\includegraphics[scale=.69]{pdf.png}}\\
  & & & & &\\
\end{tabularx}};

\begin{pgfonlayer}{background}
\draw[rounded corners,top color=red,bottom color=black,draw=white]
	($(tbl.north west)+(0.14,0)$) rectangle ($(tbl.north east)-(0.13,0.9)$);
\draw[rounded corners,top color=white,bottom color=black,
	middle color=red,draw=blue!20] ($(tbl.south west) +(0.12,0.5)$)
		rectangle ($(tbl.south east)-(0.12,0)$);
\draw[top color=blue!1,bottom color=blue!20,draw=white]
	($(tbl.north east)-(0.13,0.6)$) rectangle ($(tbl.south west)+(0.13,0.2)$);
\end{pgfonlayer}
\end{tikzpicture}
}
\end{flushleft}
%%%----------------------------------------------------------------------|


\newpage
\begin{tikzpicture}[remember picture,overlay]
	  \node [rotate=0,scale=2,text opacity=0.2]
	      at (27,1.7) {Капранов~О.~Г.~\copyright~2010~~~Luga\TeX @yahoo.com};
\end{tikzpicture}
\vglue -18pt
\hspace{187pt}
\parbox{350pt}{%
\hypertarget{preface}{\hyperlink{forms}{\mbox{%
\begin{tikzpicture}
  \colorlet{even}{cyan!60!black}
  \colorlet{odd}{orange!100!black}
  \colorlet{links}{red!70!black}
  \colorlet{back}{yellow!20!white}
  \tikzset{
    box/.style={
      minimum height=15mm,
      inner sep=.7mm,
      outer sep=0mm,
      text width=120mm,
      text centered,
      font=\small\bfseries\sffamily,
      text=#1!50!black,
      draw=#1,
      line width=.25mm,
      top color=#1!5,
      bottom color=#1!40,
      shading angle=0,
      rounded corners=2.3mm,
      drop shadow={fill=#1!40!gray,fill opacity=.8},
      rotate=0,
    },
  }
  \node [box=even]{{%
  	\huge\textbf{Загальні положення}}};
\end{tikzpicture}
}}}}
\vglue 15pt
\parbox{800pt}{%
\pdffreetextcomment[type=freetext,opacity=0.5,justification=right,height=0.5cm,
width=4.8cm,voffset=0.7cm,hoffset=1.2cm,color=red!30!yellow]{%
Адміністративне право}
%{0.045 0.278 0.643}
\pdfmarkupcomment[color={0.54 0.117 0.136},opacity=1.0,markup=Squiggly]{
є однією із профілюючих дисциплін у навчальних закладах
системи МВС України}{}. Вона передбачає вивчення комплексу\\ теоретичних положень,
правових інститутів, діяльності державних й інших органів управління, а також
широкого діапазо-\\ну суспільних відносин, що складаються в сфері виконавчої влади
з урахуванням адміністративно-правового регулювання.

\pdffreetextcomment[color={red!30!yellow},height=2.9cm,width=9cm,opacity=0.8,
voffset=15pt,hoffset=580pt,font=Georgia,fontsize=9pt,fontcolor=black,
justification=left,linewidth=2bp,bse=cloudy,bsei=1.3,type=callout,
line={100 100},icolor=blue]{%
Адміністративне право базується на положеннях Конституції України, Законів
України, указів Президента України, постанов Кабінету Міністрів України та
інших нормативних актів, а також правових положеннях суміжних юридичних
дисциплін. Воно тісно пов'язане з конституційним, цивільним, фінансовим,
трудовим, кримінальним правом й іншими галузями права.}


Разом з тим, дана дисципліна має засадниче методологічне значення для курсів
\tooltipanim{%
\tikz[baseline] \node[rectangle,inner sep=2pt,minimum height=3.1ex,rounded corners,
drop shadow,shadow scale=1,shadow xshift=.8ex,shadow yshift=-.4ex,opacity=.7,
fill=black!50,top color=blue!90!black!50,bottom color=blue!80!black!80,
draw=blue!50!black!50,very thick,text=white,text opacity=1]{%
Організація діяльності ОВС};}{21}{21},
\tooltipanim{%
\tikz[baseline] \node[rectangle,inner sep=2pt,minimum height=3.1ex,rounded corners,
drop shadow,shadow scale=1,shadow xshift=.8ex,shadow yshift=-.4ex,opacity=.7,
fill=black!50,top color=blue!90!black!50,bottom color=blue!80!black!80,
draw=blue!50!black!50,very thick,text=white,text opacity=1]{%
Адміністративна діяльність ОВС};}{22}{22},
\tooltipanim{%
\tikz[baseline] \node[rectangle,inner sep=2pt,minimum height=3.1ex,rounded corners,
drop shadow,shadow scale=1,shadow xshift=.8ex,shadow yshift=-.4ex,opacity=.7,
fill=black!50,top color=blue!90!black!50,bottom color=blue!80!black!80,
draw=blue!50!black!50,very thick,text=white,text opacity=1]{%
Адміністративний процес};}{23}{23},
\tooltipanim{%
\tikz[baseline] \node[rectangle,inner sep=2pt,minimum height=3.1ex,rounded corners,
drop shadow,shadow scale=1,shadow xshift=.8ex,shadow yshift=-.4ex,opacity=.7,
fill=black!50,top color=blue!90!black!50,bottom color=blue!80!black!80,
draw=blue!50!black!50,very thick,text=white,text opacity=1]{%
Основи управління в ОВС};}{24}{24},
що вивчаються у вищих
навчальних закладах МВС України. Зміст і структура навчального курсу
<<\textcolor{magenta}{\textbf{Адміністративне право}}>> {\color{magenta}
\underline{подаються відповідно до положень основних програмних
документів \textcolor{red}{Верховної Ради України},\textcolor{red}{Президента
України}, \textcolor{red}{Кабінету Міністрів України}}}, МВС України з урахуванням
розвитку й удосконалення організації та
функціонування системи органів внутрішніх справ.

Для глибшого висвітлення фундаментальних засад науки адміністративного права
дана дисципліна викладається з використанням результатів найновіших теоретичних
досліджень українських та зарубіжних учених, а також положень Концепції
адміністративної реформи в Україні та проекту Концепції реформи
адміністративного права. В ході вивчення курсу роз'яснюються актуальні
дискусійні питання загальної адміністративно-правової проблематики:

\begin{itemize}
\item[] \tikz[baseline] \node[ball color=magenta,circle,text=black] {};\quad
		\tikz[baseline] \node {%
	\pdfmarkupcomment[author={Сноска},color=green,opacity=1.0,markup=Highlight]{
	\textbf{загальноєвропейський контекст становлення адміністративного права
	та формування системи національних джерел галузі};}{загальної адміністративно-правової
	проблематики}};
\item[] \tikz[baseline] \node[ball color=magenta,circle,text=black] {};\quad
		\tikz[baseline] \node {%
	\pdfmarkupcomment[author={Сноска},color=green,opacity=1.0,markup=Highlight]{
	\textbf{розкриття принципово нової суспільної ролі виконавчої влади, що полягає у
	служінні інтересам людини, з метою наближення}}{загальної адміністративно-правової
	проблематики}};\\\mbox{}\hspace{15pt}
	\tikz[baseline] \node {%
	\pdfmarkupcomment[author={Сноска},color=green,opacity=1.0,markup=Highlight]{
	\textbf{українського адміністративного права до європейських стандартів};}{загальної
	адміністративно-правової проблематики}};
\item[] \tikz[baseline] \node[ball color=magenta,circle,text=black] {};\quad
		\tikz[baseline] \node {%
	\pdfmarkupcomment[author={Сноска},color=green,opacity=1.0,markup=Highlight]{
	\textbf{концептуальні засади побудови і розвитку системи органів виконавчої влади
	відповідно до вимог здійснюваної в Україні адміністративної реформи};}{
	загальної адміністративно-правової проблематики}};
\item[] \tikz[baseline] \node[ball color=magenta,circle,text=black] {};\quad
		\tikz[baseline] \node {%
	\pdfmarkupcomment[author={Сноска},color=green,opacity=1.0,markup=Highlight]{
	\textbf{розвиток дійових засобів правового захисту громадян у сфері виконавчої влади};}{
	загальної адміністративно-правової проблематики}};
\item[] \tikz[baseline] \node[ball color=magenta,circle,text=black] {};\quad
		\tikz[baseline] \node {%
	\pdfmarkupcomment[author={Сноска},color=green,opacity=1.0,markup=Highlight]{
	\textbf{особливості адміністративної відповідальності юридичних осіб,
	та інші питання}.}{загальної адміністративно-правової проблематики}};
\end{itemize}


Тематичним планом з курсу адміністративного права передбачені лекції,
семінарські та практичні заняття,\\ важливою складовою частиною навчального
процесу є самостійна робота курсантів і студентів.

\tikz[remember picture, overlay]
	\node[rectangle callout,cloud puffs=15,aspect=2.5,cloud puff arc=120,
		shading=ball,text=white,overlay,rounded corners,thick,
		 callout pointer arc=90,callout pointer width=1.3cm,
		 callout absolute pointer={(14.5,-0.4)},fill=red]
	 at (21.3,2.2) {\textbf{Семінарські і практичні заняття проводяться з метою}};

\begin{itemize}
\item[] \tikz[baseline] \node[ball color=magenta,circle,text=black] {};\quad
		\tikz[baseline] \node {%
	\pdfmarkupcomment[author={Сноска},color=green,opacity=1.0,markup=Highlight]{
\textbf{вивчення чинного адміністративного законодавства, з'ясування останніх змін і
доповнень, внесених до нього, застосовування його норм};}{Семінарські і практичні
заняття}};\\\mbox{\hspace{15pt}}
\tikz[baseline] \node {%
\pdfmarkupcomment[author={Сноска},color=green,opacity=1.0,markup=Highlight]{
\textbf{у конкретних практичних ситуаціях під час виконання службових
обов'язків};}{Семінарські і практичні заняття}};
\item[] \tikz[baseline] \node[ball color=magenta,circle,text=black] {};\quad
		\tikz[baseline] \node {%
	\pdfmarkupcomment[author={Сноска},color=green,opacity=1.0,markup=Highlight]{
\textbf{поглиблення знань з найбільш важливих теоретичних положень курсу, кращого
засвоєння змін, що відбуваються};}{Семінарські і практичні
заняття}};\\\mbox{\hspace{15pt}}
\tikz[baseline] \node {%
\pdfmarkupcomment[author={Сноска},color=green,opacity=1.0,markup=Highlight]{
\textbf{у сфері державної виконавчої влади на сучасному етапі};}{Семінарські
і практичні заняття}};
\item[] \tikz[baseline] \node[ball color=magenta,circle,text=black] {};\quad
		\tikz[baseline] \node {%
	\pdfmarkupcomment[author={Сноска},color=green,opacity=1.0,markup=Highlight]{
\textbf{прищеплення комплексу практичних вмінь і навичок, необхідних для формування
високої професійної майстерності}.}{Семінарські і практичні заняття}};
\end{itemize}		

Під час проведення практичних занять курсанти та студенти, керуючись
положеннями адміністративного законодавства, вирішують запропоновані задачі,
аналізують конкретні ситуації та дають їм правову оцінку.

\pdfmarkupcomment[author={Основна мета},color={0.15 0.25 0.15},opacity=0.4,
markup=Highlight,icolor=white,fontcolor=red]{%
Основною метою самостійної роботи є глибоке всебічне оволодіння матеріалом по
тій або іншій темі, розвиток навичок роботи з навчальною літературою 
нормативними актами}{},
\pdfmarkupcomment[author={Основна мета},color={0.15 0.25 0.15},opacity=0.4,
markup=Highlight]{%
уміння узагальнювати практичні результати, готувати
довідки, виступи, реферати й ін. При виникненні питань необхідно звертатися за
індивідуальними консультаціями до}{}
\pdfmarkupcomment[author={Основна мета},color={0.15 0.25 0.15},opacity=0.4,
markup=Highlight]{%
викладачів кафедри}{}.

У даному навчально-методичному посібнику, крім нормативних і законодавчих актів
України, основної навчальної літератури до всіх тем, наведених у спеціальному
розділі, курсантам і студентам по кожному заняттю пропонується перелік
нормативних актів, додаткової навчальної літератури й наукових статей
періодичних видань з проблематики тем, що вивчаються.

\textcolor{cyan}{\textbf{%
Підсумковим контролем якості вивчення дисципліни є складання іспиту з
урахуванням рейтингу-контролю й участі в підсумковій науково-теоретичній
конференції курсантів і студентів}}. Тематика повідомлень і доповідей на
конференцію запропонована в окремому розділі даних рекомендацій.
}
\begin{textblock}{19}(22.1,3.3)
\mbox{\includegraphics[scale=1.1]{judge.png}}
\end{textblock}
%%%-----------------------------------------------------------------------
\begin{textblock}{20}(25,-0.01)
%\begin{textblock}{5}(25,0)
\begin{tikzpicture}[even odd rule,rounded corners=2pt,x=10pt,y=10pt,drop shadow]
\filldraw[fill=yellow!90!black!40,drop shadow] (0,0)   rectangle (1,1)
	[xshift=5pt,yshift=5pt]   (0,0)   rectangle (1,1)
	[rotate=30]   (-1,-1) rectangle (2,2);
\node at (0,1.7) {\textbf{\thepage}};			      
\end{tikzpicture}
\end{textblock}
%%%--- Navigational panel top page
\begin{textblock}{21}(7.58,0.85)
\mbox{%%%--->
\Acrobatmenu{LastPage}{%
\tikz[baseline] \node[rectangle,inner sep=2pt,minimum height=3.1ex,
rounded corners,drop shadow,shadow scale=1,shadow xshift=.8ex,
shadow yshift=-.4ex,opacity=.7,fill=black!50,top color=red!90!black!50,
bottom color=red!80!black!80,draw=red!50!black!50,very thick,text=white,
text opacity=1,minimum width=3cm,font=\bfseries\sffamily] at (0,0) {К концу};
}\Acrobatmenu{GoBack}{%
\tikz[baseline] \node[rectangle,inner sep=2pt,minimum height=3.1ex,
rounded corners,drop shadow,shadow scale=1,shadow xshift=.8ex,
shadow yshift=-.4ex,opacity=.7,fill=black!50,top color=red!90!black!50,
bottom color=red!80!black!80,draw=red!50!black!50,very thick,text=white,
text opacity=1,minimum width=3cm,font=\bfseries\sffamily] at (4,0) {Назад};
}\Acrobatmenu{PrevPage}{%
\tikz[baseline] \node[rectangle,inner sep=2pt,minimum height=3.1ex,
rounded corners,drop shadow,shadow scale=1,shadow xshift=.8ex,
shadow yshift=-.4ex,opacity=.7,fill=black!50,top color=red!90!black!50,
bottom color=red!80!black!80,draw=red!50!black!50,very thick,text=white,
text opacity=1,minimum width=3cm,font=\bfseries\sffamily] at (8,0) {Предыдущий};
}\Acrobatmenu{NextPage}{%
\tikz[baseline] \node[rectangle,inner sep=2pt,minimum height=3.1ex,
rounded corners,drop shadow,shadow scale=1,shadow xshift=.8ex,
shadow yshift=-.4ex,opacity=.7,fill=black!50,top color=red!90!black!50,
bottom color=red!80!black!80,draw=red!50!black!50,very thick,text=white,
text opacity=1,minimum width=3cm,font=\bfseries\sffamily] at (12,0) {Следующий};
}\Acrobatmenu{GoForward}{%
\tikz[baseline] \node[rectangle,inner sep=2pt,minimum height=3.1ex,
rounded corners,drop shadow,shadow scale=1,shadow xshift=.8ex,
shadow yshift=-.4ex,opacity=.7,fill=black!50,top color=red!90!black!50,
bottom color=red!80!black!80,draw=red!50!black!50,very thick,text=white,
text opacity=1,minimum width=3cm,font=\bfseries\sffamily] at (16,0) {Вперед};
}\Acrobatmenu{FirstPage}{%
\tikz[baseline] \node[rectangle,inner sep=2pt,minimum height=3.1ex,
rounded corners,drop shadow,shadow scale=1,shadow xshift=.8ex,
shadow yshift=-.4ex,opacity=.7,fill=black!50,top color=red!90!black!50,
bottom color=red!80!black!80,draw=red!50!black!50,very thick,text=white,
text opacity=1,minimum width=3cm,font=\bfseries\sffamily] at (20,0) {К началу};
}\Acrobatmenu{FullScreen}{%
\tikz[baseline] \node[rectangle,inner sep=2pt,minimum height=3.1ex,
rounded corners,drop shadow,shadow scale=1,shadow xshift=.8ex,
shadow yshift=-.4ex,opacity=.7,fill=black!50,top color=red!90!black!50,
bottom color=red!80!black!80,draw=red!50!black!50,very thick,text=white,
text opacity=1,minimum width=3cm,font=\bfseries\sffamily] at (24,0) {Полный экран};
}\Acrobatmenu{Quit}{%
\tikz[baseline] \node[rectangle,inner sep=2pt,minimum height=3.1ex,
rounded corners,drop shadow,shadow scale=1,shadow xshift=.8ex,
shadow yshift=-.4ex,opacity=.7,fill=black!50,top color=red!90!black!50,
bottom color=red!80!black!80,draw=red!50!black!50,very thick,text=white,
text opacity=1,minimum width=3cm,font=\bfseries\sffamily] at (28,0) {Выход};
}
}%%%---|
\end{textblock}
%%%----------------------------------------------------------------------|

\newpage
\disableTemplate{lawordercover}
\disableTiling
\begin{tikzpicture}[remember picture,overlay]
\begin{pgfonlayer}{background}
%%%---> original size
%\clip (-1.5,-5) rectangle ++(4,10);
%\clip (-6.3,-7.8) rectangle ++(14.3,15.2);
\colorlet{upperleft}{green!50!black!25}
\colorlet{upperright}{orange!25}
\colorlet{lowerleft}{red!25}
\colorlet{lowerright}{blue!25}

% The large rectangles:
\fill [upperleft] (27,-9) rectangle ++(-28.2,14);
\fill [upperright] (27,-9) rectangle ++(8,14);
\fill [lowerleft] (27,-9) rectangle ++(-28.2,-14);
\fill [lowerright] (27,-9) rectangle ++(8,-14);

% The shadings:
\shade [left color=upperleft,right color=upperright]
([xshift=-1cm]27,-9) rectangle ++(2,14);
\shade [left color=lowerleft,right color=lowerright]
([xshift=-1cm]27,-9) rectangle ++(2,-14);
\shade [top color=upperleft,bottom color=lowerleft]
([yshift=-1cm]27,-9) rectangle ++(-28.2,2);
\shade [top color=upperright,bottom color=lowerright]
([yshift=-1cm]27,-9) rectangle ++(8,2);
\end{pgfonlayer}
\end{tikzpicture}
\begin{tikzpicture}[remember picture,overlay]
	  \node [rotate=0,scale=2,text opacity=0.2]
	      at (27,1.7) {Капранов~О.~Г.~\copyright~2010~~~Luga\TeX @yahoo.com};
\end{tikzpicture}
\vglue -18pt
\hspace{187pt}
\parbox{350pt}{%
\hypertarget{forms}{\hyperlink{admintbl1}{%
\begin{tikzpicture}
  \colorlet{even}{cyan!60!black}
  \colorlet{odd}{orange!100!black}
  \colorlet{links}{red!70!black}
  \colorlet{back}{yellow!20!white}
  \tikzset{
    box/.style={
      minimum height=15mm,
      inner sep=.7mm,
      outer sep=0mm,
      text width=120mm,
      text centered,
      font=\small\bfseries\sffamily,
      text=#1!50!black,
      draw=#1,
      line width=.25mm,
      top color=#1!5,
      bottom color=#1!40,
      shading angle=0,
      rounded corners=2.3mm,
      drop shadow={fill=#1!40!gray,fill opacity=.8},
      rotate=0,
    },
  }
  \node [box=even]{{\huge\textbf{Форми та критерії\\
  	оцінювання знань студентів}}};
\end{tikzpicture}
}}}
\vglue 15pt
\begin{flushleft}
\begin{tikzpicture}
\node [rounded corners,fill=red!20,minimum width=993pt,
	minimum height=175pt,path fading=south] at (0,0) {\mbox{}};
\end{tikzpicture}
\end{flushleft}
\begin{textblock}{22}(0.8,3.5)
\begin{tikzpicture}
\node [text width=11cm] {\parbox{350pt}{%
	\begin{center}
		\textbf{Оцінювання знань студентів здійснюється за вимогами
		кредитно-модульної системи і базується на урахуванні кількості
		балів, отриманих ними в ході проведення поточного, модульного
		та підсумкового контролю, а також індивідуальної та\\
		самостійної роботи.}
	\end{center}
}};
\end{tikzpicture}
\end{textblock}
\begin{textblock}{23}(16,5.3)
\begin{tikzpicture}
\node [chamfered rectangle, white, fill=blue, double=blue, draw, very thick]
	{\textbf{максимальна кількість балів 30}};
\end{tikzpicture}
\end{textblock}
%%%---> Table
\begin{textblock}{24}(0.5,6)
\begin{tikzpicture}
\node [text width=35cm] {{\large\textbf{Поточний контроль здійснюється під час
проведення семінарських і практичних занять, консультацій з курсу
<<\textcolor{magenta}{Адміністративного права}>> і має за мету перевірку засвоєння
знань студентами та формування необхідних вмінь і навичок з окремих тем курсу.}}};
\end{tikzpicture}
\end{textblock}
\begin{textblock}{25}(10.4,3.63)
\begin{tikzpicture}
\node (tbl) {
\begin{tabularx}{600pt}{cccccc}
\arrayrulecolor{purple}
\textcolor{white}{%
\textbf{Поточне тестування (поточна успішність)}} & \textcolor{white}{\textbf{Самостійна робота, ІНДЗ}}
& \textcolor{white}{\textbf{Підсумковий тест (модуль-контроль)}} & \textcolor{white}{\textbf{Сума}}&\\
             &  &  & &\\[-5pt]
\textbf{Всього: 30}  & & & &\down\\
 & 30 & 40 & 100 &\up\\[2pt]
\textbf{(розрахунок за формулою)}             &    & & &\down\\[-5pt]
  & & & &\\
\end{tabularx}};

\begin{pgfonlayer}{background}
\draw[rounded corners,top color=red,bottom color=black,draw=white]
	($(tbl.north west)+(0.14,0)$) rectangle ($(tbl.north east)-(0.13,0.9)$);
\draw[rounded corners,top color=white,bottom color=black,
	middle color=red,draw=blue!20] ($(tbl.south west) +(0.12,0.5)$)
		rectangle ($(tbl.south east)-(0.12,0)$);
\draw[top color=blue!1,bottom color=blue!20,draw=white]
	($(tbl.north east)-(0.13,0.6)$) rectangle ($(tbl.south west)+(0.13,0.2)$);
\end{pgfonlayer}
\end{tikzpicture}
\end{textblock}
%%%--->
\vglue 5pt
\noindent
\hspace{5pt}{\large
Результати поточного контролю (від 
\tikz[baseline] \node [draw=blue,thick,fill=blue!50] at (0,0.14) {-1}; до
\tikz[baseline] \node [draw=blue,thick,fill=blue!50] at (0,0.14) {3};
балів) заносяться викладачем
до журналів обліку відвідування занять та їх успішності.}\\[15pt]
\pdfcomment[open=true,color=yellow,icolor=red,icon=Comment,voffset=-8pt,
	subject={Аннотация},author={Система оцінювання},hoffset=35pt]{%
При підведенні підсумків загальна кількість балів перемножується з
коефіцієнтом переводу балів з модульно-рейтингової системи до системи кредитів ECTS,
який розраховується в залежності від кількості аудиторних занять та консультацій
з курсу адміністративної відповідальності за формулою: 30/(Х*Y), де 30 -
максимальна кількість балів, Х - кількість семінарських і практичних
занять, Y - кількість набраних за поточним контролем балів.}
	\begin{tikzpicture}[remember picture, note/.style={rectangle callout,fill=#1}]
\node [note=green!50,opacity=.5,overlay,text opacity=1,
	callout relative pointer={(0,-0.3)}] at (10,0) {%
	\textbf{Так, згідно з встановленою в університеті системою оцінювання:}
	};
\end{tikzpicture}
\vglue 15pt
\parbox{550pt}{%
\begin{itemize}
\item[] \tikz[baseline] \node [draw=blue,thick,fill=blue!50] at (0,0.14) {3}; 
бали виставляється у тому випадку, коли студент знає весь обсяг
пройденого програмного матеріалу; теоретичний матеріал тісно пов'язує з
практикою застосування норм адміністративного права; вільно справляється з
вирішенням задач та завдань, застосовуючи знання при вирішенні практичних
ситуацій; вміло керується при їх розв'язанні відповідними нормами; матеріал
викладає лаконічно, послідовно, не припускає помилок, дає точні
формулювання; вміє самостійно узагальнювати та викладати матеріал; не
допускає ніяких помилок.
\item[] \tikz[baseline] \node [draw=blue,thick,fill=blue!50] at (0,0.14) {2}; 
бали виставляється у тому випадку, коли студент добре знає програмний
матеріал; на питання у межах програми з
\tooltipanim{<<Адміністративного права України>>}{1}{10}
відповідає без ускладнень, грамотно і по суті; вміє застосовувати в
практичній діяльності норми, що регламентують управлінські відносини; вільно володіє
практикою користування законодавством про адміністративні правопорушення;
не відчуває труднощів під час самостійного викладення матеріалу, вирішення
практичних задач, в окремих випадках допускає неточності.
\item[] \tikz[baseline] \node [draw=blue,thick,fill=blue!50] at (0,0.14) {1}; 
бал виставляється, коли студент дає в цілому вірну, але не повну
відповідь; допускає порушення послідовності і окремі неточності у викладені
програмного матеріалу.
\item[] \tikz[baseline] \node [draw=blue,thick,fill=blue!50] at (0,0.14) {-1}; 
бал виставляється у тому випадку, коли студент проявляє незнання
більшої частини або основних положень програмного матеріалу; не вміє вирішувати
практичні задачі або проявив повне незнання програмного матеріалу; допускає
багато помилок або викладає матеріал не по суті.
\end{itemize}
}\\
\vglue 5pt
\begin{tikzpicture}[remember picture, note/.style={rectangle callout,
	fill=#1}]
	\node [note=red!50, callout relative pointer={(0,1)}] at (6,0.5)
	{\textbf{Загальна кількість балів за аудиторні заняття у рамках кожного змістового
	модулю не може перевищувати 30 балів.}};
\end{tikzpicture}	
\begin{textblock}{26}(16.3,7)
\begin{tikzpicture} [mindmap,
		every node/.style={concept, circular drop shadow, execute at begin node=\hskip0pt},
		root concept/.append style={
		concept color=black, fill=white, line width=1ex, text=black},
		text=white, grow cyclic,
		level 1/.append style={level distance=6cm,sibling angle=45},
		level 2/.append style={level distance=3cm,sibling
		angle=35},scale=0.9]
%		\clip (-8,-8) rectangle ++(16,16);
		\node [root concept, text width=115pt,
			font=\Large\bfseries\sffamily] {Форми\\
				поточного\\ контролю} % root
		child [concept color=red] { node [text width=90pt, font=\small\bfseries\sffamily]
			{опитування під час проведення
			семінарських і практичних занять}
%		child { node {Problem Measures} }
		}
		child [concept color=blue] { node [text width=80pt,
			font=\small\bfseries\sffamily] {проведення контрольних робіт}
%		child { node {Turing Machines} }
		}
		child [concept color=orange] { node [text width=90pt,
			font=\small\bfseries\sffamily] {проведення\\ опитування з\\
			застосуванням\\ спеціальних\\ комп'ютерних\\ програм}
%		child { node {Complexity Measures} }
		}
		child [concept color=green!50!black] { node [text width=90pt,
			font=\small\bfseries\sffamily] {заслуховування\\
			рефератів,\\ доповідей}
			}
%		child { node {Exact Algorithms} }
		child [concept color=magenta!50!black] { node [font=\small\bfseries\sffamily] {вирішення задач}
			}
		child [concept color=yellow!50!black] { node
		[font=\small\bfseries\sffamily] {оцінювання виконання вправ}
			}
		child [concept color=cyan!50!black] { node
		[font=\small\bfseries\sffamily] {оцінювання вирішення
		практичних завдань} };
\end{tikzpicture}
\end{textblock}
%%%---------------------------------------------------------------------------------
\begin{textblock}{27}(25,-0.01)
\begin{tikzpicture}[even odd rule,rounded corners=2pt,x=10pt,y=10pt,drop shadow]
\filldraw[fill=yellow!90!black!40,drop shadow] (0,0)   rectangle (1,1)
	[xshift=5pt,yshift=5pt]   (0,0)   rectangle (1,1)
	[rotate=30]   (-1,-1) rectangle (2,2);
\node at (0,1.7) {\textbf{\thepage}};			      
\end{tikzpicture}
\end{textblock}
%%%--- Navigational panel top page
\begin{textblock}{28}(7.58,0.85)
\mbox{%%%--->
\Acrobatmenu{LastPage}{%
\tikz[baseline] \node[rectangle,inner sep=2pt,minimum height=3.1ex,
rounded corners,drop shadow,shadow scale=1,shadow xshift=.8ex,
shadow yshift=-.4ex,opacity=.7,fill=black!50,top color=red!90!black!50,
bottom color=red!80!black!80,draw=red!50!black!50,very thick,text=white,
text opacity=1,minimum width=3cm,font=\bfseries\sffamily] at (0,0) {К концу};
}\Acrobatmenu{GoBack}{%
\tikz[baseline] \node[rectangle,inner sep=2pt,minimum height=3.1ex,
rounded corners,drop shadow,shadow scale=1,shadow xshift=.8ex,
shadow yshift=-.4ex,opacity=.7,fill=black!50,top color=red!90!black!50,
bottom color=red!80!black!80,draw=red!50!black!50,very thick,text=white,
text opacity=1,minimum width=3cm,font=\bfseries\sffamily] at (4,0) {Назад};
}\Acrobatmenu{PrevPage}{%
\tikz[baseline] \node[rectangle,inner sep=2pt,minimum height=3.1ex,
rounded corners,drop shadow,shadow scale=1,shadow xshift=.8ex,
shadow yshift=-.4ex,opacity=.7,fill=black!50,top color=red!90!black!50,
bottom color=red!80!black!80,draw=red!50!black!50,very thick,text=white,
text opacity=1,minimum width=3cm,font=\bfseries\sffamily] at (8,0) {Предыдущий};
}\Acrobatmenu{NextPage}{%
\tikz[baseline] \node[rectangle,inner sep=2pt,minimum height=3.1ex,
rounded corners,drop shadow,shadow scale=1,shadow xshift=.8ex,
shadow yshift=-.4ex,opacity=.7,fill=black!50,top color=red!90!black!50,
bottom color=red!80!black!80,draw=red!50!black!50,very thick,text=white,
text opacity=1,minimum width=3cm,font=\bfseries\sffamily] at (12,0) {Следующий};
}\Acrobatmenu{GoForward}{%
\tikz[baseline] \node[rectangle,inner sep=2pt,minimum height=3.1ex,
rounded corners,drop shadow,shadow scale=1,shadow xshift=.8ex,
shadow yshift=-.4ex,opacity=.7,fill=black!50,top color=red!90!black!50,
bottom color=red!80!black!80,draw=red!50!black!50,very thick,text=white,
text opacity=1,minimum width=3cm,font=\bfseries\sffamily] at (16,0) {Вперед};
}\Acrobatmenu{FirstPage}{%
\tikz[baseline] \node[rectangle,inner sep=2pt,minimum height=3.1ex,
rounded corners,drop shadow,shadow scale=1,shadow xshift=.8ex,
shadow yshift=-.4ex,opacity=.7,fill=black!50,top color=red!90!black!50,
bottom color=red!80!black!80,draw=red!50!black!50,very thick,text=white,
text opacity=1,minimum width=3cm,font=\bfseries\sffamily] at (20,0) {К началу};
}\Acrobatmenu{FullScreen}{%
\tikz[baseline] \node[rectangle,inner sep=2pt,minimum height=3.1ex,
rounded corners,drop shadow,shadow scale=1,shadow xshift=.8ex,
shadow yshift=-.4ex,opacity=.7,fill=black!50,top color=red!90!black!50,
bottom color=red!80!black!80,draw=red!50!black!50,very thick,text=white,
text opacity=1,minimum width=3cm,font=\bfseries\sffamily] at (24,0) {Полный экран};
}\Acrobatmenu{Quit}{%
\tikz[baseline] \node[rectangle,inner sep=2pt,minimum height=3.1ex,
rounded corners,drop shadow,shadow scale=1,shadow xshift=.8ex,
shadow yshift=-.4ex,opacity=.7,fill=black!50,top color=red!90!black!50,
bottom color=red!80!black!80,draw=red!50!black!50,very thick,text=white,
text opacity=1,minimum width=3cm,font=\bfseries\sffamily] at (28,0) {Выход};
}
}%%%---|
\end{textblock}
\begin{textblock}{66}(13.39,1.305)	%(-0.38,-0.153)
\begin{tikzpicture}[remember picture,overlay]
	\node {\mbox{\includegraphics[scale=0.99]{./laworder_bg_01}}};
\end{tikzpicture}
\end{textblock}	
%%%---------------------------------------------------------------------------------
\newpage
\AddToTemplate{lawordercover}
\enableTiling
\begin{tikzpicture}[remember picture,overlay]
	  \node [rotate=0,scale=2,text opacity=0.2]
	      at (27,1.7) {Капранов~О.~Г.~\copyright~2010~~~Luga\TeX @yahoo.com};
\end{tikzpicture}
\noindent
\textbf{Модульний контроль} 
\tikz[baseline]
\node [chamfered rectangle, white, fill=blue, double=blue, draw, very thick]
	at (0,0.14) {\textbf{максимальна кількість балів 40}};
\textbf{--- це контроль знань
студентів, який здійснюється після вивчення кожного змістового модулю
дисципліни} \tooltipanim{<<Адміністративне право України>>}{1}{10}.\\
Бали за підсумками
модуль-контролю виставляються в журнали обліку відвідування занять та їх успішності. У
випадку, коли виявлено незадовільні знання, виставляється незадовільна
оцінка.

\begin{tikzpicture}[remember picture, note/.style={rectangle callout,
	fill=#1}]
	\node [note=red!50, callout relative pointer={(0,1)}] at (0,0.1)
	{\textbf{Загальна кількість балів за кожен модуль-контроль не може
		перевищувати 40 балів.}};
\node [note=blue!50, callout relative pointer={(0,-1)}] at (16,0.1) {%
\textbf{Таблиця визначення якості знань студентів під час проведення
модуль-контролю}};
\end{tikzpicture}	
\vglue 5pt
\begin{center}
\begin{tikzpicture}
\node (tbl) {
\begin{tabularx}{.6\textwidth}{cccl}
\arrayrulecolor{purple}
\multicolumn{4}{c}{\textcolor{white}{\bf Визначення якості знань студентів}}\\[1ex]
37-40 & \textbf{балів} & \tikz \node[ball color=magenta,circle,
text=black] {}; &\parbox{480pt}{\textbf{%
	свідчить про високий рівень знань, навичок, вмінь, які отримав
	студент з адміністративної відповідальності, його вміння самостійно
	розв'язувати складні теоретичні та практичні завдання;}
}\\[1ex]
\midrule
33-36 & \textbf{балів} & \tikz \node[ball color=magenta,circle,
	text=black] {}; &\parbox{480pt}{\textbf{%
	результат, вищий за середній, але з декількома несуттєвими недоліками;}
}\\[1ex]
\midrule
29-32 & \textbf{балів} & \tikz \node[ball color=magenta,circle,
	text=black] {}; &\parbox{480pt}{\textbf{%
	свідчить, що рівень теоретичної та практичної підготовки студента
	відповідає основним вимогам навчальної програми дисципліни, протемають
	місце окремі суттєві недоліки у роботі;}
}\\[1ex]
\midrule
25-28 & \textbf{балів} & \tikz \node[ball color=magenta,circle,
	text=black] {}; &\parbox{480pt}{\textbf{%
	посередній результат, зі значними недоліками;}
}\\[1ex]
\midrule
20-24 & \textbf{балів} & \tikz \node[ball color=magenta,circle,
	text=black] {}; &\parbox{480pt}{\textbf{%
	результат задовольняє мінімальним вимогам щодо знань, умінь та
	навичок з даної дисципліни.}
}\\[1ex]
\end{tabularx}};

\begin{pgfonlayer}{background}
\draw[rounded corners,top color=red,bottom color=black,draw=white]
	($(tbl.north west)+(0.14,0)$) rectangle ($(tbl.north east)-(0.13,0.9)$);
\draw[rounded corners,top color=white,bottom color=black,
	middle color=red,draw=blue!20] ($(tbl.south west) +(0.12,0.5)$)
		rectangle ($(tbl.south east)-(0.12,0)$);
\draw[top color=blue!1,bottom color=blue!20,draw=white]
	($(tbl.north east)-(0.13,0.6)$) rectangle ($(tbl.south west)+(0.13,0.2)$);
\end{pgfonlayer}
\end{tikzpicture}
\end{center}
\vglue 5pt
\textbf{Індивідуальна самостійна робота} 
\tikz[baseline]
\node [chamfered rectangle, white, fill=blue, double=blue, draw,
	very thick, text justified]
	at (0,0.14) {\textbf{максимальна кількість балів 30}};
є невід'ємною складовою вивчення дисципліни, вона сприяє поглибленому вивченню
теоретичного матеріалу,
~\tikz[every node/.style={signal,draw,text=white,signal to=nowhere},
minimum height=2pt, minimum width=40pt,font=\bfseries\sffamily,]
\node[fill=red!65!black, signal to=east] at (0,-0.14) {Читай далее!};
~\pdfcomment[open=true,color=yellow,icolor=red,icon=Comment,
subject={Аннотация},author={Індивідуальна самостійна робота},voffset=10pt]{%
закріпленню й узагальненню отриманих знань. Індивідуальна
самостійна робота здійснюється на основі методичних розробок кафедри адміністративного
права та адміністративної діяльності ЛДУВС імені Е.О. Дідоренка. Бали за
виконання індивідуальних завдань виставляються у журнали обліку
відвідування занять студентами і їх успішності окремою графою за кожний змістовий
модуль. Індивідуальна самостійна робота складається з завдань для самостійної
роботи та індивідуальних навчально-дослідних завдань, які запропоновані до кожної
теми.
Індивідуальне навчально-дослідницьке завдання є видом самостійної роботи
студентів та курсантів, яке виконується під керівництвом викладача в межах
навчальної програми з адміністративної відповідальності. Його виконання
передбачає наявність певних знань, умінь та навичок, одержаних під час
проведення аудиторних занять.}
\vglue 5pt
\begin{tikzpicture}
\colorlet{even}{cyan!60!black}
\colorlet{odd}{orange!100!black}
\colorlet{links}{red!70!black}
\colorlet{back}{yellow!20!white}
\tikzset{
	box/.style={
		minimum height=50mm,
		inner sep=2mm,
		outer sep=1mm,
		text width=320mm,
		text centered,
%		font=\huge\bfseries\sffamily,
		font=\bfseries\sffamily,
%		text=#1!50!black,
		text=black,
		draw=#1,
		line width=.25mm,
		top color=#1!5,
		bottom color=#1!40,
		shading angle=0,
		rounded corners=2.3mm,
		drop shadow={fill=#1!40!gray,fill opacity=.8},
		rotate=0,
		opacity=.6,
		text opacity=1,
		text justified,
	},
}
\node [box=back] (th) {%
\mbox{}\\
Індивідуальні навчально-дослідницькі завдання виконуються в формі
підготовки наукових статей, виступів (доповідей) на наукових конференціях, рефератів,
доповідей, рецензування наукових статей, складання бібліографії за темою
занять тощо. За надруковану наукову статтю студент може отримати
{\colorbox{blue}{\textcolor{white}{20 балів}}}, виступ
(доповідь) на науковій конференції --- 
{\colorbox{blue}{\textcolor{white}{10 балів}}},
реферат або доповідь ---
{\colorbox{blue}{\textcolor{white}{5 балів}}}.
Реферат або доповідь повинні бути виконані на комп'ютері та роздруковані.
Аркуші повинні бути пронумеровані. На титульному аркуші повинно бути
зазначене найменування навчального закладу, назва та вид роботи, прізвище та ініціали
виконавця. У кінці роботи повинен бути поданий перелік використаної
літератури, у такій послідовності: нормативно правові акти за їх юридичною силою;
навчально-методична, наукова, спеціальна література у алфавітному порядку.
Обсяг роботи від 
{\colorbox{blue}{\textcolor{white}{10}}}
до
{\colorbox{blue}{\textcolor{white}{15}}}
аркушів формату
{\colorbox{blue}{\textcolor{white}{А-4}}},
комп'ютерного тексту. Поля верхнє, нижнє ---
{\colorbox{blue}{\textcolor{white}{20 мм}}},
ліве ---
{\colorbox{blue}{\textcolor{white}{30мм}}},
праве ---
{\colorbox{blue}{\textcolor{white}{15мм}}}.
Шрифт ---
{\colorbox{blue}{\textcolor{white}{Times New Roman}}}
кегль ---
{\colorbox{blue}{\textcolor{white}{14}}},
міжрядковий інтервал ---
{\colorbox{blue}{\textcolor{white}{1,54}}}
Кожний доповідач має викласти зміст своєї роботи усно за
{\colorbox{blue}{\textcolor{white}{1--10}}}
хвилин та
бути готовим відповідати на запитання, які будуть поставлені студентами та
викладачем.
Студенти та курсанти, які підготували найбільш глибокі та змістовні
індивідуальні дослідження, можуть згодом доробити їх до рівня наукової
доповіді для виступу на конференції.
Рецензування наукових статей --- це критичний відгук на наукову роботу, яка
пропонується з викладом її позитивних та негативних сторін. Рецензування
здійснюється у письмовому вигляді, з дотриманням таких самих вимог щодо
оформлення, які ставляться до реферату чи доповіді.
Складання бібліографії за темою заняття --- це індивідуальна робота з
підготовки та збирання (пошуку) інформації про нормативно-правові акти й літературні
джерела, що стосуються курсу
\tooltipanim{<<Адміністративне право України>>}{11}{20},
які є у бібліотеці, відбір їх за певними ознаками, систематизація та складання
переліку.};
\node[box=links,anchor=south west,xshift=3mm,yshift=1mm,
	minimum height=5pt,text width=250pt,outer sep=0mm, inner sep=1mm]
	at (th.north west) {\mbox{}\hfill\small Індивідуальні навчально-дослідницькі
	завдання\hfill\mbox{}};
\end{tikzpicture}
\vglue 25pt
\pdflinecomment[color=green,icolor=blue,,subject={Top2},type=line,opacity=1,
line={25 15 25 135},caption=top,linebegin={/Diamond},lineend={/Diamond},
linewidth=2bp,captionhoffset=-5pt,captionvoffset=15pt]{Внимательно прочтите!}
\pdflinecomment[color=green,icolor=blue,,subject={Top2},type=line,opacity=1,
line={505 15 505 135},caption=top,linebegin={/Diamond},lineend={/Diamond},
linewidth=2bp,captionhoffset=-5pt,captionvoffset=-15pt]{Внимательно прочтите!}
\parbox{450pt}{%
Загальна кількість балів, яку студент може отримати за виконання
індивідуальної або самостійної роботи за підсумками змістового модулю навчальної
програми не може перевищувати
\tikz[baseline] \node [draw=blue,thick,fill=blue!50] at (0,0.14) {30}; балів.

Підведення підсумків змістового модулю навчання студентів та курсантів
здійснюється за формулою: 
\tikz[baseline] \node [draw=red,thick,fill=red!50] at (0,0.14) {П + М + І};
де
\tikz[baseline] \node [draw=red,thick,fill=red!50] at (0,0.14) {П};
--- кількість балів, набраних у ході поточного контролю,
\tikz[baseline] \node [draw=red,thick,fill=red!50] at (0,0.14) {М};
--- кількість балів, набраних у ході модуль-контролю,
\tikz[baseline] \node [draw=red,thick,fill=red!50] at (0,0.14) {І};
--- кількість балів, набраних у ході індивідуальної та самостійної роботи за
змістовним модулем навчальної програми.

Результати підсумкового контролю змістового модулю фіксуються викладачем у
журналі успішності за
\tikz[baseline] \node [draw=red,thick,fill=red!50] at (0,0.14) {100};
--бальною системою з урахуванням суми набраних
балів за всіма видами робіт, за наступною шкалою:}
\qquad\qquad\qquad
\parbox{350pt}{%
\begin{tikzpicture}
\node (tbl) {
\begin{tabularx}{450pt}{ccr}
\arrayrulecolor{purple}
	&\textcolor{white}{\textbf{Кількість балів набраних за результатами
		вивчення дисципліни}} &
		\mbox{\textcolor{white}{\textbf{Оцінка}}}\hspace{28pt}\mbox{}\\[1ex]
\up & \textbf{90-100} &\textbf{зараховано (А);}\\[1ex]
\up & \textbf{83-89}  &\textbf{зараховано (В);}\\[1ex]
\up & \textbf{75-82}  &\textbf{зараховано (С);}\\[1ex]
\up & \textbf{68-74}  &\textbf{зараховано (D);}\\[1ex]
\up & \textbf{60-67}  &\textbf{зараховано (Е);}\\[1ex]
\down & \textbf{1-59} &\textbf{не зараховано (F);}\\[1ex]
\end{tabularx}};

\begin{pgfonlayer}{background}
\draw[rounded corners,top color=red,bottom color=black,draw=white]
	($(tbl.north west)+(0.14,0)$) rectangle ($(tbl.north east)-(0.13,0.9)$);
\draw[rounded corners,top color=white,bottom color=black,
	middle color=red,draw=blue!20] ($(tbl.south west) +(0.12,0.5)$)
		rectangle ($(tbl.south east)-(0.12,0)$);
\draw[top color=blue!1,bottom color=blue!20,draw=white]
	($(tbl.north east)-(0.13,0.6)$) rectangle ($(tbl.south west)+(0.13,0.2)$);
\end{pgfonlayer}
\end{tikzpicture}
}
%%%-----------------------------------------------------------------------
\begin{textblock}{29}(25,-0.01)
\begin{tikzpicture}[even odd rule,rounded corners=2pt,x=10pt,y=10pt,drop shadow]
\filldraw[fill=yellow!90!black!40,drop shadow] (0,0)   rectangle (1,1)
	[xshift=5pt,yshift=5pt]   (0,0)   rectangle (1,1)
	[rotate=30]   (-1,-1) rectangle (2,2);
\node at (0,1.7) {\textbf{\thepage}};			      
\end{tikzpicture}
\end{textblock}
%%%--- Navigational panel top page
\begin{textblock}{30}(7.58,0.85)
\mbox{%%%--->
\Acrobatmenu{LastPage}{%
\tikz[baseline] \node[rectangle,inner sep=2pt,minimum height=3.1ex,
rounded corners,drop shadow,shadow scale=1,shadow xshift=.8ex,
shadow yshift=-.4ex,opacity=.7,fill=black!50,top color=red!90!black!50,
bottom color=red!80!black!80,draw=red!50!black!50,very thick,text=white,
text opacity=1,minimum width=3cm,font=\bfseries\sffamily] at (0,0) {К концу};
}\Acrobatmenu{GoBack}{%
\tikz[baseline] \node[rectangle,inner sep=2pt,minimum height=3.1ex,
rounded corners,drop shadow,shadow scale=1,shadow xshift=.8ex,
shadow yshift=-.4ex,opacity=.7,fill=black!50,top color=red!90!black!50,
bottom color=red!80!black!80,draw=red!50!black!50,very thick,text=white,
text opacity=1,minimum width=3cm,font=\bfseries\sffamily] at (4,0) {Назад};
}\Acrobatmenu{PrevPage}{%
\tikz[baseline] \node[rectangle,inner sep=2pt,minimum height=3.1ex,
rounded corners,drop shadow,shadow scale=1,shadow xshift=.8ex,
shadow yshift=-.4ex,opacity=.7,fill=black!50,top color=red!90!black!50,
bottom color=red!80!black!80,draw=red!50!black!50,very thick,text=white,
text opacity=1,minimum width=3cm,font=\bfseries\sffamily] at (8,0) {Предыдущий};
}\Acrobatmenu{NextPage}{%
\tikz[baseline] \node[rectangle,inner sep=2pt,minimum height=3.1ex,
rounded corners,drop shadow,shadow scale=1,shadow xshift=.8ex,
shadow yshift=-.4ex,opacity=.7,fill=black!50,top color=red!90!black!50,
bottom color=red!80!black!80,draw=red!50!black!50,very thick,text=white,
text opacity=1,minimum width=3cm,font=\bfseries\sffamily] at (12,0) {Следующий};
}\Acrobatmenu{GoForward}{%
\tikz[baseline] \node[rectangle,inner sep=2pt,minimum height=3.1ex,
rounded corners,drop shadow,shadow scale=1,shadow xshift=.8ex,
shadow yshift=-.4ex,opacity=.7,fill=black!50,top color=red!90!black!50,
bottom color=red!80!black!80,draw=red!50!black!50,very thick,text=white,
text opacity=1,minimum width=3cm,font=\bfseries\sffamily] at (16,0) {Вперед};
}\Acrobatmenu{FirstPage}{%
\tikz[baseline] \node[rectangle,inner sep=2pt,minimum height=3.1ex,
rounded corners,drop shadow,shadow scale=1,shadow xshift=.8ex,
shadow yshift=-.4ex,opacity=.7,fill=black!50,top color=red!90!black!50,
bottom color=red!80!black!80,draw=red!50!black!50,very thick,text=white,
text opacity=1,minimum width=3cm,font=\bfseries\sffamily] at (20,0) {К началу};
}\Acrobatmenu{FullScreen}{%
\tikz[baseline] \node[rectangle,inner sep=2pt,minimum height=3.1ex,
rounded corners,drop shadow,shadow scale=1,shadow xshift=.8ex,
shadow yshift=-.4ex,opacity=.7,fill=black!50,top color=red!90!black!50,
bottom color=red!80!black!80,draw=red!50!black!50,very thick,text=white,
text opacity=1,minimum width=3cm,font=\bfseries\sffamily] at (24,0) {Полный экран};
}\Acrobatmenu{Quit}{%
\tikz[baseline] \node[rectangle,inner sep=2pt,minimum height=3.1ex,
rounded corners,drop shadow,shadow scale=1,shadow xshift=.8ex,
shadow yshift=-.4ex,opacity=.7,fill=black!50,top color=red!90!black!50,
bottom color=red!80!black!80,draw=red!50!black!50,very thick,text=white,
text opacity=1,minimum width=3cm,font=\bfseries\sffamily] at (28,0) {Выход};
}	
}%%%---|
\end{textblock}
%%%---------------------------------------------------------------------------------

%%%---> Table #1
\newpage
\begin{tikzpicture}[remember picture,overlay]
	  \node [rotate=0,scale=2,text opacity=0.2]
	      at (27,1.7) {Капранов~О.~Г.~\copyright~2010~~~Luga\TeX @yahoo.com};
\end{tikzpicture}
\vglue -18pt
\hspace{187pt}
\parbox{350pt}{%
\hypertarget{admintbl1}{\hyperlink{admintbl2}{\mbox{%
\begin{tikzpicture}
  \colorlet{even}{cyan!60!black}
  \colorlet{odd}{orange!100!black}
  \colorlet{links}{red!70!black}
  \colorlet{back}{yellow!20!white}
  \tikzset{
    box/.style={
      minimum height=15mm,
      inner sep=.7mm,
      outer sep=0mm,
      text width=120mm,
      text centered,
      font=\small\bfseries\sffamily,
      text=#1!50!black,
      draw=#1,
      line width=.25mm,
      top color=#1!5,
      bottom color=#1!40,
      shading angle=0,
      rounded corners=2.3mm,
      drop shadow={fill=#1!40!gray,fill opacity=.8},
      rotate=0,
    },
  }
  \node [box=even]{{%
  	\huge\textbf{Тематичний план за курсом <<Адміністративне право>>}}};
\end{tikzpicture}
}}}}
\hfill
\begin{flushright}
		\tikz \node [copy shadow={left color=green!50},tape,
			left color=green!50,draw=green,thick]
					 {{\large\textbf{Таблица №1}}};
\end{flushright}	
\hfill
\begin{center}
\begin{tikzpicture}
\node (tbl) {
\begin{tabularx}{976pt}{|>{\bfseries}c|>{\bfseries}p{430pt}|>{\bfseries}c|
	>{\bfseries}c|>{\bfseries}c|>{\bfseries}c|>{\bfseries}c|>{\bfseries}c|}
\arrayrulecolor{purple}
 & &\multicolumn{6}{c|}{\textcolor{white}{\bf Кількість годин за видами
 занять}}\\ \cline{3-8}\\
 & & &\multicolumn{5}{c|}{\raisebox{1.3ex}[0pt][0pt]{{\bf у тому числі:}}}\\ \cline{4-8}\\
 & & & & &\multicolumn{3}{c|}{\raisebox{1.3ex}[0pt][0pt]{{\bf із них:}}}\\ \cline{6-8}
\raisebox{5.5ex}[0pt][0pt]{Номера тем} &
 \mbox{}\hspace{130pt}\raisebox{5.5ex}[0pt][0pt]{Найменування розділів і тем} &
	\raisebox{1.5ex}[0pt][0pt]{Всього} &
		\raisebox{1.5ex}[0pt][0pt]{Самостійна робота} &
			\raisebox{1.5ex}[0pt][0pt]{Аудиторні} & 
				\raisebox{-0.8ex}[0pt][0pt]{Лекції} &
					\raisebox{-0.8ex}[0pt][0pt]{Семінарські заняття} &
						\raisebox{-0.8ex}[0pt][0pt]{Практичні заняття}\\[1ex]
\midrule
\multicolumn{8}{|>{\bfseries}c|}{ІІІ семестр}\\
\midrule
\multicolumn{8}{|>{\bfseries}c|}{ЗМІСТОВИЙ МОДУЛЬ I}\\
\midrule
\multicolumn{8}{|>{\bfseries}c|}{Загальна характеристика адміністративного
права, державного управління та виконавчої влади}\\
\midrule
1.0 & Предмет, метод і система адміністративного права & 9 & 3 & 6 & 4 & 2 &\\
\midrule
1.1 & Особливості предмета й методу адміністративно-правового регулювання & & & & 2 & &\\
\midrule
1.2 & Система адміністративного права & & & & 2 & &\\
\midrule
1.3 & Співвідношення адміністративного права з іншими галузями права & & & & & 2 &\\
\midrule
2.0 & Адміністративне право та державне управління & 9 & 5 & 4 & 2 & 2 &\\
\midrule
2.1 & Поняття та сутність державного управління & & & & 2 & &\\
\midrule
2.2 & Принципи та функції державного управління & & & & & 2 &\\
\midrule
3.0 & Адміністративно-правові норми та адміністративно-правові відносини & 9 & 5 & 4 & 2 & & 2\\
\midrule
3.1 & Поняття та особливості адміністративно-правових норм та
адміністративно-правових відносин & & & & 2 & &\\
\midrule
3.2 & Джерела адміністративного права & & & & & & 2\\
\midrule
4.0 & Суб'єкти адміністративного права & 27 & 9 & 18 & 10 & 4 & 4\\
\midrule
4.1 & Поняття та види суб'єктів адміністративного права & & & & 2 & &\\
\midrule
4.2 & Громадяни України як суб'єкти адміністративного права & & & & 2 & &\\
\midrule
4.3 & Спеціальні адміністративно-правові статуси індивідуальних суб'єктів
адміністративного права & & & & & & 2\\
\midrule
4.4 & Державні службовці --- суб'єкти адміністративного права & & & & 2 & &\\
\midrule
4.5 & Порівняльний аналіз статусів державних службовців & & & & & & 2\\
\midrule
4.6 & Органи виконавчої влади --- суб'єкти адміністративного права & & & & 2 & &\\
\midrule
4.7 & Поняття, ознаки та класифікація органів виконавчої влади & & & & & 2 &\\
\midrule
4.8 & Центральні та місцеві органи виконавчої влади & & & & 2 & &\\
\midrule
4.9 & Місце органів внутрішніх справ у системі виконавчої влади & & & & & 2 &\\[0.5ex]
\end{tabularx}};

\begin{pgfonlayer}{background}
\draw[rounded corners,top color=red,bottom color=black,draw=white]
	($(tbl.north west)+(0.14,0)$) rectangle ($(tbl.north east)-(0.13,0.9)$);
\draw[rounded corners,top color=white,bottom color=black,
	middle color=red,draw=blue!20] ($(tbl.south west) +(0.12,0.5)$)
		rectangle ($(tbl.south east)-(0.12,0)$);
\draw[top color=blue!1,bottom color=blue!20,draw=white]
	($(tbl.north east)-(0.13,0.6)$) rectangle ($(tbl.south west)+(0.13,0.2)$);
\end{pgfonlayer}
\end{tikzpicture}
\end{center}
\hfill
\centerline{%
\hyperlink{admintbl2}{\mbox{%
\tikz[baseline] \node[rectangle,inner sep=2pt,minimum height=3.1ex,rounded corners,
drop shadow,shadow scale=1,shadow xshift=.8ex,shadow yshift=-.4ex,opacity=.7,
fill=black!50,top color=blue!90!black!50,bottom color=blue!80!black!80,
draw=blue!50!black!50,very thick,text=white,text opacity=1,minimum width=4cm]{%
	Таблица №2};
	}}\qquad\qquad\qquad\qquad
\hyperlink{admintbl3}{\mbox{%
\tikz[baseline] \node[rectangle,inner sep=2pt,minimum height=3.1ex,rounded corners,
drop shadow,shadow scale=1,shadow xshift=.8ex,shadow yshift=-.4ex,opacity=.7,
fill=black!50,top color=blue!90!black!50,bottom color=blue!80!black!80,
draw=blue!50!black!50,very thick,text=white,text opacity=1,minimum width=4cm]{%
	Таблица №3};
	}}\qquad\qquad\qquad\qquad
\hyperlink{admintbl4}{\mbox{%
\tikz[baseline] \node[rectangle,inner sep=2pt,minimum height=3.1ex,rounded corners,
drop shadow,shadow scale=1,shadow xshift=.8ex,shadow yshift=-.4ex,opacity=.7,
fill=black!50,top color=blue!90!black!50,bottom color=blue!80!black!80,
draw=blue!50!black!50,very thick,text=white,text opacity=1,minimum width=4cm]{%
	Таблица №4};
	}}}
\hfill
\mbox{}
%%%-----------------------------------------------------------------------
\begin{textblock}{31}(25,-0.01)
\begin{tikzpicture}[even odd rule,rounded corners=2pt,x=10pt,y=10pt,drop shadow]
\filldraw[fill=yellow!90!black!40,drop shadow] (0,0)   rectangle (1,1)
	[xshift=5pt,yshift=5pt]   (0,0)   rectangle (1,1)
	[rotate=30]   (-1,-1) rectangle (2,2);
\node at (0,1.7) {\textbf{\thepage}};			      
\end{tikzpicture}
\end{textblock}
%%%--- Navigational panel top page
\begin{textblock}{32}(7.58,0.85)
\mbox{%%%--->
\Acrobatmenu{LastPage}{%
\tikz[baseline] \node[rectangle,inner sep=2pt,minimum height=3.1ex,
rounded corners,drop shadow,shadow scale=1,shadow xshift=.8ex,
shadow yshift=-.4ex,opacity=.7,fill=black!50,top color=red!90!black!50,
bottom color=red!80!black!80,draw=red!50!black!50,very thick,text=white,
text opacity=1,minimum width=3cm,font=\bfseries\sffamily] at (0,0) {К концу};
}\Acrobatmenu{GoBack}{%
\tikz[baseline] \node[rectangle,inner sep=2pt,minimum height=3.1ex,
rounded corners,drop shadow,shadow scale=1,shadow xshift=.8ex,
shadow yshift=-.4ex,opacity=.7,fill=black!50,top color=red!90!black!50,
bottom color=red!80!black!80,draw=red!50!black!50,very thick,text=white,
text opacity=1,minimum width=3cm,font=\bfseries\sffamily] at (4,0) {Назад};
}\Acrobatmenu{PrevPage}{%
\tikz[baseline] \node[rectangle,inner sep=2pt,minimum height=3.1ex,
rounded corners,drop shadow,shadow scale=1,shadow xshift=.8ex,
shadow yshift=-.4ex,opacity=.7,fill=black!50,top color=red!90!black!50,
bottom color=red!80!black!80,draw=red!50!black!50,very thick,text=white,
text opacity=1,minimum width=3cm,font=\bfseries\sffamily] at (8,0) {Предыдущий};
}\Acrobatmenu{NextPage}{%
\tikz[baseline] \node[rectangle,inner sep=2pt,minimum height=3.1ex,
rounded corners,drop shadow,shadow scale=1,shadow xshift=.8ex,
shadow yshift=-.4ex,opacity=.7,fill=black!50,top color=red!90!black!50,
bottom color=red!80!black!80,draw=red!50!black!50,very thick,text=white,
text opacity=1,minimum width=3cm,font=\bfseries\sffamily] at (12,0) {Следующий};
}\Acrobatmenu{GoForward}{%
\tikz[baseline] \node[rectangle,inner sep=2pt,minimum height=3.1ex,
rounded corners,drop shadow,shadow scale=1,shadow xshift=.8ex,
shadow yshift=-.4ex,opacity=.7,fill=black!50,top color=red!90!black!50,
bottom color=red!80!black!80,draw=red!50!black!50,very thick,text=white,
text opacity=1,minimum width=3cm,font=\bfseries\sffamily] at (16,0) {Вперед};
}\Acrobatmenu{FirstPage}{%
\tikz[baseline] \node[rectangle,inner sep=2pt,minimum height=3.1ex,
rounded corners,drop shadow,shadow scale=1,shadow xshift=.8ex,
shadow yshift=-.4ex,opacity=.7,fill=black!50,top color=red!90!black!50,
bottom color=red!80!black!80,draw=red!50!black!50,very thick,text=white,
text opacity=1,minimum width=3cm,font=\bfseries\sffamily] at (20,0) {К началу};
}\Acrobatmenu{FullScreen}{%
\tikz[baseline] \node[rectangle,inner sep=2pt,minimum height=3.1ex,
rounded corners,drop shadow,shadow scale=1,shadow xshift=.8ex,
shadow yshift=-.4ex,opacity=.7,fill=black!50,top color=red!90!black!50,
bottom color=red!80!black!80,draw=red!50!black!50,very thick,text=white,
text opacity=1,minimum width=3cm,font=\bfseries\sffamily] at (24,0) {Полный экран};
}\Acrobatmenu{Quit}{%
\tikz[baseline] \node[rectangle,inner sep=2pt,minimum height=3.1ex,
rounded corners,drop shadow,shadow scale=1,shadow xshift=.8ex,
shadow yshift=-.4ex,opacity=.7,fill=black!50,top color=red!90!black!50,
bottom color=red!80!black!80,draw=red!50!black!50,very thick,text=white,
text opacity=1,minimum width=3cm,font=\bfseries\sffamily] at (28,0) {Выход};
}	
}%%%---|
\end{textblock}
%%%-----------------------------------------------------------------------
%%%---> Table #2
\newpage
\begin{tikzpicture}[remember picture,overlay]
	  \node [rotate=0,scale=2,text opacity=0.2]
	      at (27,1.7) {Капранов~О.~Г.~\copyright~2010~~~Luga\TeX @yahoo.com};
\end{tikzpicture}
\vglue -18pt
\hspace{187pt}
\parbox{350pt}{%
\hypertarget{admintbl2}{\hyperlink{admintbl3}{\mbox{%
\begin{tikzpicture}
  \colorlet{even}{cyan!60!black}
  \colorlet{odd}{orange!100!black}
  \colorlet{links}{red!70!black}
  \colorlet{back}{yellow!20!white}
  \tikzset{
    box/.style={
      minimum height=15mm,
      inner sep=.7mm,
      outer sep=0mm,
      text width=120mm,
      text centered,
      font=\small\bfseries\sffamily,
      text=#1!50!black,
      draw=#1,
      line width=.25mm,
      top color=#1!5,
      bottom color=#1!40,
      shading angle=0,
      rounded corners=2.3mm,
      drop shadow={fill=#1!40!gray,fill opacity=.8},
      rotate=0,
    },
  }
  \node [box=even]{{%
  	\huge\textbf{Тематичний план за курсом <<Адміністративне право>>}}};
\end{tikzpicture}
}}}}
\hfill
\begin{flushright}
	\tikz \node [copy shadow={left color=green!50},tape,
		left color=green!50,draw=green,thick]
					 {{\large\textbf{Таблица №2}}};
\end{flushright}	
\hfill
\begin{center}
\begin{tikzpicture}
\node (tbl) {
\begin{tabularx}{976pt}{|>{\bfseries}c|>{\bfseries}p{430pt}|>{\bfseries}c|
	>{\bfseries}c|>{\bfseries}c|>{\bfseries}c|>{\bfseries}c|>{\bfseries}c|}
\arrayrulecolor{purple}
 & &\multicolumn{6}{c|}{\textcolor{white}{\bf Кількість годин за видами
 занять}}\\ \cline{3-8}\\
 & & &\multicolumn{5}{c|}{\raisebox{1.3ex}[0pt][0pt]{{\bf у тому числі:}}}\\ \cline{4-8}\\
 & & & & &\multicolumn{3}{c|}{\raisebox{1.3ex}[0pt][0pt]{{\bf із них:}}}\\ \cline{6-8}
\raisebox{5.5ex}[0pt][0pt]{Номера тем} &
 \mbox{}\hspace{130pt}\raisebox{5.5ex}[0pt][0pt]{Найменування розділів і тем} &
	\raisebox{1.5ex}[0pt][0pt]{Всього} &
		\raisebox{1.5ex}[0pt][0pt]{Самостійна робота} &
			\raisebox{1.5ex}[0pt][0pt]{Аудиторні} & 
				\raisebox{-0.8ex}[0pt][0pt]{Лекції} &
					\raisebox{-0.8ex}[0pt][0pt]{Семінарські заняття} &
						\raisebox{-0.8ex}[0pt][0pt]{Практичні заняття}\\[1ex]
\midrule
\multicolumn{8}{|>{\bfseries}c|}{Модуль-контроль №1}\\
\midrule
\multicolumn{8}{|>{\bfseries}c|}{ЗМІСТОВИЙ МОДУЛЬ II}\\
\midrule
\multicolumn{8}{|>{\bfseries}c|}{Адміністративно-правові форми і методи}\\
\midrule
5.0 & Характеристика адміністративно-правових форм і методів & 15 & 7 & 8 & 4 & 2 & 2\\
\midrule
5.1 & Поняття, види та особливості форм і методів управлінської діяльності & & & & 2 & &\\
\midrule
5.2 & Співвідношення адміністративно-правових форм і методів & & & & & 2 &\\
\midrule
5.3 & Юридичні акти управління & & & & 2 & &\\
\midrule
5.4 & Місце актів управління в системі правових актів & & & & 2 & &\\
\midrule
6.0 & Адміністративний примус & 12 & 6 & 6 & 2 & 2 & 2\\
\midrule
6.1 & Поняття, сутність та види адміністративного примусу & & & & 2 & &\\
\midrule
6.2 & Правові засади адміністративного примусу у сфері державного управління & & & & & 2 &\\
\midrule
6.3 & Адміністративний примус в діяльності органів внутрішніх справ & & & & & & 2\\
\midrule
7.0 & Адміністративно-процесуальне право & 9 & 5 & 4 & 2 & & 2\\
\midrule
7.1   & Поняття, особливості та структура адміністративного процесу & & & & 2 & &\\
\midrule
7.2 & Стадії провадження у справах про адміністративні правопорушення & & & & & & 2\\[0.5ex]
\end{tabularx}};

\begin{pgfonlayer}{background}
\draw[rounded corners,top color=red,bottom color=black,draw=white]
	($(tbl.north west)+(0.14,0)$) rectangle ($(tbl.north east)-(0.13,0.9)$);
\draw[rounded corners,top color=white,bottom color=black,
	middle color=red,draw=blue!20] ($(tbl.south west) +(0.12,0.5)$)
		rectangle ($(tbl.south east)-(0.12,0)$);
\draw[top color=blue!1,bottom color=blue!20,draw=white]
	($(tbl.north east)-(0.13,0.6)$) rectangle ($(tbl.south west)+(0.13,0.2)$);
\end{pgfonlayer}
\end{tikzpicture}
\end{center}
\hfill
\centerline{%
\hyperlink{admintbl1}{\mbox{%
\tikz[baseline] \node[rectangle,inner sep=2pt,minimum height=3.1ex,rounded corners,
drop shadow,shadow scale=1,shadow xshift=.8ex,shadow yshift=-.4ex,opacity=.7,
fill=black!50,top color=blue!90!black!50,bottom color=blue!80!black!80,
draw=blue!50!black!50,very thick,text=white,text opacity=1,minimum width=4cm]{%
	Таблица №1};
	}}\qquad\qquad\qquad\qquad
\hyperlink{admintbl3}{\mbox{%
\tikz[baseline] \node[rectangle,inner sep=2pt,minimum height=3.1ex,rounded corners,
drop shadow,shadow scale=1,shadow xshift=.8ex,shadow yshift=-.4ex,opacity=.7,
fill=black!50,top color=blue!90!black!50,bottom color=blue!80!black!80,
draw=blue!50!black!50,very thick,text=white,text opacity=1,minimum width=4cm]{%
	Таблица №3};
	}}\qquad\qquad\qquad\qquad
\hyperlink{admintbl4}{\mbox{%
\tikz[baseline] \node[rectangle,inner sep=2pt,minimum height=3.1ex,rounded corners,
drop shadow,shadow scale=1,shadow xshift=.8ex,shadow yshift=-.4ex,opacity=.7,
fill=black!50,top color=blue!90!black!50,bottom color=blue!80!black!80,
draw=blue!50!black!50,very thick,text=white,text opacity=1,minimum width=4cm]{%
	Таблица №4};
	}}}
\hfill
\mbox{}
%%%-----------------------------------------------------------------------
\begin{textblock}{33}(25,-0.01)
\begin{tikzpicture}[even odd rule,rounded corners=2pt,x=10pt,y=10pt,drop shadow]
\filldraw[fill=yellow!90!black!40,drop shadow] (0,0)   rectangle (1,1)
	[xshift=5pt,yshift=5pt]   (0,0)   rectangle (1,1)
	[rotate=30]   (-1,-1) rectangle (2,2);
\node at (0,1.7) {\textbf{\thepage}};			      
\end{tikzpicture}
\end{textblock}
%%%--- Navigational panel top page
\begin{textblock}{34}(7.58,0.85)
\mbox{%%%--->
\Acrobatmenu{LastPage}{%
\tikz[baseline] \node[rectangle,inner sep=2pt,minimum height=3.1ex,
rounded corners,drop shadow,shadow scale=1,shadow xshift=.8ex,
shadow yshift=-.4ex,opacity=.7,fill=black!50,top color=red!90!black!50,
bottom color=red!80!black!80,draw=red!50!black!50,very thick,text=white,
text opacity=1,minimum width=3cm,font=\bfseries\sffamily] at (0,0) {К концу};
}\Acrobatmenu{GoBack}{%
\tikz[baseline] \node[rectangle,inner sep=2pt,minimum height=3.1ex,
rounded corners,drop shadow,shadow scale=1,shadow xshift=.8ex,
shadow yshift=-.4ex,opacity=.7,fill=black!50,top color=red!90!black!50,
bottom color=red!80!black!80,draw=red!50!black!50,very thick,text=white,
text opacity=1,minimum width=3cm,font=\bfseries\sffamily] at (4,0) {Назад};
}\Acrobatmenu{PrevPage}{%
\tikz[baseline] \node[rectangle,inner sep=2pt,minimum height=3.1ex,
rounded corners,drop shadow,shadow scale=1,shadow xshift=.8ex,
shadow yshift=-.4ex,opacity=.7,fill=black!50,top color=red!90!black!50,
bottom color=red!80!black!80,draw=red!50!black!50,very thick,text=white,
text opacity=1,minimum width=3cm,font=\bfseries\sffamily] at (8,0) {Предыдущий};
}\Acrobatmenu{NextPage}{%
\tikz[baseline] \node[rectangle,inner sep=2pt,minimum height=3.1ex,
rounded corners,drop shadow,shadow scale=1,shadow xshift=.8ex,
shadow yshift=-.4ex,opacity=.7,fill=black!50,top color=red!90!black!50,
bottom color=red!80!black!80,draw=red!50!black!50,very thick,text=white,
text opacity=1,minimum width=3cm,font=\bfseries\sffamily] at (12,0) {Следующий};
}\Acrobatmenu{GoForward}{%
\tikz[baseline] \node[rectangle,inner sep=2pt,minimum height=3.1ex,
rounded corners,drop shadow,shadow scale=1,shadow xshift=.8ex,
shadow yshift=-.4ex,opacity=.7,fill=black!50,top color=red!90!black!50,
bottom color=red!80!black!80,draw=red!50!black!50,very thick,text=white,
text opacity=1,minimum width=3cm,font=\bfseries\sffamily] at (16,0) {Вперед};
}\Acrobatmenu{FirstPage}{%
\tikz[baseline] \node[rectangle,inner sep=2pt,minimum height=3.1ex,
rounded corners,drop shadow,shadow scale=1,shadow xshift=.8ex,
shadow yshift=-.4ex,opacity=.7,fill=black!50,top color=red!90!black!50,
bottom color=red!80!black!80,draw=red!50!black!50,very thick,text=white,
text opacity=1,minimum width=3cm,font=\bfseries\sffamily] at (20,0) {К началу};
}\Acrobatmenu{FullScreen}{%
\tikz[baseline] \node[rectangle,inner sep=2pt,minimum height=3.1ex,
rounded corners,drop shadow,shadow scale=1,shadow xshift=.8ex,
shadow yshift=-.4ex,opacity=.7,fill=black!50,top color=red!90!black!50,
bottom color=red!80!black!80,draw=red!50!black!50,very thick,text=white,
text opacity=1,minimum width=3cm,font=\bfseries\sffamily] at (24,0) {Полный экран};
}\Acrobatmenu{Quit}{%
\tikz[baseline] \node[rectangle,inner sep=2pt,minimum height=3.1ex,
rounded corners,drop shadow,shadow scale=1,shadow xshift=.8ex,
shadow yshift=-.4ex,opacity=.7,fill=black!50,top color=red!90!black!50,
bottom color=red!80!black!80,draw=red!50!black!50,very thick,text=white,
text opacity=1,minimum width=3cm,font=\bfseries\sffamily] at (28,0) {Выход};
}	
}%%%---|
\end{textblock}
%%%-----------------------------------------------------------------------
%%%---> Table #3
\newpage
\begin{tikzpicture}[remember picture,overlay]
	  \node [rotate=0,scale=2,text opacity=0.2]
	      at (27,1.7) {Капранов~О.~Г.~\copyright~2010~~~Luga\TeX @yahoo.com};
\end{tikzpicture}
\vglue -18pt
\hspace{187pt}
\parbox{350pt}{%
\hypertarget{admintbl3}{\hyperlink{admintbl4}{\mbox{%
\begin{tikzpicture}
  \colorlet{even}{cyan!60!black}
  \colorlet{odd}{orange!100!black}
  \colorlet{links}{red!70!black}
  \colorlet{back}{yellow!20!white}
  \tikzset{
    box/.style={
      minimum height=15mm,
      inner sep=.7mm,
      outer sep=0mm,
      text width=120mm,
      text centered,
      font=\small\bfseries\sffamily,
      text=#1!50!black,
      draw=#1,
      line width=.25mm,
      top color=#1!5,
      bottom color=#1!40,
      shading angle=0,
      rounded corners=2.3mm,
      drop shadow={fill=#1!40!gray,fill opacity=.8},
      rotate=0,
    },
  }
  \node [box=even]{{%
  	\huge\textbf{Тематичний план за курсом <<Адміністративне право>>}}};
\end{tikzpicture}
}}}}
\hfill
\begin{flushright}
	\tikz \node [copy shadow={left color=green!50},tape,
		left color=green!50,draw=green,thick]
					 {{\large\textbf{Таблица №3}}};
\end{flushright}	
\hfill
\begin{center}
\begin{tikzpicture}
\node (tbl) {
\begin{tabularx}{976pt}{|>{\bfseries}c|>{\bfseries}p{430pt}|>{\bfseries}c|
	>{\bfseries}c|>{\bfseries}c|>{\bfseries}c|>{\bfseries}c|>{\bfseries}c|}
\arrayrulecolor{purple}
 & &\multicolumn{6}{c|}{\textcolor{white}{\bf Кількість годин за видами
 занять}}\\ \cline{3-8}\\
 & & &\multicolumn{5}{c|}{\raisebox{1.3ex}[0pt][0pt]{{\bf у тому числі:}}}\\ \cline{4-8}\\
 & & & & &\multicolumn{3}{c|}{\raisebox{1.3ex}[0pt][0pt]{{\bf із них:}}}\\ \cline{6-8}
\raisebox{5.5ex}[0pt][0pt]{Номера тем} &
 \mbox{}\hspace{130pt}\raisebox{5.5ex}[0pt][0pt]{Найменування розділів і тем} &
	\raisebox{1.5ex}[0pt][0pt]{Всього} &
		\raisebox{1.5ex}[0pt][0pt]{Самостійна робота} &
			\raisebox{1.5ex}[0pt][0pt]{Аудиторні} & 
				\raisebox{-0.8ex}[0pt][0pt]{Лекції} &
					\raisebox{-0.8ex}[0pt][0pt]{Семінарські заняття} &
						\raisebox{-0.8ex}[0pt][0pt]{Практичні заняття}\\[1ex]
\midrule
\multicolumn{8}{|>{\bfseries}c|}{Модуль-контроль №2}\\
\midrule
\multicolumn{8}{|>{\bfseries}c|}{ЗМІСТОВИЙ МОДУЛЬ III}\\
\midrule
\multicolumn{8}{|>{\bfseries}c|}{Основи адміністративно-правової організації
	управління в економічній, соціально-культурній та адміністративно-політичній сфері}\\
\midrule
8.0  & Забезпечення законності в державному управлінні & 9 & 5 & 4 & 2 & & 2\\
\midrule
8.1  & Законність і дисципліна в сфері виконавчої влади & & & & 2 & &\\
\midrule
8.2  & Контроль і нагляд у державному управлінні & & & & & & 2\\
\midrule
9.0  & Особливості адміністративно-правової організації управління & 9 & 5 & 4 & 2 & 2 &\\
\midrule
9.1  & Адміністративно-правова організація управління економікою,
соціально-культурною та адміністративно-політичною сферами & & & & 2 & &\\
\midrule
9.2  & Сутність й особливості міжгалузевого управління & & & & & 2 &\\
\midrule
10.0 & Адміністративне право та управління економікою & 9 & 5 & 2 & 2 & &\\
\midrule
10.1 & Адміністративно-правове регулювання в окремих галузях економіки & & & & 2 & &\\
\midrule
11.0 & Адміністративне право та управління соціально-культурною сферою & 9 & 5 & 4 & 2 & 2 &\\
\midrule
11.1 & Адміністративно-правове регулювання в окремих галузях
	соціально-культурної сфери & & & & 2 & &\\
\midrule
11.2 & Особливості управління соціально-культурним комплексом & & & & & 2 &\\
\midrule
12.0 & Адміністративне право та управління адміністративно-політичною сферою & 9 & 5 & 4 & 2 & & 2\\
\midrule
12.1 & Адміністративно-правове регулювання в сфері безпеки, оборони та юстиції & & & & 2 & &\\
\midrule
12.2 & Особливості управління в адміністративно-політичній сфері & & & & & & 2\\
\midrule
12.3 & Адміністративне право та управління в сфері охорони правопорядку & 9 & 7 & 4 & 2 & 2 &\\
\midrule
13.0 & Адміністративне право та управління в сфері охорони правопорядку & 9 & 7 & 4 & 2 & 2 &\\
\midrule
13.1 & Управління в сфері внутрішніх справ & & & & 2 & &\\
\midrule
13.2 & Особливості адміністративно-правових відносин за участю ОВС & & & & & 2 &\\[0.5ex]
\end{tabularx}};

\begin{pgfonlayer}{background}
\draw[rounded corners,top color=red,bottom color=black,draw=white]
	($(tbl.north west)+(0.14,0)$) rectangle ($(tbl.north east)-(0.13,0.9)$);
\draw[rounded corners,top color=white,bottom color=black,
	middle color=red,draw=blue!20] ($(tbl.south west) +(0.12,0.5)$)
		rectangle ($(tbl.south east)-(0.12,0)$);
\draw[top color=blue!1,bottom color=blue!20,draw=white]
	($(tbl.north east)-(0.13,0.6)$) rectangle ($(tbl.south west)+(0.13,0.2)$);
\end{pgfonlayer}
\end{tikzpicture}
\end{center}
\hfill
\centerline{%
\hyperlink{admintbl1}{\mbox{%
\tikz[baseline] \node[rectangle,inner sep=2pt,minimum height=3.1ex,rounded corners,
drop shadow,shadow scale=1,shadow xshift=.8ex,shadow yshift=-.4ex,opacity=.7,
fill=black!50,top color=blue!90!black!50,bottom color=blue!80!black!80,
draw=blue!50!black!50,very thick,text=white,text opacity=1,minimum width=4cm]{%
	Таблица №1};
	}}\qquad\qquad\qquad\qquad
\hyperlink{admintbl2}{\mbox{%
\tikz[baseline] \node[rectangle,inner sep=2pt,minimum height=3.1ex,rounded corners,
drop shadow,shadow scale=1,shadow xshift=.8ex,shadow yshift=-.4ex,opacity=.7,
fill=black!50,top color=blue!90!black!50,bottom color=blue!80!black!80,
draw=blue!50!black!50,very thick,text=white,text opacity=1,minimum width=4cm]{%
	Таблица №2};
	}}\qquad\qquad\qquad\qquad
\hyperlink{admintbl4}{\mbox{%
\tikz[baseline] \node[rectangle,inner sep=2pt,minimum height=3.1ex,rounded corners,
drop shadow,shadow scale=1,shadow xshift=.8ex,shadow yshift=-.4ex,opacity=.7,
fill=black!50,top color=blue!90!black!50,bottom color=blue!80!black!80,
draw=blue!50!black!50,very thick,text=white,text opacity=1,minimum width=4cm]{%
	Таблица №4};
	}}}
\hfill
\mbox{}
%%%-----------------------------------------------------------------------
\begin{textblock}{35}(25,-0.01)
\begin{tikzpicture}[even odd rule,rounded corners=2pt,x=10pt,y=10pt,drop shadow]
\filldraw[fill=yellow!90!black!40,drop shadow] (0,0)   rectangle (1,1)
	[xshift=5pt,yshift=5pt]   (0,0)   rectangle (1,1)
	[rotate=30]   (-1,-1) rectangle (2,2);
\node at (0,1.7) {\textbf{\thepage}};			      
\end{tikzpicture}
\end{textblock}
%%%--- Navigational panel top page
\begin{textblock}{36}(7.58,0.85)
\mbox{%%%--->
\Acrobatmenu{LastPage}{%
\tikz[baseline] \node[rectangle,inner sep=2pt,minimum height=3.1ex,
rounded corners,drop shadow,shadow scale=1,shadow xshift=.8ex,
shadow yshift=-.4ex,opacity=.7,fill=black!50,top color=red!90!black!50,
bottom color=red!80!black!80,draw=red!50!black!50,very thick,text=white,
text opacity=1,minimum width=3cm,font=\bfseries\sffamily] at (0,0) {К концу};
}\Acrobatmenu{GoBack}{%
\tikz[baseline] \node[rectangle,inner sep=2pt,minimum height=3.1ex,
rounded corners,drop shadow,shadow scale=1,shadow xshift=.8ex,
shadow yshift=-.4ex,opacity=.7,fill=black!50,top color=red!90!black!50,
bottom color=red!80!black!80,draw=red!50!black!50,very thick,text=white,
text opacity=1,minimum width=3cm,font=\bfseries\sffamily] at (4,0) {Назад};
}\Acrobatmenu{PrevPage}{%
\tikz[baseline] \node[rectangle,inner sep=2pt,minimum height=3.1ex,
rounded corners,drop shadow,shadow scale=1,shadow xshift=.8ex,
shadow yshift=-.4ex,opacity=.7,fill=black!50,top color=red!90!black!50,
bottom color=red!80!black!80,draw=red!50!black!50,very thick,text=white,
text opacity=1,minimum width=3cm,font=\bfseries\sffamily] at (8,0) {Предыдущий};
}\Acrobatmenu{NextPage}{%
\tikz[baseline] \node[rectangle,inner sep=2pt,minimum height=3.1ex,
rounded corners,drop shadow,shadow scale=1,shadow xshift=.8ex,
shadow yshift=-.4ex,opacity=.7,fill=black!50,top color=red!90!black!50,
bottom color=red!80!black!80,draw=red!50!black!50,very thick,text=white,
text opacity=1,minimum width=3cm,font=\bfseries\sffamily] at (12,0) {Следующий};
}\Acrobatmenu{GoForward}{%
\tikz[baseline] \node[rectangle,inner sep=2pt,minimum height=3.1ex,
rounded corners,drop shadow,shadow scale=1,shadow xshift=.8ex,
shadow yshift=-.4ex,opacity=.7,fill=black!50,top color=red!90!black!50,
bottom color=red!80!black!80,draw=red!50!black!50,very thick,text=white,
text opacity=1,minimum width=3cm,font=\bfseries\sffamily] at (16,0) {Вперед};
}\Acrobatmenu{FirstPage}{%
\tikz[baseline] \node[rectangle,inner sep=2pt,minimum height=3.1ex,
rounded corners,drop shadow,shadow scale=1,shadow xshift=.8ex,
shadow yshift=-.4ex,opacity=.7,fill=black!50,top color=red!90!black!50,
bottom color=red!80!black!80,draw=red!50!black!50,very thick,text=white,
text opacity=1,minimum width=3cm,font=\bfseries\sffamily] at (20,0) {К началу};
}\Acrobatmenu{FullScreen}{%
\tikz[baseline] \node[rectangle,inner sep=2pt,minimum height=3.1ex,
rounded corners,drop shadow,shadow scale=1,shadow xshift=.8ex,
shadow yshift=-.4ex,opacity=.7,fill=black!50,top color=red!90!black!50,
bottom color=red!80!black!80,draw=red!50!black!50,very thick,text=white,
text opacity=1,minimum width=3cm,font=\bfseries\sffamily] at (24,0) {Полный экран};
}\Acrobatmenu{Quit}{%
\tikz[baseline] \node[rectangle,inner sep=2pt,minimum height=3.1ex,
rounded corners,drop shadow,shadow scale=1,shadow xshift=.8ex,
shadow yshift=-.4ex,opacity=.7,fill=black!50,top color=red!90!black!50,
bottom color=red!80!black!80,draw=red!50!black!50,very thick,text=white,
text opacity=1,minimum width=3cm,font=\bfseries\sffamily] at (28,0) {Выход};
}	
}%%%---|
\end{textblock}
%%%-----------------------------------------------------------------------
%%%---> Table #4
\newpage
\begin{tikzpicture}[remember picture,overlay]
	  \node [rotate=0,scale=2,text opacity=0.2]
	      at (27,1.7) {Капранов~О.~Г.~\copyright~2010~~~Luga\TeX @yahoo.com};
\end{tikzpicture}
\vglue -18pt
\hspace{187pt}
\parbox{350pt}{%
\hypertarget{admintbl4}{\hyperlink{chapter4a}{\mbox{%
\begin{tikzpicture}
  \colorlet{even}{cyan!60!black}
  \colorlet{odd}{orange!100!black}
  \colorlet{links}{red!70!black}
  \colorlet{back}{yellow!20!white}
  \tikzset{
    box/.style={
      minimum height=15mm,
      inner sep=.7mm,
      outer sep=0mm,
      text width=120mm,
      text centered,
      font=\small\bfseries\sffamily,
      text=#1!50!black,
      draw=#1,
      line width=.25mm,
      top color=#1!5,
      bottom color=#1!40,
      shading angle=0,
      rounded corners=2.3mm,
      drop shadow={fill=#1!40!gray,fill opacity=.8},
      rotate=0,
    },
  }
  \node [box=even]{{%
  	\huge\textbf{Тематичний план за курсом <<Адміністративне право>>}}};
\end{tikzpicture}
}}}}
\hfill
\begin{flushright}
	\tikz \node [copy shadow={left color=green!50},tape,
		left color=green!50,draw=green,thick]
					 {{\large\textbf{Таблица №4}}};
\end{flushright}	
\hfill
\begin{center}
\begin{tikzpicture}
\node (tbl) {
\begin{tabularx}{976pt}{|>{\bfseries}c|>{\bfseries}p{430pt}|>{\bfseries}c|
	>{\bfseries}c|>{\bfseries}c|>{\bfseries}c|>{\bfseries}c|>{\bfseries}c|}
\arrayrulecolor{purple}
 & &\multicolumn{6}{c|}{\textcolor{white}{\bf Кількість годин за видами
 занять}}\\ \cline{3-8}\\
 & & &\multicolumn{5}{c|}{\raisebox{1.3ex}[0pt][0pt]{{\bf у тому числі:}}}\\ \cline{4-8}\\
 & & & & &\multicolumn{3}{c|}{\raisebox{1.3ex}[0pt][0pt]{{\bf із них:}}}\\ \cline{6-8}
\raisebox{5.5ex}[0pt][0pt]{Номера тем} &
 \mbox{}\hspace{130pt}\raisebox{5.5ex}[0pt][0pt]{Найменування розділів і тем} &
	\raisebox{1.5ex}[0pt][0pt]{Всього} &
		\raisebox{1.5ex}[0pt][0pt]{Самостійна робота} &
			\raisebox{1.5ex}[0pt][0pt]{Аудиторні} & 
				\raisebox{-0.8ex}[0pt][0pt]{Лекції} &
					\raisebox{-0.8ex}[0pt][0pt]{Семінарські заняття} &
						\raisebox{-0.8ex}[0pt][0pt]{Практичні заняття}\\[1ex]
\midrule
\multicolumn{8}{|>{\bfseries}c|}{Модуль-контроль №3}\\
\midrule
\multicolumn{8}{|>{\bfseries}c|}{ЗМІСТОВИЙ МОДУЛЬ IV}\\
\midrule
\multicolumn{8}{|>{\bfseries}c|}{Адміністративна відповідальність}\\
\midrule
14.0 & Загально-правові засади інституту адміністративної відповідальності & 12 & 6 & 6 & 2 & 4 &\\
\midrule
14.1 & Поняття адміністративної відповідальності та її особливості & & & & 2 & &\\
\midrule
14.2 & Поняття, ознаки і юридичний склад адміністративного правопорушення & & & & & 2 &\\
\midrule
14.3 & Система адміністративних стягнень & & & & & 2 &\\
\midrule
15.0 & Антикорупційне законодавство & 6 & 2 & 4 & 2 & & 2\\
\midrule
15.1 & Характеристика корупційних правопорушень & & & & 2 & &\\
\midrule
15.2 & Адміністративна відповідальність за корупційні правопорушення & & & & & & 2\\
\midrule
16.0 & Адміністративні правопорушення, що посягають на громадський порядок,
громадську безпеку та встановлений порядок управління & 12 & 4 & 8 & 2 & 2 & 4\\
\midrule
16.1 & Юридична характеристика адміністративних право-порушень, що посягають на
громадський порядок, громадську безпеку та встановлений порядок управління & & & & 2 & 2 &\\
\midrule
16.2 & Адміністративні правопорушення, що посягають на громадський порядок та
громадську безпеку. & & & & & & 2\\
\midrule
16.3 & Адміністративні правопорушення, що посягають на встановлений порядок
управління. & & & & & & 2\\
\midrule
17.0 & Юридична характеристика адміністративних правопорушень в окремих сферах
економіки, соціально-культурного комплексу та у галузі
адміністративно-політичної діяльності & 24 & 16 & 8 & 2 & 2 & 4\\
\midrule
17.1 & Особливості складів адміністративних правопорушень в окремих галузях
економічної, соціально-культурної та адміністративно-політичної діяльності & & & & 2 & &\\
\midrule
17.2 & Адміністративні правопорушення в галузі охорони праці та здоров'я населення & & & & & 2 &\\
\midrule
17.3 & Адміністративні правопорушення на транспорті, в галузі шляхового
господарства і зв'язку. & & & & & 2 &\\
\midrule
17.4   & Адміністративні правопорушення в галузі торгівлі, фінансів і
	підприємницької діяльності & & & & & & 2\\
\midrule
\multicolumn{8}{|>{\bfseries}c|}{Модуль-контроль №4}\\
\midrule
 & Разом & 198 & 100 & 98 & 46 & 26 & 26\\[0.5ex]
\end{tabularx}};

\begin{pgfonlayer}{background}
\draw[rounded corners,top color=red,bottom color=black,draw=white]
	($(tbl.north west)+(0.14,0)$) rectangle ($(tbl.north east)-(0.13,0.9)$);
\draw[rounded corners,top color=white,bottom color=black,
	middle color=red,draw=blue!20] ($(tbl.south west) +(0.12,0.5)$)
		rectangle ($(tbl.south east)-(0.12,0)$);
\draw[top color=blue!1,bottom color=blue!20,draw=white]
	($(tbl.north east)-(0.13,0.6)$) rectangle ($(tbl.south west)+(0.13,0.2)$);
\end{pgfonlayer}
\end{tikzpicture}
\end{center}
\hfill
\centerline{%
\hyperlink{admintbl1}{\mbox{%
\tikz[baseline] \node[rectangle,inner sep=2pt,minimum height=3.1ex,rounded corners,
drop shadow,shadow scale=1,shadow xshift=.8ex,shadow yshift=-.4ex,opacity=.7,
fill=black!50,top color=blue!90!black!50,bottom color=blue!80!black!80,
draw=blue!50!black!50,very thick,text=white,text opacity=1,minimum width=4cm]{%
	Таблица №1};
	}}\qquad\qquad\qquad\qquad
\hyperlink{admintbl2}{\mbox{%
\tikz[baseline] \node[rectangle,inner sep=2pt,minimum height=3.1ex,rounded corners,
drop shadow,shadow scale=1,shadow xshift=.8ex,shadow yshift=-.4ex,opacity=.7,
fill=black!50,top color=blue!90!black!50,bottom color=blue!80!black!80,
draw=blue!50!black!50,very thick,text=white,text opacity=1,minimum width=4cm]{%
	Таблица №2};
	}}\qquad\qquad\qquad\qquad
\hyperlink{admintbl3}{\mbox{%
\tikz[baseline] \node[rectangle,inner sep=2pt,minimum height=3.1ex,rounded corners,
drop shadow,shadow scale=1,shadow xshift=.8ex,shadow yshift=-.4ex,opacity=.7,
fill=black!50,top color=blue!90!black!50,bottom color=blue!80!black!80,
draw=blue!50!black!50,very thick,text=white,text opacity=1,minimum width=4cm]{%
	Таблица №3};
	}}}
\hfill
\mbox{}
%%%-----------------------------------------------------------------------
\begin{textblock}{37}(25,-0.01)
\begin{tikzpicture}[even odd rule,rounded corners=2pt,x=10pt,y=10pt,drop shadow]
\filldraw[fill=yellow!90!black!40,drop shadow] (0,0)   rectangle (1,1)
	[xshift=5pt,yshift=5pt]   (0,0)   rectangle (1,1)
	[rotate=30]   (-1,-1) rectangle (2,2);
\node at (0,1.7) {\textbf{\thepage}};			      
\end{tikzpicture}
\end{textblock}
%%%--- Navigational panel top page
\begin{textblock}{38}(7.58,0.85)
\mbox{%%%--->
\Acrobatmenu{LastPage}{%
\tikz[baseline] \node[rectangle,inner sep=2pt,minimum height=3.1ex,
rounded corners,drop shadow,shadow scale=1,shadow xshift=.8ex,
shadow yshift=-.4ex,opacity=.7,fill=black!50,top color=red!90!black!50,
bottom color=red!80!black!80,draw=red!50!black!50,very thick,text=white,
text opacity=1,minimum width=3cm,font=\bfseries\sffamily] at (0,0) {К концу};
}\Acrobatmenu{GoBack}{%
\tikz[baseline] \node[rectangle,inner sep=2pt,minimum height=3.1ex,
rounded corners,drop shadow,shadow scale=1,shadow xshift=.8ex,
shadow yshift=-.4ex,opacity=.7,fill=black!50,top color=red!90!black!50,
bottom color=red!80!black!80,draw=red!50!black!50,very thick,text=white,
text opacity=1,minimum width=3cm,font=\bfseries\sffamily] at (4,0) {Назад};
}\Acrobatmenu{PrevPage}{%
\tikz[baseline] \node[rectangle,inner sep=2pt,minimum height=3.1ex,
rounded corners,drop shadow,shadow scale=1,shadow xshift=.8ex,
shadow yshift=-.4ex,opacity=.7,fill=black!50,top color=red!90!black!50,
bottom color=red!80!black!80,draw=red!50!black!50,very thick,text=white,
text opacity=1,minimum width=3cm,font=\bfseries\sffamily] at (8,0) {Предыдущий};
}\Acrobatmenu{NextPage}{%
\tikz[baseline] \node[rectangle,inner sep=2pt,minimum height=3.1ex,
rounded corners,drop shadow,shadow scale=1,shadow xshift=.8ex,
shadow yshift=-.4ex,opacity=.7,fill=black!50,top color=red!90!black!50,
bottom color=red!80!black!80,draw=red!50!black!50,very thick,text=white,
text opacity=1,minimum width=3cm,font=\bfseries\sffamily] at (12,0) {Следующий};
}\Acrobatmenu{GoForward}{%
\tikz[baseline] \node[rectangle,inner sep=2pt,minimum height=3.1ex,
rounded corners,drop shadow,shadow scale=1,shadow xshift=.8ex,
shadow yshift=-.4ex,opacity=.7,fill=black!50,top color=red!90!black!50,
bottom color=red!80!black!80,draw=red!50!black!50,very thick,text=white,
text opacity=1,minimum width=3cm,font=\bfseries\sffamily] at (16,0) {Вперед};
}\Acrobatmenu{FirstPage}{%
\tikz[baseline] \node[rectangle,inner sep=2pt,minimum height=3.1ex,
rounded corners,drop shadow,shadow scale=1,shadow xshift=.8ex,
shadow yshift=-.4ex,opacity=.7,fill=black!50,top color=red!90!black!50,
bottom color=red!80!black!80,draw=red!50!black!50,very thick,text=white,
text opacity=1,minimum width=3cm,font=\bfseries\sffamily] at (20,0) {К началу};
}\Acrobatmenu{FullScreen}{%
\tikz[baseline] \node[rectangle,inner sep=2pt,minimum height=3.1ex,
rounded corners,drop shadow,shadow scale=1,shadow xshift=.8ex,
shadow yshift=-.4ex,opacity=.7,fill=black!50,top color=red!90!black!50,
bottom color=red!80!black!80,draw=red!50!black!50,very thick,text=white,
text opacity=1,minimum width=3cm,font=\bfseries\sffamily] at (24,0) {Полный экран};
}\Acrobatmenu{Quit}{%
\tikz[baseline] \node[rectangle,inner sep=2pt,minimum height=3.1ex,
rounded corners,drop shadow,shadow scale=1,shadow xshift=.8ex,
shadow yshift=-.4ex,opacity=.7,fill=black!50,top color=red!90!black!50,
bottom color=red!80!black!80,draw=red!50!black!50,very thick,text=white,
text opacity=1,minimum width=3cm,font=\bfseries\sffamily] at (28,0) {Выход};
}	
}%%%---|
\end{textblock}
%%%-----------------------------------------------------------------------

\newpage
\begin{tikzpicture}[remember picture,overlay]
	  \node [rotate=0,scale=2,text opacity=0.2]
	      at (27,1.7) {Капранов~О.~Г.~\copyright~2010~~~Luga\TeX @yahoo.com};
\end{tikzpicture}
\vglue -18pt
\hspace{187pt}
\parbox{350pt}{%
\hypertarget{chapter4a}{\hyperlink{chapter4b}{\mbox{%
\begin{tikzpicture}
  \colorlet{even}{cyan!60!black}
  \colorlet{odd}{orange!100!black}
  \colorlet{links}{red!70!black}
  \colorlet{back}{yellow!20!white}
  \tikzset{
    box/.style={
      minimum height=15mm,
      inner sep=.7mm,
      outer sep=0mm,
      text width=120mm,
      text centered,
      font=\small\bfseries\sffamily,
      text=#1!50!black,
      draw=#1,
      line width=.25mm,
      top color=#1!5,
      bottom color=#1!40,
      shading angle=0,
      rounded corners=2.3mm,
      drop shadow={fill=#1!40!gray,fill opacity=.8},
      rotate=0,
    },
  }
  \node [box=even] {{%
  	\huge\textbf{Методичні рекомендації,}}
	\textbf{плани й завдання до семінарських і практичних занять}};
\end{tikzpicture}
}}}}\\[5pt]
\noindent
\begin{tikzpicture}
  \colorlet{even}{cyan!60!black}
  \colorlet{odd}{orange!100!black}
  \colorlet{links}{red!70!black}
  \colorlet{back}{yellow!20!white}
  \tikzset{
    box/.style={
      minimum height=15mm,
      inner sep=.7mm,
      outer sep=0mm,
      text width=120mm,
      text centered,
      font=\small\bfseries\sffamily,
      text=#1!50!black,
      draw=#1,
      line width=.25mm,
      top color=#1!5,
      bottom color=#1!40,
      shading angle=0,
      rounded corners=2.3mm,
      drop shadow={fill=#1!40!gray,fill opacity=.8},
      rotate=0,
    },
  }
	\node[box=links,xshift=3mm,yshift=1mm,
		minimum height=5pt,text width=325pt]
		at (0,-1.3) {\hyperlink{chapter4a}{\mbox{%
		\small \textcolor{black}{Тема 1.2. Співвідношення
			адміністративного права з іншими галузями}}}};
	\node[box=links,xshift=3mm,yshift=1mm,
		minimum height=5pt,text width=235pt]
		at (10.6,-1.3) {\hyperlink{chapter4b}{\mbox{%
		\small \textcolor{black}{Індивідуальні
		навчально-дослідницькі завдання}}}};
	\node[box=links,xshift=3mm,yshift=1mm,
		minimum height=5pt,text width=221pt]
		at (19.3,-1.3) {\hyperlink{chapter4b}{\mbox{%
		\small \textcolor{black}{
		Питання для самоконтролю та самоперевірки}}}};
	\node[box=links,xshift=3mm,yshift=1mm,
		minimum height=5pt,text width=140pt]
		at (26.35,-1.3) {\hyperlink{chapter4c}{\mbox{%
		\small \textcolor{black}{Додаткова література}}}};
\end{tikzpicture}
\vglue 5pt
\tikzfading[name=targetask, top color=transparent!90,
	bottom color=transparent!90,middle color=transparent!65]
\begin{tikzpicture}
\node [rounded corners,fill=magenta!50,minimum width=960pt,
minimum height=35pt,path fading=targetask] at (0,0) {\mbox{}};
\end{tikzpicture}
\begin{textblock}{39}(0.8,3.75)
\begin{tikzpicture}
	\node [text width=920pt] {\parbox{900pt}{%
	\textbf{Мета заняття:} оволодіння теоретичними положеннями й результатами науки
адміністративного права, її правовими категоріями щодо поняття, предмета,
методу й системи адміністративного права;
\mbox{}\hspace{35pt}розгляд організаційних і настановних питань щодо вивчення курсу
адміністративного права, порядку й форм проведення занять, форм педагогічного
контролю й т.п.}};
\end{tikzpicture}
 \end{textblock}
\begin{textblock}{40}(17,5)
\tikzstyle{abstract}=[rectangle, draw=black, rounded corners, fill=blue!40, drop shadow,
	text centered, text=white, text width=4cm,font=\large\bfseries\sffamily]
\tikzstyle{comment}=[rectangle, draw=black, rounded corners, fill=green, drop shadow,
	text centered, text=white, text width=10cm,
	font=\Large\bfseries\sffamily]
\tikzstyle{myarrow}=[->, >=open triangle 90, thick]
\tikzstyle{line}=[-, thick]
\begin{tikzpicture}[node distance=2cm]
    \node (Seminar) [abstract, rectangle split,
		rectangle split parts=2,text width=5cm] at (0,0)
        {
            Семінарське заняття
            \nodepart{second}1 год
        };

    \node (Thema) [abstract, rectangle split, rectangle split parts=2,
		text width=10cm] at (0,-2) {
            Співвідношення адміністративного права
            \nodepart{second}з іншими галузями
        };

    \node (AuxNode01) [text width=4cm] {};
    \node (Maindir) [abstract, rectangle split,
		rectangle split parts=2] at (0,-4)
        {
            Основні поняття
            \nodepart{second}:
        };

	\node (Maindirnames) [comment, rectangle split,
		rectangle split parts=2, text justified] at (0,-10)
	{
		\mbox{}
		\nodepart{second}\mbox{}
		\newline предмет права
		\newline галузь права
		\newline інститут права
		\newline публічне право
		\newline приватне право
		\newline принципи права
		\newline система права
		\newline метод правового регулювання
		\newline механізм правового регулювання
		\newline \mbox{}
	};
\end{tikzpicture}
\end{textblock}

\vglue 15pt

\tikzstyle{mybox} = [draw=red, fill=blue!20, very thick,
	rectangle, rounded corners, inner sep=10pt, inner ysep=20pt,
	font=\large\bfseries\sffamily]
\tikzstyle{fancytitle} =[fill=red, text=white,font=\large\bfseries\sffamily]

\begin{tikzpicture}
\node [mybox] (box) {%
	\begin{minipage}{0.50\textwidth}
		\begin{itemize}
			\item Особливості предмета й методу адміністративно-правового регулювання.
			\item Співвідношення адміністративного права із суміжними галузями права:
				\begin{itemize}
					\item конституційним правом;
					\item цивільним правом;
					\item кримінальним правом;
				\end{itemize}						
			\item трудовим правом.
			\item Роль адміністративного права на сучасному етапі розвитку суспільних
				відносин.
		\end{itemize}
	\end{minipage}
};
\node[fancytitle, right=10pt] at (box.north west) {Навчальні питання:};
\node[fancytitle, rounded corners] at (box.east) {{\bf ?}};
\end{tikzpicture}	

\vglue 15pt

\tikzstyle{mybox} = [draw=blue, fill=green!20, very thick,
	rectangle, rounded corners, inner sep=10pt, inner ysep=20pt,
	line width=1pt,font=\bfseries\sffamily]
\tikzstyle{fancytitle} =[fill=blue, text=white, ellipse,font=\bfseries\sffamily]

\begin{tikzpicture}[transform shape, rotate=0, baseline=-3.5cm]
	\node [mybox] (box) {%
		\begin{minipage}[t!]{0.5\textwidth}
При підготовці до заняття по даній темі курсанти та студенти повинні усвідомити
предмет, метод і систему, а також соціальне призначення адміністративного
права, ознайомитися з навчальною літературою, вміти розмежовувати категорії
„адміністративне право” як галузі права, науки й навчальної дисципліни,
з'ясувати особливості й зміни предмету, методу й системи галузі права в умовах
побудови демократичної правової держави й переходу до ринкових відносин.
Курсантам і студентам необхідно усвідомити місце адміністративного права серед
фундаментальних галузей права, його взаємозв'язок з конституційним, цивільним,
кримінальним, земельним, фінансовим, трудовим, житловим, природоохоронним й
іншими галузями права.

Визначаючи предмет галузі, необхідно визначити види суспільних відносин
управлінського характеру й підстави включення їх до предмету адміністративного
права. Важливо знати ознаки методів (прийомів і способів)
адміністративно-правового регулювання, юридичну характеристику його трьох
основних компонентів: припису, заборони й дозволу.


Особливу увагу курсанти та студенти повинні приділити питанням системи
адміністративного права, розподілу його на Загальну, Особливу й Спеціальну
частини, а також правовим інститутам державної служби, адміністративної
відповідальності, місцевого самоврядування.

З урахуванням Концепції реформи адміністративного права України варто вивчити
її положення, що стосуються збагачення й уточнення предмета
адміністративно-правового регулювання й системи адміністративного права,
викликані здійсненням адміністративної реформи й побудовою демократичної
правової держави в Україні.


Для кращого засвоєння теми пропонується виконати завдання для самостійної
роботи та індивідуальні навчально-дослідницькі завдання.


Рівень своїх знань з цієї теми пропонується перевірити шляхом надання
відповідей на питання для самоконтролю та самоперевірки.
		\end{minipage}
		};
\node[fancytitle] at (box.north) {Методичні рекомендації та пояснення};
\end{tikzpicture}
%%%-----------------------------------------------------------------------
\begin{textblock}{41}(25,-0.01)
\begin{tikzpicture}[even odd rule,rounded corners=2pt,x=10pt,y=10pt,drop shadow]
\filldraw[fill=yellow!90!black!40,drop shadow] (0,0)   rectangle (1,1)
	[xshift=5pt,yshift=5pt]   (0,0)   rectangle (1,1)
	[rotate=30]   (-1,-1) rectangle (2,2);
\node at (0,1.7) {\textbf{\thepage}};			      
\end{tikzpicture}
\end{textblock}
%%%--- Navigational panel top page
\begin{textblock}{42}(7.58,0.85)
\mbox{%%%--->
\Acrobatmenu{LastPage}{%
\tikz[baseline] \node[rectangle,inner sep=2pt,minimum height=3.1ex,
rounded corners,drop shadow,shadow scale=1,shadow xshift=.8ex,
shadow yshift=-.4ex,opacity=.7,fill=black!50,top color=red!90!black!50,
bottom color=red!80!black!80,draw=red!50!black!50,very thick,text=white,
text opacity=1,minimum width=3cm,font=\bfseries\sffamily] at (0,0) {К концу};
}\Acrobatmenu{GoBack}{%
\tikz[baseline] \node[rectangle,inner sep=2pt,minimum height=3.1ex,
rounded corners,drop shadow,shadow scale=1,shadow xshift=.8ex,
shadow yshift=-.4ex,opacity=.7,fill=black!50,top color=red!90!black!50,
bottom color=red!80!black!80,draw=red!50!black!50,very thick,text=white,
text opacity=1,minimum width=3cm,font=\bfseries\sffamily] at (4,0) {Назад};
}\Acrobatmenu{PrevPage}{%
\tikz[baseline] \node[rectangle,inner sep=2pt,minimum height=3.1ex,
rounded corners,drop shadow,shadow scale=1,shadow xshift=.8ex,
shadow yshift=-.4ex,opacity=.7,fill=black!50,top color=red!90!black!50,
bottom color=red!80!black!80,draw=red!50!black!50,very thick,text=white,
text opacity=1,minimum width=3cm,font=\bfseries\sffamily] at (8,0) {Предыдущий};
}\Acrobatmenu{NextPage}{%
\tikz[baseline] \node[rectangle,inner sep=2pt,minimum height=3.1ex,
rounded corners,drop shadow,shadow scale=1,shadow xshift=.8ex,
shadow yshift=-.4ex,opacity=.7,fill=black!50,top color=red!90!black!50,
bottom color=red!80!black!80,draw=red!50!black!50,very thick,text=white,
text opacity=1,minimum width=3cm,font=\bfseries\sffamily] at (12,0) {Следующий};
}\Acrobatmenu{GoForward}{%
\tikz[baseline] \node[rectangle,inner sep=2pt,minimum height=3.1ex,
rounded corners,drop shadow,shadow scale=1,shadow xshift=.8ex,
shadow yshift=-.4ex,opacity=.7,fill=black!50,top color=red!90!black!50,
bottom color=red!80!black!80,draw=red!50!black!50,very thick,text=white,
text opacity=1,minimum width=3cm,font=\bfseries\sffamily] at (16,0) {Вперед};
}\Acrobatmenu{FirstPage}{%
\tikz[baseline] \node[rectangle,inner sep=2pt,minimum height=3.1ex,
rounded corners,drop shadow,shadow scale=1,shadow xshift=.8ex,
shadow yshift=-.4ex,opacity=.7,fill=black!50,top color=red!90!black!50,
bottom color=red!80!black!80,draw=red!50!black!50,very thick,text=white,
text opacity=1,minimum width=3cm,font=\bfseries\sffamily] at (20,0) {К началу};
}\Acrobatmenu{FullScreen}{%
\tikz[baseline] \node[rectangle,inner sep=2pt,minimum height=3.1ex,
rounded corners,drop shadow,shadow scale=1,shadow xshift=.8ex,
shadow yshift=-.4ex,opacity=.7,fill=black!50,top color=red!90!black!50,
bottom color=red!80!black!80,draw=red!50!black!50,very thick,text=white,
text opacity=1,minimum width=3cm,font=\bfseries\sffamily] at (24,0) {Полный экран};
}\Acrobatmenu{Quit}{%
\tikz[baseline] \node[rectangle,inner sep=2pt,minimum height=3.1ex,
rounded corners,drop shadow,shadow scale=1,shadow xshift=.8ex,
shadow yshift=-.4ex,opacity=.7,fill=black!50,top color=red!90!black!50,
bottom color=red!80!black!80,draw=red!50!black!50,very thick,text=white,
text opacity=1,minimum width=3cm,font=\bfseries\sffamily] at (28,0) {Выход};
}	
}%%%---|
\end{textblock}
%%%-----------------------------------------------------------------------
%%%---> NEW PAGE ---------------------------------------------------------
\newpage
\begin{tikzpicture}[remember picture,overlay]
	  \node [rotate=0,scale=2,text opacity=0.2]
	      at (27,1.7) {Капранов~О.~Г.~\copyright~2010~~~Luga\TeX @yahoo.com};
\end{tikzpicture}
%%%-----------------------------------------------------------------------
\begin{textblock}{43}(25,-0.01)
\begin{tikzpicture}[even odd rule,rounded corners=2pt,x=10pt,y=10pt,drop shadow]
\filldraw[fill=yellow!90!black!40,drop shadow] (0,0)   rectangle (1,1)
	[xshift=5pt,yshift=5pt]   (0,0)   rectangle (1,1)
	[rotate=30]   (-1,-1) rectangle (2,2);
\node at (0,1.7) {\textbf{\thepage}};			      
\end{tikzpicture}
\end{textblock}
%%%--- Navigational panel top page
\begin{textblock}{44}(7.58,0.85)
\mbox{%%%--->
\Acrobatmenu{LastPage}{%
\tikz[baseline] \node[rectangle,inner sep=2pt,minimum height=3.1ex,
rounded corners,drop shadow,shadow scale=1,shadow xshift=.8ex,
shadow yshift=-.4ex,opacity=.7,fill=black!50,top color=red!90!black!50,
bottom color=red!80!black!80,draw=red!50!black!50,very thick,text=white,
text opacity=1,minimum width=3cm,font=\bfseries\sffamily] at (0,0) {К концу};
}\Acrobatmenu{GoBack}{%
\tikz[baseline] \node[rectangle,inner sep=2pt,minimum height=3.1ex,
rounded corners,drop shadow,shadow scale=1,shadow xshift=.8ex,
shadow yshift=-.4ex,opacity=.7,fill=black!50,top color=red!90!black!50,
bottom color=red!80!black!80,draw=red!50!black!50,very thick,text=white,
text opacity=1,minimum width=3cm,font=\bfseries\sffamily] at (4,0) {Назад};
}\Acrobatmenu{PrevPage}{%
\tikz[baseline] \node[rectangle,inner sep=2pt,minimum height=3.1ex,
rounded corners,drop shadow,shadow scale=1,shadow xshift=.8ex,
shadow yshift=-.4ex,opacity=.7,fill=black!50,top color=red!90!black!50,
bottom color=red!80!black!80,draw=red!50!black!50,very thick,text=white,
text opacity=1,minimum width=3cm,font=\bfseries\sffamily] at (8,0) {Предыдущий};
}\Acrobatmenu{NextPage}{%
\tikz[baseline] \node[rectangle,inner sep=2pt,minimum height=3.1ex,
rounded corners,drop shadow,shadow scale=1,shadow xshift=.8ex,
shadow yshift=-.4ex,opacity=.7,fill=black!50,top color=red!90!black!50,
bottom color=red!80!black!80,draw=red!50!black!50,very thick,text=white,
text opacity=1,minimum width=3cm,font=\bfseries\sffamily] at (12,0) {Следующий};
}\Acrobatmenu{GoForward}{%
\tikz[baseline] \node[rectangle,inner sep=2pt,minimum height=3.1ex,
rounded corners,drop shadow,shadow scale=1,shadow xshift=.8ex,
shadow yshift=-.4ex,opacity=.7,fill=black!50,top color=red!90!black!50,
bottom color=red!80!black!80,draw=red!50!black!50,very thick,text=white,
text opacity=1,minimum width=3cm,font=\bfseries\sffamily] at (16,0) {Вперед};
}\Acrobatmenu{FirstPage}{%
\tikz[baseline] \node[rectangle,inner sep=2pt,minimum height=3.1ex,
rounded corners,drop shadow,shadow scale=1,shadow xshift=.8ex,
shadow yshift=-.4ex,opacity=.7,fill=black!50,top color=red!90!black!50,
bottom color=red!80!black!80,draw=red!50!black!50,very thick,text=white,
text opacity=1,minimum width=3cm,font=\bfseries\sffamily] at (20,0) {К началу};
}\Acrobatmenu{FullScreen}{%
\tikz[baseline] \node[rectangle,inner sep=2pt,minimum height=3.1ex,
rounded corners,drop shadow,shadow scale=1,shadow xshift=.8ex,
shadow yshift=-.4ex,opacity=.7,fill=black!50,top color=red!90!black!50,
bottom color=red!80!black!80,draw=red!50!black!50,very thick,text=white,
text opacity=1,minimum width=3cm,font=\bfseries\sffamily] at (24,0) {Полный экран};
}\Acrobatmenu{Quit}{%
\tikz[baseline] \node[rectangle,inner sep=2pt,minimum height=3.1ex,
rounded corners,drop shadow,shadow scale=1,shadow xshift=.8ex,
shadow yshift=-.4ex,opacity=.7,fill=black!50,top color=red!90!black!50,
bottom color=red!80!black!80,draw=red!50!black!50,very thick,text=white,
text opacity=1,minimum width=3cm,font=\bfseries\sffamily] at (28,0) {Выход};
}	
}%%%---|
\end{textblock}
%%%-----------------------------------------------------------------------
\vglue -18pt
\hspace{187pt}
\parbox{350pt}{%
\hypertarget{chapter4b}{\hyperlink{chapter4c}{\mbox{%
\begin{tikzpicture}
  \colorlet{even}{cyan!60!black}
  \colorlet{odd}{orange!100!black}
  \colorlet{links}{red!70!black}
  \colorlet{back}{yellow!20!white}
  \tikzset{
    box/.style={
      minimum height=15mm,
      inner sep=.7mm,
      outer sep=0mm,
      text width=120mm,
      text centered,
      font=\small\bfseries\sffamily,
      text=#1!50!black,
      draw=#1,
      line width=.25mm,
      top color=#1!5,
      bottom color=#1!40,
      shading angle=0,
      rounded corners=2.3mm,
      drop shadow={fill=#1!40!gray,fill opacity=.8},
      rotate=0,
    },
  }
  \node [box=even] {{%
  	\huge\textbf{Методичні рекомендації,}}
	\textbf{плани й завдання до семінарських і практичних занять}};
\end{tikzpicture}
}}}}\\[5pt]
\noindent
\begin{tikzpicture}
  \colorlet{even}{cyan!60!black}
  \colorlet{odd}{orange!100!black}
  \colorlet{links}{red!70!black}
  \colorlet{back}{yellow!20!white}
  \tikzset{
    box/.style={
      minimum height=15mm,
      inner sep=.7mm,
      outer sep=0mm,
      text width=120mm,
      text centered,
      font=\small\bfseries\sffamily,
      text=#1!50!black,
      draw=#1,
      line width=.25mm,
      top color=#1!5,
      bottom color=#1!40,
      shading angle=0,
      rounded corners=2.3mm,
      drop shadow={fill=#1!40!gray,fill opacity=.8},
      rotate=0,
    },
  }
	\node[box=links,xshift=3mm,yshift=1mm,
		minimum height=5pt,text width=325pt]
		at (0,-1.3) {\hyperlink{chapter4a}{\mbox{%
		\small \textcolor{black}{Тема 1.2. Співвідношення
			адміністративного права з іншими галузями}}}};
	\node[box=links,xshift=3mm,yshift=1mm,
		minimum height=5pt,text width=235pt]
		at (10.6,-1.3) {\hyperlink{chapter4b}{\mbox{%
		\small \textcolor{black}{Індивідуальні
		навчально-дослідницькі завдання}}}};
	\node[box=links,xshift=3mm,yshift=1mm,
		minimum height=5pt,text width=221pt]
		at (19.3,-1.3) {\hyperlink{chapter4b}{\mbox{%
		\small \textcolor{black}{
		Питання для самоконтролю та самоперевірки}}}};
	\node[box=links,xshift=3mm,yshift=1mm,
		minimum height=5pt,text width=140pt]
		at (26.35,-1.3) {\hyperlink{chapter4c}{\mbox{%
		\small \textcolor{black}{Додаткова література}}}};
\end{tikzpicture}

\vfill

%%%---> Old version
%\begin{tikzpicture}[remember picture, note/.style={rectangle
%	callout,fill=#1}]
%	\node [note=green!50,opacity=.5,overlay,text opacity=1,
%		font=\large\bfseries\sffamily, callout relative pointer={(-5,-1)},
%			callout pointer width=1.3cm] at (15,1) {%
%Індивідуальні навчально-дослідницькі завдання:
%};
%\end{tikzpicture}
%
%\begin{itemize}
%	\item[] \tikz[baseline] \node[ball color=magenta,circle,text=black,
%			minimum size=4pt]
%		{1}; \quad {\large\textbf{%
%			Підготувати реферат за темою: <<Особливості предмету
%			галузі в світлі Концепції реформи адміністративного права
%			України>>.}}
%
%\item[] \tikz[baseline] \node[ball color=magenta,circle,text=black]
%		{2}; \quad {\large\textbf{%
%			Підготувати реферат за темою: <<Співвідношення методів
%			адміністративно-правового й цивільно-правового регулювання
%			суспільних відносин>>.}}
%
%\item[] \tikz[baseline] \node[ball color=magenta,circle,text=black]
%		{3}; \quad {\large\textbf{%
%			Підготувати доповідь за темою: <<Принципи адміністративного
%			права: удосконалення системи>>.}}
%\end{itemize}
%%%--->
%%%---> New version
\begin{flushleft}
\begin{tikzpicture}
\node (tbl) {
\begin{tabularx}{980pt}{l}
\arrayrulecolor{purple}
\multicolumn{1}{c}{\mbox{}\hspace{50pt}\mbox{\textcolor{white}{{%
	\large\bfseries\sffamily Індивідуальні навчально--дослідницькі
   		завдання}}}}\\[15pt]
\tikz[baseline] \node[ball color=green,circle, text=white] {{\bf 1}};\qquad
\tikz[baseline] \node[font=\large\bfseries\sffamily,
	text=black] {Підготувати реферат за темою: <<Особливості
   		предмету галузі в світлі Концепції реформи адміністративного
	   	права України>>};\\[15pt]
		\tikz[baseline] \node[ball color=green,circle, text=white] {{\bf 2}};\qquad
\tikz[baseline] \node[font=\large\bfseries\sffamily,
	text=black] {Підготувати реферат за темою: <<Співвідношення
   		методів адміністративно--правового й цивільно--правового
	   	регулювання суспільних відносин>>};\\[15pt]
		\tikz[baseline] \node[ball color=green,circle, text=white] {{\bf 3}};\qquad
\tikz[baseline] \node[font=\large\bfseries\sffamily,
	text=black] {Підготувати доповідь за темою: <<Принципи
   		адміністративного права: удосконалення системи>>};\\[15pt]
\end{tabularx}};

\begin{pgfonlayer}{background}
\draw[rounded corners,top color=red,bottom color=black,draw=white]
	($(tbl.north west)+(0.14,0)$) rectangle ($(tbl.north east)-(0.13,0.9)$);
\draw[rounded corners,top color=white,bottom color=black,
	middle color=red,draw=blue!20] ($(tbl.south west) +(0.12,0.5)$)
		rectangle ($(tbl.south east)-(0.12,0)$);
\draw[top color=blue!1,bottom color=blue!20,draw=white]
	($(tbl.north east)-(0.13,0.6)$) rectangle ($(tbl.south west)+(0.13,0.2)$);
\end{pgfonlayer}
\end{tikzpicture}
\end{flushleft}
%%%---> Old version
%\vglue 45pt
%
%\begin{tikzpicture}[remember picture, note/.style={rectangle
%	callout,fill=#1}]
%	\node [note=green!50,opacity=.5,overlay,text opacity=1,
%		font=\large\bfseries\sffamily, callout relative pointer={(-5,-1)},
%			callout pointer width=1.3cm] at (15,1) {%
%Питання для самоконтролю та самоперевірки:
%};
%\end{tikzpicture}
%
%\begin{itemize}
%\item[] \tikz[baseline] \node[ball color=magenta,circle,text=black]
%	{1}; \quad {\large\textbf{%
%У яких значеннях вживається термін <<адміністративне право України>>?
%}}
%\item[] \tikz[baseline] \node[ball color=magenta,circle,text=black]
%	{2}; \quad {\large\textbf{%
%Як співвідносяться конституційне й адміністративне права? Розкрийте їх
%поняття і значення.
%}}
%\item[] \tikz[baseline] \node[ball color=magenta,circle,text=black]
%	{3}; \quad {\large\textbf{%
%Поясніть значення термінів <<предмет адміністративного права>>, <<метод
%адміністративного права>>, <<механізм адміністративно-правового
%регулювання>>.
%}}
%\item[] \tikz[baseline] \node[ball color=magenta,circle,text=black]
%	{4}; \quad {\large\textbf{%
%Розкрийте співвідношення адміністративного права із суміжними галузями
%права.
%}}
%\item[] \tikz[baseline] \node[ball color=magenta,circle,text=black]
%	{5}; \quad {\large\textbf{%
%Розкрийте зміст поняття <<система адміністративного права>>?
%}}
%
%\item[] \tikz[baseline] \node[ball color=magenta,circle,text=black]
%	{6}; \quad {\large\textbf{%
%Дайте правову характеристику головних елементів Загальної й Особливої
%частин адміністративного права.
%}}
%\item[] \tikz[baseline] \node[ball color=magenta,circle,text=black]
%	{7}; \quad {\large\textbf{%
%Що таке Спеціальна частина адміністративного права?
%}}
%\item[] \tikz[baseline] \node[ball color=magenta,circle,text=black]
%	{8}; \quad {\large\textbf{%
%Що таке <<принципи права>>?
%}}
%\item[] \tikz[baseline] \node[ball color=magenta,circle,text=black]
%	{9}; \quad {\large\textbf{%
%Для чого необхідне вивчення адміністративного права? Аргументуйте свою
%відповідь.
%}}
%\item[] \tikz[baseline] \node[ball color=magenta,circle,text=black]
%	{10}; \quad {\large\textbf{%
%У чому виражаються публічний і приватний аспекти адміністративного права, їх
%взаємозв'язок?
%}}
%\item[] \tikz[baseline] \node[ball color=magenta,circle,text=black]
%	{11}; \quad {\large\textbf{%
%Дайте історичну характеристику розвитку галузі адміністративного права в
%Україні.
%}}
%\end{itemize}
%%%--->
%%%---> NEW Version
\vfill
\begin{flushleft}
\begin{tikzpicture}
\node (tbl) {
\begin{tabularx}{980pt}{l}
\arrayrulecolor{purple}
\multicolumn{1}{c}{\textcolor{white}{{\large\bfseries\sffamily Питання
для самоконтролю та самоперевірки}}}\\[15pt]
\tikz[baseline] \node[ball color=magenta, circle,
	minimum size=0.8cm, text=white] {{\bf 1}};\qquad
\tikz[baseline] \node[font=\large\bfseries\sffamily,
	text=black] {%
У яких значеннях вживається термін <<адміністративне право України>>?
};\\[18pt]
\tikz[baseline] \node[ball color=magenta, circle,
	minimum size=0.8cm, text=white] {{\bf 2}};\qquad
\tikz[baseline] \node[font=\large\bfseries\sffamily,
	text=black] {%
Як співвідносяться конституційне й адміністративне права? Розкрийте їх
поняття і значення.
};\\[18pt]
\tikz[baseline] \node[ball color=magenta, circle,
	minimum size=0.8cm, text=white] {{\bf 3}};\qquad
\tikz[baseline] \node[font=\large\bfseries\sffamily,
	text=black] {%
Поясніть значення термінів <<предмет адміністративного права>>, <<метод
адміністративного права>>, <<механізм адміністративно-правового
регулювання>>.
};\\[18pt]
\tikz[baseline] \node[ball color=magenta, circle,
	minimum size=0.8cm, text=white] {{\bf 4}};\qquad
\tikz[baseline] \node[font=\large\bfseries\sffamily,
	text=black] {%
Розкрийте співвідношення адміністративного права із суміжними галузями
права.
};\\[18pt]
\tikz[baseline] \node[ball color=magenta, circle,
	minimum size=0.8cm, text=white] {{\bf 5}};\qquad
\tikz[baseline] \node[font=\large\bfseries\sffamily,
	text=black] {%
Розкрийте зміст поняття <<система адміністративного права>>?
};\\[18pt]
\tikz[baseline] \node[ball color=magenta, circle,
	minimum size=0.8cm, text=white] {{\bf 6}};\qquad
\tikz[baseline] \node[font=\large\bfseries\sffamily,
	text=black] {%
Дайте правову характеристику головних елементів Загальної й Особливої
частин адміністративного права.
};\\[18pt]
\tikz[baseline] \node[ball color=magenta, circle,
	minimum size=0.8cm, text=white] {{\bf 7}};\qquad
\tikz[baseline] \node[font=\large\bfseries\sffamily,
	text=black] {%
Що таке Спеціальна частина адміністративного права?
};\\[18pt]
\tikz[baseline] \node[ball color=magenta, circle,
	minimum size=0.8cm, text=white] {{\bf 8}};\qquad
\tikz[baseline] \node[font=\large\bfseries\sffamily,
	text=black] {%
Що таке <<принципи права>>?
};\\[18pt]
\tikz[baseline] \node[ball color=magenta, circle,
	minimum size=0.8cm,text=white] {{\bf 9}};\qquad
\tikz[baseline] \node[font=\large\bfseries\sffamily,
	text=black] {%
Для чого необхідне вивчення адміністративного права? Аргументуйте свою
відповідь.
};\\[18pt]
\tikz[baseline] \node[ball color=magenta,circle,
	minimum size=0.8cm, text=white] {{\bf 10}};\qquad
\tikz[baseline] \node[font=\large\bfseries\sffamily,
	text=black] {%
У чому виражаються публічний і приватний аспекти адміністративного права, їх
взаємозв'язок?
};\\[18pt]
\tikz[baseline] \node[ball color=magenta, circle,
	minimum size=0.8cm, text=white] {{\bf 11}};\qquad
\tikz[baseline] \node[font=\large\bfseries\sffamily,
	text=black] {%
Дайте історичну характеристику розвитку галузі адміністративного права в
Україні.
};\\[18pt]
\end{tabularx}};

\begin{pgfonlayer}{background}
\draw[rounded corners,top color=red,bottom color=black,draw=white]
	($(tbl.north west)+(0.14,0)$) rectangle ($(tbl.north east)-(0.13,0.9)$);
\draw[rounded corners,top color=white,bottom color=black,
	middle color=red,draw=blue!20] ($(tbl.south west) +(0.12,0.5)$)
		rectangle ($(tbl.south east)-(0.12,0)$);
\draw[top color=blue!1,bottom color=blue!20,draw=white]
	($(tbl.north east)-(0.13,0.6)$) rectangle ($(tbl.south west)+(0.13,0.2)$);
\end{pgfonlayer}
\end{tikzpicture}
\end{flushleft}
\vfill
\begin{flushright}
\tikz[baseline] \node[rectangle,inner sep=2pt,minimum height=3.1ex,rounded corners,
drop shadow,shadow scale=1,shadow xshift=.8ex,shadow yshift=-.4ex,opacity=.7,
fill=black!50,top color=blue!90!black!50,bottom color=blue!80!black!80,
draw=blue!50!black!50,very thick,text=white,text opacity=1,minimum width=4cm]{%
Тестовые модули};\hspace{10pt}\mbox{}
\end{flushright}	
\vfill
%%%---> NEW PAGE ---------------------------------------------------------
\newpage
\begin{tikzpicture}[remember picture,overlay]
	  \node [rotate=0,scale=2,text opacity=0.2]
	      at (27,1.7) {Капранов~О.~Г.~\copyright~2010~~~Luga\TeX @yahoo.com};
\end{tikzpicture}
\vglue -18pt
\hspace{187pt}
\parbox{350pt}{%
\hypertarget{chapter4c}{\hyperlink{chapter4d}{\mbox{%
\begin{tikzpicture}
  \colorlet{even}{cyan!60!black}
  \colorlet{odd}{orange!100!black}
  \colorlet{links}{red!70!black}
  \colorlet{back}{yellow!20!white}
  \tikzset{
    box/.style={
      minimum height=15mm,
      inner sep=.7mm,
      outer sep=0mm,
      text width=120mm,
      text centered,
      font=\small\bfseries\sffamily,
      text=#1!50!black,
      draw=#1,
      line width=.25mm,
      top color=#1!5,
      bottom color=#1!40,
      shading angle=0,
      rounded corners=2.3mm,
      drop shadow={fill=#1!40!gray,fill opacity=.8},
      rotate=0,
    },
  }
  \node [box=even] {{%
  	\huge\textbf{Методичні рекомендації,}}
	\textbf{плани й завдання до семінарських і практичних занять}};
\end{tikzpicture}
}}}}\\[5pt]
\noindent
\begin{tikzpicture}
  \colorlet{even}{cyan!60!black}
  \colorlet{odd}{orange!100!black}
  \colorlet{links}{red!70!black}
  \colorlet{back}{yellow!20!white}
  \tikzset{
    box/.style={
      minimum height=15mm,
      inner sep=.7mm,
      outer sep=0mm,
      text width=120mm,
      text centered,
      font=\small\bfseries\sffamily,
      text=#1!50!black,
      draw=#1,
      line width=.25mm,
      top color=#1!5,
      bottom color=#1!40,
      shading angle=0,
      rounded corners=2.3mm,
      drop shadow={fill=#1!40!gray,fill opacity=.8},
      rotate=0,
    },
  }
	\node[box=links,xshift=3mm,yshift=1mm,
		minimum height=5pt,text width=325pt]
		at (0,-1.3) {\hyperlink{chapter4a}{\mbox{%
		\small \textcolor{black}{Тема 1.2. Співвідношення
			адміністративного права з іншими галузями}}}};
	\node[box=links,xshift=3mm,yshift=1mm,
		minimum height=5pt,text width=235pt]
		at (10.6,-1.3) {\hyperlink{chapter4b}{\mbox{%
		\small \textcolor{black}{Індивідуальні
		навчально-дослідницькі завдання}}}};
	\node[box=links,xshift=3mm,yshift=1mm,
		minimum height=5pt,text width=221pt]
		at (19.3,-1.3) {\hyperlink{chapter4b}{\mbox{%
		\small \textcolor{black}{
		Питання для самоконтролю та самоперевірки}}}};
	\node[box=links,xshift=3mm,yshift=1mm,
		minimum height=5pt,text width=140pt]
		at (26.35,-1.3) {\hyperlink{chapter4c}{\mbox{%
		\small \textcolor{black}{Додаткова література}}}};
\end{tikzpicture}
\vglue 25pt
%%%---> Old Version
%\begin{tikzpicture}
%	\node[name=s,shape=rectangle callout,
%		callout relative pointer={(1.25cm,-1cm)},
%			callout pointer width=2cm, inner xsep=2cm, inner ysep=1cm,
%				font=\Large\bfseries\sffamily, text centered, fill=black,
%					shading angle=45, drop shadow,
%						text=white] at (3,0) {Додаткова література};
%\end{tikzpicture}
%%%---<
%%%---> NEW Version
\hyperlink{chapter4d}{\mbox{%
\begin{tikzpicture}
	\node[name=s,shape=rectangle callout,
		callout relative pointer={(1.25cm,-1cm)},
		callout pointer width=2cm, inner xsep=2cm, inner ysep=1cm,
		font=\Large\bfseries\sffamily, text centered,
		shading angle=45, drop shadow,
		postaction={path fading=south,
		fading angle=45,fill=blue, opacity=.5},
		left color=black, right color=red, draw=white,
		line width=2mm, text=white, drop shadow,
		shadow scale=1.25, shadow xshift=0pt,
		shadow yshift=0pt]
	at (3,0) {Додаткова література};
\end{tikzpicture}
}}
%%%---<
\vfill
\begin{itemize}
\item[]
	\scalebox{1.9}{\iconbook}\qquad 
\tooltipanim{\large\bf Административное право зарубежных стран}{21}{21}
\quad{\large{\bf Учебник}. Под ред. А.Н. {\bf Козырина}, М.А.
{\bf Штатиной}. --- {\bf М}., {\bf 2003}. --- 464 с.}
\item[]
\scalebox{1.9}{\iconbook}\qquad
\tooltipanim{\large\bf Виконавча влада і адміністративне право}{22}{22}
\quad{\large За заг. ред В.Б.{\bf Авер'янова}. --- К.,
{\bf 2002}. --- 668 с.}
\item[]
\scalebox{1.9}{\iconarticle}\qquad~~{\large {\bf Авер'янов} В.Б.}
\tooltipanim{\large\bf Актуальні завдання реформування адміністративного права}{23}{23}
\quad{\large{\bf Право України}. --- {\bf 1999}. --- {\bf №8}. --- С. 8.}
\item[]
\scalebox{1.9}{\iconarticle}\qquad~~{\large {\bf Авер'янов} В.Б.}
\tooltipanim{\large\bf Реформування українського адміністративного права}{24}{24}
\quad{\large черговий етап. {\bf Право України}. --- {\bf 2000}. --- {\bf №7}.
--- С. 6.}
\item[]
\scalebox{1.9}{\iconarticle}\qquad~~{\large {\bf Авер`янов} В.Б.}
\tooltipanim{\large\bf Предмет адміністративного права}{25}{25}
\quad{\large нова доктринальна оцінка.
{\bf Право України}. --- {\bf 2004}. --- {\bf №10}. --- С. 25.}
\item[]
\scalebox{1.9}{\iconarticle}\qquad~~{\large {\bf Антонова} В.П.}
\tooltipanim{\large\bf Институты административного права}{23}{23}
\quad{\large (третьи <<Лазаревские чтения>>). {\bf Государство и право}.
	--- {\bf 1999}. --- {\bf №10}.  --- С. 5.}
\item[]
\scalebox{1.9}{\iconarticle}\qquad~~{\large {\bf Афанасьєв} К.К.}
\tooltipanim{\large\bf К вопросу о предмете административно--правового регулирования}{24}{24}
\quad{\large	{\bf Вісник ЛАВС МВС України}.  --- {\bf 2002}. ---  {\bf №3}.
--- С. 53-65.}
\item[]
\scalebox{1.9}{\iconarticle}\qquad~~{\large  {\bf Бельский} К.С.}
\tooltipanim{\large\bf О предмете и системе науки административного права}{25}{25}
\quad{\large	{\bf Государство и право}. --- {\bf 1998}. --- {\bf №10}. --- С. 18.}
\item[]
\scalebox{1.9}{\iconarticle}\qquad~~{\large {\bf Бельский} К.С.}
\tooltipanim{\large\bf О системе административного права}{23}{23}
\quad{\large{\bf Государство и право}.  --- {\bf 1998}. --- {\bf №3}. --- С. 5.}
\item[]
\scalebox{1.9}{\iconbook}\qquad \parbox{850pt}{%
{\large {\bf Головін} А.П., {\bf Нікітенко} О.І.}
\tooltipanim{\large\bf До питання про предмет та систему адміністративного права}{21}{21}
\quad{\large{\bf Українське адміністративне право}: актуальні проблеми реформування: {\bf Збірник наукових праць}. ---
{\bf Суми}: ВВП <<Мрія--1>> ЛТД: Ініціатива, {\bf 2000}.--282 с.}}
\item[]
\scalebox{1.9}{\iconarticle}\qquad~~\parbox{830pt}{%
{\large {\bf Данильева} И. Э.}
\tooltipanim{\large\bf Понятие и значение принципов права в регулировании административных правоотношений}{23}{23}\quad  {\large{\bf Вісник} Запорізького державного університету: Зб. наук. статей.
Юридичні науки. --- {\bf Запоріжжя}: Запорізький державний університет.
--- {\bf 2004}. --- {\bf №1}. --- С. 74--78.}}
\end{itemize}
\vfill
\begin{textblock}{45}(22.1,15)
	\hyperlink{chapter4d}{\mbox{%
	\tikz[every node/.style={font=\normalsize\bfseries\sffamily,signal,draw,
		text=white,signal to=nowhere}]
		\node[signal to=east, minimum width=35pt, postaction={path fading=south,
			fading angle=45,fill=blue,opacity=.5}, left color=black, right color=red,
				draw=white, line width=2mm, drop shadow,
					minimum height=8pt]
			{Додаткова література: 2~стр};
			}}
\end{textblock}
\vfill
\mbox{}
%%%-----------------------------------------------------------------------
\begin{textblock}{46}(25,-0.01)
\begin{tikzpicture}[even odd rule,rounded corners=2pt,x=10pt,y=10pt,drop shadow]
\filldraw[fill=yellow!90!black!40,drop shadow] (0,0)   rectangle (1,1)
	[xshift=5pt,yshift=5pt]   (0,0)   rectangle (1,1)
	[rotate=30]   (-1,-1) rectangle (2,2);
\node at (0,1.7) {\textbf{\thepage}};			      
\end{tikzpicture}
\end{textblock}
%%%--- Navigational panel top page
\begin{textblock}{47}(7.58,0.85)
\mbox{%%%--->
\Acrobatmenu{LastPage}{%
\tikz[baseline] \node[rectangle,inner sep=2pt,minimum height=3.1ex,
rounded corners,drop shadow,shadow scale=1,shadow xshift=.8ex,
shadow yshift=-.4ex,opacity=.7,fill=black!50,top color=red!90!black!50,
bottom color=red!80!black!80,draw=red!50!black!50,very thick,text=white,
text opacity=1,minimum width=3cm,font=\bfseries\sffamily] at (0,0) {К концу};
}\Acrobatmenu{GoBack}{%
\tikz[baseline] \node[rectangle,inner sep=2pt,minimum height=3.1ex,
rounded corners,drop shadow,shadow scale=1,shadow xshift=.8ex,
shadow yshift=-.4ex,opacity=.7,fill=black!50,top color=red!90!black!50,
bottom color=red!80!black!80,draw=red!50!black!50,very thick,text=white,
text opacity=1,minimum width=3cm,font=\bfseries\sffamily] at (4,0) {Назад};
}\Acrobatmenu{PrevPage}{%
\tikz[baseline] \node[rectangle,inner sep=2pt,minimum height=3.1ex,
rounded corners,drop shadow,shadow scale=1,shadow xshift=.8ex,
shadow yshift=-.4ex,opacity=.7,fill=black!50,top color=red!90!black!50,
bottom color=red!80!black!80,draw=red!50!black!50,very thick,text=white,
text opacity=1,minimum width=3cm,font=\bfseries\sffamily] at (8,0) {Предыдущий};
}\Acrobatmenu{NextPage}{%
\tikz[baseline] \node[rectangle,inner sep=2pt,minimum height=3.1ex,
rounded corners,drop shadow,shadow scale=1,shadow xshift=.8ex,
shadow yshift=-.4ex,opacity=.7,fill=black!50,top color=red!90!black!50,
bottom color=red!80!black!80,draw=red!50!black!50,very thick,text=white,
text opacity=1,minimum width=3cm,font=\bfseries\sffamily] at (12,0) {Следующий};
}\Acrobatmenu{GoForward}{%
\tikz[baseline] \node[rectangle,inner sep=2pt,minimum height=3.1ex,
rounded corners,drop shadow,shadow scale=1,shadow xshift=.8ex,
shadow yshift=-.4ex,opacity=.7,fill=black!50,top color=red!90!black!50,
bottom color=red!80!black!80,draw=red!50!black!50,very thick,text=white,
text opacity=1,minimum width=3cm,font=\bfseries\sffamily] at (16,0) {Вперед};
}\Acrobatmenu{FirstPage}{%
\tikz[baseline] \node[rectangle,inner sep=2pt,minimum height=3.1ex,
rounded corners,drop shadow,shadow scale=1,shadow xshift=.8ex,
shadow yshift=-.4ex,opacity=.7,fill=black!50,top color=red!90!black!50,
bottom color=red!80!black!80,draw=red!50!black!50,very thick,text=white,
text opacity=1,minimum width=3cm,font=\bfseries\sffamily] at (20,0) {К началу};
}\Acrobatmenu{FullScreen}{%
\tikz[baseline] \node[rectangle,inner sep=2pt,minimum height=3.1ex,
rounded corners,drop shadow,shadow scale=1,shadow xshift=.8ex,
shadow yshift=-.4ex,opacity=.7,fill=black!50,top color=red!90!black!50,
bottom color=red!80!black!80,draw=red!50!black!50,very thick,text=white,
text opacity=1,minimum width=3cm,font=\bfseries\sffamily] at (24,0) {Полный экран};
}\Acrobatmenu{Quit}{%
\tikz[baseline] \node[rectangle,inner sep=2pt,minimum height=3.1ex,
rounded corners,drop shadow,shadow scale=1,shadow xshift=.8ex,
shadow yshift=-.4ex,opacity=.7,fill=black!50,top color=red!90!black!50,
bottom color=red!80!black!80,draw=red!50!black!50,very thick,text=white,
text opacity=1,minimum width=3cm,font=\bfseries\sffamily] at (28,0) {Выход};
}	
}%%%---|
\end{textblock}
%%%-----------------------------------------------------------------------
%%%---> NEW PAGE ----------------------------------------------------------
\newpage
\begin{tikzpicture}[remember picture,overlay]
	  \node [rotate=0,scale=2,text opacity=0.2]
	      at (27,1.7) {Капранов~О.~Г.~\copyright~2010~~~Luga\TeX @yahoo.com};
\end{tikzpicture}
\vglue -18pt
\hspace{187pt}
\parbox{350pt}{%
\hypertarget{chapter4d}{\hyperlink{chapter5a}{\mbox{%
\begin{tikzpicture}
  \colorlet{even}{cyan!60!black}
  \colorlet{odd}{orange!100!black}
  \colorlet{links}{red!70!black}
  \colorlet{back}{yellow!20!white}
  \tikzset{
    box/.style={
      minimum height=15mm,
      inner sep=.7mm,
      outer sep=0mm,
      text width=120mm,
      text centered,
      font=\small\bfseries\sffamily,
      text=#1!50!black,
      draw=#1,
      line width=.25mm,
      top color=#1!5,
      bottom color=#1!40,
      shading angle=0,
      rounded corners=2.3mm,
      drop shadow={fill=#1!40!gray,fill opacity=.8},
      rotate=0,
    },
  }
  \node [box=even] {{%
  	\huge\textbf{Методичні рекомендації,}}
	\textbf{плани й завдання до семінарських і практичних занять}};
\end{tikzpicture}
}}}}\\[5pt]
\noindent
\begin{tikzpicture}
  \colorlet{even}{cyan!60!black}
  \colorlet{odd}{orange!100!black}
  \colorlet{links}{red!70!black}
  \colorlet{back}{yellow!20!white}
  \tikzset{
    box/.style={
      minimum height=15mm,
      inner sep=.7mm,
      outer sep=0mm,
      text width=120mm,
      text centered,
      font=\small\bfseries\sffamily,
      text=#1!50!black,
      draw=#1,
      line width=.25mm,
      top color=#1!5,
      bottom color=#1!40,
      shading angle=0,
      rounded corners=2.3mm,
      drop shadow={fill=#1!40!gray,fill opacity=.8},
      rotate=0,
    },
  }
	\node[box=links,xshift=3mm,yshift=1mm,
		minimum height=5pt,text width=325pt]
		at (0,-1.3) {\hyperlink{chapter4a}{\mbox{%
		\small \textcolor{black}{Тема 1.2. Співвідношення
			адміністративного права з іншими галузями}}}};
	\node[box=links,xshift=3mm,yshift=1mm,
		minimum height=5pt,text width=235pt]
		at (10.6,-1.3) {\hyperlink{chapter4b}{\mbox{%
		\small \textcolor{black}{Індивідуальні
		навчально-дослідницькі завдання}}}};
	\node[box=links,xshift=3mm,yshift=1mm,
		minimum height=5pt,text width=221pt]
		at (19.3,-1.3) {\hyperlink{chapter4b}{\mbox{%
		\small \textcolor{black}{
		Питання для самоконтролю та самоперевірки}}}};
	\node[box=links,xshift=3mm,yshift=1mm,
		minimum height=5pt,text width=140pt]
		at (26.35,-1.3) {\hyperlink{chapter4c}{\mbox{%
		\small \textcolor{black}{Додаткова література}}}};
\end{tikzpicture}
\vglue 25pt
%%%---> Old Version
%\begin{tikzpicture}
%	\node[name=s,shape=rectangle callout,
%		callout relative pointer={(1.25cm,-1cm)},
%			callout pointer width=2cm, inner xsep=2cm, inner ysep=1cm,
%				font=\Large\bfseries\sffamily, text centered,
%					shading angle=45] at (3,0) {Додаткова література};
%\end{tikzpicture}					
%%%---<
%%%---> NEW Version
\hyperlink{chapter4c}{\mbox{%
\begin{tikzpicture}
	\node[name=s,shape=rectangle callout,
		callout relative pointer={(1.25cm,-1cm)},
		callout pointer width=2cm, inner xsep=2cm, inner ysep=1cm,
		font=\Large\bfseries\sffamily, text centered,
		shading angle=45,
		postaction={path fading=south,
		fading angle=45,fill=blue, opacity=.5},
		left color=black, right color=red, draw=white,
		line width=2mm, text=white,
		shadow scale=3.25, shadow xshift=3pt,
		shadow yshift=3pt]
	at (3,0) {Додаткова література};
\end{tikzpicture}
}}
%%%---<
\vfill
\begin{itemize}
\item[]
	\scalebox{1.9}{\iconarticle}\qquad~~{\large {\bf Кондратьєв} Р., {\bf
	Чернего} О.I.}
\tooltipanim{\large\bf Принципи права та їх роль у регулюванні суспільних відносин}{23}{23}\quad {\large{\bf Право України}.  --- {\bf 2000}. --- {\bf №2}. --- С. 43.}
\item[]
	\scalebox{1.9}{\iconarticle}\qquad~~\parbox{850pt}{%
	{\large {\bf Константий} О.}
\tooltipanim{\large\bf Система адміністративного права як
	концептуальна основа здійснення адміністративного судочинства в
	Україні}{24}{24}\quad
{\large{\bf Право України}. --- {\bf 2004}. --- {\bf №12}. --- С. 20.}}
\item[]
	\scalebox{1.9}{\iconarticle}\qquad~~{\large {\bf Кравчук} І.}
\tooltipanim{\large\bf Адаптація права України до права Європейського Союзу}{25}{25}\quad
{\large	цілі, етапи, пріоритети. {\bf Право України}.  --- {\bf 2004}.
	--- {\bf №10}. --- С. 132.}
\item[]
	\scalebox{1.9}{\iconarticle}\qquad~~{\large {\bf Кубко} Є.}
\tooltipanim{\large\bf Про предмет адміністративного права}{23}{23}\quad
{\large{\bf Право України}. --- {\bf 2000}. --- {\bf №5}. --- С.3.}
\item[]
	\scalebox{1.9}{\iconarticle}\qquad~~{\large {\bf Ославський} М.}
\tooltipanim{\large\bf До питання необхідності здійснення адміністративної реформи в Україні}{24}{24}\quad {\large{\bf Право України}. --- {\bf 2004}. --- {\bf №9}. ---
	С. 40.}
\item[]
	\scalebox{1.9}{\iconarticle}\qquad~~\parbox{820pt}{%
	{\large {\bf Полешко} А.}
	\tooltipanim{\large\bf Напрями реформування адміністративного права}{25}{25}\quad
{\large	(за матеріалами Національної науковотеоретичної конференції).
	{\bf Право України}. --- {\bf 2000}. --- {\bf №8}. --- С. 35.}}
\item[]
	\scalebox{1.9}{\iconarticle}\qquad~~{\large {\bf Старилов} Ю.Н.}
\tooltipanim{\large\bf Как развивалась наука административного права в европейских странах}{23}{23}\quad {\large{\bf Журнал российского права}.  --- {\bf 1999}. ---
	{\bf №3/4}. --- С. 203.}
\item[]
	\scalebox{1.9}{\iconarticle}\qquad~~{\large {\bf Тихомиров} Ю.А.}
\tooltipanim{\large\bf О концепции развития административного права и процесса}{24}{24}\quad
{\large{\bf Государство и право}. --- {\bf 1998}. --- №1. --- С. 5.}
\item[]
	\scalebox{1.9}{\iconarticle}\qquad~~{\large {\bf Тимощук} В.}
\tooltipanim{\large\bf Адміністративне право в контексті європейського вибору України}{25}{25}\quad
{\large	(з міжнародної конференції). {\bf Право України}.  --- {\bf 2004}.
	--- {\bf №3}. --- С. 25.}
\item[]
\scalebox{1.9}{\iconarticle}\qquad~~{\large {\bf Хорощак} Н.}
\tooltipanim{\large\bf Нове в адміністративному праві}{23}{23}\quad
{\large{\bf Право України}.  --- {\bf 2004}. --- {\bf №8}.  --- С. 130.}
\item[]
\scalebox{1.9}{\iconarticle}\qquad~~{\large {\bf Шаповал} В.}
\tooltipanim{\large\bf Конституція України як форма адміністративного права}{24}{24}\quad
{\large{\bf Право України}. --- {\bf 2000}. --- {\bf №1}. --- С. 3.}
\end{itemize}
\vfill
\begin{textblock}{48}(22.1,15)
	\hyperlink{chapter4c}{\mbox{%
	\tikz[every node/.style={font=\normalsize\bfseries\sffamily,signal,draw,
		text=white}]
		\node[signal to=east,
			minimum width=35pt, postaction={path fading=south,
			fading angle=45,fill=blue,opacity=.5}, left color=black, right color=red,
			draw=white, line width=2mm, drop shadow, rotate=180]
			{\rotatebox{180}{\mbox{Додаткова література: 1~стр}}};
			}}		
\end{textblock}
\vfill
\mbox{}
%%%-----------------------------------------------------------------------
\begin{textblock}{49}(25,-0.01)
\begin{tikzpicture}[even odd rule,rounded corners=2pt,x=10pt,y=10pt,drop shadow]
\filldraw[fill=yellow!90!black!40,drop shadow] (0,0)   rectangle (1,1)
	[xshift=5pt,yshift=5pt]   (0,0)   rectangle (1,1)
	[rotate=30]   (-1,-1) rectangle (2,2);
\node at (0,1.7) {\textbf{\thepage}};			      
\end{tikzpicture}
\end{textblock}
%%%--- Navigational panel top page
\begin{textblock}{50}(7.58,0.85)
\mbox{%%%--->
\Acrobatmenu{LastPage}{%
\tikz[baseline] \node[rectangle,inner sep=2pt,minimum height=3.1ex,
rounded corners,drop shadow,shadow scale=1,shadow xshift=.8ex,
shadow yshift=-.4ex,opacity=.7,fill=black!50,top color=red!90!black!50,
bottom color=red!80!black!80,draw=red!50!black!50,very thick,text=white,
text opacity=1,minimum width=3cm,font=\bfseries\sffamily] at (0,0) {К концу};
}\Acrobatmenu{GoBack}{%
\tikz[baseline] \node[rectangle,inner sep=2pt,minimum height=3.1ex,
rounded corners,drop shadow,shadow scale=1,shadow xshift=.8ex,
shadow yshift=-.4ex,opacity=.7,fill=black!50,top color=red!90!black!50,
bottom color=red!80!black!80,draw=red!50!black!50,very thick,text=white,
text opacity=1,minimum width=3cm,font=\bfseries\sffamily] at (4,0) {Назад};
}\Acrobatmenu{PrevPage}{%
\tikz[baseline] \node[rectangle,inner sep=2pt,minimum height=3.1ex,
rounded corners,drop shadow,shadow scale=1,shadow xshift=.8ex,
shadow yshift=-.4ex,opacity=.7,fill=black!50,top color=red!90!black!50,
bottom color=red!80!black!80,draw=red!50!black!50,very thick,text=white,
text opacity=1,minimum width=3cm,font=\bfseries\sffamily] at (8,0) {Предыдущий};
}\Acrobatmenu{NextPage}{%
\tikz[baseline] \node[rectangle,inner sep=2pt,minimum height=3.1ex,
rounded corners,drop shadow,shadow scale=1,shadow xshift=.8ex,
shadow yshift=-.4ex,opacity=.7,fill=black!50,top color=red!90!black!50,
bottom color=red!80!black!80,draw=red!50!black!50,very thick,text=white,
text opacity=1,minimum width=3cm,font=\bfseries\sffamily] at (12,0) {Следующий};
}\Acrobatmenu{GoForward}{%
\tikz[baseline] \node[rectangle,inner sep=2pt,minimum height=3.1ex,
rounded corners,drop shadow,shadow scale=1,shadow xshift=.8ex,
shadow yshift=-.4ex,opacity=.7,fill=black!50,top color=red!90!black!50,
bottom color=red!80!black!80,draw=red!50!black!50,very thick,text=white,
text opacity=1,minimum width=3cm,font=\bfseries\sffamily] at (16,0) {Вперед};
}\Acrobatmenu{FirstPage}{%
\tikz[baseline] \node[rectangle,inner sep=2pt,minimum height=3.1ex,
rounded corners,drop shadow,shadow scale=1,shadow xshift=.8ex,
shadow yshift=-.4ex,opacity=.7,fill=black!50,top color=red!90!black!50,
bottom color=red!80!black!80,draw=red!50!black!50,very thick,text=white,
text opacity=1,minimum width=3cm,font=\bfseries\sffamily] at (20,0) {К началу};
}\Acrobatmenu{FullScreen}{%
\tikz[baseline] \node[rectangle,inner sep=2pt,minimum height=3.1ex,
rounded corners,drop shadow,shadow scale=1,shadow xshift=.8ex,
shadow yshift=-.4ex,opacity=.7,fill=black!50,top color=red!90!black!50,
bottom color=red!80!black!80,draw=red!50!black!50,very thick,text=white,
text opacity=1,minimum width=3cm,font=\bfseries\sffamily] at (24,0) {Полный экран};
}\Acrobatmenu{Quit}{%
\tikz[baseline] \node[rectangle,inner sep=2pt,minimum height=3.1ex,
rounded corners,drop shadow,shadow scale=1,shadow xshift=.8ex,
shadow yshift=-.4ex,opacity=.7,fill=black!50,top color=red!90!black!50,
bottom color=red!80!black!80,draw=red!50!black!50,very thick,text=white,
text opacity=1,minimum width=3cm,font=\bfseries\sffamily] at (28,0) {Выход};
}	
}%%%---|
\end{textblock}
%%%-----------------------------------------------------------------------

\newpage
\begin{tikzpicture}[remember picture,overlay]
	  \node [rotate=0,scale=2,text opacity=0.2]
	      at (27,1.7) {Капранов~О.~Г.~\copyright~2010~~~Luga\TeX @yahoo.com};
\end{tikzpicture}
\vglue -18pt
\hspace{187pt}
\parbox{350pt}{%
\hypertarget{chapter5a}{\hyperlink{chapter5b}{\mbox{%
\begin{tikzpicture}
  \colorlet{even}{cyan!60!black}
  \colorlet{odd}{orange!100!black}
  \colorlet{links}{red!70!black}
  \colorlet{back}{yellow!20!white}
  \tikzset{
    box/.style={
      minimum height=15mm,
      inner sep=.7mm,
      outer sep=0mm,
      text width=120mm,
      text centered,
      font=\small\bfseries\sffamily,
      text=#1!50!black,
      draw=#1,
      line width=.25mm,
      top color=#1!5,
      bottom color=#1!40,
      shading angle=0,
      rounded corners=2.3mm,
      drop shadow={fill=#1!40!gray,fill opacity=.8},
      rotate=0,
    },
  }
  \node [box=even] {{%
  	\huge\textbf{Методичні рекомендації,}}
	\textbf{плани й завдання до семінарських і практичних занять}};
\end{tikzpicture}
}}}}\\[5pt]
\noindent
\begin{tikzpicture}
  \colorlet{even}{cyan!60!black}
  \colorlet{odd}{orange!100!black}
  \colorlet{links}{red!70!black}
  \colorlet{back}{yellow!20!white}
  \tikzset{
    box/.style={
      minimum height=15mm,
      inner sep=.7mm,
      outer sep=0mm,
      text width=120mm,
      text centered,
      font=\small\bfseries\sffamily,
      text=#1!50!black,
      draw=#1,
      line width=.25mm,
      top color=#1!5,
      bottom color=#1!40,
      shading angle=0,
      rounded corners=2.3mm,
      drop shadow={fill=#1!40!gray,fill opacity=.8},
      rotate=0,
    },
  }
	\node[box=links,xshift=3mm,yshift=1mm,
		minimum height=5pt,text width=325pt]
		at (0,-1.3) {\hyperlink{chapter5a}{\mbox{%
		\small \textcolor{black}{Тема 2.2. Принципи й функції державного
		управління}}}};
	\node[box=links,xshift=3mm,yshift=1mm,
		minimum height=5pt,text width=235pt]
		at (10.6,-1.3) {\hyperlink{chapter5b}{\mbox{%
		\small \textcolor{black}{Індивідуальні
		навчально-дослідницькі завдання}}}};
	\node[box=links,xshift=3mm,yshift=1mm,
		minimum height=5pt,text width=221pt]
		at (19.3,-1.3) {\hyperlink{chapter5b}{\mbox{%
		\small \textcolor{black}{
		Питання для самоконтролю та самоперевірки}}}};
	\node[box=links,xshift=3mm,yshift=1mm,
		minimum height=5pt,text width=140pt]
		at (26.35,-1.3) {\hyperlink{chapter5c}{\mbox{%
		\small \textcolor{black}{Додаткова література}}}};
\end{tikzpicture}
\vglue 5pt
\tikzfading[name=targetask, top color=transparent!90,
	bottom color=transparent!90,middle color=transparent!65]
\begin{tikzpicture}
\node [rounded corners,fill=magenta!50,minimum width=960pt,
minimum height=35pt,path fading=targetask] at (0,0) {\mbox{}};
\end{tikzpicture}
\begin{textblock}{51}(0.8,3.75)
\begin{tikzpicture}
	\node [text width=920pt] {\parbox{900pt}{%
	\textbf{Мета заняття:}
	визначити поняття, сутність і види управлінської діяльності,
	співвідношення соціального й державного управління, виконавчої
	влади й державного управління, охарактеризувати види й зміст
	принципів і функцій державного управління.
}};
\end{tikzpicture}
 \end{textblock}
\begin{textblock}{52}(17,5)
\tikzstyle{abstract}=[rectangle, draw=black, rounded corners, fill=blue!40, drop shadow,
	text centered, text=white, text width=4cm,font=\large\bfseries\sffamily]
\tikzstyle{comment}=[rectangle, draw=black, rounded corners, fill=green, drop shadow,
	text centered, text=white, text width=10cm,
	font=\Large\bfseries\sffamily]
\tikzstyle{myarrow}=[->, >=open triangle 90, thick]
\tikzstyle{line}=[-, thick]
\begin{tikzpicture}[node distance=2cm]
    \node (Seminar) [abstract, rectangle split,
		rectangle split parts=2,text width=5cm] at (0,0)
        {
            Семінарське заняття
            \nodepart{second}2 год
        };

    \node (Thema) [abstract, rectangle split, rectangle split parts=2,
		text width=10cm] at (0,-2) {
            Принципи й функції
            \nodepart{second}державного управління
        };

    \node (AuxNode01) [text width=4cm] {};
    \node (Maindir) [abstract, rectangle split,
		rectangle split parts=2] at (0,-4)
        {
            Основні поняття
            \nodepart{second}:
        };

	\node (Maindirnames) [comment, rectangle split,
		rectangle split parts=2, text justified] at (0,-10)
	{
		\mbox{}
		\nodepart{second}\mbox{}
		\newline управління
		\newline організація
		\newline соціальне управління
		\newline державне управління
		\newline державна влада
		\newline виконавча влада
		\newline управлінська діяльність
		\newline управлінський цикл
		\newline управлінський апарат
		\newline компетенція
		\newline \mbox{}
	};
\end{tikzpicture}
\end{textblock}

\vglue 15pt

\tikzstyle{mybox} = [draw=red, fill=blue!20, very thick,
	rectangle, rounded corners, inner sep=10pt, inner ysep=20pt,
	font=\large\bfseries\sffamily]
\tikzstyle{fancytitle} =[fill=red, text=white,font=\large\bfseries\sffamily]

\begin{tikzpicture}
\node [mybox] (box) {%
	\begin{minipage}{0.50\textwidth}
		\begin{itemize}
			\item Поняття, сутність і види управління.
			\item Ознаки державного управління.
			\item Класифікація принципів державного управління.
			\item Система й зміст функцій управлінської діяльності.
		\end{itemize}
	\end{minipage}
};
\node[fancytitle, right=10pt] at (box.north west) {Навчальні питання:};
\node[fancytitle, rounded corners] at (box.east) {{\bf ?}};
\end{tikzpicture}	

\vglue 15pt

\tikzstyle{mybox} = [draw=blue, fill=green!20, very thick,
	rectangle, rounded corners, inner sep=10pt, inner ysep=20pt,
	line width=1pt,font=\bfseries\sffamily]
\tikzstyle{fancytitle} =[fill=blue, text=white, ellipse,font=\bfseries\sffamily]

\begin{tikzpicture}[transform shape, rotate=0, baseline=-3.5cm]
	\node [mybox] (box) {%
		\begin{minipage}[t!]{0.5\textwidth}
У ході підготовки до заняття по даній темі курсанти та студенти повинні
усвідомити поняття й види управління, роль і місце державного управління,
принципи поділу державної влади на законодавчу, виконавчу й судову. Необхідно
з'ясувати поняття і зміст функцій державного управління, уміти співвідносити
поняття <<виконавча влада>> й <<державне управління>>. При цьому на основі аналізу
нормативно--правових актів курсанти та студенти повинні показати застосування в
законодавстві цих державно-правових категорій.


Будучи різновидом соціального управління, державне управління зберігає його
характеристики. Одночасно, йому властиві численні особливості, що відображають
його специфіку. Визначаючи сутність державного управління, системність
управлінського процесу, необхідно усвідомити його принципи, класифікацію й
зміст функцій, виділяти серед них функції орієнтування, забезпечення й
оперативного управління системою.


Поділяючи принципи державного управління на правові й організаційно-правові,
слід вивчити й уміти надати характеристику принципам законності, демократизму,
планування, диференціації й фіксації повноважень, відповідальності в рамках
компетенції, єдності галузевих, міжгалузевих і територіальних, лінійних і
функціональних засад управління й ін.

Із практичної точки зору курсанти та студенти повинні вміти відмежовувати
управління від інших видів державної діяльності, визначати підсистеми
державного управління стосовно окремих його галузей, давати оцінку ролі
адміністративного права в регулюванні відносин управлінського характеру.


Для кращого засвоєння теми пропонується виконати завдання для самостійної
роботи та індивідуальні навчально--дослідницькі завдання.


Рівень своїх знань з цієї теми пропонується перевірити шляхом надання
відповідей на питання для самоконтролю та самоперевірки.
		\end{minipage}
		};
\node[fancytitle] at (box.north) {Методичні рекомендації та пояснення};
\end{tikzpicture}
%%%-----------------------------------------------------------------------
\begin{textblock}{53}(25,-0.01)
\begin{tikzpicture}[even odd rule,rounded corners=2pt,x=10pt,y=10pt,drop shadow]
\filldraw[fill=yellow!90!black!40,drop shadow] (0,0)   rectangle (1,1)
	[xshift=5pt,yshift=5pt]   (0,0)   rectangle (1,1)
	[rotate=30]   (-1,-1) rectangle (2,2);
\node at (0,1.7) {\textbf{\thepage}};			      
\end{tikzpicture}
\end{textblock}
%%%--- Navigational panel top page
\begin{textblock}{54}(7.58,0.85)
\mbox{%%%--->
\Acrobatmenu{LastPage}{%
\tikz[baseline] \node[rectangle,inner sep=2pt,minimum height=3.1ex,
rounded corners,drop shadow,shadow scale=1,shadow xshift=.8ex,
shadow yshift=-.4ex,opacity=.7,fill=black!50,top color=red!90!black!50,
bottom color=red!80!black!80,draw=red!50!black!50,very thick,text=white,
text opacity=1,minimum width=3cm,font=\bfseries\sffamily] at (0,0) {К концу};
}\Acrobatmenu{GoBack}{%
\tikz[baseline] \node[rectangle,inner sep=2pt,minimum height=3.1ex,
rounded corners,drop shadow,shadow scale=1,shadow xshift=.8ex,
shadow yshift=-.4ex,opacity=.7,fill=black!50,top color=red!90!black!50,
bottom color=red!80!black!80,draw=red!50!black!50,very thick,text=white,
text opacity=1,minimum width=3cm,font=\bfseries\sffamily] at (4,0) {Назад};
}\Acrobatmenu{PrevPage}{%
\tikz[baseline] \node[rectangle,inner sep=2pt,minimum height=3.1ex,
rounded corners,drop shadow,shadow scale=1,shadow xshift=.8ex,
shadow yshift=-.4ex,opacity=.7,fill=black!50,top color=red!90!black!50,
bottom color=red!80!black!80,draw=red!50!black!50,very thick,text=white,
text opacity=1,minimum width=3cm,font=\bfseries\sffamily] at (8,0) {Предыдущий};
}\Acrobatmenu{NextPage}{%
\tikz[baseline] \node[rectangle,inner sep=2pt,minimum height=3.1ex,
rounded corners,drop shadow,shadow scale=1,shadow xshift=.8ex,
shadow yshift=-.4ex,opacity=.7,fill=black!50,top color=red!90!black!50,
bottom color=red!80!black!80,draw=red!50!black!50,very thick,text=white,
text opacity=1,minimum width=3cm,font=\bfseries\sffamily] at (12,0) {Следующий};
}\Acrobatmenu{GoForward}{%
\tikz[baseline] \node[rectangle,inner sep=2pt,minimum height=3.1ex,
rounded corners,drop shadow,shadow scale=1,shadow xshift=.8ex,
shadow yshift=-.4ex,opacity=.7,fill=black!50,top color=red!90!black!50,
bottom color=red!80!black!80,draw=red!50!black!50,very thick,text=white,
text opacity=1,minimum width=3cm,font=\bfseries\sffamily] at (16,0) {Вперед};
}\Acrobatmenu{FirstPage}{%
\tikz[baseline] \node[rectangle,inner sep=2pt,minimum height=3.1ex,
rounded corners,drop shadow,shadow scale=1,shadow xshift=.8ex,
shadow yshift=-.4ex,opacity=.7,fill=black!50,top color=red!90!black!50,
bottom color=red!80!black!80,draw=red!50!black!50,very thick,text=white,
text opacity=1,minimum width=3cm,font=\bfseries\sffamily] at (20,0) {К началу};
}\Acrobatmenu{FullScreen}{%
\tikz[baseline] \node[rectangle,inner sep=2pt,minimum height=3.1ex,
rounded corners,drop shadow,shadow scale=1,shadow xshift=.8ex,
shadow yshift=-.4ex,opacity=.7,fill=black!50,top color=red!90!black!50,
bottom color=red!80!black!80,draw=red!50!black!50,very thick,text=white,
text opacity=1,minimum width=3cm,font=\bfseries\sffamily] at (24,0) {Полный экран};
}\Acrobatmenu{Quit}{%
\tikz[baseline] \node[rectangle,inner sep=2pt,minimum height=3.1ex,
rounded corners,drop shadow,shadow scale=1,shadow xshift=.8ex,
shadow yshift=-.4ex,opacity=.7,fill=black!50,top color=red!90!black!50,
bottom color=red!80!black!80,draw=red!50!black!50,very thick,text=white,
text opacity=1,minimum width=3cm,font=\bfseries\sffamily] at (28,0) {Выход};
}	
}%%%---|
\end{textblock}
%%%-----------------------------------------------------------------------
%%%---> NEW PAGE ---------------------------------------------------------
\newpage
\begin{tikzpicture}[remember picture,overlay]
	  \node [rotate=0,scale=2,text opacity=0.2]
	      at (27,1.7) {Капранов~О.~Г.~\copyright~2010~~~Luga\TeX @yahoo.com};
\end{tikzpicture}
%%%-----------------------------------------------------------------------
\begin{textblock}{55}(25,-0.01)
\begin{tikzpicture}[even odd rule,rounded corners=2pt,x=10pt,y=10pt,drop shadow]
\filldraw[fill=yellow!90!black!40,drop shadow] (0,0)   rectangle (1,1)
	[xshift=5pt,yshift=5pt]   (0,0)   rectangle (1,1)
	[rotate=30]   (-1,-1) rectangle (2,2);
\node at (0,1.7) {\textbf{\thepage}};			      
\end{tikzpicture}
\end{textblock}
%%%--- Navigational panel top page
\begin{textblock}{56}(7.58,0.85)
\mbox{%%%--->
\Acrobatmenu{LastPage}{%
\tikz[baseline] \node[rectangle,inner sep=2pt,minimum height=3.1ex,
rounded corners,drop shadow,shadow scale=1,shadow xshift=.8ex,
shadow yshift=-.4ex,opacity=.7,fill=black!50,top color=red!90!black!50,
bottom color=red!80!black!80,draw=red!50!black!50,very thick,text=white,
text opacity=1,minimum width=3cm,font=\bfseries\sffamily] at (0,0) {К концу};
}\Acrobatmenu{GoBack}{%
\tikz[baseline] \node[rectangle,inner sep=2pt,minimum height=3.1ex,
rounded corners,drop shadow,shadow scale=1,shadow xshift=.8ex,
shadow yshift=-.4ex,opacity=.7,fill=black!50,top color=red!90!black!50,
bottom color=red!80!black!80,draw=red!50!black!50,very thick,text=white,
text opacity=1,minimum width=3cm,font=\bfseries\sffamily] at (4,0) {Назад};
}\Acrobatmenu{PrevPage}{%
\tikz[baseline] \node[rectangle,inner sep=2pt,minimum height=3.1ex,
rounded corners,drop shadow,shadow scale=1,shadow xshift=.8ex,
shadow yshift=-.4ex,opacity=.7,fill=black!50,top color=red!90!black!50,
bottom color=red!80!black!80,draw=red!50!black!50,very thick,text=white,
text opacity=1,minimum width=3cm,font=\bfseries\sffamily] at (8,0) {Предыдущий};
}\Acrobatmenu{NextPage}{%
\tikz[baseline] \node[rectangle,inner sep=2pt,minimum height=3.1ex,
rounded corners,drop shadow,shadow scale=1,shadow xshift=.8ex,
shadow yshift=-.4ex,opacity=.7,fill=black!50,top color=red!90!black!50,
bottom color=red!80!black!80,draw=red!50!black!50,very thick,text=white,
text opacity=1,minimum width=3cm,font=\bfseries\sffamily] at (12,0) {Следующий};
}\Acrobatmenu{GoForward}{%
\tikz[baseline] \node[rectangle,inner sep=2pt,minimum height=3.1ex,
rounded corners,drop shadow,shadow scale=1,shadow xshift=.8ex,
shadow yshift=-.4ex,opacity=.7,fill=black!50,top color=red!90!black!50,
bottom color=red!80!black!80,draw=red!50!black!50,very thick,text=white,
text opacity=1,minimum width=3cm,font=\bfseries\sffamily] at (16,0) {Вперед};
}\Acrobatmenu{FirstPage}{%
\tikz[baseline] \node[rectangle,inner sep=2pt,minimum height=3.1ex,
rounded corners,drop shadow,shadow scale=1,shadow xshift=.8ex,
shadow yshift=-.4ex,opacity=.7,fill=black!50,top color=red!90!black!50,
bottom color=red!80!black!80,draw=red!50!black!50,very thick,text=white,
text opacity=1,minimum width=3cm,font=\bfseries\sffamily] at (20,0) {К началу};
}\Acrobatmenu{FullScreen}{%
\tikz[baseline] \node[rectangle,inner sep=2pt,minimum height=3.1ex,
rounded corners,drop shadow,shadow scale=1,shadow xshift=.8ex,
shadow yshift=-.4ex,opacity=.7,fill=black!50,top color=red!90!black!50,
bottom color=red!80!black!80,draw=red!50!black!50,very thick,text=white,
text opacity=1,minimum width=3cm,font=\bfseries\sffamily] at (24,0) {Полный экран};
}\Acrobatmenu{Quit}{%
\tikz[baseline] \node[rectangle,inner sep=2pt,minimum height=3.1ex,
rounded corners,drop shadow,shadow scale=1,shadow xshift=.8ex,
shadow yshift=-.4ex,opacity=.7,fill=black!50,top color=red!90!black!50,
bottom color=red!80!black!80,draw=red!50!black!50,very thick,text=white,
text opacity=1,minimum width=3cm,font=\bfseries\sffamily] at (28,0) {Выход};
}	
}%%%---|
\end{textblock}
%%%-----------------------------------------------------------------------
\vglue -18pt
\hspace{187pt}
\parbox{350pt}{%
\hypertarget{chapter5b}{\hyperlink{chapter5c}{\mbox{%
\begin{tikzpicture}
  \colorlet{even}{cyan!60!black}
  \colorlet{odd}{orange!100!black}
  \colorlet{links}{red!70!black}
  \colorlet{back}{yellow!20!white}
  \tikzset{
    box/.style={
      minimum height=15mm,
      inner sep=.7mm,
      outer sep=0mm,
      text width=120mm,
      text centered,
      font=\small\bfseries\sffamily,
      text=#1!50!black,
      draw=#1,
      line width=.25mm,
      top color=#1!5,
      bottom color=#1!40,
      shading angle=0,
      rounded corners=2.3mm,
      drop shadow={fill=#1!40!gray,fill opacity=.8},
      rotate=0,
    },
  }
  \node [box=even] {{%
  	\huge\textbf{Методичні рекомендації,}}
	\textbf{плани й завдання до семінарських і практичних занять}};
\end{tikzpicture}
}}}}\\[5pt]
\noindent
\begin{tikzpicture}
  \colorlet{even}{cyan!60!black}
  \colorlet{odd}{orange!100!black}
  \colorlet{links}{red!70!black}
  \colorlet{back}{yellow!20!white}
  \tikzset{
    box/.style={
      minimum height=15mm,
      inner sep=.7mm,
      outer sep=0mm,
      text width=120mm,
      text centered,
      font=\small\bfseries\sffamily,
      text=#1!50!black,
      draw=#1,
      line width=.25mm,
      top color=#1!5,
      bottom color=#1!40,
      shading angle=0,
      rounded corners=2.3mm,
      drop shadow={fill=#1!40!gray,fill opacity=.8},
      rotate=0,
    },
  }
	\node[box=links,xshift=3mm,yshift=1mm,
		minimum height=5pt,text width=325pt]
		at (0,-1.3) {\hyperlink{chapter5a}{\mbox{%
		\small \textcolor{black}{Тема 2.2. Принципи й функції
	   	державного управління}}}};
	\node[box=links,xshift=3mm,yshift=1mm,
		minimum height=5pt,text width=235pt]
		at (10.6,-1.3) {\hyperlink{chapter5b}{\mbox{%
		\small \textcolor{black}{Індивідуальні
		навчально-дослідницькі завдання}}}};
	\node[box=links,xshift=3mm,yshift=1mm,
		minimum height=5pt,text width=221pt]
		at (19.3,-1.3) {\hyperlink{chapter5b}{\mbox{%
		\small \textcolor{black}{
		Питання для самоконтролю та самоперевірки}}}};
	\node[box=links,xshift=3mm,yshift=1mm,
		minimum height=5pt,text width=140pt]
		at (26.35,-1.3) {\hyperlink{chapter5c}{\mbox{%
		\small \textcolor{black}{Додаткова література}}}};
\end{tikzpicture}

\vfill

%%%---> Old version
%\begin{tikzpicture}[remember picture, note/.style={rectangle
%	callout,fill=#1}]
%	\node [note=green!50,opacity=.5,overlay,text opacity=1,
%		font=\large\bfseries\sffamily, callout relative pointer={(-5,-1)},
%			callout pointer width=1.3cm] at (15,1) {%
%Індивідуальні навчально-дослідницькі завдання:
%};
%\end{tikzpicture}
%
%\begin{itemize}
%	\item[] \tikz[baseline] \node[ball color=magenta,circle,text=black,
%			minimum size=4pt]
%		{1}; \quad {\large\textbf{%
%			Підготувати реферат за темою: <<Особливості предмету
%			галузі в світлі Концепції реформи адміністративного права
%			України>>.}}
%
%\item[] \tikz[baseline] \node[ball color=magenta,circle,text=black]
%		{2}; \quad {\large\textbf{%
%			Підготувати реферат за темою: <<Співвідношення методів
%			адміністративно-правового й цивільно-правового регулювання
%			суспільних відносин>>.}}
%
%\item[] \tikz[baseline] \node[ball color=magenta,circle,text=black]
%		{3}; \quad {\large\textbf{%
%			Підготувати доповідь за темою: <<Принципи адміністративного
%			права: удосконалення системи>>.}}
%\end{itemize}
%%%--->
%%%---> New version
\begin{flushleft}
\begin{tikzpicture}
\node (tbl) {
\begin{tabularx}{980pt}{l}
\arrayrulecolor{purple}
\multicolumn{1}{c}{\mbox{}\hspace{50pt}\mbox{\textcolor{white}{{%
	\large\bfseries\sffamily Індивідуальні навчально--дослідницькі
   		завдання}}}}\\[15pt]
\tikz[baseline] \node[ball color=green,circle, text=white] {{\bf 1}};\qquad
\tikz[baseline] \node[font=\large\bfseries\sffamily,
	text=black] {
Підготувати реферат за темою: <<Роль державного управління в світлі сучасного
реформування суспільства>>.
	   	};\\[20pt]
		\tikz[baseline] \node[ball color=green,circle, text=white] {{\bf 2}};\qquad
\tikz[baseline] \node[font=\large\bfseries\sffamily,
	text=black] {
Підготувати реферат за темою: <<Управління в області внутрішніх справ як
підсистема державного управління>>.
		};\\[20pt]
		\tikz[baseline] \node[ball color=green,circle, text=white] {{\bf 3}};\qquad
\tikz[baseline] \node[font=\large\bfseries\sffamily,
	text=black] {
Підготувати доповідь за темою: <<Проблеми визначення принципів державного
управління на сучасному етапі>>.
	};\\[20pt]
		\tikz[baseline] \node[ball color=green,circle, text=white] {{\bf 3}};\qquad
\tikz[baseline] \node[font=\large\bfseries\sffamily,
	text=black] {
Скласти бібліографію за темою заняття з урахуванням нових надходжень до
бібліотеки.
	};\\[20pt]
		\tikz[baseline] \node[ball color=green,circle, text=white] {{\bf 3}};\qquad
\tikz[baseline] \node[font=\large\bfseries\sffamily, text width=820pt,
	text=black] {
	Підготувати рецензію на наукову статтю: Конопльов В. Організаційно--правовий
	механізм підвищення ефективності управлінської діяльності. Право України. ---
	2005. --- № 9. --- С. 111.
	};\\[20pt]
\end{tabularx}};

\begin{pgfonlayer}{background}
\draw[rounded corners,top color=red,bottom color=black,draw=white]
	($(tbl.north west)+(0.14,0)$) rectangle ($(tbl.north east)-(0.13,0.9)$);
\draw[rounded corners,top color=white,bottom color=black,
	middle color=red,draw=blue!20] ($(tbl.south west) +(0.12,0.5)$)
		rectangle ($(tbl.south east)-(0.12,0)$);
\draw[top color=blue!1,bottom color=blue!20,draw=white]
	($(tbl.north east)-(0.13,0.6)$) rectangle ($(tbl.south west)+(0.13,0.2)$);
\end{pgfonlayer}
\end{tikzpicture}
\end{flushleft}
%%%---> Old version
%\vglue 45pt
%
%\begin{tikzpicture}[remember picture, note/.style={rectangle
%	callout,fill=#1}]
%	\node [note=green!50,opacity=.5,overlay,text opacity=1,
%		font=\large\bfseries\sffamily, callout relative pointer={(-5,-1)},
%			callout pointer width=1.3cm] at (15,1) {%
%Питання для самоконтролю та самоперевірки:
%};
%\end{tikzpicture}
%
%\begin{itemize}
%\item[] \tikz[baseline] \node[ball color=magenta,circle,text=black]
%	{1}; \quad {\large\textbf{%
%У яких значеннях вживається термін <<адміністративне право України>>?
%}}
%\item[] \tikz[baseline] \node[ball color=magenta,circle,text=black]
%	{2}; \quad {\large\textbf{%
%Як співвідносяться конституційне й адміністративне права? Розкрийте їх
%поняття і значення.
%}}
%\item[] \tikz[baseline] \node[ball color=magenta,circle,text=black]
%	{3}; \quad {\large\textbf{%
%Поясніть значення термінів <<предмет адміністративного права>>, <<метод
%адміністративного права>>, <<механізм адміністративно-правового
%регулювання>>.
%}}
%\item[] \tikz[baseline] \node[ball color=magenta,circle,text=black]
%	{4}; \quad {\large\textbf{%
%Розкрийте співвідношення адміністративного права із суміжними галузями
%права.
%}}
%\item[] \tikz[baseline] \node[ball color=magenta,circle,text=black]
%	{5}; \quad {\large\textbf{%
%Розкрийте зміст поняття <<система адміністративного права>>?
%}}
%
%\item[] \tikz[baseline] \node[ball color=magenta,circle,text=black]
%	{6}; \quad {\large\textbf{%
%Дайте правову характеристику головних елементів Загальної й Особливої
%частин адміністративного права.
%}}
%\item[] \tikz[baseline] \node[ball color=magenta,circle,text=black]
%	{7}; \quad {\large\textbf{%
%Що таке Спеціальна частина адміністративного права?
%}}
%\item[] \tikz[baseline] \node[ball color=magenta,circle,text=black]
%	{8}; \quad {\large\textbf{%
%Що таке <<принципи права>>?
%}}
%\item[] \tikz[baseline] \node[ball color=magenta,circle,text=black]
%	{9}; \quad {\large\textbf{%
%Для чого необхідне вивчення адміністративного права? Аргументуйте свою
%відповідь.
%}}
%\item[] \tikz[baseline] \node[ball color=magenta,circle,text=black]
%	{10}; \quad {\large\textbf{%
%У чому виражаються публічний і приватний аспекти адміністративного права, їх
%взаємозв'язок?
%}}
%\item[] \tikz[baseline] \node[ball color=magenta,circle,text=black]
%	{11}; \quad {\large\textbf{%
%Дайте історичну характеристику розвитку галузі адміністративного права в
%Україні.
%}}
%\end{itemize}
%%%--->
%%%---> NEW Version
\vfill
\begin{flushleft}
\begin{tikzpicture}
\node (tbl) {
\begin{tabularx}{980pt}{l}
\arrayrulecolor{purple}
\multicolumn{1}{c}{\mbox{}\hspace{50pt}\mbox{\textcolor{white}{{\large\bfseries\sffamily Питання
для самоконтролю та самоперевірки}}}}\\[15pt]
\tikz[baseline] \node[ball color=magenta, circle,
	minimum size=0.8cm, text=white] {{\bf 1}};\qquad
\tikz[baseline] \node[font=\large\bfseries\sffamily,
	text=black] {%
Розкрийте зміст і співвідношення понять <<організація>> й <<управління>>.
};\\[18pt]
\tikz[baseline] \node[ball color=magenta, circle,
	minimum size=0.8cm, text=white] {{\bf 2}};\qquad
\tikz[baseline] \node[font=\large\bfseries\sffamily,
	text=black] {%
Назвіть основні характерні риси державного управління.
};\\[18pt]
\tikz[baseline] \node[ball color=magenta, circle,
	minimum size=0.8cm, text=white] {{\bf 3}};\qquad
\tikz[baseline] \node[font=\large\bfseries\sffamily,
	text=black] {%
Порівняєте державне управління з іншими формами державної діяльності.
};\\[18pt]
\tikz[baseline] \node[ball color=magenta, circle,
	minimum size=0.8cm, text=white] {{\bf 4}};\qquad
\tikz[baseline] \node[font=\large\bfseries\sffamily,
	text=black] {%
Наведіть характеристику принципів законності й доцільності
управлінської діяльності.
};\\[18pt]
\tikz[baseline] \node[ball color=magenta, circle,
	minimum size=0.8cm, text=white] {{\bf 5}};\qquad
\tikz[baseline] \node[font=\large\bfseries\sffamily,
	text=black] {%
Співвіднесіть поняття <<державне управління>> й <<виконавча влада>>.
};\\[18pt]
\tikz[baseline] \node[ball color=magenta, circle,
	minimum size=0.8cm, text=white] {{\bf 6}};\qquad
\tikz[baseline] \node[font=\large\bfseries\sffamily,
	text=black] {%
Визначить роль адміністративного права в здійсненні державного
управління і їх взаємозв'язок.
};\\[18pt]
\end{tabularx}};

\begin{pgfonlayer}{background}
\draw[rounded corners,top color=red,bottom color=black,draw=white]
	($(tbl.north west)+(0.14,0)$) rectangle ($(tbl.north east)-(0.13,0.9)$);
\draw[rounded corners,top color=white,bottom color=black,
	middle color=red,draw=blue!20] ($(tbl.south west) +(0.12,0.5)$)
		rectangle ($(tbl.south east)-(0.12,0)$);
\draw[top color=blue!1,bottom color=blue!20,draw=white]
	($(tbl.north east)-(0.13,0.6)$) rectangle ($(tbl.south west)+(0.13,0.2)$);
\end{pgfonlayer}
\end{tikzpicture}
\end{flushleft}
\vfill
\begin{flushright}
\tikz[baseline] \node[rectangle,inner sep=2pt,minimum height=3.1ex,rounded corners,
drop shadow,shadow scale=1,shadow xshift=.8ex,shadow yshift=-.4ex,opacity=.7,
fill=black!50,top color=blue!90!black!50,bottom color=blue!80!black!80,
draw=blue!50!black!50,very thick,text=white,text opacity=1,minimum width=4cm]{%
Тестовые модули};\hspace{10pt}\mbox{}
\end{flushright}	
\vfill
%%%-----------------------------------------------------------------------
%%%---> NEW PAGE ---------------------------------------------------------
\newpage
\begin{tikzpicture}[remember picture,overlay]
	  \node [rotate=0,scale=2,text opacity=0.2]
	      at (27,1.7) {Капранов~О.~Г.~\copyright~2010~~~Luga\TeX @yahoo.com};
\end{tikzpicture}
\vglue -18pt
\hspace{187pt}
\parbox{350pt}{%
\hypertarget{chapter5c}{\hyperlink{chapter5d}{\mbox{%
\begin{tikzpicture}
  \colorlet{even}{cyan!60!black}
  \colorlet{odd}{orange!100!black}
  \colorlet{links}{red!70!black}
  \colorlet{back}{yellow!20!white}
  \tikzset{
    box/.style={
      minimum height=15mm,
      inner sep=.7mm,
      outer sep=0mm,
      text width=120mm,
      text centered,
      font=\small\bfseries\sffamily,
      text=#1!50!black,
      draw=#1,
      line width=.25mm,
      top color=#1!5,
      bottom color=#1!40,
      shading angle=0,
      rounded corners=2.3mm,
      drop shadow={fill=#1!40!gray,fill opacity=.8},
      rotate=0,
    },
  }
  \node [box=even] {{%
  	\huge\textbf{Методичні рекомендації,}}
	\textbf{плани й завдання до семінарських і практичних занять}};
\end{tikzpicture}
}}}}\\[5pt]
\noindent
\begin{tikzpicture}
  \colorlet{even}{cyan!60!black}
  \colorlet{odd}{orange!100!black}
  \colorlet{links}{red!70!black}
  \colorlet{back}{yellow!20!white}
  \tikzset{
    box/.style={
      minimum height=15mm,
      inner sep=.7mm,
      outer sep=0mm,
      text width=120mm,
      text centered,
      font=\small\bfseries\sffamily,
      text=#1!50!black,
      draw=#1,
      line width=.25mm,
      top color=#1!5,
      bottom color=#1!40,
      shading angle=0,
      rounded corners=2.3mm,
      drop shadow={fill=#1!40!gray,fill opacity=.8},
      rotate=0,
    },
  }
	\node[box=links,xshift=3mm,yshift=1mm,
		minimum height=5pt,text width=325pt]
		at (0,-1.3) {\hyperlink{chapter5a}{\mbox{%
		\small \textcolor{black}{Тема 2.2. Принципи й функції
	   	державного управління}}}};
	\node[box=links,xshift=3mm,yshift=1mm,
		minimum height=5pt,text width=235pt]
		at (10.6,-1.3) {\hyperlink{chapter5b}{\mbox{%
		\small \textcolor{black}{Індивідуальні
		навчально-дослідницькі завдання}}}};
	\node[box=links,xshift=3mm,yshift=1mm,
		minimum height=5pt,text width=221pt]
		at (19.3,-1.3) {\hyperlink{chapter5b}{\mbox{%
		\small \textcolor{black}{
		Питання для самоконтролю та самоперевірки}}}};
	\node[box=links,xshift=3mm,yshift=1mm,
		minimum height=5pt,text width=140pt]
		at (26.35,-1.3) {\hyperlink{chapter5c}{\mbox{%
		\small \textcolor{black}{Додаткова література}}}};
\end{tikzpicture}
\vglue 25pt
%%%---> Old Version
%\begin{tikzpicture}
%	\node[name=s,shape=rectangle callout,
%		callout relative pointer={(1.25cm,-1cm)},
%			callout pointer width=2cm, inner xsep=2cm, inner ysep=1cm,
%				font=\Large\bfseries\sffamily, text centered, fill=black,
%					shading angle=45, drop shadow,
%						text=white] at (3,0) {Додаткова література};
%\end{tikzpicture}
%%%---<
%%%---> NEW Version
\hyperlink{chapter5d}{\mbox{%
\begin{tikzpicture}
	\node[name=s,shape=rectangle callout,
		callout relative pointer={(1.25cm,-1cm)},
		callout pointer width=2cm, inner xsep=2cm, inner ysep=1cm,
		font=\Large\bfseries\sffamily, text centered,
		shading angle=45, drop shadow,
		postaction={path fading=south,
		fading angle=45,fill=blue, opacity=.5},
		left color=black, right color=red, draw=white,
		line width=2mm, text=white, drop shadow,
		shadow scale=1.25, shadow xshift=0pt,
		shadow yshift=0pt]
	at (3,0) {Додаткова література};
\end{tikzpicture}
}}
%%%---<
\vfill
\begin{itemize}
\item[]
	\scalebox{1.9}{\iconbook}\qquad {\large {\bf Авер'янов} В.Б.,
	{\bf Крупчан} О.Д.}
	\tooltipanim{\large\bf Виконавча влада}{21}{21}\quad{\large конституційні
	засади і
	шляхи реформування. --- {\bf Харків}. {\bf 1998}.}
\item[]
	\scalebox{1.9}{\iconbook}\qquad {\large {\bf Атаманчук} Г.В.}
	\tooltipanim{\large\bf Теория государственного управления}{22}{22}\quad
	{\large Курс лекций. --- {\bf М}.,
	{\bf 1997}.}
\item[]
	\scalebox{1.9}{\iconbook}\qquad
	\tooltipanim{\large\bf Виконавча влада і адміністративне
	право}{21}{21}\quad {\large За заг. ред {\bf Авер'янова} В.Б. --- {\bf К}.,
	{\bf 2002}. --- 668 с.}
\item[]
	\scalebox{1.9}{\iconbook}\qquad
	\tooltipanim{\large\bf Державне управління в Україні}{22}{22}\quad
	{\large Навчальний посібник. За заг. ред. {\bf Авер'янова} В.Б.
	--- {\bf К}., {\bf 1999}. --- 266 с.}
\item[]
\scalebox{1.9}{\iconbook}\qquad
\tooltipanim{\large\bf Державне управління в Україні}{21}{21}\quad
{\large наукові, правові, кадрові та організаційні засади.
Навчальний посібник. За заг. ред.\\
\mbox{}\hspace{43pt} {\bf Нижник} Н.Р. {\bf Олуйка} В.М.  --- {\bf Львів}., {\bf 2002}. --- 352 с.}
\item[]
	\scalebox{1.9}{\iconbook}\qquad 
	\tooltipanim{\large\bf Державне управління}{22}{22}\quad
	{\large	проблеми адміністративно-правової теорії та практики.
	За заг. ред. {\bf Авер'янова} В.Б. --- {\bf К}., {\bf 2003}. --- 384 с.}
\item[]
	\scalebox{1.9}{\iconbook}\qquad
	\tooltipanim{\large\bf Державне управління та адміністративне право в сучасній Україні}{21}{21}\quad {\large актуальні проблеми
	реформування. За заг. ред. {\bf Авер'янова}, {\bf Коліушка}.
	--- {\bf К}., {\bf 1999}. --- 50 с.}
\item[]
	\scalebox{1.9}{\iconbook}\qquad {\large{\bf Фалмер} Р.М.}
	\tooltipanim{\large\bf Энциклопедия современного
	управления}{22}{22}\quad {\large В {\bf 5}-и частях. 
	--- {\bf М}., {\bf 1992}.}
\item[]
	\scalebox{1.9}{\iconbook}\qquad {\large {\bf Щекин} Г.В.}
	\tooltipanim{\large\bf Теория социального управления}{21}{21}\quad
	{\large --- {\bf К}., {\bf 1996}.}
\item[]
	\scalebox{1.9}{\iconarticle}\qquad~~{\large\bf
	Авер`янов В.} \tooltipanim{\large\bf Ще раз про зміст і співвідношення понять <<державне управління>> і <<виконавча влада>>}{23}{23}\quad
{\large проблемні нотатки.\\
\mbox{}\hspace{43pt} {\bf Право України}. --- {\bf 2004}. --- {\bf №5}. --- С. 113.}
\item[]
	\scalebox{1.9}{\iconarticle}\qquad~~{\large {\bf Бульба} О.}
\tooltipanim{\large\bf Європейська інтеграція України та питання реалізації поділу влади}{24}{24}\quad {\large{\bf Право України}. --- {\bf 2004}. --- {\bf 12}. --- С. 8.}
\item[]
	\scalebox{1.9}{\iconarticle}\qquad~~{\large {\bf Васькович} Й.}
\tooltipanim{\large\bf Проблеми та перспективи побудови правової держави в Україні}{25}{25}\quad
{\large	{\bf Право України}. --- {\bf 2000}. --- {\bf №1}. --- С. 32.}
\item[]
\scalebox{1.9}{\iconarticle}\qquad~~{\large\bf Дерець В.}
\tooltipanim{\large\bf Реординаційні відносини як окремий вид управлінських
відносин між органами виконавчої влади}{23}{23}\quad
{\large Стан і перспективи реформування\\
\mbox{}\hspace{45pt} адміністративного права.
( IV Національна конференція). {\bf Право
України}. --- {\bf 2005}. --- {\bf №5}. --- С. 35.}
\end{itemize}
\vfill
\begin{textblock}{57}(22.1,15)
	\hyperlink{chapter5d}{\mbox{%
	\tikz[every node/.style={font=\normalsize\bfseries\sffamily,signal,draw,
		text=white,signal to=nowhere}]
		\node[signal to=east, minimum width=35pt, postaction={path fading=south,
			fading angle=45,fill=blue,opacity=.5}, left color=black, right color=red,
				draw=white, line width=1mm, drop shadow,
					minimum height=8pt]
			{Додаткова література: 2~стр};
			}}
\end{textblock}
\vfill
\mbox{}
%%%-----------------------------------------------------------------------
\begin{textblock}{58}(25,-0.01)
\begin{tikzpicture}[even odd rule,rounded corners=2pt,x=10pt,y=10pt,drop shadow]
\filldraw[fill=yellow!90!black!40,drop shadow] (0,0)   rectangle (1,1)
	[xshift=5pt,yshift=5pt]   (0,0)   rectangle (1,1)
	[rotate=30]   (-1,-1) rectangle (2,2);
\node at (0,1.7) {\textbf{\thepage}};			      
\end{tikzpicture}
\end{textblock}
%%%--- Navigational panel top page
\begin{textblock}{59}(7.58,0.85)
\mbox{%%%--->
\Acrobatmenu{LastPage}{%
\tikz[baseline] \node[rectangle,inner sep=2pt,minimum height=3.1ex,
rounded corners,drop shadow,shadow scale=1,shadow xshift=.8ex,
shadow yshift=-.4ex,opacity=.7,fill=black!50,top color=red!90!black!50,
bottom color=red!80!black!80,draw=red!50!black!50,very thick,text=white,
text opacity=1,minimum width=3cm,font=\bfseries\sffamily] at (0,0) {К концу};
}\Acrobatmenu{GoBack}{%
\tikz[baseline] \node[rectangle,inner sep=2pt,minimum height=3.1ex,
rounded corners,drop shadow,shadow scale=1,shadow xshift=.8ex,
shadow yshift=-.4ex,opacity=.7,fill=black!50,top color=red!90!black!50,
bottom color=red!80!black!80,draw=red!50!black!50,very thick,text=white,
text opacity=1,minimum width=3cm,font=\bfseries\sffamily] at (4,0) {Назад};
}\Acrobatmenu{PrevPage}{%
\tikz[baseline] \node[rectangle,inner sep=2pt,minimum height=3.1ex,
rounded corners,drop shadow,shadow scale=1,shadow xshift=.8ex,
shadow yshift=-.4ex,opacity=.7,fill=black!50,top color=red!90!black!50,
bottom color=red!80!black!80,draw=red!50!black!50,very thick,text=white,
text opacity=1,minimum width=3cm,font=\bfseries\sffamily] at (8,0) {Предыдущий};
}\Acrobatmenu{NextPage}{%
\tikz[baseline] \node[rectangle,inner sep=2pt,minimum height=3.1ex,
rounded corners,drop shadow,shadow scale=1,shadow xshift=.8ex,
shadow yshift=-.4ex,opacity=.7,fill=black!50,top color=red!90!black!50,
bottom color=red!80!black!80,draw=red!50!black!50,very thick,text=white,
text opacity=1,minimum width=3cm,font=\bfseries\sffamily] at (12,0) {Следующий};
}\Acrobatmenu{GoForward}{%
\tikz[baseline] \node[rectangle,inner sep=2pt,minimum height=3.1ex,
rounded corners,drop shadow,shadow scale=1,shadow xshift=.8ex,
shadow yshift=-.4ex,opacity=.7,fill=black!50,top color=red!90!black!50,
bottom color=red!80!black!80,draw=red!50!black!50,very thick,text=white,
text opacity=1,minimum width=3cm,font=\bfseries\sffamily] at (16,0) {Вперед};
}\Acrobatmenu{FirstPage}{%
\tikz[baseline] \node[rectangle,inner sep=2pt,minimum height=3.1ex,
rounded corners,drop shadow,shadow scale=1,shadow xshift=.8ex,
shadow yshift=-.4ex,opacity=.7,fill=black!50,top color=red!90!black!50,
bottom color=red!80!black!80,draw=red!50!black!50,very thick,text=white,
text opacity=1,minimum width=3cm,font=\bfseries\sffamily] at (20,0) {К началу};
}\Acrobatmenu{FullScreen}{%
\tikz[baseline] \node[rectangle,inner sep=2pt,minimum height=3.1ex,
rounded corners,drop shadow,shadow scale=1,shadow xshift=.8ex,
shadow yshift=-.4ex,opacity=.7,fill=black!50,top color=red!90!black!50,
bottom color=red!80!black!80,draw=red!50!black!50,very thick,text=white,
text opacity=1,minimum width=3cm,font=\bfseries\sffamily] at (24,0) {Полный экран};
}\Acrobatmenu{Quit}{%
\tikz[baseline] \node[rectangle,inner sep=2pt,minimum height=3.1ex,
rounded corners,drop shadow,shadow scale=1,shadow xshift=.8ex,
shadow yshift=-.4ex,opacity=.7,fill=black!50,top color=red!90!black!50,
bottom color=red!80!black!80,draw=red!50!black!50,very thick,text=white,
text opacity=1,minimum width=3cm,font=\bfseries\sffamily] at (28,0) {Выход};
}	
}%%%---|
\end{textblock}
%%%-----------------------------------------------------------------------
%%%---> NEW PAGE ----------------------------------------------------------
\newpage
\begin{tikzpicture}[remember picture,overlay]
	  \node [rotate=0,scale=2,text opacity=0.2]
	      at (27,1.7) {Капранов~О.~Г.~\copyright~2010~~~Luga\TeX @yahoo.com};
\end{tikzpicture}
\vglue -18pt
\hspace{187pt}
\parbox{350pt}{%
\hypertarget{chapter5d}{\hyperlink{intro}{\mbox{%
\begin{tikzpicture}
  \colorlet{even}{cyan!60!black}
  \colorlet{odd}{orange!100!black}
  \colorlet{links}{red!70!black}
  \colorlet{back}{yellow!20!white}
  \tikzset{
    box/.style={
      minimum height=15mm,
      inner sep=.7mm,
      outer sep=0mm,
      text width=120mm,
      text centered,
      font=\small\bfseries\sffamily,
      text=#1!50!black,
      draw=#1,
      line width=.25mm,
      top color=#1!5,
      bottom color=#1!40,
      shading angle=0,
      rounded corners=2.3mm,
      drop shadow={fill=#1!40!gray,fill opacity=.8},
      rotate=0,
    },
  }
  \node [box=even] {{%
  	\huge\textbf{Методичні рекомендації,}}
	\textbf{плани й завдання до семінарських і практичних занять}};
\end{tikzpicture}
}}}}\\[5pt]
\noindent
\begin{tikzpicture}
  \colorlet{even}{cyan!60!black}
  \colorlet{odd}{orange!100!black}
  \colorlet{links}{red!70!black}
  \colorlet{back}{yellow!20!white}
  \tikzset{
    box/.style={
      minimum height=15mm,
      inner sep=.7mm,
      outer sep=0mm,
      text width=120mm,
      text centered,
      font=\small\bfseries\sffamily,
      text=#1!50!black,
      draw=#1,
      line width=.25mm,
      top color=#1!5,
      bottom color=#1!40,
      shading angle=0,
      rounded corners=2.3mm,
      drop shadow={fill=#1!40!gray,fill opacity=.8},
      rotate=0,
    },
  }
	\node[box=links,xshift=3mm,yshift=1mm,
		minimum height=5pt,text width=325pt]
		at (0,-1.3) {\hyperlink{chapter5a}{\mbox{%
		\small \textcolor{black}{Тема 2.2. Принципи й функції
			державного управління}}}};
	\node[box=links,xshift=3mm,yshift=1mm,
		minimum height=5pt,text width=235pt]
		at (10.6,-1.3) {\hyperlink{chapter5b}{\mbox{%
		\small \textcolor{black}{Індивідуальні
		навчально-дослідницькі завдання}}}};
	\node[box=links,xshift=3mm,yshift=1mm,
		minimum height=5pt,text width=221pt]
		at (19.3,-1.3) {\hyperlink{chapter5b}{\mbox{%
		\small \textcolor{black}{
		Питання для самоконтролю та самоперевірки}}}};
	\node[box=links,xshift=3mm,yshift=1mm,
		minimum height=5pt,text width=140pt]
		at (26.35,-1.3) {\hyperlink{chapter5c}{\mbox{%
		\small \textcolor{black}{Додаткова література}}}};
\end{tikzpicture}
\vglue 25pt
%%%---> Old Version
%\begin{tikzpicture}
%	\node[name=s,shape=rectangle callout,
%		callout relative pointer={(1.25cm,-1cm)},
%			callout pointer width=2cm, inner xsep=2cm, inner ysep=1cm,
%				font=\Large\bfseries\sffamily, text centered,
%					shading angle=45] at (3,0) {Додаткова література};
%\end{tikzpicture}					
%%%---<
%%%---> NEW Version
\hyperlink{chapter5c}{\mbox{%
\begin{tikzpicture}
	\node[name=s,shape=rectangle callout,
		callout relative pointer={(1.25cm,-1cm)},
		callout pointer width=2cm, inner xsep=2cm, inner ysep=1cm,
		font=\Large\bfseries\sffamily, text centered,
		shading angle=45,
		postaction={path fading=south,
		fading angle=45,fill=blue, opacity=.5},
		left color=black, right color=red, draw=white,
		line width=2mm, text=white,
		shadow scale=3.25, shadow xshift=3pt,
		shadow yshift=3pt]
	at (3,0) {Додаткова література};
\end{tikzpicture}
}}
%%%---<
\vfill
\begin{itemize}
\item[]
	\scalebox{1.9}{\iconarticle}\qquad~~{\large
	{\bf Конопльов} В.}
	\tooltipanim{\large\bf Організаційно-правовий механізм підвищення
	ефективності управлінської діяльності}{23}{23}\quad
	{\large{\bf Право України}.
	--- {\bf 2005}. --- {\bf №9}. --- C. 35.}
\item[]
	\scalebox{1.9}{\iconarticle}\qquad~~{\large {\bf Журавський} В.}
\tooltipanim{\large\bf Щодо реформи адміністративно-територіального устрою України}{24}{24}
\quad{\large{\bf Право України}. --- {\bf 2005}. --- {\bf №8}. --- C. 16.}
\item[]
	\scalebox{1.9}{\iconarticle}\qquad~~{\large {\bf Капустинський} В.}
\tooltipanim{\large\bf Організаційно-правовий механізм підвищення ефективності
управлінської діяльності}{25}{25}\quad
{\large{\bf Право України}. --- {\bf 2005}.  --- {\bf №9}. --- С. 111.}
\item[]
	\scalebox{1.9}{\iconarticle}\qquad~~{\large {\bf Лагода} О.}
\tooltipanim{\large\bf Законність як правовий орієнтир системи
управлінської діяльності}{23}{23}\quad
{\large	{\bf Право України}. --- {\bf 2005}. --- {\bf №10}. --- С. 95.}
\item[]
	\scalebox{1.9}{\iconarticle}\qquad~~{\large {\bf Лазарєва} Н.}
	\tooltipanim{\large\bf Державні послуги у сфері освіти}{24}{24}\quad
{\large погляди на питання. {\bf Право України}. --- {\bf 2005}. ---
{\bf №11}. --- C. 17.}
\item[]
	\scalebox{1.9}{\iconarticle}\qquad~~{\large {\bf Ней} Н.}
	\tooltipanim{\large\bf Право України}{25}{25}\quad
	{\large--- {\bf 2004}. --- {\bf №12}. --- C. 28.}
\item[]
	\scalebox{1.9}{\iconarticle}\qquad~~{\large {\bf Развадовський} В.}
\tooltipanim{\large\bf Функції державного управління транспортною системою України}{23}{23}\quad
{\large	{\bf Право України}. --- {\bf 2004}. --- {\bf №5}. --- C. 121.}
\item[]
	\scalebox{1.9}{\iconarticle}\qquad~~{\large {\bf Рыжов} В.С.}
	\tooltipanim{\large\bf К судьбе государственного
	управления}{24}{24}\quad
	{\large{\bf Государство и право}.
	--- {\bf 1999}. --- {\bf №2}. --- С. 14.}
\item[]
	\scalebox{1.9}{\iconarticle}\qquad~~{\large
	{\bf Скомороха} В.} \tooltipanim{\large\bf Адміністративна реформа в
	Україні}{25}{25}\quad
	{\large	потрібне законодавче забезпечення (правові аспекти поділу і
	розмежування влади).\\
	\mbox{}\hspace{48pt}{\bf Право України}. --- {\bf 1999} --- {\bf №8}.
	--- C. 8.}
\item[]
	\scalebox{1.9}{\iconarticle}\qquad~~{\large {\bf Титарчук} В.}
	\tooltipanim{\large\bf Вдосконалення державного апарату}{23}{23}\quad
	{\large (окремі питання).
	{\bf Право України}. --- {\bf 1999}. --- {\bf №3}. --- C. 15.}
\item[]
	\scalebox{1.9}{\iconarticle}\qquad~~{\large {\bf Фролова} О.}
\tooltipanim{\large\bf Проблеми реформування інформаційно--методичного забезпечення управління}{24}{24}\quad{\large {\bf Право України}. --- {\bf 2004}. --- {\bf №12}.
	--- С. 87.}
\end{itemize}
\vfill
\begin{textblock}{60}(22.1,15)
	\hyperlink{chapter5c}{\mbox{%
	\tikz[every node/.style={font=\normalsize\bfseries\sffamily,signal,draw,
		text=white}]
		\node[signal to=east,
			minimum width=35pt, postaction={path fading=south,
			fading angle=45,fill=blue,opacity=.5}, left color=black, right color=red,
			draw=white, line width=1mm, drop shadow, rotate=180]
			{\rotatebox{180}{\mbox{Додаткова література: 1~стр}}};
			}}		
\end{textblock}
\vfill
\mbox{}
%%%-----------------------------------------------------------------------
\begin{textblock}{61}(25,-0.01)
\begin{tikzpicture}[even odd rule,rounded corners=2pt,x=10pt,y=10pt,drop shadow]
\filldraw[fill=yellow!90!black!40,drop shadow] (0,0)   rectangle (1,1)
	[xshift=5pt,yshift=5pt]   (0,0)   rectangle (1,1)
	[rotate=30]   (-1,-1) rectangle (2,2);
\node at (0,1.7) {\textbf{\thepage}};			      
\end{tikzpicture}
\end{textblock}
%%%--- Navigational panel top page
\begin{textblock}{62}(7.58,0.85)
\mbox{%%%--->
\Acrobatmenu{LastPage}{%
\tikz[baseline] \node[rectangle,inner sep=2pt,minimum height=3.1ex,
rounded corners,drop shadow,shadow scale=1,shadow xshift=.8ex,
shadow yshift=-.4ex,opacity=.7,fill=black!50,top color=red!90!black!50,
bottom color=red!80!black!80,draw=red!50!black!50,very thick,text=white,
text opacity=1,minimum width=3cm,font=\bfseries\sffamily] at (0,0) {К концу};
}\Acrobatmenu{GoBack}{%
\tikz[baseline] \node[rectangle,inner sep=2pt,minimum height=3.1ex,
rounded corners,drop shadow,shadow scale=1,shadow xshift=.8ex,
shadow yshift=-.4ex,opacity=.7,fill=black!50,top color=red!90!black!50,
bottom color=red!80!black!80,draw=red!50!black!50,very thick,text=white,
text opacity=1,minimum width=3cm,font=\bfseries\sffamily] at (4,0) {Назад};
}\Acrobatmenu{PrevPage}{%
\tikz[baseline] \node[rectangle,inner sep=2pt,minimum height=3.1ex,
rounded corners,drop shadow,shadow scale=1,shadow xshift=.8ex,
shadow yshift=-.4ex,opacity=.7,fill=black!50,top color=red!90!black!50,
bottom color=red!80!black!80,draw=red!50!black!50,very thick,text=white,
text opacity=1,minimum width=3cm,font=\bfseries\sffamily] at (8,0) {Предыдущий};
}\Acrobatmenu{NextPage}{%
\tikz[baseline] \node[rectangle,inner sep=2pt,minimum height=3.1ex,
rounded corners,drop shadow,shadow scale=1,shadow xshift=.8ex,
shadow yshift=-.4ex,opacity=.7,fill=black!50,top color=red!90!black!50,
bottom color=red!80!black!80,draw=red!50!black!50,very thick,text=white,
text opacity=1,minimum width=3cm,font=\bfseries\sffamily] at (12,0) {Следующий};
}\Acrobatmenu{GoForward}{%
\tikz[baseline] \node[rectangle,inner sep=2pt,minimum height=3.1ex,
rounded corners,drop shadow,shadow scale=1,shadow xshift=.8ex,
shadow yshift=-.4ex,opacity=.7,fill=black!50,top color=red!90!black!50,
bottom color=red!80!black!80,draw=red!50!black!50,very thick,text=white,
text opacity=1,minimum width=3cm,font=\bfseries\sffamily] at (16,0) {Вперед};
}\Acrobatmenu{FirstPage}{%
\tikz[baseline] \node[rectangle,inner sep=2pt,minimum height=3.1ex,
rounded corners,drop shadow,shadow scale=1,shadow xshift=.8ex,
shadow yshift=-.4ex,opacity=.7,fill=black!50,top color=red!90!black!50,
bottom color=red!80!black!80,draw=red!50!black!50,very thick,text=white,
text opacity=1,minimum width=3cm,font=\bfseries\sffamily] at (20,0) {К началу};
}\Acrobatmenu{FullScreen}{%
\tikz[baseline] \node[rectangle,inner sep=2pt,minimum height=3.1ex,
rounded corners,drop shadow,shadow scale=1,shadow xshift=.8ex,
shadow yshift=-.4ex,opacity=.7,fill=black!50,top color=red!90!black!50,
bottom color=red!80!black!80,draw=red!50!black!50,very thick,text=white,
text opacity=1,minimum width=3cm,font=\bfseries\sffamily] at (24,0) {Полный экран};
}\Acrobatmenu{Quit}{%
\tikz[baseline] \node[rectangle,inner sep=2pt,minimum height=3.1ex,
rounded corners,drop shadow,shadow scale=1,shadow xshift=.8ex,
shadow yshift=-.4ex,opacity=.7,fill=black!50,top color=red!90!black!50,
bottom color=red!80!black!80,draw=red!50!black!50,very thick,text=white,
text opacity=1,minimum width=3cm,font=\bfseries\sffamily] at (28,0) {Выход};
}	
}%%%---|
\end{textblock}
%%%-----------------------------------------------------------------------

\newpage
\disableTemplate{covers}
\disableTemplate{lawordercover}
\disableTiling
\AddToTemplate{susebgcard}
\begin{tikzpicture}[remember picture,overlay]
	  \node [rotate=0,scale=1,text opacity=0.2]
	      at (3,-22.2) {Капранов~О.~Г.~\copyright~2010~~~Luga\TeX @yahoo.com};
\end{tikzpicture}

\begin{textblock}{63}(0.54,0.449)
\tikzfading[name=fade inside,
inner color=transparent!80,
outer color=transparent!30
]
\begin{tikzfadingfrompicture}[name=luga]
	\node [text=transparent!20]
	{\fontsize{65}{65}\Huge\bfseries\sffamily\selectfont Lu\emph{g}a\TeX};
\end{tikzfadingfrompicture}
\begin{tikzpicture}
	% Checker board
	\shade [ball color=white,
		path fading=fade inside] (0,0) circle (3.5);
	\shade[path fading=luga,fit fading=false,
	left color=blue,right color=black]
	(-3.5,-1) rectangle (2,1);
\end{tikzpicture}
\end{textblock}

%\begin{tikzfadingfrompicture}[name=luga]
%	\node [text=transparent!20]
%	{\fontsize{65}{65}\Huge\bfseries\sffamily\selectfont Lu\emph{g}a\TeX};
%\end{tikzfadingfrompicture}
%% Now we use the fading in another picture:
%\begin{tikzpicture}
%	\shade[path fading=luga,fit fading=false,
%	left color=blue,right color=black]
%	(-3.5,-1) rectangle (2,1);
%\end{tikzpicture}

\begin{textblock}{64}(0.1,1.2)
\begin{tikzfadingfrompicture}[name=intv]
	\node [text=transparent!1]
	{\fontsize{25}{25}\bfseries\sffamily\selectfont Инт\emph{е}рактивная
		система знаний};
\end{tikzfadingfrompicture}
% Now we use the fading in another picture:
\begin{tikzpicture}
	\node[rectangle, inner sep=2pt, minimum height=9.1ex,
		rounded corners, very thick, minimum width=13.5cm,draw=white,
		inner color=transparent!80, outer color=transparent!30,
		draw opacity=0.8, fill opacity=0.2]{\mbox{}};
	\shade[path fading=intv,fit fading=false,
	left color=magenta!80!black,right color=black]
	(-15,-1) rectangle (15,1);
\end{tikzpicture}
\end{textblock}

\vglue 155pt

%%%---> \raggedright
\begin{minipage}{7.5cm}\centering
\textbf{%
\sffamily
\color{red!40!black} Интерактивная система знаний представляет
собой интерактивно электронные формы предоставления информации.
\begin{colormixin}{25!white}
Система разработана для создания интерактивных учебных лекцый,
экзаменационно --- модульных форм в виде интерактивного заполнения
и контроля заданий.
\color{magenta!10!black}
Ответы могут быть представлены в различной форме или зашифрованы
специальной формой.
\begin{colormixin}{85!black}
Вся система может быть зашифрована с использованием пароля, есть
возможность отключить такие функции как печать, выделение текста,
сохранение документа или создание скриншотов т.е. снимков экрана,
по усмотрению преподователя.
\color{green!80!black}
Экзаменационные и модульные контроли могут быть представлены в
различных формах, с возможностью отображения правельного решения
или просто ответа или ввиде ссумы очков правельных ответов.
\end{colormixin}
\color{yellow!80!black}
Система позволяет включать интерактивные и всплывающие окна, c
возможностью отображать документы разных форматов, видио и аудио
материалы встраиваются паралельно с лекциями в единый формат.
\end{colormixin}
\color{black}
Различные интерактивные формы в учебных материаллах демонстрируют
разнообразие и простоту их использования демонстрируя наглядность
информации с различными формами предоставления данных.}
\end{minipage}\\

\vglue 25pt

\hspace{30pt}
\tikz[baseline] \node[rectangle, inner sep=2pt, minimum height=3.1ex,
rounded corners, drop shadow, shadow scale=1, shadow xshift=.8ex,
shadow yshift=-.4ex, opacity=.7, fill=black!50,
top color=cyan!90!black!50, bottom color=cyan!80!black!80,
draw=cyan!50!black!50,very thick,text=white, text opacity=1,
minimum width=5cm]{Демонстрационная версия};

\vglue 25pt
%%%--->
\hspace{-25pt}
\hypertarget{mylugatex}{%
\parbox{350pt}{%
\scalebox{1.2}{%
{\color{blue}\Telefon}~{\color{brown}$93$-$68$-$11$}\quad
{\color{blue}\ding{38}}~{\color{brown}$43$-$23$}\quad
{\color{blue}\Mobilefone}~{\color{brown}$0997170609$}\quad
{\color{blue}\ding{39}}~{\color{brown}ком.№~$611$~УК-2}}
}}
%%%--->
\begin{textblock}{65}(20.5,4)
\begin{matrixtable}{6cm}{0.6cm}{
\head{Интерактивные учебные курсы}		& \mbox{}\\
Экономика					& \down  \\
Психология					& \up    \\
Философия					& \const \\
Английский язык				& \up    \\
Украинский язык				& \down  \\
Информатика					& \up    \\
Высшая математика			& \up    \\
Теоретическая физика		& \up    \\
Теория вероятности			& \up    \\
Аналитическая геометрия		& \down  \\
Диффернциальные уравнения	& \up	 \\
Функциональный анализ		& \up	 \\
Тензорный анализ			& \up	 \\
Компактные группы Ли		& \up	 \\
Административное право		& \up	 \\
Топология					& \down	 \\
Математический анализ		& \down	 \\
Абстрактные группы			& \down	 \\		
Алгебраическая геометрия	& \down	 \\
							& \\
							& \\
							& \\
							& \\
							& \\
							& \\
							& \\
							& \\
							& \\
							& \\
							& \\
							& \\
							& \\
							& \\
							& \\
							& \\
							& \\
							& \\
							& \\
							& \\
							& \\
							& \\
							& \\
							& \\
							& \\
							& \\
							& \\
							& \\
							& \\
							& \\
							& \\
							& \\
							& \\
							& \\
							& \\
							& \\
							& \\
							& \\
							& \\
							& \\
							& \\
							& \\
							& \\
							& \\
							& \\
							& \\
							& \\
							& \\
							& \\
							& \\
							& \\
{\bf Заказывайте любой учебній курс~!}	& \const \\							
}
\end{matrixtable}
\end{textblock}

\end{document}
%%%-----------------------------------------------------------------------
