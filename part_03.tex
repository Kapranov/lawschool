\newpage
%%%---> New  Page calendar 2010
%%%---> Main baground overlay
\begin{tikzpicture}[remember picture,overlay]
	\draw [line width=1mm,opacity=.25] 
		(current page.center) circle (1cm);

		\begin{pgfonlayer}{background}
			\colorlet{upperleft}{green!50!black!25}
			\colorlet{upperright}{orange!25}
			\colorlet{lowerleft}{red!25}
			\colorlet{lowerright}{blue!25}
			
			% The large rectangles:
			\fill [upperleft] (current page.center) rectangle ++(-18.2,15);
			\fill [upperright] (current page.center) rectangle ++(18.2,15);
			\fill [lowerleft] (current page.center) rectangle ++(-18.2,-15);
			\fill [lowerright] (current page.center) rectangle ++(18.2,-15);

			% The shadings:
			\shade [left color=upperleft,right color=upperright]
			([xshift=-1cm]current page.center) rectangle ++(2,15);
			\shade [left color=lowerleft,right color=lowerright]
			([xshift=-1cm]current page.center) rectangle ++(2,-15);
			\shade [top color=upperleft,bottom color=lowerleft]
			([yshift=-1cm]current page.center) rectangle ++(-18.2,2);
			\shade [top color=upperright,bottom color=lowerright]
			([yshift=-1cm]current page.center) rectangle ++(18.2,2);
		\end{pgfonlayer}
\end{tikzpicture}
\begin{tikzpicture}[remember picture,overlay]
	  \node [rotate=0,scale=2,text opacity=0.2]
	      at (27,1.7) {Капранов~О.~Г.~\copyright~2010~~~Luga\TeX @yahoo.com};
\end{tikzpicture}
\vglue -18pt
\hspace{187pt}
\parbox{350pt}{%
\hypertarget{studyear}{\hyperlink{mymedia}{%
\begin{tikzpicture}
  \colorlet{even}{cyan!60!black}
  \colorlet{odd}{orange!100!black}
  \colorlet{links}{red!70!black}
  \colorlet{back}{yellow!20!white}
  \tikzset{
    box/.style={
      minimum height=15mm,
      inner sep=.7mm,
      outer sep=0mm,
      text width=120mm,
      text centered,
      font=\small\bfseries\sffamily,
      text=#1!50!black,
      draw=#1,
      line width=.25mm,
      top color=#1!5,
      bottom color=#1!40,
      shading angle=0,
      rounded corners=2.3mm,
      drop shadow={fill=#1!40!gray,fill opacity=.8},
      rotate=0,
    },
  }
  \node [box=odd]{{\huge\textbf{Учебный год\quad 2010--2011}}};
\end{tikzpicture}
}}}\\

\begin{center}
\sffamily
\colorlet{winter}{blue}
\colorlet{spring}{green!60!black}
\colorlet{summer}{orange}
\colorlet{fall}{red}
% A counter, since TikZ is not clever enough (yet) to handle
% arbitrary angle systems.
%
% anchor=mid
\newcount\mycount
\begin{tikzpicture}
	[transform shape,
	every day/.style={font=\fontsize{6}{6}\selectfont}]
	\node [circular drop shadow={shadow scale=1.05},minimum size=4.33cm,
	decorate, decoration=zigzag, fill=blue!20,draw,thick,circle] 
   	 (maintime)
   		{\large\bfseries\sffamily Учебный год \the\year};
	\foreach \month/\monthcolor in
	{1/winter,2/winter,3/spring,4/spring,5/spring,6/summer,
	7/summer,8/summer,9/fall,10/fall,11/fall,12/winter}
	{
	% Computer angle:
	\mycount=\month
	\advance\mycount by -1
	\multiply\mycount by 30
	\advance\mycount by -90
	% The actual calendar
	\calendar at (\the\mycount:6.4cm)
	[
	dates=\the\year-\month-01 to \the\year-\month-last,
	]
	if (day of month=1) {\color{\monthcolor}\tikzmonthcode}
	if (Sunday) [red]
	if (all)
	{
	% Again, compute angle
	\mycount=1
	\advance\mycount by -\pgfcalendarcurrentday
	\multiply\mycount by 11
	\advance\mycount by 90
	\pgftransformshift{\pgfpointpolar{\mycount}{1.4cm}}
	};
	}

%%%---> New version
%	\begin{pgfonlayer}{background}
%	\colorlet{upperleft}{green!50!black!25}
%	\colorlet{upperright}{orange!25}
%	\colorlet{lowerleft}{red!25}
%	\colorlet{lowerright}{blue!25}
%
%	\fill [upperleft] (maintime) rectangle ++(-10,8.66);
%	\fill [upperright] (maintime) rectangle ++(10,8.66);
%	\fill [lowerleft] (maintime) rectangle ++(-10,-8.66);
%	\fill [lowerright] (maintime) rectangle ++(10,-8.66);
%
%	\shade [left color=upperleft,right color=upperright]
%	([xshift=-1cm]maintime) rectangle ++(2,8.66);
%	\shade [left color=lowerleft,right color=lowerright]
%	([xshift=-1cm]maintime) rectangle ++(2,-8.66);
%	\shade [top color=upperleft,bottom color=lowerleft]
%	([yshift=-1cm]maintime) rectangle ++(-10,2);
%	\shade [top color=upperright,bottom color=lowerright]
%	([yshift=-1cm]maintime) rectangle ++(10,2);
%\end{pgfonlayer}
%%%---> Old version
%	\begin{pgfonlayer}{background}
%		\clip[xshift=-1cm] (-.5\textwidth,-.5\textheight) rectangle ++(\textwidth,\textheight);
%		\colorlet{upperleft}{green!50!black!25}
%		\colorlet{upperright}{orange!25}
%		\colorlet{lowerleft}{red!25}
%		\colorlet{lowerright}{blue!25}
%		% The large rectangles:
%		\fill [upperleft] (Computational Complexity) rectangle ++(-20,20);
%		\fill [upperright] (Computational Complexity) rectangle ++(20,20);
%		\fill [lowerleft] (Computational Complexity) rectangle ++(-20,-20);
%		\fill [lowerright] (Computational Complexity) rectangle ++(20,-20);
%		% The shadings:
%		\shade [left color=upperleft,right color=upperright]
%		([xshift=-1cm]Computational Complexity) rectangle ++(2,20);
%		\shade [left color=lowerleft,right color=lowerright]
%		([xshift=-1cm]Computational Complexity) rectangle ++(2,-20);
%		\shade [top color=upperleft,bottom color=lowerleft]
%		([yshift=-1cm]Computational Complexity) rectangle ++(-20,2);
%		\shade [top color=upperright,bottom color=lowerright]
%		([yshift=-1cm]Computational Complexity) rectangle ++(20,2);
%	\end{pgfonlayer}
\end{tikzpicture}
\end{center}

\begin{textblock}{8}(0.4,13)
\sffamily\scriptsize
\tikz
\calendar [dates=2010-01-01 to 2010-12-31,
month list,month label left,month yshift=1.25em]
if (Sunday) [white];
\end{textblock}
\begin{textblock}{9}(14,13)
\sffamily\scriptsize
\tikz
\calendar [dates=2011-01-01 to 2011-12-31,
month list,month label left,month yshift=1.25em]
if (Sunday) [white];
\end{textblock}

\begin{textblock}{10}(1,3)
\begin{tikzpicture}
	\calendar
	[
	dates=2010-10-01 to 2010-11-last,
	week list,inner sep=2pt,month label above centered,
	month text=\%mt \%y0
	]
	if (at most=2010-10-29) [nodes={strike out,draw}]
	if (weekend) [black!50,nodes={draw=none}]
	;
\end{tikzpicture}
\end{textblock}
\begin{textblock}{11}(22.2,3)
	\begin{tikzpicture}
		\calendar
		[
		dates=\year-\month-\day+-25 to \year-\month-\day+25,
		week list,inner sep=2pt,month label above centered,
		month text=\textit{\%mt \%y0}
		]
		if (at least=\year-\month-\day) {}
		else [nodes={strike out,draw}]
		if (at most=\year-\month-\day+7)
		[green!50!black]
		if (between=\year-\month-\day+8 and \year-\month-\day+10)
		[red]
		if (Sunday)
		[gray,nodes={draw=none}]
		;
	\end{tikzpicture}
\end{textblock}

\begin{textblock}{12}(4.8,12)
	\begin{tikzpicture}
	\node [draw=blue,thick,fill=blue!50,
		font=\large\bfseries\sffamily] {\bf Учебный год: 2010};
	\end{tikzpicture}
\end{textblock}
\begin{textblock}{13}(18.3,12)
	\begin{tikzpicture}
	\node [draw=red,thick,fill=red!50,
		font=\large\bfseries\sffamily] {\bf Учебный год: 2011};
	\end{tikzpicture}
\end{textblock}
%%%-----------------------------------------------------------------------
\begin{textblock}{14}(25,-0.01)
\begin{tikzpicture}[even odd rule,rounded corners=2pt,x=10pt,y=10pt,drop shadow]
\filldraw[fill=yellow!90!black!40,drop shadow] (0,0)   rectangle (1,1)
	[xshift=5pt,yshift=5pt]   (0,0)   rectangle (1,1)
	[rotate=30]   (-1,-1) rectangle (2,2);
\node at (0,1.7) {\textbf{\thepage}};			      
\end{tikzpicture}
\end{textblock}
%%%--- Navigational panel top page
\begin{textblock}{15}(7.58,0.85)
\mbox{%%%--->
\Acrobatmenu{LastPage}{%
\tikz[baseline] \node[rectangle,inner sep=2pt,minimum height=3.1ex,
rounded corners,drop shadow,shadow scale=1,shadow xshift=.8ex,
shadow yshift=-.4ex,opacity=.7,fill=black!50,top color=red!90!black!50,
bottom color=red!80!black!80,draw=red!50!black!50,very thick,text=white,
text opacity=1,minimum width=3cm,font=\bfseries\sffamily] at (0,0) {К концу};
}\Acrobatmenu{GoBack}{%
\tikz[baseline] \node[rectangle,inner sep=2pt,minimum height=3.1ex,
rounded corners,drop shadow,shadow scale=1,shadow xshift=.8ex,
shadow yshift=-.4ex,opacity=.7,fill=black!50,top color=red!90!black!50,
bottom color=red!80!black!80,draw=red!50!black!50,very thick,text=white,
text opacity=1,minimum width=3cm,font=\bfseries\sffamily] at (4,0) {Назад};
}\Acrobatmenu{PrevPage}{%
\tikz[baseline] \node[rectangle,inner sep=2pt,minimum height=3.1ex,
rounded corners,drop shadow,shadow scale=1,shadow xshift=.8ex,
shadow yshift=-.4ex,opacity=.7,fill=black!50,top color=red!90!black!50,
bottom color=red!80!black!80,draw=red!50!black!50,very thick,text=white,
text opacity=1,minimum width=3cm,font=\bfseries\sffamily] at (8,0) {Предыдущий};
}\Acrobatmenu{NextPage}{%
\tikz[baseline] \node[rectangle,inner sep=2pt,minimum height=3.1ex,
rounded corners,drop shadow,shadow scale=1,shadow xshift=.8ex,
shadow yshift=-.4ex,opacity=.7,fill=black!50,top color=red!90!black!50,
bottom color=red!80!black!80,draw=red!50!black!50,very thick,text=white,
text opacity=1,minimum width=3cm,font=\bfseries\sffamily] at (12,0) {Следующий};
}\Acrobatmenu{GoForward}{%
\tikz[baseline] \node[rectangle,inner sep=2pt,minimum height=3.1ex,
rounded corners,drop shadow,shadow scale=1,shadow xshift=.8ex,
shadow yshift=-.4ex,opacity=.7,fill=black!50,top color=red!90!black!50,
bottom color=red!80!black!80,draw=red!50!black!50,very thick,text=white,
text opacity=1,minimum width=3cm,font=\bfseries\sffamily] at (16,0) {Вперед};
}\Acrobatmenu{FirstPage}{%
\tikz[baseline] \node[rectangle,inner sep=2pt,minimum height=3.1ex,
rounded corners,drop shadow,shadow scale=1,shadow xshift=.8ex,
shadow yshift=-.4ex,opacity=.7,fill=black!50,top color=red!90!black!50,
bottom color=red!80!black!80,draw=red!50!black!50,very thick,text=white,
text opacity=1,minimum width=3cm,font=\bfseries\sffamily] at (20,0) {К началу};
}\Acrobatmenu{FullScreen}{%
\tikz[baseline] \node[rectangle,inner sep=2pt,minimum height=3.1ex,
rounded corners,drop shadow,shadow scale=1,shadow xshift=.8ex,
shadow yshift=-.4ex,opacity=.7,fill=black!50,top color=red!90!black!50,
bottom color=red!80!black!80,draw=red!50!black!50,very thick,text=white,
text opacity=1,minimum width=3cm,font=\bfseries\sffamily] at (24,0) {Полный экран};
}\Acrobatmenu{Quit}{%
\tikz[baseline] \node[rectangle,inner sep=2pt,minimum height=3.1ex,
rounded corners,drop shadow,shadow scale=1,shadow xshift=.8ex,
shadow yshift=-.4ex,opacity=.7,fill=black!50,top color=red!90!black!50,
bottom color=red!80!black!80,draw=red!50!black!50,very thick,text=white,
text opacity=1,minimum width=3cm,font=\bfseries\sffamily] at (28,0) {Выход};
}
}%%%---|
\end{textblock}
\begin{textblock}{16}(13.39,1.305)
\begin{tikzpicture}[remember picture,overlay]
	\node {\mbox{\includegraphics[scale=0.99]{./laworder_bg_01}}};
\end{tikzpicture}
\end{textblock}	
%%%----------------------------------------------------------------------|
