\newpage
\disableTemplate{lawordercover}
\disableTiling
\begin{tikzpicture}[remember picture,overlay]
\begin{pgfonlayer}{background}
%%%---> original size
%\clip (-1.5,-5) rectangle ++(4,10);
%\clip (-6.3,-7.8) rectangle ++(14.3,15.2);
\colorlet{upperleft}{green!50!black!25}
\colorlet{upperright}{orange!25}
\colorlet{lowerleft}{red!25}
\colorlet{lowerright}{blue!25}

% The large rectangles:
\fill [upperleft] (27,-9) rectangle ++(-28.2,14);
\fill [upperright] (27,-9) rectangle ++(8,14);
\fill [lowerleft] (27,-9) rectangle ++(-28.2,-14);
\fill [lowerright] (27,-9) rectangle ++(8,-14);

% The shadings:
\shade [left color=upperleft,right color=upperright]
([xshift=-1cm]27,-9) rectangle ++(2,14);
\shade [left color=lowerleft,right color=lowerright]
([xshift=-1cm]27,-9) rectangle ++(2,-14);
\shade [top color=upperleft,bottom color=lowerleft]
([yshift=-1cm]27,-9) rectangle ++(-28.2,2);
\shade [top color=upperright,bottom color=lowerright]
([yshift=-1cm]27,-9) rectangle ++(8,2);
\end{pgfonlayer}
\end{tikzpicture}
\begin{tikzpicture}[remember picture,overlay]
	  \node [rotate=0,scale=2,text opacity=0.2]
	      at (27,1.7) {Капранов~О.~Г.~\copyright~2010~~~Luga\TeX @yahoo.com};
\end{tikzpicture}
\vglue -18pt
\hspace{187pt}
\parbox{350pt}{%
\hypertarget{forms}{\hyperlink{admintbl1}{%
\begin{tikzpicture}
  \colorlet{even}{cyan!60!black}
  \colorlet{odd}{orange!100!black}
  \colorlet{links}{red!70!black}
  \colorlet{back}{yellow!20!white}
  \tikzset{
    box/.style={
      minimum height=15mm,
      inner sep=.7mm,
      outer sep=0mm,
      text width=120mm,
      text centered,
      font=\small\bfseries\sffamily,
      text=#1!50!black,
      draw=#1,
      line width=.25mm,
      top color=#1!5,
      bottom color=#1!40,
      shading angle=0,
      rounded corners=2.3mm,
      drop shadow={fill=#1!40!gray,fill opacity=.8},
      rotate=0,
    },
  }
  \node [box=even]{{\huge\textbf{Форми та критерії\\
  	оцінювання знань студентів}}};
\end{tikzpicture}
}}}
\vglue 15pt
\begin{flushleft}
\begin{tikzpicture}
\node [rounded corners,fill=red!20,minimum width=993pt,
	minimum height=175pt,path fading=south] at (0,0) {\mbox{}};
\end{tikzpicture}
\end{flushleft}
\begin{textblock}{22}(0.8,3.5)
\begin{tikzpicture}
\node [text width=11cm] {\parbox{350pt}{%
	\begin{center}
		\textbf{Оцінювання знань студентів здійснюється за вимогами
		кредитно-модульної системи і базується на урахуванні кількості
		балів, отриманих ними в ході проведення поточного, модульного
		та підсумкового контролю, а також індивідуальної та\\
		самостійної роботи.}
	\end{center}
}};
\end{tikzpicture}
\end{textblock}
\begin{textblock}{23}(16,5.3)
\begin{tikzpicture}
\node [chamfered rectangle, white, fill=blue, double=blue, draw, very thick]
	{\textbf{максимальна кількість балів 30}};
\end{tikzpicture}
\end{textblock}
%%%---> Table
\begin{textblock}{24}(0.5,6)
\begin{tikzpicture}
\node [text width=35cm] {{\large\textbf{Поточний контроль здійснюється під час
проведення семінарських і практичних занять, консультацій з курсу
<<\textcolor{magenta}{Адміністративного права}>> і має за мету перевірку засвоєння
знань студентами та формування необхідних вмінь і навичок з окремих тем курсу.}}};
\end{tikzpicture}
\end{textblock}
\begin{textblock}{25}(10.4,3.63)
\begin{tikzpicture}
\node (tbl) {
\begin{tabularx}{600pt}{cccccc}
\arrayrulecolor{purple}
\textcolor{white}{%
\textbf{Поточне тестування (поточна успішність)}} & \textcolor{white}{\textbf{Самостійна робота, ІНДЗ}}
& \textcolor{white}{\textbf{Підсумковий тест (модуль-контроль)}} & \textcolor{white}{\textbf{Сума}}&\\
             &  &  & &\\[-5pt]
\textbf{Всього: 30}  & & & &\down\\
 & 30 & 40 & 100 &\up\\[2pt]
\textbf{(розрахунок за формулою)}             &    & & &\down\\[-5pt]
  & & & &\\
\end{tabularx}};

\begin{pgfonlayer}{background}
\draw[rounded corners,top color=red,bottom color=black,draw=white]
	($(tbl.north west)+(0.14,0)$) rectangle ($(tbl.north east)-(0.13,0.9)$);
\draw[rounded corners,top color=white,bottom color=black,
	middle color=red,draw=blue!20] ($(tbl.south west) +(0.12,0.5)$)
		rectangle ($(tbl.south east)-(0.12,0)$);
\draw[top color=blue!1,bottom color=blue!20,draw=white]
	($(tbl.north east)-(0.13,0.6)$) rectangle ($(tbl.south west)+(0.13,0.2)$);
\end{pgfonlayer}
\end{tikzpicture}
\end{textblock}
%%%--->
\vglue 5pt
\noindent
\hspace{5pt}{\large
Результати поточного контролю (від 
\tikz[baseline] \node [draw=blue,thick,fill=blue!50] at (0,0.14) {-1}; до
\tikz[baseline] \node [draw=blue,thick,fill=blue!50] at (0,0.14) {3};
балів) заносяться викладачем
до журналів обліку відвідування занять та їх успішності.}\\[15pt]
\pdfcomment[open=true,color=yellow,icolor=red,icon=Comment,voffset=-8pt,
	subject={Аннотация},author={Система оцінювання},hoffset=35pt]{%
При підведенні підсумків загальна кількість балів перемножується з
коефіцієнтом переводу балів з модульно-рейтингової системи до системи кредитів ECTS,
який розраховується в залежності від кількості аудиторних занять та консультацій
з курсу адміністративної відповідальності за формулою: 30/(Х*Y), де 30 -
максимальна кількість балів, Х - кількість семінарських і практичних
занять, Y - кількість набраних за поточним контролем балів.}
	\begin{tikzpicture}[remember picture, note/.style={rectangle callout,fill=#1}]
\node [note=green!50,opacity=.5,overlay,text opacity=1,
	callout relative pointer={(0,-0.3)}] at (10,0) {%
	\textbf{Так, згідно з встановленою в університеті системою оцінювання:}
	};
\end{tikzpicture}
\vglue 15pt
\parbox{550pt}{%
\begin{itemize}
\item[] \tikz[baseline] \node [draw=blue,thick,fill=blue!50] at (0,0.14) {3}; 
бали виставляється у тому випадку, коли студент знає весь обсяг
пройденого програмного матеріалу; теоретичний матеріал тісно пов'язує з
практикою застосування норм адміністративного права; вільно справляється з
вирішенням задач та завдань, застосовуючи знання при вирішенні практичних
ситуацій; вміло керується при їх розв'язанні відповідними нормами; матеріал
викладає лаконічно, послідовно, не припускає помилок, дає точні
формулювання; вміє самостійно узагальнювати та викладати матеріал; не
допускає ніяких помилок.
\item[] \tikz[baseline] \node [draw=blue,thick,fill=blue!50] at (0,0.14) {2}; 
бали виставляється у тому випадку, коли студент добре знає програмний
матеріал; на питання у межах програми з
\tooltipanim{<<Адміністративного права України>>}{1}{10}
відповідає без ускладнень, грамотно і по суті; вміє застосовувати в
практичній діяльності норми, що регламентують управлінські відносини; вільно володіє
практикою користування законодавством про адміністративні правопорушення;
не відчуває труднощів під час самостійного викладення матеріалу, вирішення
практичних задач, в окремих випадках допускає неточності.
\item[] \tikz[baseline] \node [draw=blue,thick,fill=blue!50] at (0,0.14) {1}; 
бал виставляється, коли студент дає в цілому вірну, але не повну
відповідь; допускає порушення послідовності і окремі неточності у викладені
програмного матеріалу.
\item[] \tikz[baseline] \node [draw=blue,thick,fill=blue!50] at (0,0.14) {-1}; 
бал виставляється у тому випадку, коли студент проявляє незнання
більшої частини або основних положень програмного матеріалу; не вміє вирішувати
практичні задачі або проявив повне незнання програмного матеріалу; допускає
багато помилок або викладає матеріал не по суті.
\end{itemize}
}\\
\vglue 5pt
\begin{tikzpicture}[remember picture, note/.style={rectangle callout,
	fill=#1}]
	\node [note=red!50, callout relative pointer={(0,1)}] at (6,0.5)
	{\textbf{Загальна кількість балів за аудиторні заняття у рамках кожного змістового
	модулю не може перевищувати 30 балів.}};
\end{tikzpicture}	
\begin{textblock}{26}(16.3,7)
\begin{tikzpicture} [mindmap,
		every node/.style={concept, circular drop shadow, execute at begin node=\hskip0pt},
		root concept/.append style={
		concept color=black, fill=white, line width=1ex, text=black},
		text=white, grow cyclic,
		level 1/.append style={level distance=6cm,sibling angle=45},
		level 2/.append style={level distance=3cm,sibling
		angle=35},scale=0.9]
%		\clip (-8,-8) rectangle ++(16,16);
		\node [root concept, text width=115pt,
			font=\Large\bfseries\sffamily] {Форми\\
				поточного\\ контролю} % root
		child [concept color=red] { node [text width=90pt, font=\small\bfseries\sffamily]
			{опитування під час проведення
			семінарських і практичних занять}
%		child { node {Problem Measures} }
		}
		child [concept color=blue] { node [text width=80pt,
			font=\small\bfseries\sffamily] {проведення контрольних робіт}
%		child { node {Turing Machines} }
		}
		child [concept color=orange] { node [text width=90pt,
			font=\small\bfseries\sffamily] {проведення\\ опитування з\\
			застосуванням\\ спеціальних\\ комп'ютерних\\ програм}
%		child { node {Complexity Measures} }
		}
		child [concept color=green!50!black] { node [text width=90pt,
			font=\small\bfseries\sffamily] {заслуховування\\
			рефератів,\\ доповідей}
			}
%		child { node {Exact Algorithms} }
		child [concept color=magenta!50!black] { node [font=\small\bfseries\sffamily] {вирішення задач}
			}
		child [concept color=yellow!50!black] { node
		[font=\small\bfseries\sffamily] {оцінювання виконання вправ}
			}
		child [concept color=cyan!50!black] { node
		[font=\small\bfseries\sffamily] {оцінювання вирішення
		практичних завдань} };
\end{tikzpicture}
\end{textblock}
%%%---------------------------------------------------------------------------------
\begin{textblock}{27}(25,-0.01)
\begin{tikzpicture}[even odd rule,rounded corners=2pt,x=10pt,y=10pt,drop shadow]
\filldraw[fill=yellow!90!black!40,drop shadow] (0,0)   rectangle (1,1)
	[xshift=5pt,yshift=5pt]   (0,0)   rectangle (1,1)
	[rotate=30]   (-1,-1) rectangle (2,2);
\node at (0,1.7) {\textbf{\thepage}};			      
\end{tikzpicture}
\end{textblock}
%%%--- Navigational panel top page
\begin{textblock}{28}(7.58,0.85)
\mbox{%%%--->
\Acrobatmenu{LastPage}{%
\tikz[baseline] \node[rectangle,inner sep=2pt,minimum height=3.1ex,
rounded corners,drop shadow,shadow scale=1,shadow xshift=.8ex,
shadow yshift=-.4ex,opacity=.7,fill=black!50,top color=red!90!black!50,
bottom color=red!80!black!80,draw=red!50!black!50,very thick,text=white,
text opacity=1,minimum width=3cm,font=\bfseries\sffamily] at (0,0) {К концу};
}\Acrobatmenu{GoBack}{%
\tikz[baseline] \node[rectangle,inner sep=2pt,minimum height=3.1ex,
rounded corners,drop shadow,shadow scale=1,shadow xshift=.8ex,
shadow yshift=-.4ex,opacity=.7,fill=black!50,top color=red!90!black!50,
bottom color=red!80!black!80,draw=red!50!black!50,very thick,text=white,
text opacity=1,minimum width=3cm,font=\bfseries\sffamily] at (4,0) {Назад};
}\Acrobatmenu{PrevPage}{%
\tikz[baseline] \node[rectangle,inner sep=2pt,minimum height=3.1ex,
rounded corners,drop shadow,shadow scale=1,shadow xshift=.8ex,
shadow yshift=-.4ex,opacity=.7,fill=black!50,top color=red!90!black!50,
bottom color=red!80!black!80,draw=red!50!black!50,very thick,text=white,
text opacity=1,minimum width=3cm,font=\bfseries\sffamily] at (8,0) {Предыдущий};
}\Acrobatmenu{NextPage}{%
\tikz[baseline] \node[rectangle,inner sep=2pt,minimum height=3.1ex,
rounded corners,drop shadow,shadow scale=1,shadow xshift=.8ex,
shadow yshift=-.4ex,opacity=.7,fill=black!50,top color=red!90!black!50,
bottom color=red!80!black!80,draw=red!50!black!50,very thick,text=white,
text opacity=1,minimum width=3cm,font=\bfseries\sffamily] at (12,0) {Следующий};
}\Acrobatmenu{GoForward}{%
\tikz[baseline] \node[rectangle,inner sep=2pt,minimum height=3.1ex,
rounded corners,drop shadow,shadow scale=1,shadow xshift=.8ex,
shadow yshift=-.4ex,opacity=.7,fill=black!50,top color=red!90!black!50,
bottom color=red!80!black!80,draw=red!50!black!50,very thick,text=white,
text opacity=1,minimum width=3cm,font=\bfseries\sffamily] at (16,0) {Вперед};
}\Acrobatmenu{FirstPage}{%
\tikz[baseline] \node[rectangle,inner sep=2pt,minimum height=3.1ex,
rounded corners,drop shadow,shadow scale=1,shadow xshift=.8ex,
shadow yshift=-.4ex,opacity=.7,fill=black!50,top color=red!90!black!50,
bottom color=red!80!black!80,draw=red!50!black!50,very thick,text=white,
text opacity=1,minimum width=3cm,font=\bfseries\sffamily] at (20,0) {К началу};
}\Acrobatmenu{FullScreen}{%
\tikz[baseline] \node[rectangle,inner sep=2pt,minimum height=3.1ex,
rounded corners,drop shadow,shadow scale=1,shadow xshift=.8ex,
shadow yshift=-.4ex,opacity=.7,fill=black!50,top color=red!90!black!50,
bottom color=red!80!black!80,draw=red!50!black!50,very thick,text=white,
text opacity=1,minimum width=3cm,font=\bfseries\sffamily] at (24,0) {Полный экран};
}\Acrobatmenu{Quit}{%
\tikz[baseline] \node[rectangle,inner sep=2pt,minimum height=3.1ex,
rounded corners,drop shadow,shadow scale=1,shadow xshift=.8ex,
shadow yshift=-.4ex,opacity=.7,fill=black!50,top color=red!90!black!50,
bottom color=red!80!black!80,draw=red!50!black!50,very thick,text=white,
text opacity=1,minimum width=3cm,font=\bfseries\sffamily] at (28,0) {Выход};
}
}%%%---|
\end{textblock}
\begin{textblock}{66}(13.39,1.305)	%(-0.38,-0.153)
\begin{tikzpicture}[remember picture,overlay]
	\node {\mbox{\includegraphics[scale=0.99]{./laworder_bg_01}}};
\end{tikzpicture}
\end{textblock}	
%%%---------------------------------------------------------------------------------
\newpage
\AddToTemplate{lawordercover}
\enableTiling
\begin{tikzpicture}[remember picture,overlay]
	  \node [rotate=0,scale=2,text opacity=0.2]
	      at (27,1.7) {Капранов~О.~Г.~\copyright~2010~~~Luga\TeX @yahoo.com};
\end{tikzpicture}
\noindent
\textbf{Модульний контроль} 
\tikz[baseline]
\node [chamfered rectangle, white, fill=blue, double=blue, draw, very thick]
	at (0,0.14) {\textbf{максимальна кількість балів 40}};
\textbf{--- це контроль знань
студентів, який здійснюється після вивчення кожного змістового модулю
дисципліни} \tooltipanim{<<Адміністративне право України>>}{1}{10}.\\
Бали за підсумками
модуль-контролю виставляються в журнали обліку відвідування занять та їх успішності. У
випадку, коли виявлено незадовільні знання, виставляється незадовільна
оцінка.

\begin{tikzpicture}[remember picture, note/.style={rectangle callout,
	fill=#1}]
	\node [note=red!50, callout relative pointer={(0,1)}] at (0,0.1)
	{\textbf{Загальна кількість балів за кожен модуль-контроль не може
		перевищувати 40 балів.}};
\node [note=blue!50, callout relative pointer={(0,-1)}] at (16,0.1) {%
\textbf{Таблиця визначення якості знань студентів під час проведення
модуль-контролю}};
\end{tikzpicture}	
\vglue 5pt
\begin{center}
\begin{tikzpicture}
\node (tbl) {
\begin{tabularx}{.6\textwidth}{cccl}
\arrayrulecolor{purple}
\multicolumn{4}{c}{\textcolor{white}{\bf Визначення якості знань студентів}}\\[1ex]
37-40 & \textbf{балів} & \tikz \node[ball color=magenta,circle,
text=black] {}; &\parbox{480pt}{\textbf{%
	свідчить про високий рівень знань, навичок, вмінь, які отримав
	студент з адміністративної відповідальності, його вміння самостійно
	розв'язувати складні теоретичні та практичні завдання;}
}\\[1ex]
\midrule
33-36 & \textbf{балів} & \tikz \node[ball color=magenta,circle,
	text=black] {}; &\parbox{480pt}{\textbf{%
	результат, вищий за середній, але з декількома несуттєвими недоліками;}
}\\[1ex]
\midrule
29-32 & \textbf{балів} & \tikz \node[ball color=magenta,circle,
	text=black] {}; &\parbox{480pt}{\textbf{%
	свідчить, що рівень теоретичної та практичної підготовки студента
	відповідає основним вимогам навчальної програми дисципліни, протемають
	місце окремі суттєві недоліки у роботі;}
}\\[1ex]
\midrule
25-28 & \textbf{балів} & \tikz \node[ball color=magenta,circle,
	text=black] {}; &\parbox{480pt}{\textbf{%
	посередній результат, зі значними недоліками;}
}\\[1ex]
\midrule
20-24 & \textbf{балів} & \tikz \node[ball color=magenta,circle,
	text=black] {}; &\parbox{480pt}{\textbf{%
	результат задовольняє мінімальним вимогам щодо знань, умінь та
	навичок з даної дисципліни.}
}\\[1ex]
\end{tabularx}};

\begin{pgfonlayer}{background}
\draw[rounded corners,top color=red,bottom color=black,draw=white]
	($(tbl.north west)+(0.14,0)$) rectangle ($(tbl.north east)-(0.13,0.9)$);
\draw[rounded corners,top color=white,bottom color=black,
	middle color=red,draw=blue!20] ($(tbl.south west) +(0.12,0.5)$)
		rectangle ($(tbl.south east)-(0.12,0)$);
\draw[top color=blue!1,bottom color=blue!20,draw=white]
	($(tbl.north east)-(0.13,0.6)$) rectangle ($(tbl.south west)+(0.13,0.2)$);
\end{pgfonlayer}
\end{tikzpicture}
\end{center}
\vglue 5pt
\textbf{Індивідуальна самостійна робота} 
\tikz[baseline]
\node [chamfered rectangle, white, fill=blue, double=blue, draw,
	very thick, text justified]
	at (0,0.14) {\textbf{максимальна кількість балів 30}};
є невід'ємною складовою вивчення дисципліни, вона сприяє поглибленому вивченню
теоретичного матеріалу,
~\tikz[every node/.style={signal,draw,text=white,signal to=nowhere},
minimum height=2pt, minimum width=40pt,font=\bfseries\sffamily,]
\node[fill=red!65!black, signal to=east] at (0,-0.14) {Читай далее!};
~\pdfcomment[open=true,color=yellow,icolor=red,icon=Comment,
subject={Аннотация},author={Індивідуальна самостійна робота},voffset=10pt]{%
закріпленню й узагальненню отриманих знань. Індивідуальна
самостійна робота здійснюється на основі методичних розробок кафедри адміністративного
права та адміністративної діяльності ЛДУВС імені Е.О. Дідоренка. Бали за
виконання індивідуальних завдань виставляються у журнали обліку
відвідування занять студентами і їх успішності окремою графою за кожний змістовий
модуль. Індивідуальна самостійна робота складається з завдань для самостійної
роботи та індивідуальних навчально-дослідних завдань, які запропоновані до кожної
теми.
Індивідуальне навчально-дослідницьке завдання є видом самостійної роботи
студентів та курсантів, яке виконується під керівництвом викладача в межах
навчальної програми з адміністративної відповідальності. Його виконання
передбачає наявність певних знань, умінь та навичок, одержаних під час
проведення аудиторних занять.}
\vglue 5pt
\begin{tikzpicture}
\colorlet{even}{cyan!60!black}
\colorlet{odd}{orange!100!black}
\colorlet{links}{red!70!black}
\colorlet{back}{yellow!20!white}
\tikzset{
	box/.style={
		minimum height=50mm,
		inner sep=2mm,
		outer sep=1mm,
		text width=320mm,
		text centered,
%		font=\huge\bfseries\sffamily,
		font=\bfseries\sffamily,
%		text=#1!50!black,
		text=black,
		draw=#1,
		line width=.25mm,
		top color=#1!5,
		bottom color=#1!40,
		shading angle=0,
		rounded corners=2.3mm,
		drop shadow={fill=#1!40!gray,fill opacity=.8},
		rotate=0,
		opacity=.6,
		text opacity=1,
		text justified,
	},
}
\node [box=back] (th) {%
\mbox{}\\
Індивідуальні навчально-дослідницькі завдання виконуються в формі
підготовки наукових статей, виступів (доповідей) на наукових конференціях, рефератів,
доповідей, рецензування наукових статей, складання бібліографії за темою
занять тощо. За надруковану наукову статтю студент може отримати
{\colorbox{blue}{\textcolor{white}{20 балів}}}, виступ
(доповідь) на науковій конференції --- 
{\colorbox{blue}{\textcolor{white}{10 балів}}},
реферат або доповідь ---
{\colorbox{blue}{\textcolor{white}{5 балів}}}.
Реферат або доповідь повинні бути виконані на комп'ютері та роздруковані.
Аркуші повинні бути пронумеровані. На титульному аркуші повинно бути
зазначене найменування навчального закладу, назва та вид роботи, прізвище та ініціали
виконавця. У кінці роботи повинен бути поданий перелік використаної
літератури, у такій послідовності: нормативно правові акти за їх юридичною силою;
навчально-методична, наукова, спеціальна література у алфавітному порядку.
Обсяг роботи від 
{\colorbox{blue}{\textcolor{white}{10}}}
до
{\colorbox{blue}{\textcolor{white}{15}}}
аркушів формату
{\colorbox{blue}{\textcolor{white}{А-4}}},
комп'ютерного тексту. Поля верхнє, нижнє ---
{\colorbox{blue}{\textcolor{white}{20 мм}}},
ліве ---
{\colorbox{blue}{\textcolor{white}{30мм}}},
праве ---
{\colorbox{blue}{\textcolor{white}{15мм}}}.
Шрифт ---
{\colorbox{blue}{\textcolor{white}{Times New Roman}}}
кегль ---
{\colorbox{blue}{\textcolor{white}{14}}},
міжрядковий інтервал ---
{\colorbox{blue}{\textcolor{white}{1,54}}}
Кожний доповідач має викласти зміст своєї роботи усно за
{\colorbox{blue}{\textcolor{white}{1--10}}}
хвилин та
бути готовим відповідати на запитання, які будуть поставлені студентами та
викладачем.
Студенти та курсанти, які підготували найбільш глибокі та змістовні
індивідуальні дослідження, можуть згодом доробити їх до рівня наукової
доповіді для виступу на конференції.
Рецензування наукових статей --- це критичний відгук на наукову роботу, яка
пропонується з викладом її позитивних та негативних сторін. Рецензування
здійснюється у письмовому вигляді, з дотриманням таких самих вимог щодо
оформлення, які ставляться до реферату чи доповіді.
Складання бібліографії за темою заняття --- це індивідуальна робота з
підготовки та збирання (пошуку) інформації про нормативно-правові акти й літературні
джерела, що стосуються курсу
\tooltipanim{<<Адміністративне право України>>}{11}{20},
які є у бібліотеці, відбір їх за певними ознаками, систематизація та складання
переліку.};
\node[box=links,anchor=south west,xshift=3mm,yshift=1mm,
	minimum height=5pt,text width=250pt,outer sep=0mm, inner sep=1mm]
	at (th.north west) {\mbox{}\hfill\small Індивідуальні навчально-дослідницькі
	завдання\hfill\mbox{}};
\end{tikzpicture}
\vglue 25pt
\pdflinecomment[color=green,icolor=blue,,subject={Top2},type=line,opacity=1,
line={25 15 25 135},caption=top,linebegin={/Diamond},lineend={/Diamond},
linewidth=2bp,captionhoffset=-5pt,captionvoffset=15pt]{Внимательно прочтите!}
\pdflinecomment[color=green,icolor=blue,,subject={Top2},type=line,opacity=1,
line={505 15 505 135},caption=top,linebegin={/Diamond},lineend={/Diamond},
linewidth=2bp,captionhoffset=-5pt,captionvoffset=-15pt]{Внимательно прочтите!}
\parbox{450pt}{%
Загальна кількість балів, яку студент може отримати за виконання
індивідуальної або самостійної роботи за підсумками змістового модулю навчальної
програми не може перевищувати
\tikz[baseline] \node [draw=blue,thick,fill=blue!50] at (0,0.14) {30}; балів.

Підведення підсумків змістового модулю навчання студентів та курсантів
здійснюється за формулою: 
\tikz[baseline] \node [draw=red,thick,fill=red!50] at (0,0.14) {П + М + І};
де
\tikz[baseline] \node [draw=red,thick,fill=red!50] at (0,0.14) {П};
--- кількість балів, набраних у ході поточного контролю,
\tikz[baseline] \node [draw=red,thick,fill=red!50] at (0,0.14) {М};
--- кількість балів, набраних у ході модуль-контролю,
\tikz[baseline] \node [draw=red,thick,fill=red!50] at (0,0.14) {І};
--- кількість балів, набраних у ході індивідуальної та самостійної роботи за
змістовним модулем навчальної програми.

Результати підсумкового контролю змістового модулю фіксуються викладачем у
журналі успішності за
\tikz[baseline] \node [draw=red,thick,fill=red!50] at (0,0.14) {100};
--бальною системою з урахуванням суми набраних
балів за всіма видами робіт, за наступною шкалою:}
\qquad\qquad\qquad
\parbox{350pt}{%
\begin{tikzpicture}
\node (tbl) {
\begin{tabularx}{450pt}{ccr}
\arrayrulecolor{purple}
	&\textcolor{white}{\textbf{Кількість балів набраних за результатами
		вивчення дисципліни}} &
		\mbox{\textcolor{white}{\textbf{Оцінка}}}\hspace{28pt}\mbox{}\\[1ex]
\up & \textbf{90-100} &\textbf{зараховано (А);}\\[1ex]
\up & \textbf{83-89}  &\textbf{зараховано (В);}\\[1ex]
\up & \textbf{75-82}  &\textbf{зараховано (С);}\\[1ex]
\up & \textbf{68-74}  &\textbf{зараховано (D);}\\[1ex]
\up & \textbf{60-67}  &\textbf{зараховано (Е);}\\[1ex]
\down & \textbf{1-59} &\textbf{не зараховано (F);}\\[1ex]
\end{tabularx}};

\begin{pgfonlayer}{background}
\draw[rounded corners,top color=red,bottom color=black,draw=white]
	($(tbl.north west)+(0.14,0)$) rectangle ($(tbl.north east)-(0.13,0.9)$);
\draw[rounded corners,top color=white,bottom color=black,
	middle color=red,draw=blue!20] ($(tbl.south west) +(0.12,0.5)$)
		rectangle ($(tbl.south east)-(0.12,0)$);
\draw[top color=blue!1,bottom color=blue!20,draw=white]
	($(tbl.north east)-(0.13,0.6)$) rectangle ($(tbl.south west)+(0.13,0.2)$);
\end{pgfonlayer}
\end{tikzpicture}
}
%%%-----------------------------------------------------------------------
\begin{textblock}{29}(25,-0.01)
\begin{tikzpicture}[even odd rule,rounded corners=2pt,x=10pt,y=10pt,drop shadow]
\filldraw[fill=yellow!90!black!40,drop shadow] (0,0)   rectangle (1,1)
	[xshift=5pt,yshift=5pt]   (0,0)   rectangle (1,1)
	[rotate=30]   (-1,-1) rectangle (2,2);
\node at (0,1.7) {\textbf{\thepage}};			      
\end{tikzpicture}
\end{textblock}
%%%--- Navigational panel top page
\begin{textblock}{30}(7.58,0.85)
\mbox{%%%--->
\Acrobatmenu{LastPage}{%
\tikz[baseline] \node[rectangle,inner sep=2pt,minimum height=3.1ex,
rounded corners,drop shadow,shadow scale=1,shadow xshift=.8ex,
shadow yshift=-.4ex,opacity=.7,fill=black!50,top color=red!90!black!50,
bottom color=red!80!black!80,draw=red!50!black!50,very thick,text=white,
text opacity=1,minimum width=3cm,font=\bfseries\sffamily] at (0,0) {К концу};
}\Acrobatmenu{GoBack}{%
\tikz[baseline] \node[rectangle,inner sep=2pt,minimum height=3.1ex,
rounded corners,drop shadow,shadow scale=1,shadow xshift=.8ex,
shadow yshift=-.4ex,opacity=.7,fill=black!50,top color=red!90!black!50,
bottom color=red!80!black!80,draw=red!50!black!50,very thick,text=white,
text opacity=1,minimum width=3cm,font=\bfseries\sffamily] at (4,0) {Назад};
}\Acrobatmenu{PrevPage}{%
\tikz[baseline] \node[rectangle,inner sep=2pt,minimum height=3.1ex,
rounded corners,drop shadow,shadow scale=1,shadow xshift=.8ex,
shadow yshift=-.4ex,opacity=.7,fill=black!50,top color=red!90!black!50,
bottom color=red!80!black!80,draw=red!50!black!50,very thick,text=white,
text opacity=1,minimum width=3cm,font=\bfseries\sffamily] at (8,0) {Предыдущий};
}\Acrobatmenu{NextPage}{%
\tikz[baseline] \node[rectangle,inner sep=2pt,minimum height=3.1ex,
rounded corners,drop shadow,shadow scale=1,shadow xshift=.8ex,
shadow yshift=-.4ex,opacity=.7,fill=black!50,top color=red!90!black!50,
bottom color=red!80!black!80,draw=red!50!black!50,very thick,text=white,
text opacity=1,minimum width=3cm,font=\bfseries\sffamily] at (12,0) {Следующий};
}\Acrobatmenu{GoForward}{%
\tikz[baseline] \node[rectangle,inner sep=2pt,minimum height=3.1ex,
rounded corners,drop shadow,shadow scale=1,shadow xshift=.8ex,
shadow yshift=-.4ex,opacity=.7,fill=black!50,top color=red!90!black!50,
bottom color=red!80!black!80,draw=red!50!black!50,very thick,text=white,
text opacity=1,minimum width=3cm,font=\bfseries\sffamily] at (16,0) {Вперед};
}\Acrobatmenu{FirstPage}{%
\tikz[baseline] \node[rectangle,inner sep=2pt,minimum height=3.1ex,
rounded corners,drop shadow,shadow scale=1,shadow xshift=.8ex,
shadow yshift=-.4ex,opacity=.7,fill=black!50,top color=red!90!black!50,
bottom color=red!80!black!80,draw=red!50!black!50,very thick,text=white,
text opacity=1,minimum width=3cm,font=\bfseries\sffamily] at (20,0) {К началу};
}\Acrobatmenu{FullScreen}{%
\tikz[baseline] \node[rectangle,inner sep=2pt,minimum height=3.1ex,
rounded corners,drop shadow,shadow scale=1,shadow xshift=.8ex,
shadow yshift=-.4ex,opacity=.7,fill=black!50,top color=red!90!black!50,
bottom color=red!80!black!80,draw=red!50!black!50,very thick,text=white,
text opacity=1,minimum width=3cm,font=\bfseries\sffamily] at (24,0) {Полный экран};
}\Acrobatmenu{Quit}{%
\tikz[baseline] \node[rectangle,inner sep=2pt,minimum height=3.1ex,
rounded corners,drop shadow,shadow scale=1,shadow xshift=.8ex,
shadow yshift=-.4ex,opacity=.7,fill=black!50,top color=red!90!black!50,
bottom color=red!80!black!80,draw=red!50!black!50,very thick,text=white,
text opacity=1,minimum width=3cm,font=\bfseries\sffamily] at (28,0) {Выход};
}	
}%%%---|
\end{textblock}
%%%---------------------------------------------------------------------------------
