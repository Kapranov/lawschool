\newpage
\begin{tikzpicture}[remember picture,overlay]
	  \node [rotate=0,scale=2,text opacity=0.2]
	      at (27,1.7) {Капранов~О.~Г.~\copyright~2010~~~Luga\TeX @yahoo.com};
\end{tikzpicture}
\vglue -18pt
\hspace{187pt}
\parbox{350pt}{%
\hypertarget{preface}{\hyperlink{forms}{\mbox{%
\begin{tikzpicture}
  \colorlet{even}{cyan!60!black}
  \colorlet{odd}{orange!100!black}
  \colorlet{links}{red!70!black}
  \colorlet{back}{yellow!20!white}
  \tikzset{
    box/.style={
      minimum height=15mm,
      inner sep=.7mm,
      outer sep=0mm,
      text width=120mm,
      text centered,
      font=\small\bfseries\sffamily,
      text=#1!50!black,
      draw=#1,
      line width=.25mm,
      top color=#1!5,
      bottom color=#1!40,
      shading angle=0,
      rounded corners=2.3mm,
      drop shadow={fill=#1!40!gray,fill opacity=.8},
      rotate=0,
    },
  }
  \node [box=even]{{%
  	\huge\textbf{Загальні положення}}};
\end{tikzpicture}
}}}}
\vglue 15pt
\parbox{800pt}{%
\pdffreetextcomment[type=freetext,opacity=0.5,justification=right,height=0.5cm,
width=4.8cm,voffset=0.7cm,hoffset=1.2cm,color=red!30!yellow]{%
Адміністративне право}
%{0.045 0.278 0.643}
\pdfmarkupcomment[color={0.54 0.117 0.136},opacity=1.0,markup=Squiggly]{
є однією із профілюючих дисциплін у навчальних закладах
системи МВС України}{}. Вона передбачає вивчення комплексу\\ теоретичних положень,
правових інститутів, діяльності державних й інших органів управління, а також
широкого діапазо-\\ну суспільних відносин, що складаються в сфері виконавчої влади
з урахуванням адміністративно-правового регулювання.

\pdffreetextcomment[color={red!30!yellow},height=2.9cm,width=9cm,opacity=0.8,
voffset=15pt,hoffset=580pt,font=Georgia,fontsize=9pt,fontcolor=black,
justification=left,linewidth=2bp,bse=cloudy,bsei=1.3,type=callout,
line={100 100},icolor=blue]{%
Адміністративне право базується на положеннях Конституції України, Законів
України, указів Президента України, постанов Кабінету Міністрів України та
інших нормативних актів, а також правових положеннях суміжних юридичних
дисциплін. Воно тісно пов'язане з конституційним, цивільним, фінансовим,
трудовим, кримінальним правом й іншими галузями права.}


Разом з тим, дана дисципліна має засадниче методологічне значення для курсів
\tooltipanim{%
\tikz[baseline] \node[rectangle,inner sep=2pt,minimum height=3.1ex,rounded corners,
drop shadow,shadow scale=1,shadow xshift=.8ex,shadow yshift=-.4ex,opacity=.7,
fill=black!50,top color=blue!90!black!50,bottom color=blue!80!black!80,
draw=blue!50!black!50,very thick,text=white,text opacity=1]{%
Організація діяльності ОВС};}{21}{21},
\tooltipanim{%
\tikz[baseline] \node[rectangle,inner sep=2pt,minimum height=3.1ex,rounded corners,
drop shadow,shadow scale=1,shadow xshift=.8ex,shadow yshift=-.4ex,opacity=.7,
fill=black!50,top color=blue!90!black!50,bottom color=blue!80!black!80,
draw=blue!50!black!50,very thick,text=white,text opacity=1]{%
Адміністративна діяльність ОВС};}{22}{22},
\tooltipanim{%
\tikz[baseline] \node[rectangle,inner sep=2pt,minimum height=3.1ex,rounded corners,
drop shadow,shadow scale=1,shadow xshift=.8ex,shadow yshift=-.4ex,opacity=.7,
fill=black!50,top color=blue!90!black!50,bottom color=blue!80!black!80,
draw=blue!50!black!50,very thick,text=white,text opacity=1]{%
Адміністративний процес};}{23}{23},
\tooltipanim{%
\tikz[baseline] \node[rectangle,inner sep=2pt,minimum height=3.1ex,rounded corners,
drop shadow,shadow scale=1,shadow xshift=.8ex,shadow yshift=-.4ex,opacity=.7,
fill=black!50,top color=blue!90!black!50,bottom color=blue!80!black!80,
draw=blue!50!black!50,very thick,text=white,text opacity=1]{%
Основи управління в ОВС};}{24}{24},
що вивчаються у вищих
навчальних закладах МВС України. Зміст і структура навчального курсу
<<\textcolor{magenta}{\textbf{Адміністративне право}}>> {\color{magenta}
\underline{подаються відповідно до положень основних програмних
документів \textcolor{red}{Верховної Ради України},\textcolor{red}{Президента
України}, \textcolor{red}{Кабінету Міністрів України}}}, МВС України з урахуванням
розвитку й удосконалення організації та
функціонування системи органів внутрішніх справ.

Для глибшого висвітлення фундаментальних засад науки адміністративного права
дана дисципліна викладається з використанням результатів найновіших теоретичних
досліджень українських та зарубіжних учених, а також положень Концепції
адміністративної реформи в Україні та проекту Концепції реформи
адміністративного права. В ході вивчення курсу роз'яснюються актуальні
дискусійні питання загальної адміністративно-правової проблематики:

\begin{itemize}
\item[] \tikz[baseline] \node[ball color=magenta,circle,text=black] {};\quad
		\tikz[baseline] \node {%
	\pdfmarkupcomment[author={Сноска},color=green,opacity=1.0,markup=Highlight]{
	\textbf{загальноєвропейський контекст становлення адміністративного права
	та формування системи національних джерел галузі};}{загальної адміністративно-правової
	проблематики}};
\item[] \tikz[baseline] \node[ball color=magenta,circle,text=black] {};\quad
		\tikz[baseline] \node {%
	\pdfmarkupcomment[author={Сноска},color=green,opacity=1.0,markup=Highlight]{
	\textbf{розкриття принципово нової суспільної ролі виконавчої влади, що полягає у
	служінні інтересам людини, з метою наближення}}{загальної адміністративно-правової
	проблематики}};\\\mbox{}\hspace{15pt}
	\tikz[baseline] \node {%
	\pdfmarkupcomment[author={Сноска},color=green,opacity=1.0,markup=Highlight]{
	\textbf{українського адміністративного права до європейських стандартів};}{загальної
	адміністративно-правової проблематики}};
\item[] \tikz[baseline] \node[ball color=magenta,circle,text=black] {};\quad
		\tikz[baseline] \node {%
	\pdfmarkupcomment[author={Сноска},color=green,opacity=1.0,markup=Highlight]{
	\textbf{концептуальні засади побудови і розвитку системи органів виконавчої влади
	відповідно до вимог здійснюваної в Україні адміністративної реформи};}{
	загальної адміністративно-правової проблематики}};
\item[] \tikz[baseline] \node[ball color=magenta,circle,text=black] {};\quad
		\tikz[baseline] \node {%
	\pdfmarkupcomment[author={Сноска},color=green,opacity=1.0,markup=Highlight]{
	\textbf{розвиток дійових засобів правового захисту громадян у сфері виконавчої влади};}{
	загальної адміністративно-правової проблематики}};
\item[] \tikz[baseline] \node[ball color=magenta,circle,text=black] {};\quad
		\tikz[baseline] \node {%
	\pdfmarkupcomment[author={Сноска},color=green,opacity=1.0,markup=Highlight]{
	\textbf{особливості адміністративної відповідальності юридичних осіб,
	та інші питання}.}{загальної адміністративно-правової проблематики}};
\end{itemize}


Тематичним планом з курсу адміністративного права передбачені лекції,
семінарські та практичні заняття,\\ важливою складовою частиною навчального
процесу є самостійна робота курсантів і студентів.

\tikz[remember picture, overlay]
	\node[rectangle callout,cloud puffs=15,aspect=2.5,cloud puff arc=120,
		shading=ball,text=white,overlay,rounded corners,thick,
		 callout pointer arc=90,callout pointer width=1.3cm,
		 callout absolute pointer={(14.5,-0.4)},fill=red]
	 at (21.3,2.2) {\textbf{Семінарські і практичні заняття проводяться з метою}};

\begin{itemize}
\item[] \tikz[baseline] \node[ball color=magenta,circle,text=black] {};\quad
		\tikz[baseline] \node {%
	\pdfmarkupcomment[author={Сноска},color=green,opacity=1.0,markup=Highlight]{
\textbf{вивчення чинного адміністративного законодавства, з'ясування останніх змін і
доповнень, внесених до нього, застосовування його норм};}{Семінарські і практичні
заняття}};\\\mbox{\hspace{15pt}}
\tikz[baseline] \node {%
\pdfmarkupcomment[author={Сноска},color=green,opacity=1.0,markup=Highlight]{
\textbf{у конкретних практичних ситуаціях під час виконання службових
обов'язків};}{Семінарські і практичні заняття}};
\item[] \tikz[baseline] \node[ball color=magenta,circle,text=black] {};\quad
		\tikz[baseline] \node {%
	\pdfmarkupcomment[author={Сноска},color=green,opacity=1.0,markup=Highlight]{
\textbf{поглиблення знань з найбільш важливих теоретичних положень курсу, кращого
засвоєння змін, що відбуваються};}{Семінарські і практичні
заняття}};\\\mbox{\hspace{15pt}}
\tikz[baseline] \node {%
\pdfmarkupcomment[author={Сноска},color=green,opacity=1.0,markup=Highlight]{
\textbf{у сфері державної виконавчої влади на сучасному етапі};}{Семінарські
і практичні заняття}};
\item[] \tikz[baseline] \node[ball color=magenta,circle,text=black] {};\quad
		\tikz[baseline] \node {%
	\pdfmarkupcomment[author={Сноска},color=green,opacity=1.0,markup=Highlight]{
\textbf{прищеплення комплексу практичних вмінь і навичок, необхідних для формування
високої професійної майстерності}.}{Семінарські і практичні заняття}};
\end{itemize}		

Під час проведення практичних занять курсанти та студенти, керуючись
положеннями адміністративного законодавства, вирішують запропоновані задачі,
аналізують конкретні ситуації та дають їм правову оцінку.

\pdfmarkupcomment[author={Основна мета},color={0.15 0.25 0.15},opacity=0.4,
markup=Highlight,icolor=white,fontcolor=red]{%
Основною метою самостійної роботи є глибоке всебічне оволодіння матеріалом по
тій або іншій темі, розвиток навичок роботи з навчальною літературою 
нормативними актами}{},
\pdfmarkupcomment[author={Основна мета},color={0.15 0.25 0.15},opacity=0.4,
markup=Highlight]{%
уміння узагальнювати практичні результати, готувати
довідки, виступи, реферати й ін. При виникненні питань необхідно звертатися за
індивідуальними консультаціями до}{}
\pdfmarkupcomment[author={Основна мета},color={0.15 0.25 0.15},opacity=0.4,
markup=Highlight]{%
викладачів кафедри}{}.

У даному навчально-методичному посібнику, крім нормативних і законодавчих актів
України, основної навчальної літератури до всіх тем, наведених у спеціальному
розділі, курсантам і студентам по кожному заняттю пропонується перелік
нормативних актів, додаткової навчальної літератури й наукових статей
періодичних видань з проблематики тем, що вивчаються.

\textcolor{cyan}{\textbf{%
Підсумковим контролем якості вивчення дисципліни є складання іспиту з
урахуванням рейтингу-контролю й участі в підсумковій науково-теоретичній
конференції курсантів і студентів}}. Тематика повідомлень і доповідей на
конференцію запропонована в окремому розділі даних рекомендацій.
}
\begin{textblock}{19}(22.1,3.3)
\mbox{\includegraphics[scale=1.1]{judge.png}}
\end{textblock}
%%%-----------------------------------------------------------------------
\begin{textblock}{20}(25,-0.01)
%\begin{textblock}{5}(25,0)
\begin{tikzpicture}[even odd rule,rounded corners=2pt,x=10pt,y=10pt,drop shadow]
\filldraw[fill=yellow!90!black!40,drop shadow] (0,0)   rectangle (1,1)
	[xshift=5pt,yshift=5pt]   (0,0)   rectangle (1,1)
	[rotate=30]   (-1,-1) rectangle (2,2);
\node at (0,1.7) {\textbf{\thepage}};			      
\end{tikzpicture}
\end{textblock}
%%%--- Navigational panel top page
\begin{textblock}{21}(7.58,0.85)
\mbox{%%%--->
\Acrobatmenu{LastPage}{%
\tikz[baseline] \node[rectangle,inner sep=2pt,minimum height=3.1ex,
rounded corners,drop shadow,shadow scale=1,shadow xshift=.8ex,
shadow yshift=-.4ex,opacity=.7,fill=black!50,top color=red!90!black!50,
bottom color=red!80!black!80,draw=red!50!black!50,very thick,text=white,
text opacity=1,minimum width=3cm,font=\bfseries\sffamily] at (0,0) {К концу};
}\Acrobatmenu{GoBack}{%
\tikz[baseline] \node[rectangle,inner sep=2pt,minimum height=3.1ex,
rounded corners,drop shadow,shadow scale=1,shadow xshift=.8ex,
shadow yshift=-.4ex,opacity=.7,fill=black!50,top color=red!90!black!50,
bottom color=red!80!black!80,draw=red!50!black!50,very thick,text=white,
text opacity=1,minimum width=3cm,font=\bfseries\sffamily] at (4,0) {Назад};
}\Acrobatmenu{PrevPage}{%
\tikz[baseline] \node[rectangle,inner sep=2pt,minimum height=3.1ex,
rounded corners,drop shadow,shadow scale=1,shadow xshift=.8ex,
shadow yshift=-.4ex,opacity=.7,fill=black!50,top color=red!90!black!50,
bottom color=red!80!black!80,draw=red!50!black!50,very thick,text=white,
text opacity=1,minimum width=3cm,font=\bfseries\sffamily] at (8,0) {Предыдущий};
}\Acrobatmenu{NextPage}{%
\tikz[baseline] \node[rectangle,inner sep=2pt,minimum height=3.1ex,
rounded corners,drop shadow,shadow scale=1,shadow xshift=.8ex,
shadow yshift=-.4ex,opacity=.7,fill=black!50,top color=red!90!black!50,
bottom color=red!80!black!80,draw=red!50!black!50,very thick,text=white,
text opacity=1,minimum width=3cm,font=\bfseries\sffamily] at (12,0) {Следующий};
}\Acrobatmenu{GoForward}{%
\tikz[baseline] \node[rectangle,inner sep=2pt,minimum height=3.1ex,
rounded corners,drop shadow,shadow scale=1,shadow xshift=.8ex,
shadow yshift=-.4ex,opacity=.7,fill=black!50,top color=red!90!black!50,
bottom color=red!80!black!80,draw=red!50!black!50,very thick,text=white,
text opacity=1,minimum width=3cm,font=\bfseries\sffamily] at (16,0) {Вперед};
}\Acrobatmenu{FirstPage}{%
\tikz[baseline] \node[rectangle,inner sep=2pt,minimum height=3.1ex,
rounded corners,drop shadow,shadow scale=1,shadow xshift=.8ex,
shadow yshift=-.4ex,opacity=.7,fill=black!50,top color=red!90!black!50,
bottom color=red!80!black!80,draw=red!50!black!50,very thick,text=white,
text opacity=1,minimum width=3cm,font=\bfseries\sffamily] at (20,0) {К началу};
}\Acrobatmenu{FullScreen}{%
\tikz[baseline] \node[rectangle,inner sep=2pt,minimum height=3.1ex,
rounded corners,drop shadow,shadow scale=1,shadow xshift=.8ex,
shadow yshift=-.4ex,opacity=.7,fill=black!50,top color=red!90!black!50,
bottom color=red!80!black!80,draw=red!50!black!50,very thick,text=white,
text opacity=1,minimum width=3cm,font=\bfseries\sffamily] at (24,0) {Полный экран};
}\Acrobatmenu{Quit}{%
\tikz[baseline] \node[rectangle,inner sep=2pt,minimum height=3.1ex,
rounded corners,drop shadow,shadow scale=1,shadow xshift=.8ex,
shadow yshift=-.4ex,opacity=.7,fill=black!50,top color=red!90!black!50,
bottom color=red!80!black!80,draw=red!50!black!50,very thick,text=white,
text opacity=1,minimum width=3cm,font=\bfseries\sffamily] at (28,0) {Выход};
}
}%%%---|
\end{textblock}
%%%----------------------------------------------------------------------|
