\AddToTemplate{lawordercover}
\enableTiling
\newpage
\disableTemplate{covers}
\begin{tikzpicture}[remember picture,overlay]
	  \node [rotate=0,scale=2,text opacity=0.2]
	      at (27,1.7) {Капранов~О.~Г.~\copyright~2010~~~Luga\TeX @yahoo.com};
\end{tikzpicture}
%%%-----------------------------------------------------------------------
\begin{textblock}{17}(25,-0.01)
%\begin{textblock}{5}(25,0)
\begin{tikzpicture}[even odd rule,rounded corners=2pt,x=10pt,y=10pt,drop shadow]
\filldraw[fill=yellow!90!black!40,drop shadow] (0,0)   rectangle (1,1)
	[xshift=5pt,yshift=5pt]   (0,0)   rectangle (1,1)
	[rotate=30]   (-1,-1) rectangle (2,2);
\node at (0,1.7) {\textbf{\thepage}};			      
\end{tikzpicture}
\end{textblock}
%%%--- Navigational panel top page
\begin{textblock}{18}(7.58,0.85)
\mbox{%%%--->
\Acrobatmenu{LastPage}{%
\tikz[baseline] \node[rectangle,inner sep=2pt,minimum height=3.1ex,
rounded corners,drop shadow,shadow scale=1,shadow xshift=.8ex,
shadow yshift=-.4ex,opacity=.7,fill=black!50,top color=red!90!black!50,
bottom color=red!80!black!80,draw=red!50!black!50,very thick,text=white,
text opacity=1,minimum width=3cm,font=\bfseries\sffamily] at (0,0) {К концу};
}\Acrobatmenu{GoBack}{%
\tikz[baseline] \node[rectangle,inner sep=2pt,minimum height=3.1ex,
rounded corners,drop shadow,shadow scale=1,shadow xshift=.8ex,
shadow yshift=-.4ex,opacity=.7,fill=black!50,top color=red!90!black!50,
bottom color=red!80!black!80,draw=red!50!black!50,very thick,text=white,
text opacity=1,minimum width=3cm,font=\bfseries\sffamily] at (4,0) {Назад};
}\Acrobatmenu{PrevPage}{%
\tikz[baseline] \node[rectangle,inner sep=2pt,minimum height=3.1ex,
rounded corners,drop shadow,shadow scale=1,shadow xshift=.8ex,
shadow yshift=-.4ex,opacity=.7,fill=black!50,top color=red!90!black!50,
bottom color=red!80!black!80,draw=red!50!black!50,very thick,text=white,
text opacity=1,minimum width=3cm,font=\bfseries\sffamily] at (8,0) {Предыдущий};
}\Acrobatmenu{NextPage}{%
\tikz[baseline] \node[rectangle,inner sep=2pt,minimum height=3.1ex,
rounded corners,drop shadow,shadow scale=1,shadow xshift=.8ex,
shadow yshift=-.4ex,opacity=.7,fill=black!50,top color=red!90!black!50,
bottom color=red!80!black!80,draw=red!50!black!50,very thick,text=white,
text opacity=1,minimum width=3cm,font=\bfseries\sffamily] at (12,0) {Следующий};
}\Acrobatmenu{GoForward}{%
\tikz[baseline] \node[rectangle,inner sep=2pt,minimum height=3.1ex,
rounded corners,drop shadow,shadow scale=1,shadow xshift=.8ex,
shadow yshift=-.4ex,opacity=.7,fill=black!50,top color=red!90!black!50,
bottom color=red!80!black!80,draw=red!50!black!50,very thick,text=white,
text opacity=1,minimum width=3cm,font=\bfseries\sffamily] at (16,0) {Вперед};
}\Acrobatmenu{FirstPage}{%
\tikz[baseline] \node[rectangle,inner sep=2pt,minimum height=3.1ex,
rounded corners,drop shadow,shadow scale=1,shadow xshift=.8ex,
shadow yshift=-.4ex,opacity=.7,fill=black!50,top color=red!90!black!50,
bottom color=red!80!black!80,draw=red!50!black!50,very thick,text=white,
text opacity=1,minimum width=3cm,font=\bfseries\sffamily] at (20,0) {К началу};
}\Acrobatmenu{FullScreen}{%
\tikz[baseline] \node[rectangle,inner sep=2pt,minimum height=3.1ex,
rounded corners,drop shadow,shadow scale=1,shadow xshift=.8ex,
shadow yshift=-.4ex,opacity=.7,fill=black!50,top color=red!90!black!50,
bottom color=red!80!black!80,draw=red!50!black!50,very thick,text=white,
text opacity=1,minimum width=3cm,font=\bfseries\sffamily] at (24,0) {Полный экран};
}\Acrobatmenu{Quit}{%
\tikz[baseline] \node[rectangle,inner sep=2pt,minimum height=3.1ex,
rounded corners,drop shadow,shadow scale=1,shadow xshift=.8ex,
shadow yshift=-.4ex,opacity=.7,fill=black!50,top color=red!90!black!50,
bottom color=red!80!black!80,draw=red!50!black!50,very thick,text=white,
text opacity=1,minimum width=3cm,font=\bfseries\sffamily] at (28,0) {Выход};
}
}%%%---|
\end{textblock}
%%%----------------------------------------------------------------------|
\vglue -18pt
\hspace{187pt}
\parbox{350pt}{%
\hypertarget{mymedia}{\hyperlink{preface}{%
\begin{tikzpicture}
  \colorlet{even}{cyan!60!black}
  \colorlet{odd}{orange!100!black}
  \colorlet{links}{red!70!black}
  \colorlet{back}{yellow!20!white}
  \tikzset{
    box/.style={
      minimum height=15mm,
      inner sep=.7mm,
      outer sep=0mm,
      text width=120mm,
      text centered,
      font=\small\bfseries\sffamily,
      text=#1!50!black,
      draw=#1,
      line width=.25mm,
      top color=#1!5,
      bottom color=#1!40,
      shading angle=0,
      rounded corners=2.3mm,
      drop shadow={fill=#1!40!gray,fill opacity=.8},
      rotate=0,
    },
  }
  \node [box=even]{{\huge\textbf{Мультимедийные лекции}}};
\end{tikzpicture}
}}}\\
\begin{flushleft}
\parbox{520pt}{%
%%%---> Old version
%	\mbox{\includegraphics[scale=.99]{video_02.png}}\par
%%%----------------------------------------------------------------------|
\begin{tikzfadingfrompicture}[name=myvideoa]
	\node [text=transparent!1]
	{\fontsize{25}{25}\bfseries\sffamily\selectfont Видеол\emph{е}кция};
\end{tikzfadingfrompicture}
\begin{tikzfadingfrompicture}[name=myvideob]
	\node [text=transparent!1]
	{\fontsize{25}{25}\normalsize\bfseries\sffamily\selectfont полный курс};
\end{tikzfadingfrompicture}
\begin{tikzfadingfrompicture}[name=myvideoc]
	\node [text=transparent!1]
	{\fontsize{25}{25}\large\bfseries\sffamily\selectfont Административное право};
\end{tikzfadingfrompicture}
\begin{tikzpicture}
	\node {%
	\mbox{\includegraphics[scale=.99]{video_02.png}}	
	};
	\node[rectangle, inner sep=2pt, very thick, draw=black,
	inner color=transparent!80, outer color=transparent!30,
	draw opacity=0.8, fill opacity=0.2, line width=1.6pt,
	minimum width=16.48cm, minimum height=51.4ex
	] at (-0.32,0.26) {\mbox{}};
%	\shade[path fading=myvideoa,fit fading=false,
%	left color=blue!80!black,right color=black]
%	(-15.8,-1) rectangle (3.7,1);
%	\shade[path fading=myvideob,fit fading=false,
%	left color=blue!80!black,right color=black]
%	(-2.8,-1) rectangle (2.7,1);
%	\shade[path fading=myvideoc,fit fading=false,
%	left color=red!80!black,right color=black]
%	(-2.8,-1) rectangle (2.7,1);
	\node [%
	minimum height=65mm,
	inner sep=.7mm,
	outer sep=0mm,
	text width=65mm,
	text centered,
	font=\huge\bfseries\sffamily,
	text=white!100!black,
	line width=.25mm,
	rounded corners=3.3mm,
	%rounded corners=2.3mm,
	shading angle=45,
%	draw=orange!100!black,
%	top color=orange!100!black!5,
%	bottom color=orange!100!black!40,
%	drop shadow={fill=orange!100!black!40!gray,fill opacity=.8},
	draw=gray,
	inner color=transparent!80,
	outer color=transparent!30,
	draw opacity=0.8,
	fill opacity=0.2,
%	opacity=1,
	text opacity=1,
%	fill=blue!65!black
	] at (-4.7,0.5) {Видеолекция\\[-13pt]
	{\small 02:45:59}\\
	{\normalsize <<Административное право>>}\\[-13pt]
	{\small полный курс}\\
	{\normalsize Горбачев Борис Иванович}\\[-13pt]
	{\small Доцент, кандидат юридических наук}
	};
\end{tikzpicture}\par
\begin{center}
\parbox{420pt}{%
	Видеолекция в формате \textbf{AVI} по предмету
	<<\textcolor{magenta}{\bf Административное право}>>,
	лекция была записана в течении семестра с сентября по
	декабрь 2010 года, и смонтирована в единый файл.
	Для просмотра, нажмите мышкой по центру окна Видеолекция
	полный курс.}
\vglue 25pt
\parbox{420pt}{%
Аудиолекция по предмету <<\textcolor{magenta}{\bf Административное право}>>
с комментариями, звуковое сопровождение Капланов Олег Георгиевич,
в формате \textbf{MP3}, продолжительность \textbf{8} часов \textbf{43}
минуты. Для прослушивания лекции подвидите мышку к ссылке
\textbf{Аудиолекция}, и нажмите на нее.
Аудиолекция воспроизводится автоматически в фоновом режиме, вы можете
одновременно прослушивать лекцию и работать с интерактивными документами.
}
\vglue 25pt
\mbox{%
\begin{tikzpicture}[rounded corners,ultra thick]
	\node [rectangle,very thick,
		bottom color=blue!80!black!30,
		top color=white,
		draw=blue!50!black!50,
		minimum width=9cm,
		drop shadow,
		] at (0,0) {%
		\textcolor{black}{\textbf{Аудиолекция в формате MP3}}};
	\pattern [path fading=north] ;
	\pattern [top color=transparent!50,bottom color=transparent!50,color=transparent!20] ;
\end{tikzpicture}
}
\vglue 25pt
\parbox{420pt}{%
Электронный навчально-методичний посібник по предмету
<<\textcolor{magenta}{\bf Административное право}>>,
с автоматическим перелистованием страниц. Подвидите мышку к кнопке
<<\textbf{Интерактивный навчально-методичний посібник}>>,
и документ будет отображатся с задержкой для каждой страницы
\textbf{30 секунд}.
}
\vglue 25pt
\mbox{}\hspace{5pt}\mbox{1}{43}\par
\begin{tikzpicture}[rounded corners,ultra thick]
	\node [rectangle,very thick,
		bottom color=blue!80!black!30,
		top color=white,
		draw=blue!50!black!50,
		minimum width=9cm,
		drop shadow,
		] at (0,0) {%
		\textcolor{black}{\textbf{Интерактивный навчально-методичний посібник}}};
	\pattern [path fading=north] ;
	\pattern [top color=transparent!50,bottom color=transparent!50,color=transparent!20] ;
\end{tikzpicture}
}{1}{10}
}
\end{center}
}
\hspace{48pt}
\parbox{410pt}{%
\begin{tikzpicture}
\node (tbl) {
\begin{tabularx}{400pt}{rccccc}
\arrayrulecolor{purple}
 & \textcolor{white}{\bf Автор} & \textcolor{white}{\bf Название} & \textcolor{white}{\bf WMV}&
	\textcolor{white}{\bf MP3}&\textcolor{white}{\bf PDF}\\[0.5ex]
 & & & & &\\
	& \parbox{65pt}{\includegraphics[scale=.69]{faces_01.jpg}}
	& \parbox{175pt}{Арлинский Юрий Моисеевич\par Зав. кафедрою, професор,\par
		доктор фіз.-мат. наук\par
			Функціональний аналіз} & 
	\mbox{\includegraphics[scale=.69]{wmv.png}} &
	\mbox{\includegraphics[scale=.69]{mp3.png}} &
	\mbox{\includegraphics[scale=.69]{pdf.png}}\\
  &  & & & &\\
	& \parbox{65pt}{\includegraphics[scale=.69]{faces_02.jpg}}
	& \parbox{175pt}{Балицька Татьяна Юревна\par Доцент, кандидат технічних наук\par
	Математичний аналіз} & 
	\mbox{\includegraphics[scale=.69]{wmv.png}} &
	\mbox{\includegraphics[scale=.69]{mp3.png}} &
	\mbox{\includegraphics[scale=.69]{pdf.png}}\\
  & & & & &\\
	& \parbox{65pt}{\includegraphics[scale=.69]{faces_03.jpg}}
	& \parbox{175pt}{Кочевський А.О.\par Доцент, кандидат технічних наук\par
		Чисельні методи} & 
	\mbox{\includegraphics[scale=.69]{wmv.png}} &
	\mbox{\includegraphics[scale=.69]{mp3.png}} &
	\mbox{\includegraphics[scale=.69]{pdf.png}}\\
  & & & & &\\
	& \parbox{65pt}{\includegraphics[scale=.69]{faces_04.jpg}}
	& \parbox{175pt}{Щестюк Н.Ю.\par Доцент, канд. фіз.-мат. наук\par
		Теорія функцій комплексної змінної} & 
	\mbox{\includegraphics[scale=.69]{wmv.png}} &
	\mbox{\includegraphics[scale=.69]{mp3.png}} &
	\mbox{\includegraphics[scale=.69]{pdf.png}}\\
  & & & & &\\
	& \parbox{65pt}{\includegraphics[scale=.69]{faces_05.jpg}}
	& \parbox{175pt}{Деордиця Ю.С.\par Професор, кандидат технічних наук\par
		Економiко-математичне моделювання} & 
	\mbox{\includegraphics[scale=.69]{wmv.png}} &
	\mbox{\includegraphics[scale=.69]{mp3.png}} &
	\mbox{\includegraphics[scale=.69]{pdf.png}}\\
  & & & & &\\
	& \parbox{65pt}{\includegraphics[scale=.69]{faces_06.jpg}}
	& \parbox{175pt}{Чабанова Т.Д.\par Доцент, кандидат педагогічних наук\par
		Методика викладання математики} & 
	\mbox{\includegraphics[scale=.69]{wmv.png}} &
	\mbox{\includegraphics[scale=.69]{mp3.png}} &
	\mbox{\includegraphics[scale=.69]{pdf.png}}\\
  & & & & &\\
\end{tabularx}};

\begin{pgfonlayer}{background}
\draw[rounded corners,top color=red,bottom color=black,draw=white]
	($(tbl.north west)+(0.14,0)$) rectangle ($(tbl.north east)-(0.13,0.9)$);
\draw[rounded corners,top color=white,bottom color=black,
	middle color=red,draw=blue!20] ($(tbl.south west) +(0.12,0.5)$)
		rectangle ($(tbl.south east)-(0.12,0)$);
\draw[top color=blue!1,bottom color=blue!20,draw=white]
	($(tbl.north east)-(0.13,0.6)$) rectangle ($(tbl.south west)+(0.13,0.2)$);
\end{pgfonlayer}
\end{tikzpicture}
}
\end{flushleft}
%%%----------------------------------------------------------------------|

